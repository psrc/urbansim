%% $Id: yp.tex,v 1.17 2003/06/02 02:58:39 borning Exp $

%% Tempest paper, converted to use IEEE latex macros

\documentclass[times, 10pt,twocolumn]{article} 
\usepackage{latex8}
\usepackage{times,url}

\newcommand{\tight}{\itemsep 0pt}

%------------------------------------------------------------------------- 
% take the % away on next line to produce the final camera-ready version 
% for IEEE publication
% \pagestyle{empty}

%------------------------------------------------------------------------- 
\begin{document}

\title{YP and Urban Simulation: Applying an Agile Programming Methodology \\
in a Politically Tempestuous Domain}

\author{Bjorn Freeman-Benson and Alan Borning \\
Department of Computer Science \& Engineering \\
University of Washington, Box 352350 \\
Seattle, Washington 98195 \\
\{bnfb,borning\}@cs.washington.edu}

\maketitle
\thispagestyle{empty}

\begin{abstract}
YP is an agile programming methodology that has evolved over the past 15 years.
Many of its features are common to other agile methodologies; its novel
features include using a highly visible, physical software status indicator
(a real traffic light), and a well-defined nested set of development
cycles.  It is also an exceptionally open process, with the current status
of the development process visible to the customers, as well as the code
and documentation.  We are using YP in developing the software for
UrbanSim, a sophisticated simulation system for modeling urban land use,
transportation, and environmental impacts over periods of 20 or more years
under alternate possible scenarios.  Our purpose in developing UrbanSim is
to support public deliberation and debate on such issues as building a new
light rail system or freeway, or changing zoning or economic incentives, as
well as on broader issues such as sustainable, livable cities, economic
vitality, social equity, and environmental preservation.  The domain of use
is thus politically charged, with different stakeholders bringing strongly
held values to the table.  Our goal is to not favor particular stakeholder
values in the simulation or its output, but rather to let different
stakeholders evaluate the results in light of what is important to them.
There are several implications of this for the development process.  First,
having credible, reliable code is important --- and further, both the code
itself and the development process that produced it should be open and
inspectable, not a black box.  Second, to allow us to respond quickly to
different stakeholder values and concerns, a flexible agile development
process is required.
\end{abstract}

% comment this out to produce the camera-ready version
\begin{tabular}{|c|} \hline
Preprint -- to appear in \\
\emph{Proceedings of the} \\
\emph{2003 Agile Development Conference} \\
Salt Lake City, June 2003 \\
\hline
\end{tabular}

\newpage

\section{Introduction}
\label{introduction}

\begin{quote}
\begin{center}
[\emph{On a ship at sea:}] \emph{a tempestuous noise of thunder and lightning
heard} \\
\emph{Enter a {\rm Ship-Master} and a {\rm Boatswain}}
\end{center}

\emph{Master.} B{\sc oatswain}! \\
\emph{Boatswain.} Here, master; what cheer? \\
\emph{Master.} Good; speak to the mariners.  Fall to 't, 
yarely\footnote{yarely, \emph{adv.} Quickly, promptly; nimbly, briskly.
\cite{oed-2003}},
or we run ourselves aground.  Bestir, bestir.  \emph{Exit.}

\begin{center}
\emph{Enter} Mariners
\end{center}

\emph{Boatswain.} Heigh, my hearts!  cheerly, cheerly, my hearts! yare, yare!
Take in the topsail.  Tend to the master's whistle.---Blow till thou burst
thy wind, if room enough!

\begin{center}
William Shakespeare, \emph{The Tempest}, Act I Scene I
\end{center}
\end{quote}

Yare Programming (YP) is an agile programming methodology that has evolved
over the past 15 years.  It was originally developed in industrial
settings, and is now being used in a major academic software development
effort.  Many of its features are common to other agile methodologies; some
are unique.  We describe some of the more interesting (such as the traffic
light) in Section \ref{features}.  Then, in Section \ref{urbansim}, we
describe our application domain, and in Section \ref{implications} the
implications of working in such a politically tempestuous domain for the
development process, including our adoption of exceptionally open process.
Our development is done in Java; but the methodology is applicable to a
variety of languages.

\section{Features of the YP Methodology}
\label{features}

\subsection{Testing}
\label{testing}

\begin{quote}
\emph{Prospero.} 
All thy vexations \\
Were but my trials of thy love, and thou \\
Hast strangely stood the test. \\
\hspace*{1cm} Act IV Scene I
\end{quote}

As in other agile development processes, we employ test-first development
with an extensive battery of tests, including both unit tests and
acceptance tests.   The unit tests are written using 
JUnit (\url{www.junit.org})
and Ward Cunningham's Framework for Integrated Tests (FIT, at
\url{http://fit.c2.com/}).
There are two dimensions of testing for the project: who
writes the tests, and how often the tests are run.

The tests could be written by either the clients or by the developers.  We
have tried to encourage the clients to write tests as part of their feature
specification process, but have had less success than we desire.  We have
incorporated tools such as FIT
to make it easier for users to read and write tests, but have still
not achieved the agile ``holy grail'' of having our clients write tests as
executable specifications.
We have had more success using Ward Cunningham's technique of sitting
side-by-side with the clients to develop the acceptance tests.  We ask
gentle questions about what a good test might cover, thus prompting the
client to think about and explain what the correct inputs and outputs for
such a test would be.  Then we do the actual typing to create the test
using the FIT framework.

There is clearly a social barrier that we are approaching with our requests
for acceptance tests from the clients.  In each application of YP to date,
the clients have seen the value of having the tests, and have been willing
to help generate them, but they have not been willing to create them on
their own.  Developing better processes and technical support to help
overcome this reluctance is an important area for future research.

The second dimension of testing is how often the tests are run.  In a
perfect world, tests would take neglible time, and thus all the tests could
be run all the time.  In the real world, some tests are quick and others
are very time consuming.  Thus, YP divides the tests into two suites:

\begin{itemize}
\tight

\item The \emph{Commit-Level Tests} are those run every time someone
commits code to the repository. These are the tests considered ``most
likely to fail.''  Another future research area for YP is to incorporate an
automated ``most likely'' determination system, such as the one developed
by Microsoft Research \cite{gates-oopsla-2002}.

\item The \emph{Nightly Tests} consider the other complicated cases that
would be tedious and time consuming to run during the course of ordinary
development.  This category includes such tests as the database driver
tests, and others that aren't likely to be affected by small changes to
the code base or test data.  The complete set of all tests are run
automatically each night.

\end{itemize}

YP follows XP by insisting on always having a working, releasable system.
Developers work by repeatedly taking small, testable steps, rather than
large ones.
One of the benefits of an extensive, and automatically checked, regression
suite is to ensure exactly that: a working system in which features that
were working do not suddenly stop working.
As with other agile processes, YP uses the process equivalent of 
programming-by-contract.  The contract is that the test suite (equivalent
to the assertions) succeed at the beginning and the end of each unit of 
work (equivalent to a method).  Thus if the suite fails after a change,
the developer can be sure that it was his or her modifications that caused
the problem.

YP also follows XP by using Test First programming.  The benefits of
writing the tests first rather than second are well described elsewhere.
We have experimented with both techniques in settings in which we have been
using YP, and then measured the defect rates.  The survey results strongly
supported using Test First, and so we have adopted this methodology
throughout our process.

\subsection{Framework for Integrated Tests}

\begin{quote}
\mbox{\emph{Caliban.} You taught me language; and my profit on 't} \\
Is, I know how to curse. The red plague rid you \\
For learning me your language! \\
\hspace*{1cm} Act I Scene II
\end{quote}

Two of the problems we have encountered in creating extensive test suites are:
\begin{itemize}
\tight
\item Writing the test scaffolding and objects takes a significant amount
  of time; and 
\item Java code is not easy for non-programmers to read.
\end{itemize}

To mitigate these problems, we incorporated the Framework for Integrated
Tests (FIT), as described above.  This has allowed us to write
\emph{literate tests}.  Literate tests are similar to literate programming
in that they mix descriptive prose and executable tests.  Both prose and
tests are written in HTML and displayed with a standard browser.  The tests
are written as HTML tables: in our case, each row of the table defines an
action to be performed, such as ``use this input database'' or ``expect the
following errors.''  Some of table rows contain nested tables that define
input or output objects.

Because the FIT tests are HTML pages, the tests are easy for our customers
to read and understand.  Additionally, through the use of a WYSIWYG HTML
editor and the simple table structure, the FIT tests are easy to create and
modify.  Our input and output simulation objects are sets, arrays,
dictionaries, and other very regular structures, so nested HTML tables have
proven to be a much easier mechanism for describing them than Java
constructors and methods in JUnit.

\subsection{The Traffic Light}

We display the results of our testing very visibly: using a physical
traffic light (a used one from a transportation department surplus
program\footnote{We purchased our traffic lights from Scott Signal
Company (\url{www.trafficlights.com/scottsig.htm).}}).  
Actually there are
several: one in the hallway of our laboratory, two more in developer
offices, and one virtual traffic light on the web.  We use four color
combinations: green when the build and all the tests have succeed, yellow
when the build and tests are in progress, yellow and green together when
the build has passed the point where it is likely to fail (in practice,
this means that all the tests have passed and that the final installer and
distribution are being produced), and red if any part of the build or tests
has failed.  The lights are controlled by our Tinderbox-like continuous
build script (named Fireman)\footnote{There are many equivalent
continuous build
systems; we are currently switching from our internal tool (Fireman) 
to the open source
Cruise Control system \url{http://cruisecontrol.sourceforge.net/}.}, 
which sends TCP packets to selected PCs. Each
of physical traffic lights is connected to one of the PCs using a Weeder
Technologies (\url{www.weedtech.com}) WTSSR board attached 
to the serial
port. Each PC runs a simple server (written in Visual Basic) that waits for
packets (e.g., ``+yellow'') and then sends serial control information
(e.g., ``AW00100'') to the WTSSR board.  Because the physical traffic
lights are controlled via TCP, we can easily support a geographically
distributed team.

The traffic light is a powerful symbol of the current state of the software.
The web version at \url{www.urbansim.org/fireman} makes the status visible to
anyone else who might be interested (including you, Gentle Reader, if you
wish), and in particular for our customers, although in a less compelling
way than the physical device.  The physical traffic light has a number of
advantages over the web version or other status indicators: 
\begin{itemize}
\tight
\item It is a push technology, and thus the stakeholders don't have to 
remember to check the status - it just \emph{there}.
\item It uses culturally familiar set of colors, and communicates with the
stakeholders on both 
a conscious and a subconcious level, reassuring them that all is well (green),
and strongly suggesting a repair action when all is not (red).

\pagebreak

\item Seeing, and due to our implementation using relays, hearing, frequent
cycles in a day gives a good feeling of progress.
\end{itemize}

We have used the traffic light as part of applying YP in four
other development projects as well. Interestingly, only one team in
addition to ours is still using the light --- the other teams disconnected
their lights because they didn't like them being red so much of 
the time.  Rather
than fixing the underlying problem (that their system was sufficiently
unstable that their regression tests would not pass reliably), they chose
to ``eliminate the messenger.''  Only one of the teams acknowledged this
decision explicitly.  We plan to do a user study across multiple
industrial and academic projects to investigate this interesting, and
apparently counter-productive, behavior.

YP is by no means the first process to use a physical project status
indicator, but \emph{is} probably the first to
use a traffic light for this purpose.  In conversations with
other team leads and software managers, we learned that a number of them
had tried ``failure indicators,'' such as sirens, flashing lights, red
lights, and so forth, and that each had cancelled the experiment after a
few days.  Apparently the use of negative reinforcement (a red light)
without the corresponding positive reinforcement (a green light) was too
damaging to morale.  Our experience with the traffic light is quite the
opposite: everyone who joins a YP team immediately reports a sense of
comfort at the large (8-12'') green light glowing its message of ``all is
well with the build.''  Or on the rare occasions when the software has
failed the nightly build tests, as the staff arrives in the morning, in the
winter's gloom, with the lab hallway illuminated by the red glow of the
traffic light, it's clear that a) something has gone wrong, and b) it
should get fixed as soon as possible.  Another possible difference between
our use of the traffic light and other projects' uses of indicators is that
we acknowledge that mistakes happen and 
thus do not penalize a (repaired) red light.
This approach differs from the standard "build breakage penalty" 
(donuts, dunce caps, etc.) that many development organization utilize.

\subsection{Iteration Planning and the Dashboard}

\begin{quote}
\emph{Prospero.} 
Thou shalt be as free \\
As mountain winds; but then exactly do \\
All points of my command. 

\emph{Ariel.} To the syllable. \\
\hspace*{1cm} Act I Scene II
\end{quote}

As is common with agile processes, the YP planning process uses nested
iterations:

\pagebreak

\begin{itemize}
\tight
\item Long-Term Cycle (Multi-Year) 
\item Release Cycle (Three Month) 
\item Iteration Cycle (Two Week) 
\item Daily Cycle (One Day) 
\item Task Cycle (Few Hours) 
\end{itemize}

Each of the cycles has checklists, duties, and roles associated with
it.  For example, the Task Cycle includes ``synchronize with the CVS
repository, run the tests, and write a test-first test for the task.''
The Daily Cycle includes ``check the dashboard, verify any tasks that have
been resolved, and start a new task.''
We use two week iterations: long enough to provide stability 
for the developers to avoid thrashing, but also short enough to provide
flexibility for the modelers to adjust the end point without wasting effort.
As is common, we freeze the iteration task lists and specifications to prevent
creeping features from distracting the developers.

We use Taskzilla, a locally-modified version of the Bugzilla
defect-tracking system (\url{www.bugzilla.org/}), to keep track of work
requests: not only bugs, but also enhancements, administrative requests,
and other tasks.  As with the traffic light, Taskzilla is visible to anyone
who browses our website.  Our customers, as well as members of our own
staff, enter bugs and work requests into it.  We then use a Wiki page with
a table to record the tasks assigned to current and future iterations.
Finally, we use a number of cron scripts to merge the Taskzilla and Wiki
information together and update our project status ``dashboard'' web pages.

Originally, only the developers used Taskzilla, but now all the team
members (including the modelers and HCI group) make use of it.  This is of
course useful for group coordination, but has also been important in
fostering our value of cooperation (see Section \ref{implications}).

The dashboard has been used in industrial settings as well, as part of
applying YP there.  Our experience with it has been similar to our traffic
light experience: three-quarters of the companies that had eagerly adopted
a dashboard turned it off after a few months because they did not want the
status of the project to be that visible.

One of the missing bits in our tool set is that our work request system
(Taskzilla) does not have a hierarchical request structure to match our
nested iterations.  Thus, while we have a good tracking system for
Iterations, we do not have an equivalent one for Releases or other
longer-term planning.  We believe that this lack of visibility into the
tracking is a direct cause of the relative de-emphasis of long-term planning
in each of five YP instances to date.

% sigh ... I guess this has to go ...
%
% \subsection{Software Reuse}
% 
% \begin{quote}
% \emph{Prospero.} 
% My library was dukedom large enough \ldots \\
% \hspace*{1cm} Act I Scene II 
% \end{quote}
% 
% One of the project's development aphorisms is ``write less code!''  When a
% high-quality, open-source software solution is available that solves a
% problem, we make every effort to use that, rather than writing our own.
% The result is less code to write, test, and maintain.

\pagebreak

\subsection{Student Involvement and Turnover}

\begin{quote}
\mbox{\emph{Prospero.} Our revels now are ended.  These our actors,} \\
As I foretold you, were all spirits, and \\
Are melted into air, into thin air \ldots \\
\hspace*{1cm} Act IV Scene I
\end{quote}

The development team includes two professional software engineers, plus
(currently) three undergraduate majors in Computer Science \& Engineering.
The traditional XP methodology assumes that all developers are more or less 
equal.  In a university environment, however, it is 
important to involve undergraduates
in research, and while they are bright, enthusiastic, and hard-working,
they don't have the depth of experience of the senior developers.  Our
development processes accomodate this well --- we have had excellent
success with undergraduates working on the project, and the experience has
been beneficial both for the project and for the undergraduates and their
education.  

Another attribute of undergraduates is, of course, that they come and go,
both on a daily basis (they have classes to attend) and on a yearly basis
(they graduate).
However, unlike in the play (``And, like this insubstantial pageant faded,
Leave not a rack behind'') it is essential that all team members
participate in maintaining a solid, well-documented and tested code base
that others may build on, so that when an undergraduate graduates and moves
on, his or her contribution remains.  
We handle this bright, inexperienced, and transient workforce with 
a more traditional Chief Programmer or Animal Farm\footnote{All developers are
equal, but some are more equal than others.} team organization rather than the
XP pair programming organization.  The two staff members provide
mentoring and design guidance to the rest of the team, as well as
periods of pairing, but the intermittent schedules do not permit
full-time pairing.  We have compensated by including mandatory 
pre-checkin design and code reviews, wherein an experienced developer 
is given a tour of the design, documentation, tests, and code.  Because
the review is done just before checkin,
all the tests should already have passed.
The reviewer uses a checklist, and they pair-browse the code base using the 
Eclipse comparison tool.

We have made some informal comparisons between the review-based technique
and full-time pairing (using both our own prior experience with full-time
pairing, as well as that of colleagues).  Unfortunately, these comparisons
indicate that our reviews do not ensure the same architectural integrity,
code knowledge and quality as full-time pairing.
We believe some of the reasons for this lower quality include the lack of time 
(less than an hour a day versus eight hours a day), the context changes 
(the reviewer is typically working on his or her own task as well), and 
the lack of ownership (the reviewer does not have the same emotional 
investment in getting it right).
However, in spite of these shortcomings, the reviews seem to be a
reasonable compromise given the constraints of our academic setting.

\subsection{Refactoring}

\begin{quote}
\emph{Trinculo.} What have we here?  A man or a fish?  Dead or alive?  A fish;
he smells like a fish; a very ancient and fish-like smell \ldots \\
\hspace*{1cm} Act II Scene II 
\end{quote}

As in other agile methodologies, an important aspect of our development
work is nearly-continuous refactoring of the code, to maintain quality and
to avoid ancient and fish-like smells of any kind.\footnote{N.B.:
unfortunately, this does not extend to our current physical space.  (Note
to self: remember to speak to landlord about strange odor in back room.)}
This refactoring is, of course, integrated with our testing methodology and
with the philosophy of taking small steps.

We end up doing more refactoring than other projects of comparable size,
due to our staff attributes and turnover.  However, because we consider
refactoring to be a first-class citizen in the planning process, the
long-term effect has been to keep the code comprehensible by non-programmers.
One of the important goals of the UrbanSim project is to provide a 
transparent system (see below), and thus providing not only a 
valid simulation, but also the correct set of abstractions that
are understandable by the customers.

\subsection{Status Meetings and Today Email}

YP, like XP, eschews status meetings.  Instead, we report and plan
with two different mechanisms.  First, the daily ``Stand Up'' is a time to
briefly discuss what one is planning to do that day, and to coordinate with
others on those tasks.  The stand up meeting is not used for status: it is
a forward looking meeting.  (It is a stand-up meeting to keep it short ---
if someone is leaning on the wall or sitting down, this indicates the
meeting is going on too long.)  Second,
at the end of each day, a daily `today' email is used for status.
Each person sends a brief email to the entire group describing what they
accomplished that day.  The idea is that group status is essential to know,
but boring to listen to.  Since reading is faster than
listening, we use short descriptive email status messages rather than
status meetings.  Additionally, the `today' emails are low overhead and can
be archived and reviewed, unlike the discussions in status meetings.  

In our team, `today' messages were initially used just by the developers,
but have now spread to all project participants.  In addition to their use
as a status report, they had the unexpected additional effect of increasing
group task awareness at very low cost.  We discuss this phenomenom
(including survey results) in a short paper \cite{brush-today-msgs-2002}.

\subsection{Making It All Work}

\begin{quote}
\emph{Sebastian.}
Look, he's winding up the watch of his wit; by and by it will strike. \\
\hspace*{1cm} Act II Scene I
\end{quote}

One aspect of software development that YP does not solve nor, as far as we
can tell, does any other agile method solve, is that it's still work.  Keeping
the process live and healthy requires continual diligence.  Just because YP is
an agile process does not mean that it's easy, nor a magic bullet for software
development. We still require a strong process leader and/or coach to keep the
team on track and to evolve the process to meet the project's changing needs.

\section{UrbanSim}
\label{urbansim}

Patterns of land use and available transportation systems play a critical
role in determining the economic vitality, livability, and sustainability
of urban areas.  Transportation interacts strongly with land use.  For
example, automobile-oriented development induces demand for more roads and
parking (which in turn induces more automobile-oriented development), while
compact, pedestrian-friendly urban environments can induce more walking and
demand for transit.  Both land use and transportation have strong
environmental effects, in particular on emissions, resource consumption,
and conversion of rural to suburban or urban land.

The process of planning and constructing a new light rail system or freeway,
setting an urban growth boundary, changing tax policy, or modifying zoning
and land use plans is often politically charged.  Strong technical support
can play a critical role in fostering informed civic deliberation and
debate on these issues, as well as on broader issues such as sustainable,
livable cities, economic vitality, social equity, and environmental
preservation.  We want urban planners and stakeholders to be able to
consider different scenarios --- packages of possible policies and
investments --- and then, based on these alternatives, model the resulting
patterns of urban growth and redevelopment, of transportation usage, and of
resource consumption and other environmental impacts, over periods of
twenty or more years.

\subsection{Technical Characteristics}

UrbanSim \cite{waddell-japa-2002,waddell-nse-2003} is a simulation system
that is intended to provide such technical support.  It performs
simulations of urban development, including transportation, land use,
environmental impacts, and their interactions.  It is a moderate-sized Java
program (around 60,000 lines of code), and is distributed as Open Source
software (available from \url{www.urbansim.org}).  UrbanSim consists of a
set of interacting component models that simulate different actors or
processes within the urban environment.  For example, the Residential
Location Choice model simulates the process of household location --- of a
household deciding whether to rent or buy a dwelling, what kind (detached
house, townhouse, apartment, etc.), and in what part of the city.  This
choice is modeled using a logit model.  This is a probabilistic decision
tree, in which the probability of a given choice is determined by both
household characteristics (number of workers, number of children, household
income, etc.), and the characteristics of a potential site.  These
probabilities are calibrated to observed data.  Another model is the
Developer Model, which simulates the activities of real estate developers
(not software developers!) as they decide whether to develop new housing or
commercial space, or redevelop existing space.

Although the total code size is moderate, many of the component models are
complex conceptually and mathematically, drawing on results from urban
economics, sociology, civil engineering, and other disciplines.  The system
is still undergoing considerable development, with the developers working
in close collaboration with the domain experts.  To support rapid software
evolution and development, these component models are written as
independently as possible, and communicate via a shared database rather
than by direct method invocation \cite{noth-ceus-2003}.

\subsection{Applications}
\label{applications}

To date, UrbanSim has been applied in the metropolitan regions that include
Eugene/Springfield, Oregon; Honolulu, Hawaii; Houston, Texas; and Salt Lake
City, Utah; application to the Puget Sound region (which includes Seattle,
Washington) is under way.  As part of the process of validating the model,
we also performed a historical validation with the Eugene data, starting
UrbanSim with the 1980 data, simulating until 1994, and then comparing with
what actually happened.

In the United States, local or regional government planning agencies are
typically charged with performing travel demand forecasting and land use
modeling and forecasting.  Each metropolitan region over a given size is
required to have an identified ``Metropolitan Planning Organization'' (MPO)
to receive federal funding.  For example, in the Seattle area the MPO is
the Puget Sound Regional Council; in the Salt Lake City area it is the
Wasatch Front Regional Council.  These MPOs are our primary customers.

UrbanSim has been recently been brought into the middle of a land use and
transportation dispute in Salt Lake City.  A new freeway had been planned
for the region, and after years of controversy, construction was imminent.
In response, two environmental groups (the Sierra Club and Utahns for
Better Transportation), joined by the mayor of Salt Lake City, brought a
lawsuit, wherein they claimed that the potential land use and environmental
impacts of the proposed freeway had not been adequately evaluated, as
required by law.  In a June 2002 out-of-court settlement, all parties
agreed to test the application of UrbanSim in the region, so that it could
be used, for example, to model building or not building projects such as
the freeway.  Given pressures of this kind, reliable, high-quality,
credible software is essential.

\subsection{Using Value Sensitive Design}
\label{vsd}

Clearly, the application domain is politically charged.  Different
stakeholders, such as business owners, members of advocacy or
neighborhoods, elected officials, and planners, as well as other citizens
of the region, may bring to the table widely divergent values about land
use, transportation, and environmental impacts.  To handle these issues in
a comprehensive way, we are applying the emerging technique of Value
Sensitive Design \cite{friedman-tr-2002}.  Value Sensitive Design is a
theoretically grounded approach to the design of technology that accounts
for human values in a principled and comprehensive manner throughout the
design process.

For UrbanSim, we distinguish between \emph{explicitly supported values}
(i.e., ones that we explicitly want to embed in the simulation) and
stakeholder values (i.e., ones that are important to some but not
necessarily all of the stakeholders).  Three explicitly supported values to
which we have committed are fairness, accountability, and democracy.
Examples of stakeholder values are environmental sustainability, walkable
neighborhoods, space for business expansion, affordable housing, freight
mobility, minimal government intervention, minimal commute time, open space
preservation, property rights, and environmental justice.  In contrast to
the explicitly supported values, these stakeholder values may often be in
conflict.  

In more detail, one explicitly supported value is fairness, and more
specifically freedom from bias.  The simulation should not discriminate
unfairly against any group of stakeholders.  A second is accountability.
Insofar as possible, stakeholders should be able to confirm that their
values are reflected in the simulation, evaluate and judge its validity,
and develop an appropriate level of confidence in its output.  The third is
democracy.  The simulation should support the democratic process in the
context of land use, transportation, and environmental planning.  In turn,
as part of supporting the democratic process, we decided that the model
should not a priori rule out any one set of stakeholder values, but
instead, should allow different stakeholders to evaluate the alternatives
according to the values that are important to them.

A major issue is how to present the results of the simulation in useful
ways for different stakeholders, particularly in light of widely divergent
values about land use, transportation, and environmental impacts.  We are
relying heavily on the use of \emph{indicators} for this purpose: numeric
quantities that concisely distill attributes of concern about a situation.
For example, for stakeholders interested in open space, an obvious
indicator is the percentage of open space in a given area.  For
stakeholders concerned about environmental impacts, an indicator of air
quality could be the number of days per year that air quality falls below
EPA minimum requirements.  For stakeholders concerned about freight
mobility, an appropriate indicator might be the number of minutes of
congestion delay per year per ton of freight.  Other indicators are useful
in evaluating the effectiveness of policies.  For example, a pair of
indicators to help evaluate the effectiveness of an urban growth boundary
would be the population growth inside and outside the boundary.

Many of the technical choices in the design of the UrbanSim software are in
response to the need to generate indicators and other evaluation measures
that respond to different stakeholder values.  For example, for some
stakeholders, walkable, pedestrian-friendly neighborhoods are very
important.  But being able to model walking as a transportation mode makes
difficult demands on the underlying simulation, requiring a finer-grained
spatial scale than is needed for modeling automobile transportation alone.
In turn, being able to answer questions about walking as a transportation
mode supports two explicitly supported values: fairness (not to privilege
one transportation mode over another), and democracy (since it is an 
important value to a significant number of stakeholders).

\section{Implications for the Software Development Process}
\label{implications}

Given the high stakes and pressures of the kind described in Section
\ref{applications}, reliable, high-quality, credible software is essential.
To help establish credibility, \emph{repeatable} high quality is also
essential.

UrbanSim's software architecture is designed to support rapid evolution in
response to changed or additional requirements. For instance, the component
models are designed to be easily reconfigured.  Also, the system writes the
simulation results into an SQL database, so that we can conveniently query
it to produce new indicators quickly and as needed, as opposed to embedding
the indicator code directly in the component models.  Our YP development
process is similarly tuned to agility and flexibility in the face of
changing requirements.

We identify a set of important values for the development process:

\begin{description}

\item[openness and accountability] This is in support of our explicitly
  supported value of accountability (Section \ref{vsd}), and the need for
  credibility of the software.  To enhance the openness and accountability
  of our development process, we use technical artifacts described in
  Section \ref{features}, such the Taskzilla task management system, the
  traffic light, the dashboard, and the automated testing regime.  We 
  also make these visible via the web to our customers as well as to
  our development team --- for example, Taskzilla, the web version of the
  traffic light, and the dashboard are all available from our web page.

\item[collective ownership] This is important for at least two reasons.
  First, collective ownership mitigates the difficulty of developer
  turnover (particularly among students).  Second, it increases reliability
  and quality, by bringing more eyes on each part of the code.

\item[cooperation] This is essential in creating and maintaining a 
  productive, satisfying work
  environment.  Cooperation is primarily fostered by the actions and
  attitudes of the members of the team, but in some cases changes to the
  procedures and technical artifacts help, such as including tasks for all
  of the team in Taskzilla and in iteration cycle planning, not just those
  of the developers.  (This has helped foster cooperation, since now, for
  example, a problem can be entered as a bug in Taskzilla and then
  scheduled, without worrying whether it
  was due to a specification bug, an implementation bug, or some
  combination: it's just something that the team needs to fix.)

\end{description}

\subsection{Role of Open Source}

\begin{quote}
\emph{Prospero.}
Of temporal royalties \\
He thinks me now incapable \ldots \\
\hspace*{1cm} Act I Scene II
\end{quote}

The UrbanSim software is all Open Source, licensed under the GNU General
Public License \cite{gpl-web}.  Open Source licensing has significant
benefits for software reliability, robustness, and support for sharing and
collaboration, and is of course well-understood in the software development
community.  However, it is not yet a common model for land use or
transportation modeling software --- but we believe that it is quite
appropriate for this domain, in which the development effort is publicly
funded, the customers are government agencies, and the requirements for
openness and credibility are so strong.  Our intent is that allowing and
encouraging access to the model without proprietary restrictions will
stimulate rapid innovation and sharing among agencies.

As described in Section \ref{applications}, the immediate clients for
UrbanSim and systems like it are typically Metropolitan Planning
Organizations.  These MPO's often contract with consulting companies to
write or customize modeling software.  Launching an operational land use
and transportation model is a complex undertaking, Open Source software or
not, and will involve considerable work in preparing the data and
customizing the model to local conditions.  This means that an Open Source
platform, such as UrbanSim, can still provide a solid basis for a
service-oriented business model for a consultancy.

Open Source doesn't just mean that the code is available, but it also
implies that others can contribute to the project in a number of well
defined roles: Users, Developers, Committers, and Architects (as defined in
\url{www.eclipse.org/eclipse/eclipse-charter.html}); another role for our
project is Modeler.  (Also see e.g.\
\url{www.mozilla.org/about/roles.html} and \url{www.opensource.org/}.)

\subsection{Beyond Open Source}

YP is nicely suited to the ``All Open'' Development style of the UrbanSim
project.  By ``All Open,'' we mean not just Open Source, but Open
Everything:

\begin{description}

\item[Open Source] means that availability and access to the source code,
as well as the distribution rights (free redistribution, integrity of the
author's source code,
etc.)\footnote{\url{www.opensource.org/docs/definition.php}} are 
not compromised.

\item[Open Code] means that the source code is readable, documented, and
understandable.  Not only is the source code \emph{not} obfuscated (an Open
Source requirement), but that the source code is deliberately written to be
as understandable by as many people as possible.  (We are working on a
complex and difficult problem, so there are obviously limits here, but we
try to achieve this goal as well as we can.)  Open Code also means that
this principle of clarity is not compromised.

\item[Open Design] means that the design documents are available and easily
accessed.  Further, the design is deliberately created to be clear to as
large an audience as possible.

\item[Open Process] means that the process documents and status are also
available and easily accessed, and are (again) open, inspectable,
and written for clarity.  Open Process implies that such access cannot be
compromised.

\end{description}

YP has a number of attributes that make it an Open Process:

\begin{itemize}
\tight
\item The status of the project is open and available for all to see via
the project dashboard and the traffic light.
\item The future directions of the project are open and available via
the iteration and release plans on the website, as well as the
Taskzilla list of tasks and defects.
\item The history of the project, including all the dirty laundry of 
schedule slips and incorrect design decisions, is open and available via 
the website and Taskzilla.  We are planning to create tools and webpages
that will make it easier to examine the project history and plans, but all the
information is available even without these tools.
\end{itemize}

Additionally, the process itself is open to changes. For example, we originally
had only the most current successful build available on the download site.
Based on feedback from the MPOs, we changed the process to include both
stable builds and current builds
(see Section \ref{constant-change} that follows).
Further input has indicated that we need
to include three levels of builds: releases, stable builds, and current builds.
Thus an Open Process means that not only is the current, future, and
historical status available for view, but that the community can contribute
to the planning and the process.

\subsection{Social Difficulties With Constant Change}
\label{constant-change}

One of the advantages that XP, YP, and other agile development processes
have is the constantly updated and tested current build.  We provide our
latest successful build via a download site and an ``update'' menu item in
the installed product.  In the context of UrbanSim, we felt that this would
be a big benefit to our customers, since they could report a problem (for
example, a small bug in the simulation algorithm) one day and download a
patched version the next day.

However, we discovered that many organizations, including the Metropolitan
Planning Organizations who are our primary customers, prefer fewer
releases, and so we have provided that option.  We hypothesize that this
may be due to one or more of the following reasons.  First, they might
prefer a known to an unknown set of defects.  Second, their
internal process for evaluating a new release have a longer cycle time.  Or
third, they might have developed a distrust of software due to its general
unreliability, and are unaware of the extensive testing done on each
release.  We plan to gather data on this issue through empirical
investigation (semi-structured interviews and surveys), and plan to
report the results in a future paper.

\subsection{Efficiency Considerations}
\label{efficiency}

\begin{quote}
\emph{Prospero.}
Fetch us in fuel; and be quick; \ldots \\
\hspace*{1cm} Act I Scene II
\end{quote}

UrbanSim is a fine-grained simulation, with much more spatial detail than
other land use models.  This makes significant demands on our software,
both for execution speed and memory usage.  Despite these demands, we have
been satisfied with our decision to develop in Java rather than C or C++,
because of the gains in programmer productivity (in particular not having
to do explicit memory management).  However, some additional work has been
required, which has added to the programmer burden and the complexity of
the code; but the overall tradeoff has still supported our choice.

Execution speed has been a particular issue in running the logit
computations for the more complex component models, in particular for
Residential Location Choice and the Real Estate Developer models.  To
compensate, we make extensive use of caching to store intermediate logit
computations and reuse them.

Memory useage has also been an issue.  The fine-grained simulation creates
very large numbers of small objects.  In Java, each object has 10-20 bytes
of overhead, making it expensive to use small objects.  In a technique
developed by Michael Noth, we have handled this issue by ``exploding'' the
small objects into large parallel arrays.  For example, rather than
10,000,000 grid cell objects, each with an $x$ field, a $y$ field, and so
forth, we use an array with 10,000,000 floats to hold the $x$ values,
another array to hold the $y$ values, and so forth.  In our Version 1
implementation of Urbansim \cite{noth-ceus-2003}, this exploded object
representation was all-too-visible to the programmer --- we achieved a
substantial reduction in memory useage, but at a cost in code readability
and maintainability, and perhaps more importantly, decreased openness.

Note that exploded objects are not the same as the Flyweight pattern
\cite{gamma-book-1995}: the Flyweight pattern is a way to reduce storage
for the shared fields, whereas exploded objects are a way to reduce storage
for the instance specific fields.  For example, the Flyweight pattern would
share the storage for the grid cell \emph{width} and \emph{height} values
(common to all grid cells), but not the $x$ and $y$ values.  Exploded
objects do not share storage, but rather move the information to a location
where it can be stored more efficiently.

In the current implementation, we have retained the exploded object
representation, but have wrapped it in a much cleaner interface.  The array
implementation of exploded objects is no longer visible to the programmer.
Instead, programmers use a specialized iterator object, which returns a
real Java object for each invocation of {\tt next()}.  To avoid the
overhead of large amounts of object creation and garbage collection, this
object is reused for successive invocations of {\tt next()}.  Therefore,
the programmer should not keep other references to the object --- lest the
object be altered unexpectedly by another call to {\tt next()} --- but in
all other respects it can be used as any other Java object.  This technique
has worked reasonably well.

As a spinoff project, for his Ph.D. dissertation Michael Noth is developing
a Java language extension that supports such exploded objects with
convenient syntax and runtime support, which will make it more
straightforward to use this exploded object representation.

\section{Conclusion}

\begin{quote}
\emph{Miranda.} Your tale, sir, would cure deafness. \\
\hspace*{1cm} Act I Scene II
\end{quote}

YP (``one step beyond XP'') is an agile programming methodology that has
been slowly evolving over the past 15 years, first in several industrial
settings and most recently in a large academic software project.  It
includes a number of novel features (for example, the traffic light and the
today emails), and has proven effective in helping to manage a substantial
and rapidly changing development project.

We believe that it represents an interesting and useful point in the agile
process space.  The choices of what to include in --- and what to exclude
from --- YP were made deliberately to tailor the process to the particular
environment in which YP is being used: a team with some permanent and some
student members, developing software for a research system that
nevertheless must meet stringent quality and reliability requirements, and
that is being applied in politically charged environments.  Many of YP
practices are identical to XP practices: short iterations, continuous
integration, test first programming, collective code ownership, writing
documentation as part of development, and so on.  Some of the YP practices
differ from XP, such as code reviews rather than pair programming, and
using a online bug database rather than cards on the corkboard.

In addition to other agile methodologies, YP is also related the method used
in the M.A.D. project \cite{christensen-ecoop-1998}.
In that project, an interdisciplinary
team from Aarhus University in Denmark collaborated with a large shipping
company in developing a prototype for a global customer service system.
This project was notable in its adoption of a highly agile programming
methodology, and integrating this methodology with ethnography and
Participatory Design \cite{bjerknes-book-1987,greenbaum-pd-1991}.  As in
the UrbanSim project, the researchers comprised a multidisciplinary team,
which was simultaneoulsy concerned with software development methodology,
object-oriented design, and value issues (in this case arising from
Participatory Design and its tradition of workplace democracy).

YP has evolved in a number of interesting ways in the process of its
transition from industry to the academic environment, and in its current
application to a politically tempestuous domain.  First, the process has moved
toward increasing openness.  In its present form, the current build, the
status of the build, our task list, and our progress are all publicly
visible.  The intent with this strong visibility is to foster credibility
and confidence --- critical issues for our domain.  Second, the process has
expanded.  Originally, only the developers participated in it; now the
modelers and customers work with the same nested iteration cycles and
Taskzilla management system.  

We hope that others will be able to apply and adapt parts of the YP
methodology to their own domains.

\subsection*{Acknowledgments}

We would particularly like to thank each and every member of the UrbanSim
research team for their work on the project.  We would like to thank the 
anonymous referees and the onymous David Socha for their helpful comments:
you have helped make this a better paper.  The Earl of Oxford provided
apt quotations.  (Or maybe it \emph{was} Will himself; we are agnostic on
this question.)  This research has been funded in part by National Science
Foundation Grants EIA-0090832 and EIA-0121326.

\bibliographystyle{latex8}
\bibliography{urbansim}

\end{document}

% LocalWords: YP Bjorn Borning bnfb borning Yare UrbanSim stakeholders yarely
% LocalWords: stakeholder oatswain aground Heigh yare topsail nse Caliban
% LocalWords: fishlike japa VSD tr workflow XP EIA urbansim CVS checkin logit
% LocalWords:  sustainability Exp Prospero
