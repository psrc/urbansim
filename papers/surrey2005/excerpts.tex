\section{Old Introduction}
Regional transportation planning has historically been poorly
integrated with land use planning in the United States, where
local municipalities control land use policy.  However, there is
growing interest in coordinating these policies to achieve
overarching regional objectives, such as economic competitiveness,
high levels of accessibility, social equity, and environmental
protection.  Metropolitan Planning Organizations are charged with
responsibility for coordinating the development of long-term
regional transportation plans, and make substantial use of travel
demand model systems and aggregate macroeconomic forecasts in
preparing these plans.   To date, however, the integration of
macroeconomic models and transportation models with models of
urban land use development has been rare.  This paper describes
the application of a land use model system to the metropolitan
region surrounding Seattle, Washington, and its integration with a
regional transportation model system and a regional macroeconomic
model system.

UrbanSim is a model system has been developed over the past
several years to support intra-metropolitan simulation and policy
analysis of transportation, land use, and environmental policies.
The system is available as Open Source software at
http://www.urbansim.org. As the theory, specifications and
software architecture of UrbanSim have been documented previously
(Waddell, 2002; Waddell et al., 2003; Noth et al., 2003), this
paper focuses on aspects that have not been previously documented
in the literature, and condenses the discussion of the theory and
specification of the model system in the following section. The
focus in this paper is on the interfacing of the travel models
with UrbanSim via alternative accessibility measures, the linkage
with a regional macroeconomic model, development of the database,
and results from the model estimation for the Central Puget Sound
region.  The model development effort described here represents
the initial phase of development of an integrated model system for
operational use in regional planning by the Puget Sound Regional
Council and affiliated agencies.  The system is currently being
tested in preparation for use in the development of an updated
regional transportation plan.

\section{Model Specification}

UrbanSim simulates annually key events and choice processes
related to urban spatial development and location of activities,
at the level of individual households, jobs, and locations defined
by small grid cells of 150 by 150 meters. The main model
components are the economic and demographic transition models, the
household and employment mobility (intra-urban relocation) models,
the household and employment location choice models, the real
estate development model, and the land price model.

It is worth noting that a key feature of the model system, in
addition to its microsimulation representation, is that it
represents the processes modeled as either short-term (less than
one year) or long-term (more than one year), with the long-term
processes fixed for the short-term processes.  Specifically,
household and firm relocation and location choices, reflecting the
demand side of the real estate market, are treated as elastic in
the short-term, while the supply-side of the real estate market
represented by real estate development, is treated as a long-term
process due to the time lags involved in real estate development.
This means that real estate prices adjust in the short-term to
account for short-term imbalances between real estate demand and
supply.  This formulation marks a clear departure from
full-equilibrium treatments that are more common in urban
simulation, which abstract the adjustments over time to reflect a
temporally-indefinite result that exhibits no path-dependence. The
approach we use reflects current understanding of the implications
of durability of real estate, the lag times in real estate
development, and the development and price volatility that
accompany local business cycles.

Exploiting the Random Utility Maximization (RUM) models pioneered
by McFadden (1974, 1981), UrbanSim implements the residential
location choice, employment location choice, and real-estate
development choice models as discrete choice models\footnote{This
section draws substantially on a longer description of the model
theory and specification provided in Waddell \emph{et al}, 2004.}.
We describe this common framework before discussing each model
component individually. We assume that each alternative $i$ has
associated with it a utility $U_i$ that can be separated into a
systematic part and a random part:
\begin{equation}
    U_i = u_i + \epsilon_i,
    \label{eq:utility}
\end{equation}
where $u_i = \vk{\beta}\cdot\vk{x}_i$ is a linear-in-parameters
function, $\vk{\beta}$ is a vector of $k$ estimable coefficients,
$\vk{x}_i$ is a vector of observed, exogenous, independent
alternative-specific variables that may be interacted with the
characteristics of the agent making the choice (e.g. household
characteristics in the residential location choice model), and
$\epsilon_i$ is an unobserved random term. Assuming the unobserved
term in (\ref{eq:utility}) to be distributed with a Gumbel
distribution (Type I extreme value distribution) leads to the
familiar multinomial logit model (McFadden 1974, 1981):
\begin{equation}
    P_i = \frac{\mathrm{e}^{u_i}}{\sum_j \mathrm{e}^{u_j}},
    \label{eq:mnl}
\end{equation}
where $j$ is an index over all possible alternatives. The
estimable coefficients of (\ref{eq:mnl}), $\vk{\beta}$, are
estimated with the method of maximum likelihood (see for example
Greene, 2002).

\subsubsection{Employment Location Choice}

The Employment Location Choice model simulates location choices
for new jobs created as a byproduct of economic expansion,
predicted by an external macroeconomic model, and for jobs that
have been predicted to move by the employment relocation model
component of UrbanSim. To arrive at a choice model for employment
location we assume that 1) each job belongs to a firm (whose
characteristics other than industry sector remain latent) which is
faced with a choice between alternative locations for the job, 2)
that each location, indexed by $i$, has attached to it some
utility, $U_i$, for the firm, and 3) that the location with the
highest utility has been chosen (maximization of utility).

We refer to the more general concept of utility maximization
rather than profit maximization, since the utility may be based
largely or exclusively on expectations of profit for some sectors,
but profit may represent a small or nonexistent part of the
utility for other sectors, such as governmental and educational
establishments. We only observe the current location of jobs, and
do not observe the alternative locations open to the employer
before locating the job, nor do we observe the utility. We proceed
with the multinomial logit assumptions for the utility
(\ref{eq:utility}) leading to (\ref{eq:mnl}).

The systematic component of the utility of a particular location
(dropping the index $i$ for simplicity) is specified as a function
of an array of characteristics at the \emph{site} ($\vk{x}_S$),
including the real estate characteristics (land value, residential
units, commercial sq. ft., land use) and proximity of the site to
freeways and arterials; characteristics of the land use mix and
value (quantity of residential units, average land values, average
improvement values) in the immediate \emph{neighborhood}
surrounding the site ($\vk{x}_N$); agglomeration economies from
geographic \emph{clustering} (employment by sector within 600 m)
of firms of the same and each of the other sectors ($\vk{x}_C$);
and multi-modal \emph{accessibility} to labor, consumers, the
Central Business District (CBD), and the regional airport
($\vk{x}_A$):
\begin{equation}
    u = \vk{\beta}_S \vk{x}_S
     + \vk{\beta}_N \vk{x}_N
     + \vk{\beta}_C \vk{x}_C
     + \vk{\beta}_A \vk{x}_A.
     \label{eq:uemploc}
\end{equation}

The probability, $P_i$, represents the probability of the firm
choosing location alternative $i$ for a particular job. We
estimate one choice model (i.e. one set of coefficients) for each
of the 14 industry sectors shown in Table\ref{ta:sectors} on a
random sample of 5000 observed jobs in each sector. To estimate
the model coefficients we use data for business establishments in
1997, geo-coded to grid cells. The database links individual jobs
to job spaces. The job spaces can be either nonresidential square
footage, or a residential housing unit to account for home-based
employment.

To arrive at a set of alternatives we allow each job in a sector
the choice of all locations in the universe open to that industry
sector. This generates a very large choice set. We use a uniform
distribution to randomly sample a set of nine alternatives in
addition to the chosen location and estimate a model using this
random sample of alternatives. This makes it impossible to
estimate alternative-specific coefficients or alternative-specific
constants. However, it can be proven that the coefficients of a
choice model estimated from a random sample of alternatives,
selected with a uniform distribution, are consistent, as explained
by McFadden (1978) in his paper on residential location choice,
which faces a similar issue.

Since the coefficients are based on random sampling of
alternatives, there are no alternative-specific constants, and no
base alternative. The coefficients are therefore interpretable in
terms of the direction of the influence of a variable on the
utility and the probability of a location choice. In addition,
coefficients can be compared across industry sectors, since the
same specification is used for all sectors, with the exception of
insignificant variables, which were restricted to zero.

\subsubsection{Residential Location Choice}

The residential location choice model predicts the probability
that a household will select a location, defined in the current
specification by a grid cell of 150 by 150 meters. Each grid cell
can include zero or more housing units and households can only
select cells with vacant housing. All housing units on a cell are
assumed to be identical and we therefore do not assign the
household to a particular unit.

The model is therefore a disaggregate choice model with over one
million housing units. The number of alternatives that households
can select form is the total number of available housing units,
though noting that we only specify the location to a grid cell.
This creates a very large choice set and in this implementation of
the model we reduce that set by simply selecting a random sample
of nine alternatives from the universe of vacant housing, in
addition to the observed choice for each household, yielding 10
alternatives in the model.

The model is specified as a multinomial logit model (\ref{eq:mnl})
with a systematic utility for a particular location (we drop the
index $i$ for convenience) on the form:
\begin{equation}
    u =  \alpha
     + \vk{\beta}_H \vk{x}_H
     + \vk{\beta}_R \vk{x}_R
     + \vk{\beta}_N \vk{x}_N,
    \label{eq:uresloc}
\end{equation}
where each utility term is a linear combination of variables that
have been grouped in to categories: $H$ indicates housing
characteristics (e.g. prices, density, age), $R$ indicates
regional accessibility, and $N$ reflects neighborhood-scale
effects (socioeconomic composition, land use mix, density, local
accessibility).

\subsubsection{Real Estate Development}

The real estate development model simulates the actions of real
estate developers to intensify the real estate development within
any given location, or grid cell.  The unit of analysis in this
model is the grid cell, and the choice outcomes are alternative
development types.  The probability of each alternative (no
development, increasing density of current cell within its current
development type, and transitions to other development types)
being chosen is calculated using a discrete choice model. We draw
on discrete choice theory and random utility maximizing models,
following the work of McFadden (1974, 1981), to design a
multinomial logit model.

To arrive at a choice model for development we assume that 1) each
cell has a developer agent, 2) that each development alternative,
indexed by $i$, has attached to it some utility, $U_i$, for the
developer, based principally on profit expectations, and 3) that
the development event with the highest utility has occurred
(maximization of utility). We only observe events that have
occurred and do not observe the utility.

To form the estimation data we take all the development event
cells, i.e. cells with a known development event and look up the
values for a set of independent variables from the grid cell
database. The independent variables in the real estate development
model include characteristics of the \emph{site} ($\vk{x}_S$),
including current development, land use plan, environmental
constraints, policy constraints, land and improvement value,
proximity to highways, arterials, existing development, and recent
development; characteristics of the land use mix, property values,
and local accessibility measures in the \emph{neighborhood}
surrounding the site ($\vk{x}_N$); and multi-modal
\emph{accessibility} ($\vk{x}_A$), including access to population
and employment and travel time to the central business district
and airport:
%
\begin{equation}
    u = \alpha
     + \vk{\beta}_S \vk{x}_S
     + \vk{\beta}_N \vk{x}_N
     + \vk{\beta}_A \vk{x}_A.
     \label{eq:uemploc}
\end{equation}
%
We proceed with the multinomial logit assumptions for the utility
(\ref{eq:utility}) leading to (\ref{eq:mnl}). The probability,
$P_i$, represents the probability of a developer agent for a
particular cell choosing development alternative $i$.

We now need to take into account the much larger set of cells that
didn't experience a development event. We take a random sample of
these cells to generate a set of similar size as the development
event set. This gives us a choice-based sample of cells.
Choice-based sampling only biases the alternative-specific
constants but other coefficients remain consistent (Manski and
McFadden, 1981). We adjust the alternative-specific constants
after estimation to account for this bias.

We estimate one choice model (i.e. one set of coefficients) for
each development type, since the types are very different and the
development alternatives open to each development type vary. To
estimate the model coefficients we need data for cells
experiencing no development and for cells with development events
of all types.


\subsection{Land Price}

Land prices represent the interaction between demand and supply
sides of the model system, with prices fluctuating in response to
short-term (intra-year) shifts in demand and long-term
(inter-year) shifts in supply, and work in the traditional
economic sense to ration scarce supply of land and clear the
market in the short term. A semi-log specification is used to
define a hedonic regression model. The model is a linear
multiple-regression of the log of land prices, $\ln(v_i)$, for
each cell $i$, on an array of housing structural ($\vk{x}_{Si}$),
neighborhood ($\vk{x}_{Ni}$), and accessibility ($\vk{x}_{Ai}$)
characteristics:
\begin{equation}
    \ln(v_i) = \alpha
        + \vk{\beta_S}\vk{x_{Si}}
        + \vk{\beta_N}\vk{x_{Ni}}
        + \vk{\beta_A}\vk{x_{Ai}}
        + \epsilon_i,
\end{equation}
where $\alpha$ is the estimable intercept term; $\vk{\beta_S}$,
$\vk{\beta_N}$, and $\vk{\beta_A}$ are the estimable coefficient
vectors on the housing structural, neighborhood, and accessibility
characteristics, respectively; $\epsilon_i$ is an unobserved error
term, assumed to be normally distributed with mean zero and
variance $\sigma^2$.

The full set of grid cells in the study area is used in model
estimation, using base year characteristics and values. As such,
this is a cross-sectional estimation of the market hedonic price
function, rather than an estimation of a dynamic price function.
Dynamics are introduced through the process of annual changes in
the characteristics of grid cells due to simulated results from
the real estate development, residential location and employment
location models, and the external transportation model system, all
of which combine to change the characteristics of grid cells on an
annual basis.

\section{Overview of the Database Development Process}
The data preparation process is a collection of procedures, both
automated and manual, to create the baseyear database used in the
application of UrbanSim to the central Puget Sound region. Our
approach has emphasized using the best available local data, and
undertaking as much effort as feasible within time and budget
constraints to improve the quality and consistency of the data for
use in model estimation and application.

Details regarding the structure of  the database produced in the
data preparation process is described in the UrbanSim User Manual,
available at www.urbansim.org/docs. The database contains three
primary tables (gridcells, households and jobs) and numerous
supporting tables. The model documentation specifies the required
contents, names, and formats for each field in these tables. These
details are omitted from this paper in order to focus the
discussion on the model specification and estimation results.

The development of the base year database consisted of a series of
steps.  The major components of this process are summarized below.
The tools used to undertake the processes described in the report
include ArcGIS 8.3 (GIS), the R statistical package, the MySQL
database, Structured Query Language (SQL), and Java, Perl, and
Python programming languages. All scripts, queries, and programs
are available from www.urbansim.org.

\begin{description}
\item{\textbf{Define Study Area}}

The determination of the study area boundary is a critical first
step, since it determines the geographic extent of the data
collection effort. The area defined for the Puget Sound Regional
Council model application was based on the 4-county region for
which the PSRC does transportation planning: King, Kitsap, Pierce
and Snohomish. An additional 4 counties outside this region were
anticipated for potential extension of the model: Island, Mason,
Skagit, and Thurston.  %A map of this area is shown in Figure
%\ref{fig:AreaMap}.

%\begin{figure}[h]
%\centering
%\includegraphics[width=5in]{../graphics/psrc_p1.jpg}
%\caption{The Study Area} \label{fig:AreaMap}
%\end{figure}

\item{\textbf{Generate Grid Over Study Area}}

The current specification of UrbanSim uses a grid as its
geographic units of analysis.  The database development process
therefore involves the creation of a GIS grid (typically using 150
meter cell size) that covers the study area, and assigning unique
identifying numbers to each cell in the grid. Water and other
areas that are not relevant for the model application could be
suppressed from the grid at this point to make the database more
compact.

Even though the study area for this project comprised the four
main counties of the central Puget Sound region (King, Pierce,
Snohomish, and Kitsap), the grid created included the greater
Puget Sound region consisting of eight counties (Skagit, Island,
Mason, Thurston, King, Pierce, Snohomish, and Kitsap).  This
larger grid was created to allow for possible expansion of the
study area in the future.  The eight-county grid was clipped to
the four central counties of the initial study area, and all
further processing was done only for the four core counties.

%Figure \ref{fig:GridcellExample} shows the resulting grid in the
%Green Lake area of Seattle, with Traffic Analysis Zones (TAZ) and
%Forecast Analysis Zones (FAZ) superimposed to show the relative
%sizes of these geographies.  The grid cells at 150 by 150 meters
%cover approximately 5.56 acres, somewhat smaller than a suburban
%square block.

%\begin{figure}[h]
%\centering
%\includegraphics[width=5in]{../graphics/grid_greenlake.jpg}
%\caption{Grid Sample} \label{fig:GridcellExample}
%\end{figure}

\item{\textbf{Assemble and Standardize Parcel Data}}

Parcel data form a foundation for the description of the land use
and real estate inventory within the study area  These files are
increasingly available from County Tax Assessor offices in GIS
format, with associated attribute tables containing the main
characteristics relevant to urban modeling.  These attributes
include lot size, land use, housing units, built square footage,
year built, zoning, land use plan designation, assessed land
value, assessed improvement value, and others.  An important step
in this process was to standardize the names of and coding
standards used for these fields.

\item{\textbf{Impute Missing Data on Parcels}}

Parcel data tend to have both systematic and random errors, as
well as missing data.  For example, tax exempt properties, such as
government-owned properties, tend to have incomplete data in the
assessor files since there is no compelling reason for assessors
to collect these items.  The data are massive in volume, and rely
on individual property assessors to input data, so the resulting
databases are often riddled with missing data and data errors.
However, much of the data is likely to be correct, and serves as
the most widely available and complete representation of land use
and real estate data that is collected and maintained. As a result
of these limitations in the data and the difficulty of manually
verifying the data, it was necessary to develop automated
procedures to impute missing data for key fields such as land use
codes, year built, and housing units.

\item{\textbf{Assemble Employment Data}}

Assembling and processing employment data was one of the most
challenging aspects of the database development process. The
principal source of employment data used in this process was the
ES202 unemployment insurance database from the Washington State
Department of Employment Security (ESD). These records are
confidential, and great care was taken to ensure that the final
data derived from them did not violate confidentiality agreements.
The ESD data contained addresses, Standard Industrial
Classification codes (SIC) and the newer NAICS classifications of
sectors, and the number of jobs at each establishment address.
These data had been geocoded by the PSRC, usually using a
combination of automatic and manual address-matching techniques.
Additional employment sources included an inventory of government
and educational establishments developed by the Puget Sound
Regional Council, and estimates of proprietor and self-employed
developed by comparing the wage and salary employment totals to
total employment estimates by county.

\item{\textbf{Assign Employment to Parcels}}

Once geocoded, these establishments and the jobs they contain
needed to be allocated to the available real estate in the parcel
data. Systematic and random errors occur in both the employment
data and parcel data, making their reconciliation difficult. The
geocoding of business establishments had been conducted by the
PSRC using an address-matching process that frequently placed
employer near, but not on within, the parcel in which the employer
is truly located. This was most often due to the fact that the
employers had been geocoded to the street centerline as opposed to
the parcel.  Also, the ESD data represented only \emph{covered}
employment, and did not include workers who were self-employed. In
order to account for these jobs, county-level estimates of total
employment were obtained from the PSRC. Several algorithms were
developed to allocate the ESD records, government and educational
establishments, and proprietors to parcels, and to attempt to
resolve many of the inconsistencies within and between these data
sources.

\item{\textbf{Convert Parcel Data to Grid}} In order to support
the grid-cell based specification of the model, the parcel data
were converted to a grid. To avoid data loss at cell resolutions
coarser than small parcels, a GIS polygon overlay operation was
used. Once this was completed, the fraction of land area of a
parcel that falls within each grid cell was used to allocate the
quantities within the parcel to the cells it overlays.

\item{\textbf{Convert Other GIS Layers to Grid}} Each grid cell
was assigned to any environmental or planning areas that it falls
into, such as wetlands, floodplains, steep slope areas, traffic
analysis zones, cities, counties, urban growth boundaries. Simple
GIS overlay processing is used to make these assignments.

\item{\textbf{Assign Development Types}}

The model system uses the real estate contents of grid cells to
classify cells into Development Types based on the quantities of
housing units and square feet of commercial, industrial and
governmental/institutional space in each grid cell produced in the
conversion of parcel data to grid cells.  One development type is
identified as Undevelopable based on characteristics such as
water, wetlands, steep slopes, federal lands, or other
user-specified characteristics. It is important to note that
designating a cell as Undevelopable means that the model system
will not consider the cell for development, regardless of what
policies the user later applies to the cell (unless they change
the Development Type to something other than Undevelopable).  The
remaining Development Types are assigned using a lookup table of
units and square feet of nonresidential space in the cell.  For
those cells that have nonresidential square footage and virtually
no residential units, the relative magnitudes of commercial,
industrial and government space are used to further differentiate
the Development Type.

\item{\textbf{Synthesize Household Database}}

Since there is no inventory of households at the household level
in the United States (or most anywhere else), we must resort to
synthesizing this database of households from publicly available
census data sources. We have adapted a process developed by
Richard Beckman as part of the TRANSIMS traffic microsimulation
model system, to integrate census tabulations of household
characteristics at the census tract level (in Summary File 3A)
with a sample of actual households available from the Public Use
Microdata Sample (PUMS).

\item{\textbf{Diagnose Data Quality and Make Refinements}}

Part of the process of developing the baseyear database involves
diagnosing problems in the integration of the data such as missing
or invalid data, or inconsistencies within or between tables. A
series of checks of the data were used to determine the
suitability of the database for use in model development and
application.
\end{description}

\section{Accessibility Measures}

The development of the land use model system depends heavily on
measures of accessibility, which influence location choices, real
estate development, and land prices.  Several alternative
accessibility measures have been evaluated as part of this project
by interfacing UrbanSim with a travel demand model system.  In the
Puget Sound Region, the travel models used by the Puget Sound
Regional Council for developing regional transportation plans are
the basis for this model integration.  This section describes
alternative accesibility measures derived from the integration of
UrbanSim and the travel model system which are used as independent
variables in UrbanSim.

\subsection{Logsum-weighted Access to Activity}

The initial approach to measuring accessibility by interacting
UrbanSim with a travel model is to use the Logsum, or composite
utility information from the mode choice model, and interacting
this with the spatial distribution of activities in UrbanSim to
form summary measures of accessibility to general types of
activities, such as jobs.   The form of this calculation is:

\begin{equation}
A_i = \sum_{j=1}^{J} D_j e^{L_{aij}} \end{equation}

\begin{tabbing}

where: \= \\
\> $D_j$ \= is the quantity of activity in location $j$ \\

\> $L_{aij}$ \=is composite utility, or logsum, for vehicle
ownership
level $a$ households, from \\
\> \> location $i$ to $j$.

\end{tabbing}

where the Logsum is scaled to ensure it is non-positive.

The exponentiated logsum provides a weight between 0 and 1 to
apply to the quantity of activity in a particular destination,
discounting the activity more heavily in destinations that are
relatively less accessible.  Since the range of logsum values is
occasionally slightly positive in practice (though theoretically
the values should all be negative since this is representing the
disutility of travel), we scale the logsums by subtracting a
constant equivalent to the maximum value of the logsums across all
ij pairs, resulting in a new maximum equal to zero, and a weight
equal to 1 for the highest accessibility destination.

The main advantage of the logsum term is that it provides a
single, comprehensive measure of travel disutility across modes,
and incorporates all the factors considered in the mode choice
model that have an impact on utility, such as tolls, fares, wait
times, transfer penalties, comfort, etc.  It is theoretically
elegant, in that it potentially provides a way to incorporate into
broader consumer welfare measures the effects of land use and
transportation changes.

The following are some disadvantages of this formulation, in the
context of use in land use modeling:

\begin{itemize}

\item Logsums are difficult to interpret, compared to simpler
measures such as travel times.

\item The Logsums must be scaled to ensure that they are not
positive values, since exponentiating positive values would
produce very large weights for specific destinations (the behavior
of the weight differs drastically for positive as compared to
negative values of Logsums) distorting the access measure
significantly.  This scaling is problematic, since it is
arbitrary, and is an artifact of the scale of the single maximum
value, which is much more volatile than any measures of central
tendency, such as the mean Logsum.

\item The Logsum\-weighted Access measure will be inflated by
growth in population and jobs in a region, unless Logsum values
degrade at a greater rate.  This can introduce unintended behavior
in the models in which the measure is used, such as a real estate
price model, causing predictions to shift quite far from
conditions on which the model was calibrated.

\item The Access measure is somewhat inconsistent with the travel
model, since it does not consider the probability of making trips
to different destinations from the origin zone, as predicted by
the trip distribution model.  One minor variant of the above
formulation that would modestly improve the consistency with the
trip distribution is to multiply the Logsum by a coefficient that
is calibrated as the friction factor in the trip distribution
model - but this also depends on the structure of the trip
distribution model, and does not account fully for the patterns of
travel predicted by the travel model.

\item Mode choice models vary in their structure, and particularly
in their stratification, with some models using auto ownership,
others using a relationship of autos and workers, etc. This
complicates the use of the measure, in that the measure should be
applied in the land use model using the same stratification, for
consistency.

\end{itemize}

\subsection{Travel Time Threshold Measures}

A simple measure of accessibility that is widely used in the
literature, and in practical modeling applications, is a count of
activities that can be reached within a specified threshold radius
of travel time, such as 30 minutes. This is an intuitively
appealing measure, because it is easy to understand.  It may also
therefore be valuable, to the extent that it mimics how people
consider accessibility in actual decision-making situations such
as choosing a residence location.

\begin{equation}
A_i = \sum_{j=1}^{J} D_j \forall j \in C_{ij} < X
\end{equation}

\begin{tabbing}

where: \= \\
\> $D_j$ \= is the quantity of activity in location $j$ \\

\> $C_{ij}$ \=is the travel impedance from location $i$ to $j$ \\

\> $X$ \=is a threshold value.

\end{tabbing}


This measure has limitations also, such as:

\begin{itemize}

\item The arbitrariness of the threshold.  What threshold should
be used? Why should an activity one minute beyond the threshold
not count at all, while an activity one minute within the
threshold counts in full?  Are activities one minute away no more
accessible than activities just within the threshold?

\item It only accounts for time.  What about tolls or fares?

\item Representing the effect of different modes would either
require using a similar threshold measure for each mode, or using
some arbitrary method of averaging.

\item It is even less consistent with the travel model than the
first measure.

\end{itemize}

\subsection{Trip-weighted Travel Disutility}

An alternative access measure that might address some of the
concerns raised about both of the preceding measures is one that
computes a weighted average of a measure of travel disutility,
such as Logsum, travel time, or generalized cost, where the
weighting is based on the probability of making a trip to a
particular destination.  Variants on this measure include
stratification by trip purpose and/or by mode.

\begin{equation}
A_i = {\sum_{j=1}^{J} T_{ij} L_{pij} \over \sum_{j=1}^{J} T_{ij}}
\end{equation}

\begin{tabbing}

where: \= \\
\> $D_j$ \= is the quantity of activity in location $j$ \\

\> $L_{pij}$ \=is composite utility for trip purpose p (Home-Based
Work) from location $i$ to $j$.

\end{tabbing}

An advantage of this measure is that it uses not only the
composite utility of travel for specific interchanges computed by
the mode choice model, but it also uses information on trip
distribution from the travel model, increasing the consistency of
the accessibility measure used in land use modeling with the
information in the travel model.

A second advantage of the measure is that the scale of the
disutility measure need not be adjusted artificially, as in the
first measure.  Nor does it inflate over time due to population or
employment growth.

\subsection{Accessibility Measures Used in Puget Sound Model Estimation}

After considerable testing within the model estimation process
described in the following sections, the accessibility measures
that proved most useful in explaining location and development
choices in the region were:

\begin{itemize}

\item Logsum-weighted Access to Jobs

\item Travel time to the Central Business District of Seattle

\item Trip\-weighted Travel Time via Single-Occupancy Vehicle for
the Home-Based Work purpose, and the same measure described for
destinations

\item Trip\-weighted Travel Time via Transit with Walk Access for
the Home-Based Work purpose

\item Trip\-weighted Composite Utility via Single-Occupancy
Vehicle for the Home-Based Work purpose

\item Trip\-weighted Composite Utility via Transit with Walk
Access for the Home-Based Work purpose

\end{itemize}

\section{Macroeconomic Model Interface}

The macroeconomic model to which UrbanSim is coupled in this
application is the STEP model (Conway, 1990).  It is a hybrid
input-output/econometric model that represents a localization of
the Washington State economic model it is derived from.  The
sectors used in this model are the basis for the sector
definitions in UrbanSim, and the total employment by sector, total
population, and households predicted by the STEP model are used as
control totals for use in UrbanSim.  UrbanSim adds or deletes jobs
by sector, or households, from its database on an annual basis, in
order to match these predicted totals.  In short, this is a
trivial one-way connection of the models.

There has been ongoing interest in examining the economic effects
of transportation projects for several decades (Chinitz, 1960). To
date the issue of feedback from local policies to macro-scale
economic or demographic outcomes has not been addressed within our
modelling framework.  The changes required to the macroeconomic
model to accommodate this feedback would be substantial, and were
outside the scope of this project. It is a significant topic for
future research, and should be a priority for further development
of the UrbanSim system.

There may be several viable alternative strategies to pursue
integration of spatial and macro-scale modeling. One is to
continue to use an aggregate macroeconomic model similar to the
STEP input-output/econometric model, and simply add variables that
would be generated by the land use and travel demand model system,
such as average housing prices, vehicle miles travelled, or total
hours of congestion delay.  Of course, these kinds of
transportation measures, though they may have significant
aggregate effect on economic growth, are not available
historically, and make the likelihood of incorporation into the
current macro scale models unlikely.

A second approach to integrating macro-scale modeling would be to
examine the possibility of implementing the behavior of the
macro-scale model at the disaggregate level, through
microsimulation.  While this has been done fairly extensively for
demongraphic modeling, I am not aware of any operational systems
designed for microsimulation of emergent regional macroeconomic
outcomes. Experimentation in the Agent-based Computational
Economics community may eventually lead to some useful building
blocks for such an approach (c.f. Tesfatsion, 2004).

\section{Model Estimation}
\subsection{Land Price Model}

The Land Price model is a linear regression model, with the
natural logarithm of the total land value within a grid cell set
as the dependent value, and a series of physical, real estate, and
locational variables used as independent variables.  Figures
\ref{fig:totlvbase} and \ref{fig:totlvpred} show the observed and
predicted land values per acre in 2000.

The coefficients were estimated using Ordinary Least Squares (OLS)
estimation on a sample of 100,000 gridcells, randomly selected
from cells with non-zero land values from across the region.  The
model had an adjusted R\(^{2}\) of 0.779, indicating that the
model explains approximately 80 percent of the variation in land
values in the base year. Table \ref{tab:LP_coefficients} lists the
coefficients used, along with their values and descriptive
statistics.

\subsection{Household Location Choice Model}

The Household Location Choice Model predicts the location choices
of households that are moving into or within the region.  Location
choices are defined as grid cells of 150 meters, representing an
area of approximately 5.5 acres.  Since there are a very large
number of alternative locations, the model estimation process uses
a random sample of 9 non-chosen alternatives, in addition to the
location the household is observed to occupy, to construct a
choice set.  This random sampling of alternatives technique is
commonly used to estimate multinomial logit models with large
numbers of alternatives, and has been shown to produce unbiased
coefficient estimates.

The coefficients for the Household Location Choice Model were
calculated using data from the PSRC Household Activity Survey. All
households that had moved within the previous five years were
included, which provided us with a sample size of 2,364
observations.  The restriction to households which had moved
within the past 5 years was made to reflect the choices of
households in similar circumstances to those being modelled.
Figure \ref{fig:HH2} displays the locations of households used in
the model estimation. Table \ref{tab:HLC_coefficients} lists the
variable codes and coefficients that were included in the final
model.

Once an acceptable model specification was determined and its
coefficients were estimated, several summary tables were
constructed to aid in the analysis of results.  These tables took
the form of summaries of the prediction success rates by
geographical units we called FAZ districts (so called because they
were made up of agglomerations of the PSRC's Forecast Analysis
Zones).  We defined nine FAZ districts regionally such that they
represented broad areas of relatively homogenous, or at least
internally similar, social areas.  We then calculated several
summaries of the results of the estimation by these districts.
These summaries can be grouped into  the following three main
classifications:

\begin{description}
    \item[Sums of Probabilities] -- During estimation, predicted
    probabilities are assigned to each grid cell in a household's
    choice set.  This value represents the expected probability that
    the household would choose that grid cell to move to were it to
    move to a grid cell in the choice set.  These probabilities were
    summed by FAZ district, with the expectation that the sums would
    be comparable to the counts of observed households.
    \item[Counts of Predictions -- Maximum Probability]  Within a
    choice set, we declared the grid cell with the highest
    probability to be the predicted choice for that household.
    These sums of predicted locations can be compared against the
    sums of observed locations.
    \item[Counts of Predictions -- Random Choice Simulation] The third approach to implementing a choice prediction is to
    stochastically select a grid cell from the choice set in such
    a way that the probability of selecting each cell is
    proportional to that cell's predicted probability.  This is
    the approach used in the UrbanSim application.
\end{description}

Each of these prediction success definitions were summarized in
several ways, as can be observed in table
\ref{tab:HLC_fazdist_matrix_probability}, which summarizes the
results by FAZ district.  FAZ districts predicted are summarized
here by FAZ districts observed.  The predictions are then
expressed in three summary columns: A) as percentages correctly
placed (where correct placement is defined as placement in the
correct FAZ disrict), B) as counts of households placed in the
wrong FAZ district, and C) as totals placed divided by totals
observed, regardless of which district each individual chose.


\subsection{Employment Location Choice Model}

% Document sampling process for employment

Employment sectors were defined based on Standard Industrial
Classification (SIC) codes, grouped into 18 \emph{sectors} that
correspond to aggregations of those defined in the STEP
macroeconomic model used by PSRC to predict metropolitan
employment and population levels over a thirty-year planning
horizon, as shown in Table \ref{tab:Sectors}.

Model estimation has been performed for all sectors except
military, government, and education.  These sectors are
distinguished from the private sectors in that the location of
jobs in these sectors are considered as outcomes of governmental
decisions that are outside the scope being modeled within the real
estate market. An additional pragmatic reason for this design
choice is that the quality of real estate information is very poor
for tax-exempt properties, resulting in many public facilities
lacking information in the parcel file on square footage. Without
such data, it would be impractical to model the location and use
of space by these governmental sectors.  Location of employment in
public facilities is therefore regarded as an exogenous policy
input. To facilitate use of the model, simplified algorithms for
the allocation of jobs in these sectors have been implemented to
allocate growth in any of these sectors to locations in proportion
to the existing employment in the same sector.  The procedure
samples an existing job in the database randomly, and assigns its
location to the current job in the queue to be placed.

Home-based jobs, principally in the form of proprietors and small
contractors with the home address as the principal work address,
will be handled separately from jobs occupying non-residential
real estate.  The use of a fixed ratio of units to home-based jobs
was considered to be too simplistic and will be replaced by a new
home-based job location model.  An initial split of jobs in each
sector into home-based and non-home based categories is applied,
based on proportions found in the 2000 Public Use Microdata Sample
data.  A new model to assign home-based jobs to households is
currently in development, but will be based on the probability of
households of differing characteristics (such as income, age, and
number of workers) having a home-based job.  The remainder of this
section describes the procedure for estimating the probability of
location choices for non-home based jobs, by sector.

The data used for estimating the Employment Location Choice Model
were based on 1995 and 2000 employment databases maintained by
PSRC and derived from the Washington Employment Securities
Division ES202 records.  The total employment by Traffic Analysis
Zone and employment sector were compared between 1995 and 2000.
Net gains in employment by zone and sector were used to define the
universe of recent job location choices.  From this set of job
gains by zone and sector, samples of 5,000 jobs were randomly
selected in each sector for use in model estimation.

Estimation for the Employment Location Choice Model used a similar
process as the Household Location Choice Model. That is, a choice
set of ten grid cells was selected for each non-home based job in
the estimation data set (hereafter we will simply refer to
'non-home based jobs' as 'jobs'). The alternatives include the
grid cell on which the job was observed (i.e. the "chosen" grid
cell), and nine non-chosen alternatives selected at random from
the set of all grid cells in proportion to the quantity of
non-home-based job spaces in those grid cells.  This random
sampling of alternatives technique is commonly used to estimate
multinomial logit models with large numbers of alternatives, and
has been shown to produce unbiased coefficient estimates.

Job spaces are calculated for each grid cell as a function of
nonresidential square footage in the cell. Nonhome-based job
spaces are determined by a gridcell's total nonresidential square
footage.  Total job spaces are the sum of home-based jobs and
nonhome-based jobs supported by a given cell's units and
nonresidential square footage.

Coefficient estimation for the Employment Location Choice model
was performed on a subset of jobs, which was selected on a
stratified random basis from all jobs in the region.  The
construction of this subset took place in a two-part process.
First, jobs were divided into sector-TAZ groups and were selected
from each group in proportion to the number of jobs that had been
added since 1995.  This was done in order that the set of jobs
used for estimation would reflect the jobs that had recently
moved.  Job samples of 5,000 jobs per sector were then selected at
random from this distribution.

The estimation process for the Employment Location Choice model is
somewhat different from the previously discussed models in that it
comprises a number of submodels, one for each employment sector.
Each submodel was estimated independently.  Most of the variable
sets were similar across submodels, although the do vary slightly
owing to the principal that jobs of divergent sectors might
predicate their location choice on different factors.  Table
\ref{tab:ELC_coefficients} lists the variables and their
coefficients for the Finance, Insurance and Real Estate sector
submodel\footnote{results for all other sectors are available in
Waddell \emph{et al} 2004a}.  Prediction-success results for this
sector follow in Table \ref{tab:ELC_prediction_10}.

\subsection{Developer Model}

The Developer Model consists of a number of multinomial logit
models that predict potential development events on each grid
cell. Whereas the previous logit models predict an agent's choice
from a set of grid cells, the Developer model predicts a grid
cell\'s choice from a set of possible development type change
events.  The choice set for each grid cell is therefore the set of
all possible development types it could convert to, including the
most common alternative: no change. This choice set size varies by
starting development type, since some development type transitions
are unlikely to occur.

Estimation was conducted independently for grid cells of each
starting development type.  This approach was used because many of
the development types are so different from one another, and the
development of them would thus be expected to depend on very
different factors. This is reflected in the SUB\_MODEL\_ID field
in the developer\_model\_specification and
developer\_model\_coefficients tables; Each submodel number
corresponds to a development type ID before a development event.
Within a given submodel there are usually several EQUATION\_ID\'s,
each of which correspond to a different ending development type
ID.

The data used to estimate the developer model equations draw
heavily from the base year database.  The year built values are
used to construct estimates of the location and type of
development events over a historical period, in this case from
1995 through 1999.  The model estimation then attempts to find
coefficients that explain the variations in the timing, nature,
and location of development events.

The base year distribution of housing units and total
nonresidential square footage, and changes in these quantities
from 1990 to 2000 are shown below in Figure \ref{fig:units_base}
through \ref{fig:nonres_added}.  Maps depicting industrial and
commercial sqft in 2000 and added over the preceding decade are
shown in Figures \ref{fig:ind_base} through \ref{com_added}.

The estimation process for the Developer Choice model, like the
Employment Location Choice Model, contains a number of submodels,
one for each development type. Each submodel was estimated
independently. Table \ref{tab:DEV_coefficients} lists the
variables and their coefficients for only the submodel
representing conversions from vacant land, development type 24.
Although this is clearly an important source of development, in
the Puget Sound region more than half the development events
occurred on cells that were already developed.  These other 23
development types are also estimated, but due to the volume of
results, the interested reader is referred to Waddell \emph{et al}
(2004) for these estimation results.  Prediction success results
are not reported here since this model includes
alternative-specific constants, and the prediction success results
therefore match the observed counts of transitions almost exactly.

\section{Conclusions}

The results reported here represent a work in progress, and may be
substantially modified before the UrbanSim model is put into
production use by the Puget Sound Regional Council and its
constituents.  The next phase of this project is to undertake a
substantial effort to validate and test the model system,
historical validation if sufficient data are available, and
augmenting this with a rigorous set of sensitivity tests to
examine the sensitivity of the model system to policy changes and
other exogenous shocks.

Much more remains to be done in improving the integration of
UrbanSim with transportation and macroeconomic models.  Due to the
interactions of household short-term travel choices and their
longer-term choices of residence location, workplace, schools, and
vehicle ownership, it is likely that the integration of land use
and travel modeling will need to be done at the individual choice
level to allow for sufficient representation of substitutability
across these choice outcomes.  The recent emergence of
activity-based travel models provides new opportunities for
micro-level integration of these processes.


\section*{Acknowledgements}

This material is based upon work supported by the National Science
Foundation under Grants CMS-9818378, EIA-0090832, BCS-0120024, and
EIA-0121326, and by the Puget Sound Regional Council.  I also wish
to acknowledge the important contributions of members of the Puget
Sound Regional Council Technical Advisory Committee, and
particularly Larry Blain, Mark Simonson and Chris Peak at the
PSRC, who each contributed substantially to the analysis in this
paper.  Acknowledgement is also due to Peter Caballero, who did
much of the database development, Robert Duisberg, who developed
automation tools to facilitate the estimation of the model system
and development of reports, Jack Kim, who contributed to the
analysis of non-residential real estate, and Liming Wang, who
contributed analyis and automation support throughout.  Last, but
not least, credit is due to Alan Borning, co-Director of the
Center for Urban Simulation and Policy Analysis, and David Socha
and Bjorn Freeman-Benson, who designed and developed the software
implementation for UrbanSim.

\section*{References}

\newlength{\lengthstorage}
\lengthstorage=\parindent
\parindent=0pt
\parskip=2mm

Ben-Akiva M, and Lerman S, 1987 \textit{Discrete Choice Analysis:
Theory and Application to Travel Demand}  The MIT Press,
Cambridge, MA

Chinitz B, 1960 "The Effect of Transportation Forms on Regional
Economic Growth" \textit{Traffic Quarterly} \textbf{14} pp
129--142

Conway, R, 1990 "The Washington Projection and Simulation Model: A
regional interindustry econometric model" \textit{International
Regional Science Review} \textbf{13} (1)

DiPasquale D, and Wheaton W, 1996 \textit{Urban economics and real
estate markets} Prentice Hall, Englewood Cliffs, NJ

Greene W, 2002 \textit{Econometric Analysis} 5th Ed. Pearson
Education

Krugman P, 1991 "Increasing Returns and Economic Geography"
\textit{Journal of Political Economy} \textbf{99} pp 483--499


McFadden D, 1974 "Conditional logit analysis of qualitative choice
behavior" In \textit{Frontiers in Econometrics} Zarembka P, ed.,
Academic Press, New York, NY

McFadden D, 1978 "Modeling the choice of residential location in
spatial interaction theory and planning models" In \textit{Spatial
Interaction Theory and Planning Models} Karlqvist A, Lundqvist L,
Snickars F, and Wiebull J W, eds, North Holland, Amsterdam, pp
75--96

McFadden D, 1981 "Econometric Models of Probabilistic Choice" In
\textit{Structural Analysis of Discrete Data with Econometric
Applications} Manski C F, and McFadden D, eds, MIT Press,
Cambridge, MA, pp 198--272


Noth M, Borning A, and Waddell P, 2003 "An Extensible, Modular
Architecture for Simulating Urban Development, Transportation, and
Environmental Impacts" \textit{Computers, Environment and Urban
Systems} \textbf{27} (2) pp 181--203

O'Sullivan A, 2000 \textit{Urban Economics} McGraw-Hill, New York,
NY

Tesfatsion L (2004)  "Agent-Based Computational Economics Growing
Economies from the Bottom Up" Web Portal for Agent-Based
Computational Economics, accessed at
http://www.econ.iastate.edu/tesfatsi/ace.htm.

Waddell P, 2000 "A behavioral simulation model for metropolitan
policy analysis and planning: residential location and housing
market components of UrbanSim" \textit{Environment and Planning B:
Planning and Design} \textbf{27} (2) pp 247--263

Waddell P, 2002 "UrbanSim: Modeling Urban Development for Land
Use, Transportation and Environmental Planning" \textit{Journal of
the American Planning Association} \textbf{68} 3 pp 297--314

Waddell P, and Nourzad F, 2002 "Incorporating Non-Motorized Mode
and Neighborhood Accessibility in an Integrated Land Use and
Transportation Model System" \textit{Transportation Research
Record 1805} pp 119--127

Waddell P, Borning A, Noth M, Freier N, Becke M and Ulfarsson G F,
2003a "Microsimulation of Urban Development and Location Choices:
Design and Implementation of UrbanSim" \textit{Networks and
Spatial Economics} \textbf{3} (1) pp 43--67


\parindent=\lengthstorage
\parskip=0pt

\appendix
\newpage
