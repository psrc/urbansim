\section{Urban Modelling and Formative Policy Evaluation}
The use of models to assist in the formation of policy has a long
tradition in the broad arena of urban policy.  Much of the early
work in the 1960's focused principally on transportation policy,
and led to the development of the precursors to current 4-step
travel demand model systems that are now a standard part of the
metropolitan planning process \cite{}. Models of urban development
using spatial interaction techniques came into moderately
widespread use in the 1970's, for predicting land use change
associated with transportation improvements
\cite{putman-book-1983}.  Other land use models were developed
around the same time that used a spatial extension of the
Leontieff Input-Output model of the U.S. national economy
\cite{echenique-transport-reviews-1990,delabarra-book-1995}. There
were high expectations in the early modelling projects that these
projects would rapidly change the policy-making process, making
urban models the centerpiece of urban policy.  Expensive projects
needed to be evaluated efficiently prior to making substantial
capital investments, and cost-benefit analysis was promoted to
assess the relative cost-effectiveness of alternative projects.

The design of early operational urban models was generally
aggregate in the representation of agents and geographies, and
cross-sectional in the representation of time.  They imposed some
form of iterative procedure to converge to a time-abstract
equilibrium, and assigned this to a specific year in the long-term
planning horizon for transportation projects, with no
path-dependence representing the evolution of the urban system
from the current to the predicted state.  In 1973, Douglas Lee
wrote "Requiem for Large-Scale Urban Models," a scathing critique
of the urban modelling efforts of the time
\cite{lee-1973,lee-1994}. This critique heavily influenced federal
attitudes towards investment in urban modelling research, and
helped to trigger an era of skepticism about their potential that
persisted for more than two decades.

Most of the criticism raised by Lee was well-founded.  The models
came to be known as `black-box' models because their behavior was
not at all transparent, they were unnecessarily complex and
abstract, and this significantly limited their effectiveness as
tools for communicating with policy-makers and the public about
policy alternatives.

Many things have changed regarding the environment for urban
modelling since Lee's critique in 1973, including a tendency to
forget the concerns he raised.  Much of the obvious change is on
the 'supply side' of modelling, or the capacity for
model-building. Computational capacity has dramatically increased,
information management in the form of database and Geographic
Information Systems has revolutionized data processing, data
available for fitting urban models has become more widely
available and more detailed, sophisticated statistical methods
have emerged, theoretical and empirical research on the underlying
processes has progressed, and new frameworks for microsimulation
and agent-based modelling at the individual level have appeared.

Other changes have come on the 'demand-side' of modelling, and
these have also heavily shaped recent modelling projects.  The
practice of urban policy development and planning has been
subjected to substantial pressure to become more transparent and
democratic, with greater public participation in all stages of the
policy process.  This pressure has met varying degrees of
responsiveness among public agencies, of course, but the pressure
is widespread.  Further, the nature of the questions being raised
in the policy process are increasingly diverse, and considerably
more complex than the civil-engineering oriented processes of the
middle twentieth century that focused on efficiently sizing
capacity to meet anticipated demand on the transportation system.
Questions of economic efficiency are now generally balanced in the
public discourse with concerns about social equity and
environmental health and sustainability, with widely varying
representation of these stakeholder interests from place to place.

The convergence of these supply-side and demand-side trends has
led to a resurgent interest in urban modelling for formative
policy evaluation. In 1995, a conference was convened by the
Travel Model Improvement Project of the Federal Highway
Administration to assess he state of urban modelling.  Many of the
recommendations echoed criticisms raised two decades earlier by
Lee, but there was also a clear sense of renewed enthusiasm in
developing new modelling approaches, and the ensuing decade has
generated a proliferation of modelling projects.

In the late 1990's, the development of UrbanSim  was begun as an
effort to create an operational urban modelling system that would
be a fundamental departure from the aggregate, time-abstract
models that had been the mainstream approach in operational urban
models, and which would respond to the above-mentioned changes on
the demand and supply side of modelling.  UrbanSim was intended to
promote a more open, participatory approach to formative policy
evaluation, so it emphasized behavioral transparency in its
design.  In keeping with an emphasis on transparency, it was
released as Open Source software, and has been continually updated
since its initial release in 1998.  It is available from
\url{www.urbansim.org}.

The emphasis on transparency led to a design focus on
representation of agents, their choices, and their dynamic
interaction over time. The intent was to make the modelling
reflect as transparently as possible the real world aspects the
model represented, so that communication about the model would
reflect as much as possible the syntax used to describe the real
phenomenon of interest: households and firms locating, real estate
developers developing and redeveloping real estate, and the role
of governments in creating transportation infrastructure that
influences accessibility and setting land use regulations that
constrain development.  The representation of policies, ranging
from transportation, to land use, to environmental, should be
explicit, and their effects on outcomes of interest manifested
through the sensitivity of the agents to changes in their
environment caused by these policies.
