% $Id: related-work.tex,v 1.6 2006/06/05 05:47:20 borning Exp $

\subsection{Related Work}

UrbanSim has been developed over the past decade, and has both benefited
from prior work and, we think, substantially extended prior modeling
approaches.  While space prevents a complete treatment of how UrbanSim
compares to other modeling approaches, we offer here a concise treatment,
along with pointers to more in-depth reviews of urban models, including
UrbanSim itself
\cite{dowling-nchrp-2005,miller-tcrp-1999,southworth-1995,waddell-ulfarsson-2004}.
We begin by
contrasting UrbanSim with two alternative modeling approaches that were
available by the mid-1990's, when the design and development of UrbanSim
began.  The approach that has been most widely used to do land use modeling
is based on the Lowry gravity model \cite{lowry-1964}, which adapted the
law of gravity from physics and applied it to predict the flow of travel
between locations as a function of the sizes of origin and destination and
the ease of travel between them.  This approach, also widely referred to as
a spatial-interaction approach, was used widely in transportation modeling
to predict trip distribution patterns, was later refined and adapted to
predict land use patterns in the DRAM/EMPAL model system developed around
1970 \cite{putman-book-1983}.  This approach has been criticized as
over-emphasizing the role of transportation in location choices, and
lacking behavioral content such as a representation of real estate markets
(housing supply, demand, and prices are not considered, only household and
job locations in zones).  An alternative approach is the spatial
input-output approach that was also initially developed around 1970,
adapting the input-output models developed to describe monetary flows in
the U.S. economy to predict monetary flows between economic sectors and
between zones \cite{delabarra-book-1995,echenique-transport-reviews-1990}.
This is roughly analogous to an international trade model in which the
countries are replaced by some geographic subdivisions of the study area.
These spatial input-output models translate from monetary flows into tons
of freight, and by dividing the monetary flow from labor to a sector by the
average wage rate in the sector, derive the number of workers and the
implied commuting flows.  While this approach is intriguing for large-scale
areas where there is interest in predicting inter-metropolitan freight
movement, it seems less suited to developing a
behaviorally-transparent model of how households choose where to live or
business choose to locate, or of the behavior of real estate developers.
Unlike the spatial-interaction approach, the spatial input-output approach
does recognize the role of markets, and treats demand, supply, and
prices.  Both of these modeling approaches share the approach used by
virtually previous all urban models: they formulate an equilibrium set of
conditions and solve for an equilibrium that is not path-dependent, rather
than attempting to predict dynamic changes over time.

UrbanSim differs significantly from these and other earlier modeling
approaches in several respects.  First, it emphasizes behavioral
theory and transparency.  This leads to a more explicit treatment of
individual agents such as households, jobs, and locations, and to a
microsimulation of the choices that these agents make over time.
UrbanSim was the first operational modeling system to use very small
geographic units such as parcels or small grid cells, and to
microsimulate individual agents and their choices in a framework
that includes an explicit representation of real estate demand,
supply, and prices and their evolution over time.  One other
approach that has emerged recently warrants comment.  The use of
rule-based models that are embedded in highly graphical environments
such as Geographic Information Systems has increased in the past
several years. These are not behavioral models in the sense that
they do not attempt to reflect the pattern of agent behavior
observed in the real world. Instead, these tools allow a user to
impose on the system a set of assumptions, such as the distribution
of population twenty years in the future, and then to visually
examine the result of applying those assumptions.  While the intent
is quite different than the ones we have outlined above, there is
some risk that users may misinterpret the output of such tools as
containing more behavioral information than it in fact does.  (In the above
example, simply asserting that a given neighborhood will have a
given population density is very different from simulating the
results of policies, such as transportation and zoning changes,
designed to achieve that result. After all, the policies might or
might not achieve the desired result.)  There may in fact be some
useful complementarity between these kinds of visually-engaging
tools and more behaviorally-focused models: the visual tools could
provide a useful interface to obtain information from stakeholders
about their preferences, and the behavioral models could provide
stakeholders feedback about how effective alternative policy actions
might be in achieving those objectives, and to help examine what the
trade-offs are among the objectives.

% LocalWords:  tex pwaddell UrbanSim Lowry EMPAL microsimulate
