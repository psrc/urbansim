% $Id: conclusion.tex,v 1.6 2006/06/02 19:02:35 borning Exp $

\section{Conclusion and Future Work}

In this chapter we have presented an introduction to the domain of urban
modeling and to some of the uses and controversies around employing these
models to inform public decision-making, including a taxonomy of refinements
to urban models and to the process of applying them.  We then presented a
case study of the UrbanSim model system, including principal areas of
research and some applications to planning activities in different regions.
This domain represents a significant opportunity for digital government
research: hard technical problems, unmet demand from government users, and
important issues around supporting a more democratic planning process.

There is work that remains to be done.  Most importantly, our goal
of producing a system that is in routine and widespread use in
informing the planning process is not yet achieved.  UrbanSim is
being transitioned to operational use in a number of regions, and
there are a fair number of additional research applications of the
system.  However, it is not yet in routine policy use.  Beyond that,
the development of the Opus platform should enable a rich set of
collaborations among researchers world-wide, including the
development of open-source travel models and environmental models
closely integrated with the land use models in UrbanSim.  Finally,
we have touched on two other major open areas of ongoing research:
first, increasing access to the results of modeling for a wide range
of stakeholders, and ultimately to simulating additional
alternatives; and second, providing a principled modeling of
uncertainty in land use and transportation simulations.

% LocalWords:  borning pwaddell UrbanSim
