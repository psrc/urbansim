% $Id: introduction.tex,v 1.16 2006/06/05 15:50:58 borning Exp $

\section{Introduction}

Decisions regarding major urban transportation investments, as well as
regarding policies to improve air quality and to manage urban development
to reduce the adverse effects of low-density urban sprawl, are critical,
interdependent choices that shape the long-term quality of life in urban
areas.  These choices and the problems they attempt to address have
important social, economic and environmental impacts that spill over
jurisdictional boundaries and that are impacted by decisions made by a wide
range of institutions.  In the United States, they fall into the scope of
metropolitan governance, where the institutional frameworks for forming and
implementing policy are less robust than at higher or lower levels of
government.  These metropolitan governance structures hover between the
vise-grip of local governments' control of land use decisions, and the
state and federal control of resources for transportation and environmental
regulations.  In the gap, Metropolitan Planning Organizations (MPOs) have
been created by states under federal requirements to better coordinate the
allocation of federal investments in transportation, and air quality
planning.  These MPOs generally do not have any taxing or direct
implementation or operational responsibility, but are charged with creating
regional transportation plans and coordinating these with land use and air
quality planning.  It is a tall order.

Putting institutional difficulties aside for the moment, the task of
developing regional transportation plans is complex enough at a technical
level.  How can an almost infinite list of alternative transportation
investments proposed by local governments, states, and other entities be
examined systematically and an investment plan adopted that reflects a
democratic process based on a robust assessment of the alternatives?  Over
the past several decades, MPOs and their predecessor institutions have used
simulation models to predict the volumes of traffic on the transportation
network, given assumptions about the land use patterns that would generate
patterns of travel demand on this network.  The traditional models are
called `four-step models' because they break this task into (1) predictions
of the number of trips generated and attracted in each zone of the
metropolitan area, (2) the trip distribution patterns from zone to zone,
(3) the mode choice of trips (automobile, transit, etc.)\ between
any two zones, and ultimately, (4) how
these trips are assigned to the capacity-constrained network, leading to
patterns of travel time and congestion.  These four-step models were
originally developed within the discipline of civil engineering in the late
1950's and early 1960's to address a very specific problem: how to estimate
the amount and location of additional road capacity needed to satisfy a
given demand for transportation. They became ingrained into the planning
process for transportation, reinforced by federal investment and
regulation.  In the 1960's and 1970's, the four-step travel models were
brought into mainstream use and became the mainstay analysis tool used to
support decisions on alternative road investments.

Since the 1980's, however, the models and the decision-making process have
come under increasing scrutiny and criticism, leading to substantial
pressure to revise both \cite{beimborn-1996}.  One of the central
criticisms is that the models, and the way they have been generally used,
assume that changes in land use result in different demands on the
transport system, but that changes in the transportation system do not
cause land use changes to occur.  Aside from the mountain of theoretical
and empirical evidence to the contrary, this assumption violates common
sense.  Building a major highway through farmlands cannot be expected to
have absolutely no impact on the probability that sites along the new
highway, or accessible to it, will develop.  And if there is an impact on
development, the logical extension is that it will in turn impact travel
demand.  This idea is what has been referred to as induced demand, and one
of the reasons scholars have become increasingly skeptical that it is
possible to ``build your way out of congestion'' (see \cite{downs-2004} for
example).  Since the U.S. Clean Air Act Amendments of 1990 and the Intermodal
Surface Transportation Efficiency Act of 1991, federal policy has
recognized the need to link transportation and land use, in order to
account for this relationship.  Since that time, refinement of
transportation planning practice has been slow, partly due to the technical
difficulties of accounting for the interactions, and partly due to
political constraints and the increasing role of public involvement in
decision-making processes such as these.

Early use of technology such as transportation models to support
transportation investments dates to a conception of planners as technocrats
who provide answers that are to be taken at face value and used as an
objective basis for public decisions.  Public participation in these
decisions, and in the technical analyses behind them, was decidedly not on
the agenda.  Much has changed since then, especially at the local
government level.  An increasingly sophisticated and skeptical set of
stakeholders demands public participation, as well as transparency and
access to information about the decision-making process and the assumptions
and analyses behind it.  Conflicting interests are played out in public
meeting after public meeting and in committee after committee that is
deliberating land use policies or transportation investments.
Environmental advocates have increasingly come to use the courts to prod
planning agencies to refine their analyses to address shortcomings such as
the omission of land use feedback effects \cite{garret-1996}.

%% Though the mandate for setting and managing land use policy has been
%% claimed unambiguously by local governments, the mandates for setting
%% policies that cut across local jurisdictions are much less clear.  The
%% devolution of federal responsibilities to state and local governments makes
%% the setting of these policies increasingly a metropolitan agenda, but our
%% institutional organizations at metropolitan scales are not fully developed
%% to address the these policies.  In the 1970's, a movement towards
%% regionalism spawned the creation of Councils of Governments (COG) to
%% oversee some limited aspects of coordination of local government decisions
%% and investments.  But COGs had no taxing and no real operation authority,
%% and were often criticized as being incapable of taking strong positions due
%% to their construction as agents of local governments.  Beginning around
%% 1990, federal legislation such as the Clean Air Act Amendments and ISTEA
%% began to reinvigorate metropolitan governance in the form of Metropolitan
%% Planning Organizations (MPO), charged principally with the role of
%% coordinating federal investments in transportation within their
%% metropolitan areas.  Still, these MPOs generally lack taxing and
%% operational mandates beyond coordinating long-term plans for transportation
%% investments, and have boards that are not directly elected.  Making
%% difficult decisions about the allocation of transportation funds, and the
%% even more difficult tasks of coordinating land use and environmental
%% policies with these, has generally proven to be challenging within the
%% current institutional framework for metropolitan governance.

\subsection{Urban Modeling as a Digital Government Research Area}

The domain of land use and transportation modeling thus provides an
significant opportunity for digital government research: it is of great
interest to government agencies, and it includes a set of hard, open
problems, both technical and procedural.  This chapter is intended for
digital government researchers and students who are generally computer- and
policy-literate, but who are not necessarily expert in either the domain of
urban modeling or of land use and transportation policy.  In the chapter,
we first present a taxonomy of needed refinements to urban models
themselves, and to the process of applying them.  We then present a case
study of UrbanSim, an urban modeling system that our group has been
developing at the University of Washington, including a short history, more
recent research initiatives, and some significant applications to planning
activities.

Our focus in this chapter is primarily on the U.S. context.  However,
controversies regarding land use and transportation occur world-wide, and
analogous issues arise around using models to inform decision-making in
other countries.

\subsection{A Taxonomy of Model and Process Refinements}
\label{sec:refinement-taxonomy}

Our research is intended to contribute both to improving the technical
modeling capacity to address issues such as the land use consequences of
transportation investments, as well as to improving the process of using
models in a democratic decision-making context. To help structure this case
study, as well as providing a framework for evaluating urban models, we
offer the following taxonomy of model and process refinements
(Table \ref{taxonomy-table}).  We hope
that this taxonomy will be of value beyond this particular case study as
well, for other studies of modeling and simulation in the policy arena.  In
developing this framework, we draw on and extend earlier work that has
criticized earlier urban models (for example \cite{lee-1973}).  We then
describe how our project and several research initiatives within it have
emerged to address these challenges.

\begin{table}[t]
\begin{itemize}
\item Refinement of Models
   \begin{itemize}
   \item Validity
      \begin{itemize}
      \item Accuracy
      \item Handling of uncertainty
      \item Policy sensitivity
      \end{itemize}
   \item Comprehensiveness
      \begin{itemize}
      \item Real estate development and prices
      \item Employment location
      \item Household location
      \item Transportation system
      \item Environmental impacts
      \end{itemize}
   \end{itemize}
\item Refinement of Process
   \begin{itemize}
   \item Refinement of the model construction and application process
      \begin{itemize}
      \item Feasibility of data preparation
      \item Performance
      \item Usability
      \item Support for software evolution
      \end{itemize}
   \item Support for a more effective democratic process
      \begin{itemize}
      \item Responding to stakeholder interests and concerns
      \item Transparency
      \item Fairness
      \item Facilitating stakeholder access to models and their output
      \end{itemize}
   \end{itemize}
\end{itemize}

\caption{Model and Process Refinements}
\label{taxonomy-table}
\end{table}

\subsubsection{Refinement of Models}

At the top level, we distinguish between \emph{refinement of models} and
\emph{refinement of process}.  Refinement of models focuses on the models
themselves.  In turn, we can classify the work on refinement of models
as work on \emph{validity} and on \emph{comprehensiveness}.

Validity includes improving the accuracy of the models, and also their
sensitivity to policies of interest.  Accuracy means that the predicted
values (for example, of population density in different neighborhoods) are
close to the observed values.  This raises the obvious problem of how to
evaluate the accuracy of predictions of events in the future.  One
technique is \emph{historical validation}, in which the model is run on
historical data, and the results compared with what actually transpired
(see \cite{waddell-japa-2002} for example). This has the clear merit of
comparing with real outcomes.  There are
difficulties as well, however.  First, in many cases the needed historical
data is not available.  Also, for the relatively small number of regions
for which data is available, there may not have been major land use and
transportation changes over the period being tested, so that the model in
effect isn't being used to simulate major decisions.  An alternative
technique that is
often used is to run the model system with fairly extreme scenarios
(e.g.\ doubling the capacity of selected roadways, or removing zoning
restrictions on height limits in a neighborhood).  The results are then
evaluated by an expert review panel.

Predicting the future is a risky business.  There are numerous,
complex, and interacting sources of uncertainty in urban simulations
of the sort we are developing, including uncertainty regarding
exogenous data, the model structure and specification, the
parameters of the model, and from the stochastic nature of the
simulation. Nevertheless, citizens and governments do have to make
decisions, using the best available information. Ideally we should
represent the uncertainty in our conclusions as well as possible,
both for truthfulness and as important data to assist in selecting
among alternatives.  However, to date there has been only a small
amount of work done on handling uncertainty in urban modeling in a
principled fashion \cite{sevcikova-trb-2006}.

We often also want to improve the sensitivity of the model to policies of
interest.  For example, if a region is interested in policies that foster
walkable neighborhoods, then the model should be able to model walking for
transportation as well as for health and recreation.  Which policies are of
interest is of course a political and societal question; but given such
policies, whether the model responds suitably to them becomes a question of
validity.

Yet another sort of refinement of the models is increasing their
comprehensiveness to include other actors and processes in the urban
environment.  For example, for households, we might model additional
demographic processes, such as household formation and dissolution.
Or for environmental impacts, we might model consumption of
additional kinds of resources, or the impacts of decisions on
biodiversity as well as on particular species of interest (for
example, due to Endangered Species Act considerations).

There are important pitfalls and tensions associated with the goal of
increasing the comprehensiveness of models: namely what Lee \cite{lee-1973}
called the problem of hyper-comprehensiveness.  One aspect of this is
pressure to model more and more aspects of the urban environment because
these aspects are important to someone --- even though they might have
little relation to land use and transportation.  For example, there might
be demands to model voter turnout rates.  These pressures are relatively
straightforward to deal with, by reminding stakeholders of the purpose of
the modeling work and the need to remain focused.  A more difficult issue
is that a seemingly endless number of factors influence urban land use and
transportation.  For example, crime is clearly an important factor in
residential location choice, in transportation choice, and others.  But we
need not just data on current crime rates --- and perhaps more importantly,
on people's perceptions of crime --- but also a predictive model of crime
in the future under different possible scenarios.  This is both difficult
and controversial.  For example, what are the major determinants of the
crime rate?  Economic conditions?  Family stability and moral instruction?
The nature of the criminal justice system?  How far should the modeler go
down this path?  Or as another example (relevant to the region around
Seattle), suppose we want to model the return rate of wild salmon in rivers
and streams that flow through urban regions.  There are many factors
affecting this: the amount of impervious surface, pollutants from
agricultural runoff, the number of fish caught by both commercial and sport
fishers, oceanic conditions (including temperature, since the salmon grow
to maturity in the ocean before returning to fresh water streams to spawn),
and many others.  Among the pitfalls of overly ambitious modeling are
increasing model complexity, additional data requirements, and in some
cases the credibility of the overall modeling effort.

\subsubsection{Refinement of Process}

Returning to the top level of the taxonomy, refinement of process includes
first, improving the process of developing, extending, and applying models;
and second, supporting their more effective use in a democratic society.
The first of these is concerned with instrumental values such as usability
and feasibility: data preparation issues, adequate performance, usability
of the software, and accommodating changes in requirements, data, and the
like.  It must be feasible to prepare the data needed to run the model.
Typically, this implies that the data must already be in hand ---
collecting new data is enormously expensive.  But the data in hand may be
of varying quality, or in the wrong format, and so forth.  Performance should
be adequate, and the software must be usable by the technical staff at the
planning organization.  Also, the model system architecture should allow
for the system to evolve as requirements and the questions asked change
over time.

Another set of process issues revolve around the desire to use modeling as
part of a more participatory, open, and democratic process, rather than in
a technical, back-room exercise.  One aspect of this is improving the
relevance of the modeling and output to the diverse range of stakeholder
concerns (in other words, increasing its comprehensiveness in response to
stakeholder values).  Transparency of the model itself, of the input data
preparation process, and of the overall context in which it is used also
play an important role as well.  Another aspect of this is improving the
fairness of the model (for example, in not omitting an important
transportation mode, or short-changing the interests of renters as compared
with home owners).  Again, this can result in additional demands for
refinement of the model (either its validity, comprehensiveness, or both).
The results of running the model, and ideally even the ability to
experiment with alternatives, should be opened up to a wider range of
stakeholders, rather than being restricted to the technical modelers.
System performance is relevant here also: for example, if the model takes
weeks to run, clearly this would make it difficult to use to support
deliberation, in which model results are discussed, and in response new
questions are asked of the model or new scenarios are proposed for testing.

There are some obvious tensions among these objectives for refinement of
models and process.  Pressures to increase policy sensitivity in order to
avoid bias from omission of certain policies from consideration, for example
pedestrian and bicycling modes, increase the need for
a very high level of behavioral and spatial detail.  This
will certainly come at a cost in performance, and
quite possibly also at a cost of some reduction in the accuracy of the
results.  How can model sensitivity, data requirements, transparency,
computing performance, and accuracy be compared against each other?  How are
the interests of different stakeholders served by alternative compromises
among these?  How do these choices affect the legitimacy of the model
system and the process for using it in the decision-making process?  These
are difficult problems, and ones that have not received sufficient
attention to date.  We seek to address these concerns in our project in
addition to the more purely technical issues of model refinement.

% LocalWords:  borning MPOs Intermodal analyses UrbanSim pwaddell
