% $Id: applications.tex,v 1.8 2006/06/05 05:47:20 borning Exp $

\subsection{Applying UrbanSim}
\label{sec:urbansim-applications}

An important part of our research agenda has been close collaborations with
a number of regional government agencies in applying UrbanSim to their
regions.  In this subsection we call out the Salt Lake City and Puget Sound
applications in particular.  These two cases present substantially
different aspects, both in terms of characteristics of the region, and of
the legal and political context.  The Salt Lake City application is
described in detail in reference \cite{waddell-wfrc-2006}, and more briefly
in reference \cite{waddell-sscr-2004}.  The PSRC application is being
studied, particularly with respect to the PSRC/UW collaboration, the
decision-making regarding adoption, and their implications for use of the
model in practice, as a component of Ruth F\"{o}rster's forthcoming
Ph.D. dissertation \cite{foerster-phd}.

During 2002--20003, we worked intensively with the Wasatch Front
Regional Council in applying UrbanSim to Salt Lake City and the
surrounding areas, building on earlier work with the agency.  The
Greater Wasatch Area, containing 80\% of Utah's population and
centered on Salt Lake City, is a rapidly growing metropolitan area,
with population predicted to increase by 60\% in the year 2020.  In
order to deal with projected increases in population and travel,
Utah officials developed a series of transportation improvement
plans, one component being the Legacy Parkway, a controversial
four-lane highway extending 14 miles north from Salt Lake City.  The
highway project precipitated a lawsuit, which ultimately resulted in a
settlement that included terms requiring the Wasatch Front Regional
Council to test the integration of UrbanSim with their regional
travel model system, and if successful, to bring this into their
operational use in transportation planning.

The assessment of the integration of UrbanSim with the regional
travel model system was launched in 2003 with the formation of a
Peer Review Panel, consisting of technical experts in land use and
transportation modeling, along with a Management and Policy
Committee and a Scenarios Committee.  A very tight schedule was
specified in the out-of-court settlement, requiring that the entire
review be completed by the end of 2003.  The Peer Review Panel
decided on a validation of the combined UrbanSim -- Travel Model
System using a series of tests.  One test was to model the effects
of the existing long-range plan (LRP) in 2030.  There were also five
other ``sensitivity tests,'' each of which involved making a simulated major
change to the adopted plan and assessing the results, for example,
removing a major highway link included in the LRP, removing a major
transit link, or adding a significant land use policy such as an
urban growth boundary.  The overall evaluation focused on issues of
model validity and usability.

The Peer Review Panel concluded that UrbanSim produced credible
results for tests involving large changes (e.g.\ the urban growth
boundary test), but not for relatively smaller changes (e.g.\
removing a major highway link).  It was during this process that we
discovered the dispersion problem mentioned earlier. Their summary
assessment supported the implementation and application of UrbanSim
by the WFRC, with the understanding that important refinements and
improvements were needed.  Subsequently, during 2004, staff of the
WFRC made additional efforts to improve the data and the model
specification, and the agency decided to move the model into
operational use.

Another important collaboration has been with the Puget Sound
Regional Council (PSRC), to apply UrbanSim to that region (which
includes Seattle and other surrounding cities).  In this project, a
collaborative agreement was developed between the PSRC and the
Center for Urban Simulation and Policy Analysis (CUSPA), to develop
the database and apply UrbanSim to the Central Puget Sound region
containing Seattle.  A Technical Advisory Committee, consisting of
planners and analysts from cities and counties in the region, was
engaged with the process to review the development of the data and
model, and to provide refinements to the data and feedback on the
model development.  The process was managed closely by two staff
members of the PSRC, and during the project two PSRC staff members
were hired by CUSPA for the intensive database development effort,
and then re-hired by PSRC afterwards.  The project was funded by the
PSRC and structured in annual contracts with clear work scope for
each period.  The first two years were essentially spent on database
development and refinement, and the third on model estimation and
testing.

During this period, as we learned from both the Salt Lake City
project and the Puget Sound project about several problems in the
model specification and software implementation in UrbanSim 3, we
confronted a difficult choice of whether to attempt to resolve these
problems in the production system (UrbanSim 3 in Java), or invest
all of our effort on completing more quickly the conversion to the
more modular Python implementation in OPUS and UrbanSim 4.  We had
already learned that we could more readily solve modeling problems
using the incomplete Python version that we had been unable to
address in the UrbanSim 3 code base (mainly due to the complexity of
debugging the code).  At the same time, there were considerable
risks to attempting to rapidly complete and test the conversion to
Python without putting the PSRC project well behind schedule.
Ultimately, after extensive consultation with the PSRC, we made the
decision to freeze further investment in the UrbanSim 3 code, other
than minor maintenance, and to put all our effort into the new
platform. Since that decision, we have fully implemented the PSRC
model application using the new UrbanSim 4 code in OPUS, and have
done extensive testing with the system.  It may have added as much
as a year to the project schedule over what had been expected, but
it is not clear that the schedule would have been any earlier if we
had continued working on the UrbanSim 3 platform to attempt to
resolve the problems we faced.

PSRC staff worked with CUSPA to develop criteria for evaluating the model
results and to determine when it would be ready to put into production use.
Unfortunately, longitudinal data were not available to undertake a
historical validation as was done in Eugene-Springfield
\cite{waddell-japa-2002}, so the focus of the evaluation was shifted to
sensitivity analysis and comparison with previous results. The PSRC has used
the DRAM/EMPAL model to prepare land use allocations for use in its
transportation planning process for many years, and though they have
numerous concerns about it, they have managed to find workable solutions by
overriding the model results based on local review procedures using their
Technical Advisory Committee and a Regional Technical Forum.  The
assessment of UrbanSim involved comparing the predictions from 2000 (the
base year) to 2030 using UrbanSim with the heavily reviewed and adjusted
predictions of DRAM/EMPAL\@.  Since the two models use different levels of
geography (DRAM/EMPAL uses approximately 200 Forecast Analysis Zones and
UrbanSim uses approximately 800,000 grid cells), UrbanSim data were
aggregated to levels that could be compared to the previous results. The
initial focus was on population and employment predictions, since these are
the data used in the travel model system.  A set of indicators was selected
for use in diagnosing the performance of the model and to identify issues
and problems, in a process of negotiation among the participants.

Based on this, the UW/PSRC team classified the issues into critical,
significant, and cosmetic.  (Critical issues were ones that had to be
resolved before the PSRC would be comfortable placing the model system into
operational use.  Significant issues were ones that PSRC regarded as
important, but would not block putting the system into operational use.)

In addition to comparing the predictions of UrbanSim with previously
adopted forecasts, a series of sensitivity tests were conducted to
determine if the results from the model were sensitive in the
direction and magnitude experts would expect when an input was
changed significantly.  For these tests, we compared results for the baseline
scenario, which was based on currently adopted land use policies and
the adopted Regional Transportation Plan, with results for
scenarios that included
doubling highway capacity in Snohomish County on the north end of
the study area, relaxing the Urban Growth Boundary, and constraining
development capacity in King County (the central county in the study
area).  These tests 
were selected to probe both the scientific robustness of the
model and its sensitivity to policies of interest in the region.
This can thus be seen as addressing both the accuracy and policy
sensitivity aspects of model validity, as discussed in the taxonomy
presented in Section \ref{sec:refinement-taxonomy}.

The land use policies showed the expected effects, but the
highway scenario showed less sensitivity than was expected, in the
sense that it did not produce much redistribution of population and
employment into Snohomish County.  This was consistent with a
pattern in the comparison of the UrbanSim baseline scenario against
previously adopted forecasts, which showed considerably lower growth
rates in Snohomish County than the earlier results did.  These
remain on the agenda for further examination to determine whether
these results are plausible or are an artifact of the model
specification or input data.  In spite of these remaining issues,
the PSRC is moving ahead with the process of bringing UrbanSim into
operational use, and is preparing for a final phase of work to do so
during 2006-07.

In addition to the Salt Lake City and Puget Sound applications, we
have worked with other agencies in applying UrbanSim in the urban
areas around Detroit, Eugene, Honolulu, and Houston.  There have
also been research and pilot applications in Amsterdam, Paris,
Phoenix, Tel Aviv, and Zurich. We have also been working actively to
form a user community.  The first UrbanSim Users Group Meeting was
in San Antonio, Texas, in January 2005. This meeting attracted some
30 participants from Metropolitan Planning Organizations around the
country, a number of academic researchers, and one participant from
Europe.  The second UrbanSim Users Group meeting will be held in
July 2006, with one important goal being to help existing UrbanSim 3
users to migrate to UrbanSim 4.  We hope to increasingly engage
model users and other research groups in the collaborative
development of Opus and of
UrbanSim itself, by extensions and refinements to the current model
system, and the addition of new tools and models.

% LocalWords:   borning UrbanSim FHWA COE NGO Utahns WFRC WFRC's LRP PSRC San
% LocalWords:  pwaddell UW rster's CUSPA afterwards EMPAL Snohomish
