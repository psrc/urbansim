% $Id: history.tex,v 1.7 2006/06/05 05:47:20 borning Exp $

\subsection{Early Development of UrbanSim}

Paul Waddell originally became interested in the problem of land use
and transportation modeling in the 1980's, when he worked for a
metropolitan planning organization and encountered first-hand the
unsatisfactory state of modeling practice.  At that time (and it is
still widely the case today), the models available for use in
metropolitan planning lacked behavioral realism, had insufficient
theoretical underpinnings, were largely insensitive to important
public policies, and as a cumulative result of these considerations,
lacked credibility with both technical planners and with
policy-makers. These models were, in short, ineffective.

In the mid-1990's Waddell began designing and developing UrbanSim to
address some of the clear shortcomings of existing models to support
metropolitan planning.  The design choices in this early work
focused on improving behavioral realism, validity, and policy
relevance.  UrbanSim was designed based on a micro-simulation
approach to modeling household choices of residential location,
business location choices, and real estate development and prices,
and the use of a dynamic, annual simulation of the evolution of
cities over time.  It attempted to make explicit the role of policy
inputs, such as comprehensive land use plans and transportation
infrastructure, in order to support the evaluation of alternative
policy scenarios and their effects.  Waddell supervised the
development of a prototype version in Java for testing in Eugene,
Oregon \cite{waddell-env-and-planning-2000}, supported by a contract
with the Oregon Department of Transportation.

In 1998, Alan Borning began collaborating with him on this work,
based on common interests in using information technology to improve
the process of metropolitan planning. In 1999 we obtained the first
of a series of grants from the National Science Foundation to
support the project, this one from the Urban Research Initiative.
This more flexible funding allowed us to put together a team
consisting of graduate and undergraduate computer science students,
along with students from urban planning, civil engineering, and
economics, who rewrote the model and its interface (still in Java).

% LocalWords:  pwaddell UrbanSim Waddell Borning borning
