% $Id: additional-information.tex,v 1.7 2006/06/05 05:47:20 borning Exp $

\section{Additional Information}

\subsection{Acknowledgments}

We would like to thank all of the UrbanSim research team members and
collaborators.  This research has been funded in part by grants from
the National Science Foundation (EIA-0121326 and IIS-0534094), in
part by a grant from the Environmental Protection Agency (R831837),
in part by a partnership with the Puget Sound Regional Council, and
in part by gifts from IBM and Google.

\subsection{Suggested Readings}

% The directions say:
%
% Please list 2-10 essential readings (books or articles) for readers who
% wish to gain more in-depth knowledge in this topic area. Please provide 1-3
% sentences of brief explanation for each reading.

\begin{itemize}

\item Paul Waddell and Gudmundur F. Ulfarsson, ``Introduction to Urban
Simulation: Design and Development of Operational Models,'' \emph{Handbook
of Transport, Volume 5: Transport Geography and Spatial Systems},
P. Stopher et al., eds., Pergamon Press, 2004.  Available from
\url{www.urbansim.org/papers}.  An introduction to urban simulation
(interpreted broadly to mean operational models that attempt to represent
dynamic processes and interactions of urban development and
transportation).

\item Paul Waddell, ``UrbanSim: Modeling Urban Development for Land Use,
Transportation, and Environmental Planning,'' \emph{Journal of the American
Planning Association}, Vol.\ 68 No.\ 3, Summer 2002, pages 297--314.  An
early but accessible overview of the UrbanSim system.

\item Paul Waddell and Alan Borning, ``A Case Study in Digital Government:
Developing and Applying UrbanSim, a System for Simulating Urban Land Use,
Transportation, and Environmental Impacts,'' \emph{Social Science Computer
Review}, Vol.\ 22 No.\ 1, February 2004, pages 37--51.  As the title
suggests, presents UrbanSim as a case study in digital government research.
Emphasizes the various tensions between conducting university research and
applying it to a regional government application in Salt Lake City
in a highly visible, contested domain.

\item Batya Friedman, Peter H. Kahn Jr., and Alan Borning, ``Value Sensitive
Design and Information Systems: Three Case Studies,'' in
\emph{Human-Computer Interaction and Management Information Systems:
Foundations}, Ping Zhang and Dennis Galletta, eds., M.E. Sharpe, Armonk,
NY, 2006.  A description of the Value Sensitive Design theory and
methodology, including an UrbanSim case study.

\end{itemize}

\subsection{Online Resources}

\begin{itemize}

\item \url{http://www.urbansim.org} The UrbanSim website, including
  papers, downloads, and other information.

\item \url{http://www.urbansimcommons.org} The UrbanSim Commons is a
  companion site, designed and maintained by UrbanSim users.

\item \url{http://www.ischool.washington.edu/vsd} The home page for the
  Value Sensitive Design project.

\end{itemize}

\subsection{Questions for Discussion}

Here are a number of questions that would be suitable for classroom
discussion (as well as future research).

\begin{itemize}

\item Consider D. B. Lee's 1973 critique ``Requiem for Large-Scale Models''
  \cite{lee-1973}.  Which of the problems raised by Lee have been addressed
  by current models, and which are still open problems?  Of the open
  problems, which will it be feasible to address in the next decade?

\item What issues arise in applying UrbanSim to urban regions in countries
  other than the United States, including both developed and developing
  countries?  Consider issues of alternate land use laws and ownership,
  transportation patterns, data availability, and others.

\item There are huge uncertainties about some of the exogenous assumptions
  used by UrbanSim, including the future price of oil, the macroeconomy,
  possible technological shifts in transportation or telecommuting, and
  others.  How should these uncertainties best be communicated to users of
  UrbanSim?  What are appropriate controls for allowing them to change
  these assumptions?

\item Suppose UrbanSim were being designed for use by authoritarian
  governments rather than in a democratic context.  Are there things that
  the designers should do differently?  If so, what?

\end{itemize}

% LocalWords:  tex borning UrbanSim EIA IIS Google Waddell Gudmundur Ulfarsson
% LocalWords:  Stopher al Batya Kahn Zhang Galletta Online macroeconomy
% LocalWords:  pwaddell
