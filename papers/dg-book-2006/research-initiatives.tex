% $Id: research-initiatives.tex,v 1.14 2007/02/16 18:08:11 borning Exp $

\subsection{Subsequent Research Initiatives}

After the initial development work on the UrbanSim prototype and a first
effort to improve the software engineering and design of the system, three
priorities emerged.  First, we wanted to further refine the models, to
improve their validity and comprehensiveness (see the taxonomy in Section
\ref{sec:refinement-taxonomy}).  Second, we wanted to explore more
systematically the issues around opening the process of using the model to
more participation, as well as carefully considering issues of
transparency, fairness, and the range of stakeholders affected by the
decisions that were to be informed by information coming from the model
system.  These priorities in turn gave rise to the third priority, namely,
addressing the significant computational and software engineering
challenges that then arose due to the level of detail the system was
attempting to represent, the complexity of the models, and the need to
support system evolution and change.

Guided by these priorities, we submitted two additional NSF proposals.  In
2001 we were fortunate enough to be awarded a 5-year NSF grant under the
Information Technology Research Initiative, administered through the NSF
Digital Government program, along with a companion grant from the Digital
Government program itself.  This increased level of funding and longer
funding period enabled us take on a substantially larger and more risky
research agenda, while still pursuing our overall strategy of integrating
the research with close collaboration with local and metropolitan
governments and planning processes.  In the remainder of this section we
summarize these initiatives.  Relating back to the taxonomy, these
initiatives all apply to various items in the taxonomy of refinements, but
often an initiative will pertain to several --- there is usually not a
one-to-one mapping between initiatives and items in the taxonomy.

\subsubsection{Improving Model Validity}

A significant amount of the work in the past several years has concerned
improving the validity of the models.  Much of this work has been quite
technical, and for purposes of this case study, we mention two of these
activities, along with providing references to papers that provide
additional detail.

The location choice models (for Household Location Choice and
Employment Location Choice), for example, have undergone a series
of improvements. Despite this, throughout the UrbanSim 3 efforts
these two models suffered from a problem of dispersion. For
example, highly desirable residential areas
--- e.g.\ with a lake or ocean view --- typically have higher real estate
values and wealthier residents.  Absent some significant external
cause we would expect this to continue; yet in the simulation
results we were seeing a dispersion of wealthy households from
these neighborhoods, as well as poorer households moving into
these desirable areas. (While for policy reasons we might desire
more mixed neighborhoods, achieving this would require policy
changes
--- the effect we were seeing was due to some combination of model 
specification and software implementation.)
This shortcoming was an issue in an assessment of the application
of UrbanSim in Salt Lake City (see Section
\ref{sec:urbansim-applications}, and a full description in reference
\cite{waddell-wfrc-2006}). It was only in our most recent version
of the system, UrbanSim 4 (Section \ref{sec:opus}), that we were able to
address this. This problem was one of several that precipitated a
complete redesign of the software implementation, since diagnosing
and addressing it proved to be too challenging in the UrbanSim 3
implementation.

The Developer Model, which simulates the actions of real estate
developers, is one of the most complex of UrbanSim's component
models.  We have had more versions of this model component than
any other.  In part this reflects the relatively smaller amount of
academic research on real estate supply than on demand.
Two approaches are discussed in the real estate
literature: the site looking for a use and the use looking for a
site.  The former approaches the problem from the land owner's
perspective: whether to develop a parcel, and if so, into what
use.  The latter approaches the problem from the perspective of a
specialized developer, say, of office space, who needs to choose a
profitable location for the project.  Through UrbanSim 3 we
adopted the site looking for a use perspective, which was
consistent with the literature in land cover change. We learned
through several model applications, however, that this approach
was problematic from the perspective of policy sensitivity.  If we
significantly changed the land area available for development, for
example, modeling the introduction of an Urban Growth Boundary to
constrain low-density development, then the amount of development
was significantly reduced, which does not conform to expectations.
Instead, we would expect development to continue, but at higher
prices, and focused in the areas available for development.  So in
the development of UrbanSim 4, we re-designed the developer model
from the use looking for a site perspective, which has a natural
and behaviorally-correct response to a significant change in the
area of land available for development: it scales the
probabilities among the remaining available sites.


\subsubsection{Increasing Model Comprehensiveness --- Modeling
Environmental Impacts}

Another aspect of model refinement is increasing the comprehensiveness of
UrbanSim's component models to encompass additional phenomena.  In some
cases this is of primary interest as a means of improving model validity.
For example, one such effort has concerned improving the Demographic
Transition Model, which simulates the creation and removal of households
(to simulate the real world processes of births, deaths, children leaving
home, and movement into and out of the study area).  The primary purpose of
this activity is to improve the validity of the location choice models.

%% *** omitted for now - continue to leave out?  Put someplace else?
%%
%% In addition to model refinement, a companion aspect of this work has been
%% improving technical characteristics of our modeling process, by putting
%% parameters of this model into a set of Household Control Total tables in a
%% SQL database, allowing easier access and modification, and improving model
%% transparency.  These steps have significantly improved the situation
%% relative to where we were with respect to this model in 2000.  However,
%% shortcomings remain --- in particular, our demographic transition model
%% still does not simulate aging and other changes within each household, but
%% simply adds or removes households with the appropriate characteristics.  We
%% plan to address these shortcomings in future work.

Another reason to model additional phenomena is to capture important
impacts of land use and transportation decisions, even if these impacts
don't feed back directly into the operation of the other component models.
Environmental impacts are the most significant class of such phenomena.  In
our current implementation, we produce indicators of greenhouse gas
emissions from transportation, and have a component model that simulates
land cover change.  (Land cover is important in its own right, and is also
useful as a factor in producing other environmental indicators --- for
example, impervious surface affects water runoff characteristics.)  We plan
to add additional indicators in the near future, beginning with air quality
and resource consumption indicators.

As noted above, these indicators do not currently feed back into the
operation of the other component models --- although adding such feedback
(e.g.\ to the residential location choice model) is a research topic for
the future.  In any case, these indicators of environmental impact become
important considerations in the deliberative process in which the model is
used.

\subsubsection{Value Sensitive Design}
\label{sec:vsd}

The domain of urban planning is both value laden and rife with
long-standing disagreements.  In addition, as described in the
introduction, there is increasing desire to move the modeling and planning
process from a technically-focused ``back room'' operation to a more open
and participatory one, which immediately leads to questions of who are the
stakeholders and who should participate, along with other issues of
transparency, accountability, and openness.  In terms of our taxonomy,
these issues initially fall into the category of ``Support for a more
effective democratic process,'' but fairly quickly lead to issues regarding
the process of constructing and applying the models, as well as to
pressures for refinements to the model.

To approach these value issues in a principled fashion, we rely on the
Value Sensitive Design theory and methodology \cite{friedman-amis-2006}.
Value Sensitive Design is an approach to the design of information systems
that seeks to account for human values in a principled and comprehensive
way throughout the design process.  Key features are its interactional
perspective, tripartite methodology (consisting of conceptual, empirical,
and technical investigations), and emphasis on indirect as well as direct
stakeholders.

%% *** long version of the above paragraph: ***
%%
%% To approach these value issues in a principled fashion, we rely on the
%% Value Sensitive Design theory and methodology \cite{friedman-amis-2006}.
%% Value Sensitive Design is an approach to the design of information systems
%% that seeks to account for human values in a principled and comprehensive
%% way throughout the design process.  Key features are its interactional
%% perspective, tripartite methodology, and emphasis on indirect as well as
%% direct stakeholders.  Value Sensitive Design is an interactional theory:
%% people and social systems affect technological development, and
%% technologies --- such as urban simulation systems --- shape (but do not
%% rigidly determine) individual behavior and social systems.  Value Sensitive
%% Design employs a tripartite methodology, consisting of conceptual,
%% empirical, and technical investigations.  Conceptual investigations
%% comprise philosophically informed analyses of the central constructs and
%% issues under investigation.  Empirical investigations focus on the human
%% response to the technical artifact, and on the larger social context in
%% which the technology is situated, using quantitative and qualitative
%% methods from social science research.  Technical investigations focus on
%% the design and performance of the technology itself.  A third key aspect of
%% Value Sensitive Design is its focus on both direct and indirect
%% stakeholders.  The direct stakeholders are the users of the system; the
%% indirect stakeholders are those who don’t use the system directly, but who
%% are affected by it --- a group often overlooked in other design
%%% methodologies.

For UrbanSim in its current form, the direct stakeholders are the
urban modelers and technical planners who use UrbanSim and
manipulate its results.  The indirect stakeholders are those who
do not use the system directly, but who are affected by it.  They
include for example elected officials, members of advocacy and
business groups, and more generally all the residents of the
region being modeled, as well as residents of nearby regions.  One
way of framing our goal of opening up the planning process and use
of models to wider participation is to move more people from the
indirect to the direct stakeholder category: to provide meaningful
direct access to the results from the models, and ultimately to
the tools for simulating other possible plans and futures.

Early in our conceptual investigations, we made a sharp distinction between
explicitly supported values (i.e., ones that we explicitly want to support
in the model system) and stakeholder values (i.e., ones that are important
to some but not necessarily all of the stakeholders).  Next, we committed
to several key moral values to support explicitly: fairness and more
specifically freedom from bias \cite{friedman-tois-1996},
representativeness, accountability, and support for a democratic society.
In turn, as part of supporting a democratic society, we decided that the
system should not a priori favor or rule out any given set of stakeholder
values, but instead should allow different stakeholders to articulate the
values that are most important to them, and evaluate the alternatives in
light of these values.  We identified comprehensibility, and subsequently 
legitimation and transparency, as key instrumental values to be supported
explicitly as well.

In terms of specific Value Sensitive design research activities, one
project involved carefully documenting and presenting the indicators that
portray key results from the simulations \cite{borning-ecscw-2005} using a
web-based Indicator Browser.  We also developed the Indicator Perspective
framework, which allows a set of organizations with a diverse set of
positions and interests to put forth different perspectives on what is most
important in the results from UrbanSim, and how it should be interpreted.
Our design for the Indicator Browser and Indicator Perspectives addresses a
number of challenges, including responding to the values and interests of
diverse stakeholders, and providing relatively neutral technical
information and at the same time supporting value advocacy and opinion.  A
third activity has been the development of ``Personal Indicators'' that
allow individuals and households to ask how different alternatives will
affect them personally, for example, in terms of housing costs or
individual commute times \cite{davis-chi-2006,davis-phd}.  Together, these
projects represent a significant set of steps toward making the modeling
components of the planning process more open and participatory, supported
by a solid theoretical foundation.

%% \subsubsection{Other User Interface Issues}
%% GUIs for modelers
%% eclipse project - subsequently abandoned
%% run monitor

\subsubsection{Software Engineering}

The significant performance demands of the system that arise due to the
level of detail being represented, the complexity of
the models, the close scrutiny of the simulation results, and the need to
support system evolution and change, all give rise to pressures for good
software engineering and software development practices.  The ITR grant,
with its funding level and five-year term, enabled us to hire two software
engineering professional with substantial industry experience, who were
then able to provide much stronger guidance on our software development
process, as well as being able to work closely with the students involved
with the software side of the project over the years, including a large
number of computer science and computer engineering undergraduates.

We put into place an agile software development process
\cite{beck-book-2000}, which relies on small, incremental development steps,
automated testing, and responsiveness to change.  \emph{Unit tests} ---
small, self-contained tests that exercise a particular part of the code ---
play an essential role in the process
\cite{hunt-book-2003,noonan-sigcse-2002}.  Following a
modified eXtreme Programming approach \cite{beck-book-2003}, we carry this
further and use a test-first development strategy, in which the software
developer first writes the unit tests, and then writes the code to be
tested.  We also rely on an automatic build system that runs all the tests
whenever new code is checked into the code repository.  A novel feature of
our development environment is a set of traffic lights (real ones), placed
in the hallway of our lab and in developers' offices.  When the light is
green, the checked-in code has passed all tests; when it is yellow, the
tests are currently being run on newly-submitted code; and when it is red,
one or more tests have failed.  A green light serves as a reassuring status
indicator of the state of the software, while a red light is a powerful
social incentive to fix the software problem, as a top priority.  The
traffic light and other novel features of our development methodology are
described in a paper in the 2003 Agile Development Conference
\cite{freeman-benson-agile-2003}, while another novel feature of our
process, the use of `today' messages as a coordination tool for software
developers, is discussed in more detail in reference
\cite{brush-hicss-2005}.  Finally, a recent paper
\cite{sevcikova-issta-2006} describes a novel extension to the unit testing
framework to handle testing stochastic models (such as many of the UrbanSim
models).

\subsubsection{Opus and UrbanSim 4}
\label{sec:opus}

In 2002 we rewrote UrbanSim again (still in Java), resulting in
UrbanSim 3, which was our production system for several years.  The
use of Java as an implementation language for UrbanSim was a good
choice in many respects, but there were some problems as well, in
particular, inaccessibility of the code to domain experts, and
performance issues.  Regarding accessibility, our hope was that, by
using suitable abstractions and a very clear coding style, that
modelers (domain experts), as well as software engineers, could
experiment with the models.  Alas, due to the inherent complexities
of the Java implementation of UrbanSim 3 and the Java development
environment, this did not happen. There were also some intractable
modeling problems (regarding dispersion of housing and employment)
that we were having great difficulty with in the Java version
--- but which we were able to quickly resolve in some experimental code in the
Python programming language.  We also found that, unlike Java, the
domain experts (the modelers) \emph{were} willing to read and
modify Python code.  Python itself has poorer performance than
Java.  However, in recent years, a number of packages have been
developed, including Numeric and numarray, that can be called from
Python and that allow one to manipulate large arrays and matrices very
efficiently.  UrbanSim simulations usually process a huge amount of
data, which in the Java version was done using iterations written in
Java, and is now handled using calls to these efficient packages.

Another problem was the existence of a set of independent research
projects at different institutions, all working in the domain of
land use and transportation modeling, but that had incompatible
platforms and implementations, making it hard to share and build on
each other's work.  A growing consensus emerged among researchers in
the urban modeling community that a common, open-source platform
would greatly facilitate sharing and collaboration.  In response, an
international group of researchers, including groups at the
University of Toronto, the Technical University of Berlin, and ETH
(the Swiss Federal Institute of Technology in Zurich), began
collaborating with the UrbanSim team to developed a new software
architecture and framework --- Opus, the Open Platform for Urban
Simulation \cite{waddell-opus-2005}.

At the same time that we were designing and implementing Opus, we began
implementing UrbanSim~4 using the new platform (using UrbanSim as
one of the test cases for the design and implementation of Opus). We
continued using our agile, test-driven development methodology with
Opus/UrbanSim. We now have a well-functioning system, written in
Python, that makes heavy use of efficient matrix and array
manipulation libraries.  Opus and UrbanSim 4
contain far less code than the previous implementation, yet
implement a much more modular and user-extensible system that also
runs faster. UrbanSim 4 also incorporates key functional extensions
such as integrated model estimation and visualization.  In terms of
our taxonomy (Section \ref{sec:refinement-taxonomy}), these
extensions are process refinements that address technical and
engineering issues. Model estimation, for example, was formerly an
error-prone process involving using external econometric packages,
which required several months of skilled staff time.  With the
integrated estimation capabilities, models can now be estimated in
less than a day.  The integrated visualization facilities don't
provide quite such a dramatic gain in efficiency, but nevertheless
make it much easier to produce simple visualizations of indicator
results for diagnosis and policy application.

%*** drop the following paragraph? (well, comment it out?) ***

Another component of the Opus work --- our data storage
representation --- illustrates the ongoing tension between ease of
use and transparent access on the one hand, and efficiency on the
other.  In UrbanSim 2, input data for the simulation was held in a
single large binary file that used a format specific to UrbanSim
(and similarly for output data).  This was efficient but not
particularly transparent.  In addition, we regularly changed the
format to meet new requirements, introducing versioning problems.
For UrbanSim 3, we moved to a SQL database, using MySQL, an
efficient, widely-used open source database.  The database schema
was carefully documented in our web-based documentation.  Indicators
were defined using SQL queries, which were part of ``live''
technical documentation \cite{borning-ecscw-2005}.  This greatly
enhanced transparency, but at a cost in efficiency.  In particular,
computing a set of indicators could take longer than the run time
for the simulation itself.  In Opus and UrbanSim 4, we support a variety of
data storage formats, including a SQL database, binary files (using
the format defined by the numarray package), tab-delimited files,
and others. Indicators can now be defined using Opus variable
definitions in addition to SQL queries. This is currently a
considerably less transparent format, but indicator values can now
be computed in minutes rather than hours.  We expect to be able to
design ways to increase the transparency of this more efficient
approach by developing appropriate user interfaces for browsing and
computing indicators.

We released Opus and UrbanSim 4 in June 2006. UrbanSim is currently in use
in projects in the U.S. for the metropolitan regions around Detroit, El
Paso, Honolulu, Houston, and Seattle, and internationally in Amsterdam,
Paris, Tel Aviv, and Zurich. Additional projects have been launched in
Burlington, Durham, Phoenix, and San Francisco, and internationally in
Melbourne, Australia. Many of the current projects, as well as all of the
new ones, are now using UrbanSim 4. In July 2006 we hosted the second
UrbanSim Users Group meeting, which included 35 participants from across 18
Chapter 22 the U.S., as well as Netherlands and Israel. The primary purpose
of this meeting was to assist current users in transitioning to UrbanSim 4.

\subsubsection{Uncertainty}
\label{sec:uncertainty}

We recently started a project to provide a principled statistical
analysis of uncertainty in UrbanSim, and to portray these results in
a clear and useful way to the users of the system.  In this work, we are
leveraging a promising technique, Bayesian melding,
which combines evidence and uncertainty about the inputs and outputs
of a computer model to yield distributions of quantities of policy
interest.  From this can be derived both best estimates and
statements of uncertainty about these quantities. This past year we
have had some initial success in employing this technique, applying
it to calibrate the model using various sources of uncertainty with
an application in Eugene-Springfield, Oregon. These results are
reported in a recent journal article \cite{sevcikova-trb-2006}. Once
the analysis infrastructure is in place, we will extend the
Indicator Browser and other interaction and
visualization tools to include appropriate depictions of
uncertainty.  We also intend to extend the uncertainty analysis to
address simultaneously the problems of developing data for use in
modeling (which is currently a very tedious and error-prone process)
and analyzing the role of uncertainty in input data on model
estimation and on simulation results.

% LocalWords:  tex pwaddell UrbanSim ITR Biocomplexity interactional analyses
% LocalWords:  integratively Socha Numarray MPOs San borning eXtreme ETH SQL
% LocalWords:  numarray UrbanSim's versioning MySQL FHWA COE NGO Utahns WFRC
% LocalWords:  WFRCs LRP PSRC amis
