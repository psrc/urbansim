%% $Id: intro.tex,v 1.19 2001/08/16 18:01:27 pwaddell Exp $

\section{Introduction}
\label{sec:introduction}

Patterns of land use and available transportation systems play a
critical role in determining the economic vitality, livability,
and sustainability of urban areas.  Transportation interacts
strongly with land use.  For example, automobile-oriented
development may induce demand for more roads and parking (which in
turn induces more automobile-oriented development), while compact
urban environments may induce more walking and demand for transit.
Both land use and transportation have significant environmental
effects, in particular on emissions, resource consumption, and
conversion of rural to suburban or urban land.

Good technical support can play an important role in fostering informed
civic deliberation and debate on these issues.  To aid urban planners,
residents, and elected officials in evaluating alternate
scenarios---packages of policies and investments---we want to simulate the
effects of these scenarios on patterns of urban growth and redevelopment,
of transportation usage, and resource consumption, over periods of twenty
or more years.

Early attempts at comprehensive urban simulations in the 1960s and
early 1970s were largely unsuccessful \citep{lee-1973,lee-1994}.
Much has changed since then, both on the supply side (including
dramatically improved hardware, theoretical and methodological
advances such as discrete choice choice modeling
\citep{mcfadden-1973,mcfadden-iatbr-2000}, and the emergence of a
commercial GIS market), and on the demand side (including public
concern over sprawl, legal challenges to transportation plans made
without considering their land use implications
\citep{garret-1996}, and regulatory requirements such as the Clean
Air Act Amendments of 1990).  As a result, there has been somewhat
of a renaissance in interest in urban simulation modeling over the
past decade.

However, in terms of planning agency practice, land use planning
is still often poorly integrated with transportation planning,
despite their strong interactions.  While transportation models
have been in routine use by metropolitan planning organizations
for decades, the state of common practice in land use modeling,
and in integrated land use and transportation modeling, is much
less advanced than that for transportation modeling alone.  For
example, the Travel Model Improvement Project sponsored by the
U.S. Department of Transportation and the Environmental Protection
Agency has focused a substantial investment on TRANSIMS, a new
traffic microsimulation model \citep{TRANSIMS-1999}, but almost no
federal investment has occurred on land use modeling to integrate
with these new travel models.

The UrbanSim system has been designed and implemented in response
to these needs.  It is a system for simulating the development of
urban areas, including land use, transportation, and environmental
impacts, over periods of twenty or more years
\citep{waddell-env-and-planning-2000,urbansim-reference-2000,waddell-nse-2001}.
From a software perspective, it is a large, complex application,
with heavy demands for excellent space efficiency and support for
software evolution. The system is fully operational and freely
available via our web site at {\sf www.urbansim.org}.  It consists
of around 130,000 lines of Java code for the core UrbanSim system;
including the visualization, data preparation, and calibration
tools, the total is approximately 200,000 lines, plus another
100,000 lines of automatically generated code.  It has been
applied to Eugene-Springfield, Oregon; Salt Lake City, Utah; and
Honolulu, Hawaii, working with the planning organizations in those
metropolitan regions.  Application to other regions is underway.
We have also done a historical validation of the system, starting
UrbanSim with 1980 data for Eugene-Springfield, running it through
1994, and comparing the results with what actually transpired
\citep{waddell-oregon-2000}.  Correlations between results of the
15 year simulation and observed data were generally above 0.8 at
the level of the grid cell, and were higher for spatial
aggregations such as traffic analysis zones.

% LocalWords:  noth UrbanSim sustainability Exp pwaddell GIS
