%% $Id: abstract.tex,v 1.9 2001/08/18 00:56:25 borning Exp $

%\subsection*{Abstract}
\begin{abstract}

UrbanSim simulates the development of urban areas, including land
use, transportation, and environmental impacts, over periods of
twenty or more years.  Its purpose is to aid urban planners,
residents, and elected officials in evaluating the long-term
results of alternate plans, particularly as they relate to such
issues as housing, business and economic development, sprawl, open
space, traffic congestion, and resource consumption.  From a
software perspective, it is a large, complex, system, with heavy
demands for excellent space efficiency and support for software
evolution.  It consists of a collection of models that represent
different urban actors and processes, an object store that holds
the state of the simulated urban environment, a model coordinator
that schedules models to run and notifies them when data of
interest has changed, and a translation and aggregation layer that
performs a range of data conversions to mediate between the object
store and the models.  The paper concludes with a discussion of
the lessons learned regarding software architecture to support
rapid evolution within the field of urban simulation.

\end{abstract}

% LocalWords:  noth UrbanSim Exp borning pwaddell
