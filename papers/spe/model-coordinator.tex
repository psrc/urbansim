%% $Id: model-coordinator.tex,v 1.16 2001/08/18 00:56:25 borning Exp $

\subsection{Model Coordinator}
\label{sec:ModelCoordinator}

The Model Coordinator is responsible for managing the collection of models
present in a simulation.  It is responsible for determining the execution
order of models, resolving any data dependencies one model may have on
another, and notifying a model when another model has changed data it is
monitoring.

Some key methods defined by the Model Coordinator class are:

\begin{description}

\item[\varnm{runSimulation}] Run the simulation once the event queue has been
populated.

\item[\varnm{executeEvent}] Execute a single event (provided as an argument
to this method).

\item[\varnm{getOrdering}] Determine a total ordering among a collection of
events (provided as an argument to this method).

\end{description}

\subsubsection{The Event Queue}
\label{sec:event-queue}

The Model Coordinator maintains an event queue containing timestamped
events.  These events include requests by a model to execute at a future
time, development events scheduled to occur at a future time, database
updates that were created by exogenous events that did not occur
instantaneously, and policy events that indicate planned changes in
regional or local policy.  Running the simulation consists of gathering the
set of events that are to occur at the current timestep, determining a
total order for those events that preserves any data or ordering
dependencies they may have, and then executing them in that order.

The event queue is thus the traditional data structure used in discrete
event simulations, except for the additional consideration of breaking ties
among events scheduled to occur at the same time.  Any number of models or
simulation events may be scheduled to execute at the same instantaneous
timestep.  However, the Model Coordinator may \emph{not} then execute these
events in an arbitrary order---there may be dependencies among them.  For
example, if the Household Mobility Model and the Household Location Choice
Model are both scheduled to execute at time $t$, the Household Mobility
Model must be run first, determining a set of households that decide to
move, and then the Household Location Choice Model must be run to find
available housing units for them.  In other words, the choice to move from
a current dwelling and the choice to look for a new dwelling are not
independent; this dependency is reflected by the constraints on the order
in which the models are run.

Since models are not restricted to running at regular intervals, in general
it is not possible to determine execution orders until run-time.  This
introduces an enormous amount of complexity not found in most other urban
modeling systems, which typically have a fixed ordering of execution.  When
more than one event is to occur at a given timestep, it is necessary to
determine a total order of the events that preserves any order dependencies
that may exist between them.  Dependencies are of two types, data-level
dependencies, and model-level dependencies between model execution events.

A data dependency exists between two models when one model writes to a
field that another model reads from.  In such cases, it is essential that
the reading model reads the correct version of the data, and the writing
model overwrites data only when it is safe to do so from another model's
perspective.  In the absence of other ordering dependencies, we assume that
all reads to a field occur before any writes to it,
and that writes can occur in any order.
This reflects the typical access pattern of models, which generally read
from many objects and write to a small number of fields of a small number
of objects.  (The fields written to generally have a very small overlap
with reads from other models.)

A model-level dependency is an ordering dependency explicitly introduced by
a model's author, in the form of a set of partial orderings between two or
more models.  This provides a mechanism to order models based on their
semantics rather than their syntax (data reads/writes).  For example, the
Land Developer Model and the Land Price Model could be executed in either
order, based on their inputs and outputs, but we schedule the latter to
execute after the former so that adjustments in land price due to this
year's demographic and economic changes do not affect development.  This is
based on the simplification of market dynamics that we adopt: that
development decisions are made once per year, looking at the state of the
region in the prior year to decide what should be built in the current
year.

To determine a valid total order for a collection of events, a
directed acyclic graph is constructed.  Events are represented as
nodes in the graph, and directed edges are created between events
that access the same data fields of objects or that have
model-level dependencies, with the direction of the edge
indicating that the source node (event) must occur before the
destination.  When determining the possible dependencies between
two model execution events, we must compute the transitive closure
of all models and all fields that may potentially be read from or
written to on the basis of the notification mechanisms. To ensure
correctness, the full chain of potential reads and writes must be
considered. Model-level dependencies override data-level
dependencies in the case of conflicts.  A topological sort is used
to generate a valid total ordering of the events to be executed.

\subsubsection{Development, Employment, and Policy Events}
\label{sec:DevPolicyEvents}

UrbanSim supports events that create, modify, or delete buildings;
create, move, or delete businesses or households; or change the
urban growth boundary or re-zone land. Many of these events are
generated by models.  (For example, the developer model generates
building development events.)  However, events can also be read
from an external file, allowing the modeler to introduce
development projects and policy changes into the event pipeline
that are exogenous to the models.  For example, a scenario author
may wish to simulate the effects of a major business leaving the
region, a shopping center being constructed, or a modification to
the urban growth boundary that occurs at a particular 
time in the simulated future.

%This can be used for calibration purposes as well, by introducing
%major events from historical data.
%This capability was useful, for instance, in performing our
%historical validation of the system for Eugene-Springfield, Oregon
%(starting the model with 1980 data and comparing the simulated
%results for 1994 with what actually transpired).  We used
%Development Events to model the phased closure of a large
%Weyerhauser lumber mill in the 1980s, and the opening of the
%Gateway regional shopping mall in 1990.

%As a somewhat philosophical aside, the reader may wonder why these sorts of
%large events are exogenous and not produced by models themselves.  The
%reason is that UrbanSim is intended to model the dynamics of a single urban
%region---but events in that region are influenced by the larger world.  For
%example, the closing of a lumber mill might be due to declining timber
%stocks, changing world markets, or other external factors.  Other
%examples of exogenous inputs are population control totals, based on
%predictions by demographers for the region, and overall economic
%predictions.  This gives rise to another question.  If we need to introduce
%a Business Event (such as a plant closure) for the system to give an
%accurate simulation of the historical development of a region, how can we
%have confidence in the system's predictions about a region in the year
%2020?  Might there not be some major event in 2015---for example, a global
%economic downturn---that will drastically influence the region?  The answer
%is, of course, that UrbanSim provides no crystal ball regarding the global
%economy.  Planners must use expert judgment in using the model, generally
%testing it under alternate scenarios and showing the results for all of
%them.  For example, when evaluating the effects of a major transportation
%infrastructure project, it would be prudent to perform this evaluation
%based on several alternative scenarios for general economic conditions.

%A related issue concerns policy events, such as moving an urban
%growth boundary or re-zoning land.  These are also input as
%external events, and are not the output of a model.  In contrast
%to development events representing global influences, these might
%be purely local policy changes. Our reason for representing policy
%events as exogenous inputs is philosophical rather than pragmatic:
%UrbanSim is intended as a tool to aid civic deliberation and
%debate, not as a tool to model the behavior of voters or
%governments.  We want it to be used to say ``if you adopt the
%following policy, here are the consequences,'' but not to say
%``UrbanSim predicts that in 5 years the city will adopt the
%following policy.''

% LocalWords:  Exp UrbanSim  timestamped timestep noth runSimulation C's
% LocalWords:  executeEvent getOrdering borning pwaddell
