%% $Id: relatedwork.tex,v 1.17 2001/08/18 01:15:58 borning Exp $

\section{Related Work}
\label{sec:relatedwork}

There is a huge body of work on urban transportation modeling,
land use modeling, and integrated land use/transportation
modeling.  Reviews and assessments of existing systems are given in
references
\citet{dowling-nchrp-2000,epa-report-2000,miller-tcrp-1999,parsons-1998,southworth-1995},
among others. Considerable progress has recently been made in land
use modeling in both experimental and deployed systems. However,
except for UrbanSim, all the operational models in use by planning
agencies rely on a cross-sectional, aggregate, equilibrium
approach.  Such models include DRAM/EMPAL
\citep{putman-book-1983}, TRANUS \citep{delabarra-book-1995},
MEPLAN \citep{echenique-transport-reviews-1990}, METROSIM
\citep{anas-book-1994}, and 5-LUT
\citep{martinez-env-planning-1992}.  The cross-sectional,
equilibrium framework implies that there are no relevant temporal
dynamics to the processes of urban change; rather, one can model
urban development as a static process that represents an economic
or a transportation optimization problem.  In other words, these
models could be run for the year 2050 without needing to model the
dynamics of evolution between the current time and the year 2050.
Clearly, this is a severe simplification, and makes problematic
the potential integration of these models with models of dynamic
environmental processes, or even of the dynamic evolution of human
behavior with respect to the built environment.  The approach
taken in UrbanSim more closely compares to the dynamic
disequilibrium HUDS model \citep{kain-book-1985} and the DORTMUND
microsimulation model
\citep{wegener-dortmund-1983,wegener-spiekermann-1996}, but
differs from these in having substantially greater spatial detail
and incorporating the nonresidential dimensions of urban
development.

% omitted for new journal:
% Reference \cite{beimborn-1996} is a short, useful introduction to the area
% for the nonspecialist.

Another substantial body of related work concerns Integrated
Assessment Models (IAMs), which model the interactions between
human and ecological systems in an integrated way.  A major
motivation for models of this kind is the assessment of global
environmental change
\citep{alberti-envplanning-1999,dowlatibadi-1995,parson-1995,rotmans-1995,weyant-1996}.
While the first generation of operational IAMs has emerged in the
mid-eighties, their roots can be traced back to earlier modeling
work in the late sixties and early seventies
\citep{forrester-book-1971,isard-1969,meadows-book-1982,odum-book-1983}.
Not surprisingly, all of these global-scale models are quite
aggregate, predicting environmental disturbances from broad
measures of economic growth and urbanization.  The UrbanSim
approach, by contrast, uses substantial spatial detail, and a
clearer behavioral approach grounded in discrete-choice theory.

In addition to global models, spatially-explicit regional integrated models
are now emerging, such as the Patuxent Landscape Model
\citep{voinov-envmodeling-1999}.  The Patuxent Landscape Model contains an
economic land use conversion model that uses a statistical process to
determine probabilities that grid cells will be allocated to forest,
agricultural, or urban usage.  The resulting conversion probabilities are
used to predict land use patterns which determine the land cover values
used as an input to the PLM's hydrology component.  Communication between
the land conversion and hydrology models is implicit through changed data
values in grid cells.  Several of the factors used in its land use
conversion component are similar to ones used in UrbanSim (e.g., access to
infrastructure, historical tax assessor data), but UrbanSim explicitly
models agents and their actions rather than using statistical or
finite-element processes.

Finally, another area of related work concerns agent-based
modeling, artificial life, and cellular automata.  In agent-based
modeling in its pure form, individual agents and their actions are
simulated, with each agent having local knowledge; global behavior
then emerges from these agent-level interactions.  Agent-based
modeling has been used for a wide range of applications, including
economic, sociological, biological, and physical simulations.  Two
that are closely related to UrbanSim are Sugarscape
\citep{epstein-book-1996,sugarscape-web}, a simulation of a small,
artificial society, and Aspen, a microanalytic simulation of the
entire U.S. economy \citep{pryor-sandia-1996,sandia-aspen-web}.
These approaches attempt to produce plausible macro-level behavior
as emergent properties of micro-level behavior.  This approach has
not yet evolved to the point of operational use in applied
planning settings, but represents a significant area of ongoing
research.

Cellular automata have been used for simulating urban development
\citep{batty-envplanning-1998,batty-computers-environment-1999,clark-envplanning-1997},
as well as for other applications such as simulating change in
land cover, freeway traffic, or the spread of wildfires.  In its
classic form, a cellular automaton consists of a regular array of
cells, each of which has a finite number of states.  Each state
change must be local, depending only on the states of neighboring
cells.  Urban processes, such as sprawl or urban decay, can emerge
from simple local rules.  However, these restrictions do not
always mesh well with our goal of supporting deliberation about
public policy.  For example, rather than viewing the conversion of
rural areas to urban ones as an analog of a biological process in
which the suburb grows and occupies increasingly wider areas, in
UrbanSim we view this process as the result of interactions among
the Land Developer Model (which simulates developers actively
seeking out development opportunities throughout the region in
response to market conditions, zoning regulations, taxes and
incentives, and the like), the location choice models (which
simulate residents or businesses seeking housing and commercial
space), and the Land Price Model.  More recently, researchers have
experimented with extensions of the cellular automata formalism
that incorporate extensions such as more agent-like behavior or
non-local search \citep{batty-jiang-1999,osullivan-2000}.

The UrbanSim approach assimilates aspects of these recent
developments in highly disaggregate agent-based and cellular
automata models, while retaining the flexibility to use
macro-scale model components when appropriate.  This
assimilative approach requires that the software architecture
support multiple modeling approaches, and not be optimized or
restricted to only one.  Models may be designed to operate at
different temporal and spatial scales, requiring unusual
flexibility from the software architecture. 

One 
implication of this for the software architecture is the need for a flexible
mechanism for assimilating model components and coordinating their
behaviors.  To meet this need, we use
\emph{implicit invocation},  
a software engineering technique in which different system components
communicate indirectly, rather than directly using procedure calls.  In our
realization of implicit invocation,
models communicate by registering
interest in objects and fields held in the Object Store, and by
being notified when such an object or field has been changed by
another model; but not by invoking each other explicitly.
This allows models to be developed more independently of each other.
(See Section \ref{sec:implicit-invocation} for details.)
Additional advantages and other applications of
implicit invocation are described in references
\citet{garland-aske-1993,sullivan-tse-1992,sullivan-tse-1996}.
Implicit invocation is essentially an event mechanism; related
concepts include active variables in
LOOPS~\citep{stefik-software-1986}, active databases such as
AP5~\citep{cohen-sigmod-1989}, and the Smalltalk-80
Model-View-Controller~\citep{krasner-joop-1988} and Field
integration mechanisms~\citep{reiss-book-1994}.  A discipline of
defining and using event-based programming mechanisms is
evolving~\citep{barrett-tse-1996,carzinga-saw-1998,
garlan-vdm-1991}.

The UrbanSim simulation approach, in summary, differs along
several lines from prior urban simulation models.  It is far more
disaggregate than any operational model implemented to date.  It
uses a dynamic, path-dependent approach that does not impose
simplifying assumptions of general equilibrium.  It is designed
for operational use to examine the effects of land use,
transportation, and the environmental plans and policies. And it
adopts an assimilative approach that draws from multiple streams
of ongoing research in urban simulation, including multi-agent,
cellular automata, and macro-scale models. The software
architecture described in this paper provides a modular and
extensible simulation environment that facilitates developing and
integrating urban models with varying temporal and spatial scales.

% LocalWords:  Exp UrbanSim AP Smalltalk GIS EMPAL TRANUS MEPLAN METROSIM  LUT
% LocalWords:  IAMs PLM's borning Patuxent Sugarscape wildfires StarLogo noth
% LocalWords:  aske sullivan tse stefik cohen sigmod krasner joop reiss barrett
% LocalWords:  carzinga garlan vdm tcrp southworth putman delabarra echenique
% LocalWords:  anas martinez env alberti envplanning dowlatibadi rotmans weyant
% LocalWords:  forrester isard odum voinov envmodeling epstein sugarscape pryor
% LocalWords:  sandia batty clark jiang osullivan pwaddell dowling nchrp epa
% LocalWords:  HUDS kain wegener dortmund spiekermann
