% $Id: eparticipation.tex,v 1.3 2006/04/15 04:11:41 borning Exp $

\documentclass{acm_proc_article-sp}
\usepackage{url}
\sloppy

\begin{document}

\title{Informing eParticipation with Results \\ 
from Simulations of Urban Development}

\numberofauthors{2}

\author{
\alignauthor Alan Borning and Janet Davis\\
       \affaddr{Dept.\ of Computer Science \& Engineering}\\
       \affaddr{University of Washington}\\
       \affaddr{Box 352350}\\
       \affaddr{Seattle, Washington 98195}\\
       \email{\{borning,jlnd\}@cs.washington.edu}
\alignauthor Batya Friedman\\
       \affaddr{The Information School}\\
       \affaddr{University of Washington}\\
       \affaddr{Box 352840 }\\
       \affaddr{Seattle, Washington 98195}\\
       \email{batya@u.washington.edu}
}
\date{28 March 2006}
\maketitle

\section{Overview}

UrbanSim is a system for modeling urban development over periods of 20 or
more years.  Its purpose is to support making more informed decisions about
such issues as building new transit systems or freeways, or adopting
alternative growth management regulations and incentives, as well as on
broader issues such as sustainable, livable cities, economic vitality,
social equity, and environmental preservation.  When presented with
alternate scenarios --- packages of possible policies and investments ---
UrbanSim models the resulting patterns of urban growth and redevelopment, of
transportation usage, and of resource consumption and other environmental
impacts.

Indicators provide the principal mechanism for presenting results from
UrbanSim simulations to staff at government agencies, elected officials,
members of advocacy groups, and interested citizens generally.  Presenting
information about indicators to better inform public deliberation presents
several challenges.  One such challenge is how to provide
relatively neutral technical information, and at the same time support
value advocacy and opinion.  Another is how to support the value of
fairness (and more specifically freedom from bias).  A third challenge (and
one most connected with the workshop) is how to encourage participation and
deliberation --- both electronic and other --- around the simulation
results and their implications.

To approach these issues in a principled
fashion, we rely on the Value Sensitive Design theory and methodology
\cite{friedman-amis-2006}, an approach to the design of information systems
that seeks to account for human values in a principled and comprehensive
way throughout the design process.  Key features of the methodology are its
interactional perspective, tripartite methodology, and emphasis on direct
and indirect stakeholders.

\section{UrbanSim}

The process of planning and constructing a new light rail system or
freeway, setting an urban growth boundary, changing tax policy, or
modifying zoning and land use plans is often politically charged.  Our goal
in the UrbanSim project is to provide tools for planners and stakeholders
to be able to consider different scenarios --- packages of possible
policies and investments --- and then to evaluate these policies by
modeling the resulting patterns of urban growth and redevelopment, of
transportation usage, and of resource consumption and other environmental
impacts, over periods of twenty or more years.  UrbanSim
\cite{waddell-nse-2003,waddell-ulfarsson-2004}, combined with
transportation models and macroeconomic inputs, performs simulations of the
interactions among urban development, transportation, land use, and
environmental impacts. It consists of a set of interacting component models
that simulate different actors or processes within the urban environment.

An integral part of our research program is facilitating the application of
UrbanSim to real metropolitan regions, working with regional government
agencies, and using this real-world experience to extend the research.  So
far, UrbanSim is being transitioned into operational use in the Puget Sound
region (Seattle and surrounding areas), Honolulu, and Salt Lake City; and
has already been used operationally in Houston.  There have also been
research and pilot applications in Amsterdam, Detroit, Eugene, Paris,
Phoenix, Tel Aviv, and Zurich.

The current version of the system, UrbanSim 4, is written using a new
software architecture and framework named Opus, the Open Platform for Urban
Simulation \cite{waddell-opus-2005}.  The system is written in Python, and
makes heavy use of efficient matrix and array manipulation libraries
(principally numarray).

\section{Indicators and Indicator Perspectives}

In urban planning, indicators \cite{gallopin-1997,hart-book-1999} are often
used to monitor changes in a region with respect to specific attributes of
concern.  In UrbanSim, simulation results can be presented using the same
set of selected indicators for all the policy alternatives being
considered, thus aiding the assessment and comparison of different
scenarios.  For example, suppose that a number of stakeholders are
interested in fostering compact, walkable, more densely populated
neighborhoods within the urban area, and curbing low-density, auto-oriented
development (``sprawl'').  In the urban planning literature, population
density is regarded as one of the key indicators of the character of
development (e.g., dense urban, low-density suburban, rural, etc.).  They
can then monitor population density to understand current trends, and also
use UrbanSim to assess and compare the impacts of different policies on
population density 30 years in the future.

The choice and description of indicators can be value-laden and politically
sensitive.  As noted above, our work in this area is guided by the Value
Sensitive Design methodology.  One project in the Value Sensitive
Design/UrbanSim area involved carefully documenting and presenting the
indicators that portray key results from the simulations
\cite{borning-ecscw-2005}.  Our design addresses a number of challenges,
including responding to the values and interests of diverse stakeholders,
and balancing the value of fairness with presenting a diverse set of
advocacy positions.

Among other things, we found that it was difficult to present a single
description of an indicator that simultaneously provided relatively neutral
technical information, and at the same time supported value advocacy and
opinion.  In response to this, we developed the Indicator Perspective
framework, which allows a set of organizations with a diverse set of
positions and interests to put forth different perspectives on what is most
important in the results from UrbanSim, and how it should be interpreted.
Our initial partners in this are Northwest Environment Watch, the King
County Benchmarks Program, and the Washington Association of Realtors.

We recently completed an empirical investigation of how well this framework
achieves two goals: first, of simultaneously providing relatively neutral
technical information and at the same time supporting value advocacy and
opinion, and second, of supporting the value of fairness (and more
specifically freedom from bias).  Data analysis is just getting under way,
and we hope to present some findings from this research at the workshop.

\section{Facilitating eParticipation around Indicator Perspectives}

In its current form, the Indicator Perspectives framework strongly supports
participation by organizations --- each organization is in charge of its
perspective, and determines what content is appropriate, with technical
support being provided by the UrbanSim team.  However, it doesn't support
electronic discussion and participation by individuals.  

As part of this recent study, we also did some formative evaluation of an
interface to support such participation, including a facility for adding
comments and discussion to different Indicator Perspectives, as well as a
novel mechanism that will allow users to upload evocative photos (which
they take themselves) to perspectives.  We envision, for example, users
adding photos of such places and situations as walkable,
pedestrian-friendly streets, or of graffiti, or congested freeways, or
parks.  In the future, we plan to implement and deploy such an interface,
and evaluate its effectiveness.

\subsection*{Acknowledgments}

We would like to thank all of the UrbanSim research team members and
collaborators, in particular Peyina Lin, who worked closely with us on the
empirical investigation of the Indicator Browser and Indicator
Perspectives.  This research has been funded in part by grants from the
National Science Foundation (EIA-0121326, IIS-0325035, and IIS-0534094), in
part by a partnership with the Puget Sound Regional Council, and in part by
gifts from IBM and Google.

\bibliographystyle{abbrv}
\bibliography{../bibliographies/urbansim}

%\balancecolumns

\end{document}

% LocalWords:  eParticipation Borning UrbanSim numarray interactional EIA IIS
% LocalWords:  Google
