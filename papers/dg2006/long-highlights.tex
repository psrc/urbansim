% $Id: long-highlights.tex,v 1.1 2006/01/23 02:02:28 borning Exp $
%

\documentclass{acm_proc_article-sp}
\usepackage{url}
\sloppy

\begin{document}

\title{Interaction and Participation in Integrated Urban Land Use,
Transportation, and Environmental Modeling: \\
UrbanSim Project Highlights}

\numberofauthors{2}

\author{
\alignauthor Alan Borning\\
       \affaddr{Dept.\ of Computer Science \& Engineering}\\
       \affaddr{University of Washington}\\
       \affaddr{Box 352350}\\
       \affaddr{Seattle, Washington 98195}\\
       \email{borning@cs.washington.edu}
\alignauthor Paul Waddell\\
       \affaddr{Daniel J. Evans School of Public Affairs}\\
       \affaddr{University of Washington}\\
       \affaddr{Box 353055}\\
       \affaddr{Seattle, Washington 98195}\\
       \email{pwaddell@u.washington.edu}
}
\date{20 January 2006}
\maketitle

\section{Project Overview and Impacts}

The process of planning and constructing a new light rail system
or freeway, setting an urban growth boundary, changing tax policy,
or modifying zoning and land use plans is often politically
charged.  Our goal in the UrbanSim project is to provide tools for
planners and stakeholders to be able to consider different
scenarios --- packages of possible policies and investments ---
and then to evaluate these policies by modeling the resulting
patterns of urban growth and redevelopment, of transportation
usage, and of resource consumption and other environmental
impacts, over periods of twenty or more years.  UrbanSim
\cite{waddell-nse-2003,waddell-ulfarsson-2004}, combined with
transportation models and macroeconomic inputs, performs
simulations of the interactions among urban development,
transportation, land use, and environmental impacts. It consists
of a set of interacting component models that simulate different
actors or processes within the urban environment.

An integral part of our research program is facilitating the
application of UrbanSim to real metropolitan regions, working with
regional government agencies, and using this real-world experience
to extend the research.  So far, UrbanSim is being transitioned
into operational use in the Puget Sound region (Seattle and
surrounding areas), Honolulu, and Salt Lake City; and has already
been used operationally in Houston.  There have also been research
and pilot applications in Amsterdam, Detroit, Eugene, Paris,
Phoenix, Tel Aviv, and Zurich.

\section{Recent Research Activities}

In this section we briefly describe some of our major activities during the
past year.

\subsection{Opus and UrbanSim 4}
\label{opus}

One project this past year has been collaboratively
developing a new software architecture and framework --- Opus, the Open
Platform for Urban Simulation --- and rewriting UrbanSim in that framework.
There were several factors that led us to take this step: a growing
consensus among researchers in the urban modeling community that a common,
open-source platform would greatly facilitate sharing systems and packages,
the desire to make the system code more accessible to domain experts, and
some intractable problems with some of our previous component models (which
were hard to solve due to the inaccessibility of the source code to domain
experts, making rapid experimentation and testing hard).

After preliminary testing and design work that begain in January
2005, we began implementing Opus and UrbanSim 4 (the latest
version of the system) in March, and now have a working version of
both \cite{waddell-opus-2005}. The system is written in Python,
and makes heavy use of efficient matrix and array manipulation
libraries (principally numarray).  The implementation of Opus and
UrbanSim 4 contains far less code than the previous
implementation, yet implements a much more modular and
user-extensible system, and runs faster.  It also incorporates key
functional extensions such as integrated model estimation and
visualization.

Opus has been designed in collaboration with groups at the
University of Toronto, Technical University of Berlin, and ETH,
the Swiss Federal Institute of Technology, in Zurich. The Toronto
group has also been active in developing a new open-source travel
model implementation in Opus; we plan to use that in our own work,
both directly and to do baseline comparisons with an experimental
activity-based travel model.

% ** longer paragraph (too long for the highlight, but saved in case it is
% useful someplace else:
%
% Our experience so far is that our domain experts (urban modelers) have been
% willing to read and modify code written in Python.  (In contrast, with our
% previous Java implementation, even though we paid careful attention to
% using good abstractions and a clear coding style, they invariably relied on
% our software developers to write and even explain the code.)  The system
% runs at about the same speed as the previous version, even though the new
% algorithms are in some cases are doing considerably more.  This is due to
% our extensive use of efficient matrix and array manipulation libraries
% (numarray and Numeric), which are written in C but callable from Python,
% allowing us to manipulate large sets of objects efficiently.

\subsection{Statistical Analysis of Uncertainty}

Predicting the future is a risky business.  There are numerous,
complex, and interacting sources of uncertainty in urban
simulations of the sort we are developing, including measurement
errors, uncertainty regarding exogenous data and other input
parameters, and uncertainty arising from the model structure and
from the stochastic nature of the simulation. Nevertheless,
citizens and governments do have to make decisions, using the best
available information.  At the same time, we should represent the
uncertainty in our conclusions as well as possible, both for
truthfulness and as important data to assist in selecting among
alternatives.

We are starting a new project to provide a principled statistical
analysis of uncertainty in UrbanSim, and to portray these results
in a clear and useful way to the users of the system.  We are
leveraging in this work a promising technique, Bayesian melding
\cite{poole-jasa-2000,raftery-jasa-1995}, which combines evidence
and uncertainty about the inputs and outputs of a computer model
to yield distributions of quantities of policy interest.  From
this can be derived both best estimates and statements of
uncertainty about these quantities. This past year we have had
some initial success in employing this new technique, applying it
to calibrate the model using various sources of uncertainty with
an application in Eugene-Springfield, Oregon. These results are
reported in a journal article recently submitted to Transportation
Research B: Methodology \cite{sevcikova-trb-2006}.

\subsection{Indicators and Stakeholder Interaction}

Another set of activities concerns presenting the results of simulations to
different stakeholders, including elected officials, members of
neighborhood, business, and advocacy groups, and engaged citizens more
generally, in ways that are clear and that speak to the issues of concern
for those stakeholders.  Our work in this area is guided by the Value
Sensitive Design methodology \cite{friedman-amis-2006}, an approach to the
design of information systems that seeks to account for human values in a
principled and comprehensive way throughout the design process.

One project involved carefully documenting and presenting the indicators
that portray key results from the simulations. Our design addresses a
number of challenges, including responding to the values and interests of
diverse stakeholders, making documentation ready-to-hand, and balancing the
value of fairness with presenting a diverse set of advocacy positions.  We
published the results of this work, including empirical evaluations, in the
European Computer Supported Cooperative Work conference
\cite{borning-ecscw-2005}.  Our work contributes to CSCW as an example of
designing a system for effective use in an environment with multiple
stakeholders who have fundamental disagreements.  These are of course
characteristics shared by important other Digital Government applications,
and the conference paper includes a discussion of lessons for other systems.

Another project has been the development of ``Personal Indicators,'' which
distill the simulation results down in ways that speak to concerns of
individual citizens (for example, ``what will my commute time be like under
Scenario A?''  ``Will my children be able to afford housing in the region
in 2020?'')  A preliminary description of this has been submitted to the
ACM Computer Human Interaction Conference, and will form a section
of Janet Davis's forthcoming Ph.D. dissertation.

\subsection{Testing Stochastic Systems}

Agile software development methodologies and extensive testing have been a
hallmark of our software engineering practices on UrbanSim for some years
\cite{freeman-benson-agile-2003}.  However, we have had consistent problems
adequately testing stochastic algorithms, which may give different results
each time they are run.  (And many of the key UrbanSim algorithms are
stochastic.)  We recently made major progress in this area, developing a
set of design patterns for tests of stochastic systems that include
distributional tests on the results of running the test repeatedly.  This
is supported by a sound statistical analysis of how to interpret the
results from such tests, and a unit test framework that implements it.
These results are being written up for submission to the premier software
testing conference, the ACM International Symposium on Software Testing and
Analysis.

\section{Collaborations and Funding}

One set of collaborations is with government planning agencies that want to
apply UrbanSim to their regions.  Our primary effort at present is with
Puget Sound Regional Council, the metropolitan planning organization for
our own region.  We have also collaborated actively with MPOs in Salt Lake
City, Eugene, Honolulu, Houston, and Detroit; and since the UrbanSim system
is open source, under the GNU Public License and available for download
from our website, we've also had other groups use and apply the system in
their own regions.  The first UrbanSim Users Group meeting in San Antonio,
Texas, in January 2005, attracted some 30 participants from MPOs around the
country, a number of academic researchers, and one participant from the
Netherlands.

Another set of collaborations concerns the development of Opus,
the Open Platform for Urban Simulation described in Section
\ref{opus}, with an emerging consortium of research teams from
Canada, France, Germany, Japan, Switzerland, and the United
States. We are also working with researchers at the University of
Massachusetts in Amherst on the ``UrbanSim Commons,'' a web portal
to facilitate exchange and collaboration among UrbanSim users.

A substantial portion of our funding continues to come from the
Digital Government program.  This past year, the Puget Sound
Regional Council continued their partnership with us, also
contributing \$150,000 toward the application of UrbanSim to this
region. We also began working in this past year on a new \$700,000
grant from the Environmental Protection Agency under the Science
to Achieve Results (STAR) program, to extend UrbanSim by
integrating new travel modeling capabilities and linking to
emissions models. We are participating as well in a new NSF
Biocomplexity project, in collaboration with Arizona State
University, to explore emergent properties of urban landscapes.
Finally, we have consistently employed computer science
undergraduates as research assistants on the project, where they
have gained an in-depth exposure to Digital Government research
and to agile software development methodologies.  In recognition
of the quality of UrbanSim students they were seeing (and in some
cases hiring), this past year Google gave \$15,000 as an
unrestricted gift in support of the project.

\section{Plans and Challenges for the Coming Year}

We plan to release Opus and UrbanSim 4 early this year.  A
challenge has been balancing this constant software evolution,
driven by the research agenda and problems that we encounter, with
the needs of our government partners, who, after all, want a
stable, working system that they can use as an ongoing part of
their operational decision-making procedures. We hope that Opus
will provide a workable platform for them, and are putting a great
deal of our effort towards that end.

A more risky area of research will our emerging work on
statistical analysis and representation of uncertainty using
Bayesian melding, which is supported by a new Digital Government
grant.  As discussed above, our preliminary results are promising
--- but there are significant challenges and risks, including
being able to adequately uncover the uncertainties in the input
data and models, and being able to achieve adequate performance.
(The technique requires running the model many times with varying
inputs.)

In the area of stakeholder presentation and interaction, we plan
to complete the implementation and deployment of a web-based
Indicator Browser, which will let interested stakeholders browse
through simulation results.  The choice and description of
indicators can be value-laden and politically sensitive.  In
response to this, we have been developing \emph{Indicator
Perspectives}, partnering with different groups and agencies to
put forth a variety of perspectives on what is most important in
the results from UrbanSim, and how it should be interpreted.  Our
initial partners in this are Northwest Environment Watch, the King
County Benchmarks Program, and the Washington Association of
Realtors.  In the coming year we will be extending this
collaboration to other groups, and evaluating this as a way of
addressing the problem of providing both facts and relatively
neutral technical documentation, and also supporting value
advocacy and opinion.

\subsection*{Acknowledgments}

We would like to thank each and every member of the UrbanSim research team
for their work on the project.  Thanks also to the NSF Digital Government
program, which has provided a solid home for this work within NSF.  In
particular, the Digital Government program has supported the kind of
mixture of basic and applied, and highly interdisciplinary, research that
is needed for this domain.

This research has been funded in part by grants from the National Science
Foundation (EIA-0121326 and IIS-0534094), in part by a partnership with the
Puget Sound Regional Council, and in part by gifts from IBM and Google.

\bibliographystyle{abbrv}
\bibliography{../bibliographies/urbansim}

%\balancecolumns

\end{document}
