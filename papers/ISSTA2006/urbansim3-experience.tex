% $Id: urbansim3-experience.tex,v 1.25 2006/05/10 07:04:11 hana Exp $

\section{Experience With UrbanSim Unit Tests}
\label{prior-experience}

Our prior implementation of UrbanSim was written in Java, using agile
development techniques, such as unit tests (using JUnit, integrated with
the Eclipse IDE), FIT tests (using Ward Cunningham's Framework for
Integrated Testing), small steps with frequent check-ins, test-first
development, nested planning iterations, regular refactoring, automated
builds, and so forth.  Central to this approach is having good unit tests,
and making it easy to know when they fail.  To facilitate this, in our lab
we installed traffic lights (real ones) that provide ambient indicators of
the most recent results of these tests~\cite{freeman-benson-agile-2003}.

We continue to use many aspects of this software development methodology in
our current work on UrbanSim 4, including extensive testing (now using PyUnit,
since we are working in Python), and the traffic lights.  One aspect that did
not work well, however, was our unit tests of the parts of the system
that exhibited stochastic behavior.

\subsection{Problems Writing Unit Tests for Stochastic Models}

A good
unit test typically runs the component with just enough data to exercise the
part of the algorithm and implementation being tested, and is simple enough
that the expected output values can be computed by hand for checking against
the result from running the algorithm.  We were able to write such tests for
components of the system involving deterministic algorithms, and these tests
were very useful for detecting problems there.

However, neither the domain
experts (the urban modelers) nor the developers were able to devise really
satisfactory tests for the stochastic models.  So instead we generally
resorted to regression testing, checking whether the output values from a new
version of an algorithm was identical to that from the previous version, or if
there was a change that \emph{should} have changed the output values,
convincing ourselves that the new output values were correct, and then
installing this as the new standard against which future runs would be
checked.

In practice, however, it was difficult to decide whether the new result was
correct when there was a change.  We used both a realistic test data set
(for Eugene, Oregon), as well as a contrived one.  But the results from the
Eugene set were too complex to feasibly do more than check for identical
results.  On the other hand, the results from the small, contrived set were
unrealistic, and the modelers could not reason about them.

Another problem was that for comparing exact results, the test scenarios were
fragile.  The scenario required running the algorithm with a fixed seed for the
random number generator, and tiny algorithmic changes (for example, a change
that caused the random number generator to be called one more time), would give
different output values.

As a result, over time we slowly reduced the power of the
tests as their fragility caused increasing pain for the developers.
Eventually, many of the unit tests for models
degraded to simply testing that the
number of rows of data produced were as expected.  (Not surprisingly, we
subsequently discovered that there were still bugs in the model algorithms
and implementation, which were not found by such tests.)  We did, however, continue to
have extensive and reasonable unit tests of the deterministic aspects of
the system.

\subsection{Toward Statistically-Based Unit Tests}

In response to these problems, in UrbanSim 4 we set out to
design statistically-based tests that would be reliable, intuitive to the
modelers, and straightforward to write for the software developers.

When running in production mode, UrbanSim (like many other simulation
systems) produces voluminous, multidimensional outputs, for example, the
number of households in each grid cell.  However, for small test cases,
even though different runs in general will produce different values, the
expected values can be readily found --- in fact they often can be computed
by hand, as in the case of simple deterministic tests.
But we have no, or very little, information about the spread of the
actual results around the expected ones.  We could test whether the actual
results from a unit test are within a certain tolerance of the expected
results, but what tolerance should be used to decide whether the test has
succeeded?  Furthermore, the spread can differ across different outputs,
for example it may increase with larger expected values. Under such
conditions, even a solution of running the test several times, and letting
it pass if it succeeds at least once, is obviously \emph{ad hoc}.

For our very real-world application, these are not purely theoretical
questions.  When we first began to use such unit tests, we used a tolerance
to determine whether the test had succeeded, but we kept needing to
increase the tolerance, or to increase the number of times the test was
run, to get some of our tests to pass consistently.  This was especially
true as we added more tests to the system, which caused our traffic lights
to go red more often. This was clearly unsatisfying: did we still have
good unit tests, or had we relaxed them to the point where we were masking
errors?  Such considerations motivated the development of a more
mathematically grounded approach.

% LocalWords:  UrbanSim tex borning YP XP JUnit IDE wgb CVS PyUnit jan logit

%%% Local Variables:
%%% mode: latex
%%% TeX-master: "main"
%%% End:
% LocalWords:  hana gridcell socha
