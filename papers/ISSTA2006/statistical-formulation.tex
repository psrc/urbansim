% $Id: statistical-formulation.tex,v 1.39 2006/05/11 22:08:27 hana Exp $

\section{A Statistical Formulation for Automated Tests}
\label{sec:statform}

To properly support our goal of automated testing, we put the matter
of testing our stochastic systems on a firm theoretical basis by using
statistical hypothesis testing (see e.g.\ \cite{mood-book-1974}, Chapter 9).
We define the null hypothesis $H_0$ and the alternative hypothesis
$H_1$ as follows:

\begin{tabular}{ll}
$H_0$: & program behaves as expected\\
$H_1$: & program does not behave as expected
\end{tabular}

Using various properties of the results of running the test, we can test
whether there is strong evidence to reject $H_0$. In such a case, the unit
test should fail.

There are many ways to construct the test statistics; the best choice depends
primarily on the properties of the distribution of output values.  The
methodology is based on comparing the distribution of the output values to the
expected distribution. This requires multiple runs of the program.
Furthermore, one has to be able to model the expected output properties by a
probability distribution with known density function. Often, various
transformation functions can be applied to the data in order to obtain an
approximation of a known distribution (see e.g.~\cite{Afifi&2004}, Chapter~4).
As in the case of deterministic unit test, the approach is applicable to test
cases for which one can calculate expected values, which are then the
parameters of the expected distribution.

In the following subsections, we present different ways of constructing a test
statistics for the above hypothesis, namely for normally distributed and for
Poisson distributed data (Sections \ref{sec:normal} and \ref{sec:poisson}
respectively).  These two commonly used distributions have proved sufficient
for the cases that arise in our application.  We believe that further they
will cover the vast majority of cases that researchers and practitioners must
deal with.  However, the methodology is easily extensible, so that tests can
be constructed for data with other kinds of distributions.

%
\subsection{Normally Distributed Data}
\label{sec:normal}
%
The normal distribution is typically the most commonly used distribution. (It
is actually not so widespread in our application --- for UrbanSim, it arises
for real-valued quantities, especially averages, and the like.)  For output
values that are independent and normally distributed, we suggest using the
following test.  The program is assumed to be correct if the
actual distribution of outputs for each dimension\footnote{In this paper we
  use the term ``dimension'' to refer to the number of outputs for one test.
  This is standard terminology in statistics when used in joint probability
  distribution formulas.}  has the same mean as the expected distribution.
Thus, we will perform a test on means of normal distributions with unknown
variance using a likelihood-ratio test statistic (LRTS).

This test requires the variance to be constant over all dimensions, which is not
necessarily the case. A simple solution is to find an appropriate
transformation of the output values prior to the statistical testing that will
stabilize the variance. Often a square root or log transformation is a good
candidate if the variance varies with the mean.  Plotting the variances
against means (one point per dimension) before and after the transformation
can be helpful in finding the right function (see for
example~\cite{sevcikova-trb-2006}, Figures 2 and 3).

More formally, we denote the number of replications by $R$ and the number of
dimensions by $K$\@. $y_{kr}$ denotes the $k$-th output from $r$-th
replication.  Note that all $K \times R$ ouputs are produced using the same
inputs. The differences in $y$ along the $r$ axis are due to the
nondeterminism of the code. $x_{kr}$ is either equal to $y_{kr}$ if no
transformation is necessary, or $x_{kr} = g(y_{kr})$ where $g(\cdot)$ denotes
the transformation function.  Suppose that $x_{kr}$ is independent normally
distributed
\[
x_{kr} \sim N(\mu_k, \sigma^2)\;, k=1,\dots, K,\; r=1,\dots,R
\]
where $\mu_k$ denotes the mean of the distribution for dimension $k$
and $\sigma^2$ denotes the variance which is constant over all dimensions.

We translate the above formulated hypothesis test into:
\[
\begin{array}{lc}
H_0: & \forall k: \, \mu_{k} = \mu_{k}^{(0)}\\
H_1: & \exists k: \, \mu_{k} \not= \mu_{k}^{(0)}
\end{array}
\]
Here, $\mu_{k}^{(0)}$ denotes the known mean for the output dimension $k$
(i.e. the expected value for $k$-th output). Using the formulas for normal distribution,
we can define the likelihood for each hypothesis as
\[
L_{H_0} = \prod_{k,r}\frac{1}{\sqrt{2\pi\hat{\sigma}_0^2}} \exp\left[ \frac{-1/2(x_{kr}-\mu_{k}^{(0)})^2}{\hat{\sigma}_0^2}\right]
\]
and
\[
L_{H_1} = \prod_{k,r}\frac{1}{\sqrt{2\pi\hat{\sigma}_1^2}} \exp\left[
  \frac{-1/2(x_{kr}-\hat{\mu}_{k})^2}{\hat{\sigma}_1^2}\right]
\]
where
\[
\hat{\mu}_{k} = \frac{1}{R}\sum_{r=1}^R x_{kr}\,.
\]
$\hat{\sigma}_0^2$ and $\hat{\sigma}_1^2$ are estimates of the
variance for the null and alternative hypothesis, respectively. They are
obtained by
\[
\hat{\sigma}_0^2 = \frac{1}{KR}\sum_{k,r}(x_{kr}-\mu_{k}^{(0)})^2
\]
and
\[
\hat{\sigma}_1^2 = \frac{1}{KR}\sum_{k,r}(x_{kr}-\hat{\mu}_{k})^2\,.
\]

The likelihood-ratio test statistic
\[
LRTS_{normal} = 2(\log L_{H_1} - \log L_{H_0}) = KR
\log\left(\frac{\hat{\sigma}_0^2}{\hat{\sigma}_1^2}\right)
\]
has a $\chi^2$ distribution with $K$
degrees of freedom, asymptotically. If the corresponding $p$-value is smaller than a selected level
of significance $\alpha$, the null hypothesis
will be rejected.

%
\subsection{Poisson Distributed Data}
\label{sec:poisson}
%
The Poisson distribution can be used if the outputs are integers,
especially if they represent counts.  (This is the more common case in
UrbanSim and other systems using discrete choice models.  Examples of such
outputs in UrbanSim are the number of households per grid cell, jobs in a
particular employment sector, and the like.)  We assume that the data $x_{kr}$
are independent Poisson distributed
\[
x_{kr} \sim Poisson(\lambda_k),\; \lambda_k>0, \; k=1,\dots, K,\; r=1,\dots,R
\]
where $\lambda_k$ denotes the mean and variance of the distribution for
dimension $k$.

Similarly to the case of normal distribution, we set the hypothesis as
\[
\begin{array}{lc}
H_0: & \forall k: \, \lambda_{k} = \lambda_{k}^{(0)}\\
H_1: & \exists k: \, \lambda_{k} \not= \lambda_{k}^{(0)}
\end{array}
\]
where $\lambda_{k}^{(0)}$ denotes the known mean (and variance)
for dimension $k$.

A likelihood-ratio test statistic is constructed as follows:

\begin{eqnarray*}
\log L_{H_0}& =& \log \prod_{k,r} \frac{(\lambda_k^{(0)})^{x_{kr}}
    \exp(-\lambda_k^{(0)})}{x_{kr}!} \\
 &= &\sum_{k,r} (x_{kr}\log \lambda_k^{(0)}
    - \lambda_k^{(0)}) - \sum_{k,r} \log (x_{kr}!)
\end{eqnarray*}


 \begin{eqnarray*}
\log L_{H_1} &= &\log \prod_{k,r} \frac{\hat{\lambda}_{k}^{x_{kr}}
    \exp(-\hat{\lambda}_{k})}{x_{kr}!} \\
&= &\sum_{k,r} (x_{kr}\log \hat{\lambda}_{k}
    - \hat{\lambda}_{k}) - \sum_{k,r} \log (x_{kr}!)
\end{eqnarray*}
where $\hat{\lambda}_{k}$ is the maximum likelihood estimator of $\lambda_{k}$:
\[
\hat{\lambda}_{k} = \frac{1}{R}\sum_{r=1}^R x_{kr}\,.
\]

This gives the likelihood ratio test statistic
\begin{eqnarray*}
LRTS_{poisson}  & = & 2(\log L_{H_1} - \log L_{H_0}) \\
&= &2 \sum_{k,r} \left[ x_{kr} \log
\left(\frac{\hat{\lambda}_{k}}{\lambda_k^{(0)}}\right) - \hat{\lambda}_{k} + \lambda_k^{(0)} \right]
\end{eqnarray*}
which has a $\chi^2$ distribution with $K$
degrees of freedom, asymptotically.

A common alternative to this likelihood-ratio test statistic for the Poisson
distribution is the Pearson $\chi^2$ test:
\[
\text{Pearson } \chi^2 = \sum_{k,r} \frac{(x_{kr} - \lambda_k^{(0)})^2}{\lambda_k^{(0)}}
\]
This test statistic is also $\chi^2$ distributed and has $KR$
degrees of freedom, asymptotically. In Section~\ref{sec:power} we will provide a
comparison of those tests.


\subsection{An Example}
\label{sec:example}

We apply the above methodology to a simple but realistic example, by
writing a unit test for a model that is based on multinomial logit theory
\cite{ben-akiva-lerman-1987,train-book-2003}.  UrbanSim contains several
such models, including the \emph{Residential Location Choice model}, the
\emph{Developer model}, and others.  For example, the Residential Location
Choice model simulates the decision-making process of households deciding
where to live.  For each household that is moving to a new house or
apartment, the probability of moving to each vacant unit is computed, based
on characteristics both of the household (income, number of children, age
of head, \ldots), and of the potential dwellings (cost, percent residential
within walking distance, \ldots).

More generally, these models represent a
situation in which agents make a choice from a set of alternatives. The choice
process is based on probabilities for each of the alternatives which are
computed using the multinomial logit formula.  Thus, the deterministically
computed probabilities are the basis for determining the known means
$\mu_k^{(0)}$ or $\lambda_k^{(0)}$ for $k=1,\dots,K$, where $K$ denotes the
number of alternatives.

Suppose we have a set of $100$ agents, each of which
chooses one of $10$ available locations.  Of these locations,
$5$ cost \$$1000$ each, while the other $5$ locations are less costly, say
\$$100$ each. (In production use, these models employ
many parameters in addition to cost, but
for purposes of writing a unit test, we use just the one parameter.
This follows the unit test philosophy of using a minimal set of data that
nevertheless exercises the code.)

We set the cost coefficient
to $\beta = -0.001$ and denote the cost variable for location $i$ by
$c_i$. The multinomial logit formula
\[
 P_{k} = \frac{e^{\beta c_k}}{\sum_{i=1}^{K}e^{\beta c_i}}
\]
yields probabilities that suggest that $5\cdot 14.2 \% = 71 \%$ of the agents
in total will choose the less expensive locations, whereas $5 \cdot 5.8 \% =
29 \%$ will choose the more expensive alternatives. Our quantity of interest
is the number of agents in each location. From the above computations we know
that the expected number of agents in an expensive location is $100 \cdot
0.058$ and in a less expensive location is $100 \cdot 0.142$.  For illustrative
purposes, we will apply all
three tests described in Sections~\ref{sec:normal} and~\ref{sec:poisson}.  As
a transformation function in the likelihood-ratio test for normal distribution
we choose the square root function.

Running our program $5$ times gives the intermediate values
shown in Table \ref{good-results-table}, which can be used to compute
$LRTS_{normal}$, $LRTS_{poisson}$ and Pearson $\chi^{2}$. Note that in the
two Poisson tests $x_{kr}=y_{kr}$, whereas in the test for normal distribution
$x_{kr}=\sqrt{y_{kr}}$.

\begin{table*}[t]
\[
\begin{array}{lr|rrrrrrrrrr}
% experiment made with seed(1,1) (stochastic_test_power.py)
&& \multicolumn{5}{c}{\text{less expensive locations}} & \multicolumn{5}{c}{\text{more expensive
      locations}} \\
& r \backslash k& 1 & 2 & 3 & 4 & 5 & 6 & 7 & 8 & 9 & 10 \\\hline
& 1 & 19& 15& 18&  9 & 9 & 7 & 11 & 5 & 3 & 4 \\
& 2 & 17 &16 &10 &16 &10 & 5 & 5 & 6 &11 & 4 \\
y_{kr} & 3 & 11 & 22& 12& 19& 14 & 4 & 5 & 5 & 6 & 2 \\
& 4 & 13 & 9 &21 &12 &13 & 6 & 7 & 3 & 7 & 9 \\
& 5 & 14 &17 &17 & 7 &15 & 5 & 6 & 3 & 7 & 9 \\\hline
\mu_{k}^{(0)} & & 3.77 & 3.77 & 3.77 & 3.77 & 3.77 & 2.41 & 2.41 &2.41
      &2.41 & 2.41\\
\hat{\mu}_{k} & & 3.83 &   3.94 &  3.91 &  3.49 &  3.48 &
  2.31 &  2.58 &  2.08 &  2.56 &  2.28 \\\hline
\lambda_{k}^{(0)} & & 14.2  & 14.2 & 14.2 & 14.2 & 14.2
& 5.8 &  5.8 &  5.8 &  5.8 &  5.8\\
\hat{\lambda}_{k} & & 14.8 &  15.8 &  15.6 &  12.6 &  12.2 &
   5.4 &    6.8 &   4.4 &    6.8 &   5.6 \\
\\
\multicolumn{12}{l}{\hat{\sigma}_0^2=0.2555$, $\hat{\sigma}_1^2=0.2155}
\end{array}
\]

\caption{Results from running a correct location choice model}
\label{good-results-table}
\end{table*}

The three different test statistics are:
\[
\begin{array}{l|rrr}
& \text{test statistic} &\text{df} & p\text{-value} \\\hline
LRTS_{\text{normal}} & 8.5026 & 10 & 0.5799 \\
LRTS_{\text{poisson}} & 7.7336 & 10 & 0.6548 \\
\text{Pearson } \chi^2 & 50.2234 & 50 & 0.4645
\end{array}
\]
where df denotes degrees of freedom.  All three $p$-values suggest that there
is no strong evidence to reject the null hypothesis --- employing the commonly
used significance level $\alpha=0.05$ we would accept the hypothesis that the
program behaves as expected (and so it would pass the unit test) in all three cases.

To see what happens when there is an error in the model code, we will use an
example bug that our stochastic unit test case recently exposed.
Initially, the test only
examined the first half of the choice set, namely the less expensive locations,
and succeeded.  When we expanded it to examine the whole choice set, it failed
every time.  Upon investigation, it turned out that one of the locations was,
by mistake, being excluded from the set of alternatives, due to an indexing
error in the model code.

Repeating the above experiment for the model code with this error gives the
values shown in Table \ref{bad-results-table}, which produces these test
statistics values:
\[
\begin{array}{l|rrr}
& \text{test statistics} &\text{df} & p\text{-value} \\\hline
LRTS_{\text{normal}} & 78.9807 & 10 & 0.0000 \\
LRTS_{\text{poisson}} & 68.4220 & 10 & 0.0000 \\
\text{Pearson } \chi^2 & 75.1238 & 50 & 0.0123
\end{array}
\]

\begin{table*}[t]
\[
\begin{array}{lr|rrrrrrrrrr}
% experiment made with seed(1,1) (stochastic_test_power.py)
&& \multicolumn{5}{c}{\text{less expensive locations}} & \multicolumn{5}{c}{\text{more expensive
      locations}} \\
& r \backslash k& 1 & 2 & 3 & 4 & 5 & 6 & 7 & 8 & 9 & 10 \\\hline
& 1 &19 & 16 & 19 & 7 & 14& 12 & 6 & 3 & 4 & 0\\
& 2 &19 &15& 14 &16 & 9 & 6 & 5 & 12 & 4 & 0\\
y_{kr} & 3 &14 &20 &18 &15 &15 & 5 & 4 & 6 & 3 & 0\\
& 4 &13 &13 &18 &13 &16 & 6 & 5 & 7 & 9 & 0\\
& 5 &14 &18 &17 & 9 &17 & 6 & 3 & 5 &11 & 0\\\hline
\mu_{k}^{(0)} & & 3.77 & 3.77 & 3.77 & 3.77 & 3.77 & 2.41 & 2.41 &2.41
      &2.41 & 2.41\\
\hat{\mu}_{k} & & 3.96 &  4.04 &  4.14 &  3.42 &  3.75 &
  2.61 &  2.13 &   2.51 & 2.41 &   0.00 \\\hline
\lambda_{k}^{(0)} & & 14.2  & 14.2 & 14.2 & 14.2 & 14.2
& 5.8 &  5.8 &  5.8 &  5.8 &  5.8\\
\hat{\lambda}_{k} & & 15.8 &  16.4 &  17.2 &  12.0 & 14.2 &   7.0 &
   4.6 &    6.6 &    6.2 &   0.0 \\
\\
\multicolumn{12}{l}{\hat{\sigma}_0^2 = 0.7929$, $\hat{\sigma}_1^2 = 0.1634}
\end{array}
\]
\caption{Results from running a location choice model with a bug}
\label{bad-results-table}
\end{table*}

In this case, the tests correctly detect that the means for $k=10$ departs from
the expected value, leading to small $p$-value.  The unit test would fail
in all three cases for level of significance $\alpha=0.05$.

% LocalWords:  UrbanSim gridcell numarray closeto kr lc LRTS logit rrrrrrrrrr
% LocalWords:  cccc tex borning TODO gridcells hana poisson lr rrr df

%%% Local Variables:
%%% mode: latex
%%% TeX-master: main
%%% TeX-master: "main"
%%% End:
