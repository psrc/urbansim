% $Id: conclusion.tex,v 1.12 2006/05/11 22:08:27 hana Exp $

\section{Conclusion}
\label{sec:conclusion}

We have been very satisfied so far with the results of using automated,
statistically-based unit tests in UrbanSim.  For example, the tests exposed an
error in our implementation of the \emph{Residential Location Choice model}
--- the same bug that we used as an example in this paper.  Also, the
stochastic test framework has allowed us to deal effectively with the problem
of tests failing when they should have passed. Switching from a normal
distribution to a Poisson distribution reduced these incorrect failures
significantly for the tests that are checking counts. Over the period of last
few months we have experienced a very few errors, in fact their frequency
confirms our initial experiment for choosing $\alpha$ (in step~7 of
Guidelines). The tests are now much more comprehensible to both modelers and
developers.  Finally, we have more confidence in our system now that we are
using tests based upon rigorous statistical theory.  What became an
intractable problem with our former version of UrbanSim seems quite tractable
and pleasant now.

We plan to continue to refine our use of this testing framework, and in the
process, continue to gather data and case studies regarding the real-world
utility of such tests.  The framework will also be part of our upcoming
release of Opus, making it available to all researchers using Opus for
their modeling systems.  This methodology is of course not restricted to
urban modeling --- it is applicable to testing stochastic algorithms of all
kinds.  We look forward to seeing the types of design patterns and agile
software development practices that emerge from its application.

% LocalWords:  borning hana UrbanSim

%%% Local Variables: 
%%% mode: latex
%%% TeX-master: "main"
%%% End: 
