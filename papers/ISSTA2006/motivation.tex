% $Id: motivation.tex,v 1.5 2006/01/25 03:58:04 borning Exp $

% *** THIS SECTION IS NO LONGER USED
% (some of it is folded into the ``Prior Experience with UrbanSim'' section)


\section{Motivation for a Statistical Approach}

In response to the problems with testing described in the previous section, we 
set out to design robust, statistically-based tests that would be intuitive to 
the modelers and straightforward to write for the software developers.  This 
section describes how our solutions evolved, and in the process highlights some 
of the types of problems that we needed to solve.  The following section
describes the statistical mechanism and justification for the final solution. 
And the section after that describes some generalizations and design patterns
that derive from our solution.

Our current use of stochasting testing focuses on testing individual land-use ``models''
in isolation.  Each run of our simulation system sequentially runs a set of
models.  The question we were addressing was whether each individual model was
behaving correctly in isolation.  

For instance, the \emph{Household Location Choice model} allows households to choose a 
new place to live.  The basic idea is that each such household gets presented a 
small set of alternative locations from which they may choose.  For each 
location, the household computes ``attractiveness'' measure, and then uses that 
to determine which location to choose.

This process has two stochastic steps.  First, randomness is used to draw the 
set of alternative locations from the total universe of available locations. 
Second, randomness is used to select from the available locations based upon 
their ``attractiveness'' -- this allows locations that are nearly equally 
attractive to be choosen with nearly equal probability.

To test that households preferentially choose attractiveness locations, we 
created a set of 100 locations.  Half of these had a ``cost'' of 100; the others 
a ``cost'' of 1000.  Presumably, households would prefer to choose the less 
expensive locations.  How much more they would choose locations with a cost of 
100 versus those with a cost of 1000 depends upon the ``attractiveness'' 
formula used by the households.  For this test, we defined a very simple 
attractiveness formula where a location $l$'s attactiveness, $a_{l}$, is a 
simple function of that location's cost, $c_{l}$: $a_{l} = -0.001 * c_{l}$.  
The higher the cost, the less attactive the location.  Using this formula, we 
determined that $71\%$ of the households should choose the less expensive 
locations, and the remaining $29\%$ the more expensive locations.

The next step was to determine how many households to use for this test. 
Clearly, using a single household would be unsatisfactory, since it would not
provide us enough resolution about the distribution.  We decided to use 10000
households, since we thought that would be a large enough sample.

These model tests are similar to unit tests, so we built our stochastic test 
system by extending the standard PyUnit unit-testing framework.

\subsection{First solution: Tolerance}

Our first test technique ran the model once on the test data to place all 10000 
households, and then compared the actual proportion of households that choose
the less expensive locations to the expected ratio of $0.71$.  If it was within
some tolerance of the expected value, we were happy enough.  The Python code for
this was:

\begin{verbatim}
... run model ...
n = # of households in more attractive grid cells
self.assert_(allclose(n, 10000*0.71, rtol=0.05))
\end{verbatim}

When such a test occassionally failed, we ran it again.  

But what was the appropriate amount of tolerance to allow?  Some tests used 
0.05, others 0.40. This made us nervous -- were we really catching errors?  Or 
were some tests so loose they were ineffective?  When tests failed more than 
was comfortable, we would increase the relative tolerance value to some 
arbitrarily higher value.  Very unsatisfying.

We also looked for solutions to this problem elsewhere, but could not find any 
-- at least not in the software engineering or testing literature.

\subsection{Second solution: Multiple Runs}

Some tests continued to fail more than we liked.  But we did not want to 
increase the tolerance.  Instead, we decided to run the test multiple times and 
see if it passed at least once:

\begin{verbatim}
succeeded = False
for i in range(2):
    ... run model ...
    n = # of households in more attr. grid cells
    if allclose(n, 10000*0.71, rtol=0.05):
        succeeded = True
self.assert_(succeeded)
\end{verbatim}

Again, we knew this was a hack, but we did not know how to do it correctly.  We 
hoped it would work well enough until we could think it through.

[2006-jan-24-wgb: I would place the following paragraph somewhere else since it
suggests that particularly the above code is referenced. From knowing the project it is
not this particular one but the idea of the first and second solution in general.]

As we increased the number of such tests in the system, the automated build 
system started to fail more and more frequently simply because one of the 
stochastic tests failed.  Also, people were worrying that the code was broken, 
when it was only our test techniques that were broken (or at least we hoped it 
was only the latter).

\subsection{Third solution: Stochastic Foundation}

At last, we knew we had to find the right solution.  After a series of 
conversations with Hana, our statistician, one day she came in and filled our 
white board with the solution.  It was a relief, since it appeared to be a 
solution grounded in statistical theory, and one that eliminated much of the 
guesswork.  
[2006-jan-24-wgb: as a non-native speaker the words "appeared to be" 
read as if the authors are uncertain. Is this correct/intended?]

We implemented the solution in a class {\sf StochasticTestCase} that extended 
PyUnit's {\sf TestCase} class with a single new method: 
{\sf run\_stochastic\_test}.  
This method takes a ``run method'' that returns an 
array of values by running a model, an array of expected results, and the 
number of times to run the ``run method'' before doing the comparison.  We made 
sure it worked in one case, and then returned to our other work.  Every time 
one of our naively constructed stochastic tests failed, we would replace it 
with the new {\sf StochasticTestCase} solution.  

For instance, for the \emph{Household Location Choice model}, the code looks like the 
following.  Note that we halved the number of households to improve the run 
time, now that we were more confidence in the test.

\begin{verbatim}
def run_model():
    ... run model ...
expected_results = array(50*[5000*0.71/50])
self.run_stochastic_test(run_model, 
                         expected_results, 10)
\end{verbatim}

This solution felt much better.  It was based upon solid statistics.  And it 
also tested that the distribution of households was uniform across the 50 
attractive grid cells.

We were still nervous, though, since some of the test cases continued to fail 
more than would be expected.  How should we determine the number of times to 
run the ``run method''?  How many households and locations did we need to use? 
This required us to know more about the power of the tests.

