$Id: patterns.tex,v 1.6 2006/01/26 18:29:27 borning Exp $

% *** THIS SECTION IS NO LONGER USED
% the last paragraph may well be useful elsewhere though!

\section{Patterns of Statistical Automated Tests}

[2006-jan-24-wgb: what is the purpose of this chapter? it is fairly short and - as far
as I can see - does not functions as a binding segment between the preceeding and the
following section.  In this current state I would suggest to put the discovered
patterns in the discussion section and move the "doubt paragraphs" at the end to the
outlook and future work section.]

As we developed robust, statistically-based tests, we developed a collection of 
design patterns for such tests, refining it as we gained experience with the 
methodology.

% TODO: check names for these patterns

One pattern is for a \emph{uniform distributional test}, in which there are
a set of equally-likely choices.  In this case, we expect the outcomes to
be roughly equally distributed among the possibilities.  For example, we
might have a collection of 100 gridcells, each with 20 vacant residential
units, and 1000 households.  All households and residential units have the
same characteristics.  In this case we'd expect that we'd end up with
approximately 10 households placed in each grid cell.

[2006-jan-24-wgb: in each section we are using different examples with different numbers (100, 1000, 10000). could we use the same example and numbers in each sections to make the paper more easy for the reader?]

[2006-jan-24-wgb: is it OK to have the short form of would "'d" in a paper?]

Another pattern is for a \emph{highly likely choice}, in which there is one
choice that is much better than the others.  For example, we might have one
unplaced household to be placed, and 10 vacant units, one of which is much
more attractive than the others.  We'd expect the household to be placed in
the more attractive unit nearly all of the time.
(Additional patterns, and discussions of our experience with them, are
presented in Section \ref{test-patterns}.)

In stochastic models different runs produce different results when using
random numbers generator with different seeds.  Nevertheless, for test
patterns such as those described above, the expected result is known --- in
fact it often can be computed by hand, as in the examples above.  (More
generally, if the formula for computing probability of a certain behavior
is known, the resulting quantity measuring that behavior will be
proportional to that probability.)

However, we have no or very little information about the spread of the
results around these expected quantities.  Furthermore, UrbanSim (like many
other simulation systems) produces a large number of outputs, for
example, the number of households for each of the
800,000 gridcells in Puget Sound.  The spread of the results around the expected values can
differ for each of these outputs.  These facts make it infeasible to design a
test that is based on a comparison using a single specific tolerance.

If we were running and assessing the test results manually, typical
practice would be to inspect the results and ask a domain expert if they
looked reasonable, and if they did not, perhaps run the test again (in case
we were unlucky with the random number seed that time).  This isn't a
feasible solution for an automated test environment.  (For example, how do
we decide if the results are reasonable?  How often should we repeat the
test?)

[2006-jan-24-wgb: having said the above, I/one could also suggest to write (much) more about how
these new test patterns fit into a continuous development cycle.]

%%% Local Variables: 
%%% mode: latex
%%% TeX-master: "main"
%%% End: 
