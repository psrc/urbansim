% $Id: guidelines.tex,v 1.20 2006/05/11 22:08:27 hana Exp $

\section{Guidelines}
\label{sec:guidelines}

We turn now to some practical considerations for constructing and using
stochastic unit tests, including suggestions for setting the various
parameters.  Note that these are guidelines rather than an
algorithm for constructing the test, and some adaptation may be needed for
other situations.

\begin{enumerate}

\item
{\bf Separate the unit test code into two parts.}  Part A prepares input to the
stochastic algorithm, while Part B includes the stochastic aspects of the
algorithm.  Part B will be run repeatedly by the stochastic test system,
while part A only needs to be run once --- assuming that part B does not
modify the values prepared by part A\@.  For instance, if a model moves
households, part B must include the initial setup of the households so that
every time part B is run it starts with the same set of households in the
same locations.  Sometimes it is pragmatically convenient to include in B
some deterministic tests about quantities that should always have the same
value every time B is run.  Implement part B as a separate function or
method that returns $K$ values.

\item
{\bf Determine the desired dimensionality $K$ of the output data.}  In
general, it is best to use small test sets that are hand-crafted to test
specific stochastic aspects of the algorithm, and small enough so you can
compute the expected values by hand.  Our experiments suggest that the
tests are independent of the number of dimensions (see
Figure~\ref{fig:error1-k-agents}), which is the result we would expect from
statistical theory.  Thus, keeping this number small for run-time
efficiency should not affect the results if the hypothesis test is
constructed properly.  Recall that the tests assume independence, both for
values within each dimension and across all dimensions.  If the dimensions
are not independent, increasing their number might help to approximate
independence.  (However, as shown by the comparison between
Figure~\ref{fig:power-T1} and Figure~\ref{fig:power-T2}, using fewer
dimensions may increase the chance of detecting errors in the code.)

\item
\label{assess-type-of-distribution}
{\bf Assess the type of distribution} of each of the $K$ outputs that B will
return when called repeatedly.  For example, the Poisson distribution is
likely to be a good choice when dealing with count data, while in other cases
the normal distribution may be preferable.  Other candidates to look at when
dealing with practical applications are the binomial and gamma
distributions.  It is also possible to apply a transformation function of your
choice to the output values in order to obtain an approximate normal
distribution (see e.g.~\cite{Afifi&2004}, Chapter~4).



\item
\label{assess-variances}
{\bf Assess whether the variances are independent of the means} if your
choice is the normal distribution.  This can be done, for example, by
calling part B of the test $R$ times and storing all $K\times R$ output
values. Then create a scatter plot, with the means of the values for each
of the $K$ rows computed over $R$ columns on the $x$ axis, and the
variances computed on the same data on the $y$-axis.  If the points lie
approximately along a horizontal line, no action needs to be taken.
Otherwise apply a transformation function on the output data and repeat
this experiment.  Common transformations are the square root, log, or the
inverse function.

\item
\label{choose-hypothesis-test}
{\bf Choose a hypothesis test.}  In Section~\ref{sec:statform} we described
tests that can be used with normally and Poisson distributed data. If those
tests are not appropriate, a variety of methods for finding a hypothesis test
is available (see for example~\cite{mood-book-1974}, Chapter IX). The simplest
way is to construct a likelihood ratio test by plugging the particular density
function into the formula $2(\log L_{H_1} - \log L_{H_0})$, as was done in
the techniques described in Sections~\ref{sec:normal} and~\ref{sec:poisson}.

\item
{\bf Choose the number of replicates.}  Figure~\ref{fig:error1} suggests
that the frequency of a type I error is independent of the number of
replicates unless this number is very small. On the other hand, the higher
the number of replicates, the better the power of the test (see
Figures~\ref{fig:power-T1} and~\ref{fig:power-T2}). We recommend between 10
and 20 replicates, based on the experience with our application.  (A
different number of replicates may be necessary for other applications with
substantially different characteristics, however.)

\item
\label{choose-significance-level}
{\bf Choose a significance level.}  There is trade-off in setting the
significance level $\alpha$. The higher the $\alpha$ value, the higher the
probability of type I error (compare the left to the right panel of
Figure~\ref{fig:error1}). On the other hand, the higher the $\alpha$ value,
the higher the power (compare the left to the right panel of
Figures~\ref{fig:power-T1} and~\ref{fig:power-T2}). If $\alpha=0.05$ we
expect the test to falsely fail once out of 20 runs.  If there were a 100
stochastic tests in the system, our automated build would fail once every 5
runs, on average.  In our own development environment, this would very
likely lead to problems --- either the modelers and developers would become
increasingly annoyed at the automated test system, or worse, start ignoring
red lights (which might result in real problems being neglected).
  
Given this tradeoff, it is difficult to proscribe a universally acceptable
value for the significance level.  Instead, it will be a pragmatic decision
for different projects.  In our case, we use a smaller value of
$\alpha=0.01$ to minimize the number of false failures, while still not 
reducing the power of the tests too severely.

Our decision for $\alpha=0.01$ in our real-world application is supported by
the following experiment: We constructed a suite with 11 different
$LRTS_{poisson}$ tests covering stochastic qualities of four different types
of models.  We ran each of these 11 tests 100 times with significance level of
$0.01$.  In these $1100$ test runs there were only 3 type I errors, i.e.\ the
probability of a type I error occurrence is $0.003$.

\item
{\bf Invoke the test.}  Suppose the hypothesis test is implemented in a
unit test method, as is the case in Opus. Such an implementation typically
takes as arguments a reference to the function B, the expected results, and
a significance level.  It calls the function B $R$ times, then using the
$K\times R$ outputs, it computes the test statistics and the corresponding
$p$-values, as described in Section~\ref{sec:statform}. There are
statistical packages for various programming languages that provide
functions for obtaining $p$-values.  (For Python, we recommend the module
pstat.) The method fails if the $p$-value is smaller than the given
significance level.  In our implementation, the StochasticTestCase in Opus
extends PyUnit's TestCase class with a run\_stochastic\_test method that
performs this step.

\item
{\bf Check the test behavior.}
Run the test multiple times and count the number of failures. If the frequency
of failing is significantly higher then $\alpha$, before adjusting the test
parameters, check whether there is some other cause for the failures, such as:
\begin{itemize}
\item The hypothesis test itself is implemented incorrectly.
\item There is a bug in the code.
\item There is an error in the expected values.
\item The data are not modeled properly (i.e. the assumptions of the
  chosen hypothesis test are not met, such as the underlying distribution of
  the data or independence). 
\end{itemize}

After elimating these other possible causes, then, if necessary, adjust the
test parameters --- the number of replicates, the significance level, the
number of outputs, or the test data --- in order to produce a satisfactory
test.

\end{enumerate}


In practice, we do not need to repeat each of these steps for every new
test.  The results that are tested in unit tests often have similar
characteristics, such as being count data with a Poisson distribution, or
continuous data with a normal distribution.  In these cases, it often is
sufficient to do Step~\ref{assess-variances} just once for each such
variety of data.  And if the unit test framework already contains methods
for hypothesis testing of classes or methods with those characteristics,
Steps~\ref{assess-type-of-distribution} and \ref{choose-hypothesis-test}
reduce to characterizing the data and selecting an appropriate statistical
test method.  This significantly reduces the effort necessary for using
this stochastic testing methodology.


%%% Local Variables: 
%%% mode: latex
%%% TeX-master: "main"
%%% End: 

% LocalWords:  pstat socha al tradeoff StochasticTestCase PyUnit's TestCase
% LocalWords:  LRTS poisson borning
