% $Id: test-framework.tex,v 1.14 2006/01/25 08:47:13 borning Exp $

*** SECTION NO LONGER USED ***

\section{A Framework for Stochastic Tests}

\subsection{Patterns for Automated Tests}
\label{test-patterns}

  \begin{itemize}
    \item standard UnitTests for deterministic models (e.g. LandPriceModel)
    \item distributional tests (1000 households) [Is there a better name for this?]
    \item forced choice, e.g. have one unplaced household to be placed, a vacancy exists, no
    sampling, so the model must place the household in that location.
    \item highly likely choice, e.g. have one unplaced household to be placed, 10 vacancies,
    one is much more attractive than the others, the household is highly likely to be 
    placed in that place.
  \end{itemize}


\subsection{The Stochastic System Test Case}

Patterns
\begin{itemize}
  \item run once (UnitTest)
  \item run until test succeeds, maximum n times
  \item run n times, number of successes must be between j and k for the test to succeed
\end{itemize}

Support to help modelers and developers choose the correct test parameters (e.g. number of times
to run a test until it succeeds, or bounds on the number of successes.  This support could include
a tool that runs a small test many times and gathers statistics, and also something that tells the
modeler or developer how likely Type 1 and Type 2 errors are.

\subsection{Practical Aspects}

\begin{itemize}
  \item how difficult is it to write these tests?
  \item who can write them?  only domain experts? 
  \item how much maintenance do these tests require?
  \item how fragile are these tests?
  \item how useful are these tests?  Are they practical, lightweight, 
     understandable, and convincing?
  \item do they fit into the unit test philosophy (adapted it to accomodate
  stochastic models of these kinds)?
\end{itemize}

\subsection{Results So Far}

So far, our use of these stochastic tests have had three positive results. 
Firstly, it exposed an error in our implementation of the \emph{Household Location 
Choice model} -- the same bug that we used as example in this paper.  Secondly, 
the stochastic test framework reduced the frequency of the tests failing when 
they should have passed.  Switching from a Normal distribution to a Poisson 
distribution reduced these incorrect failures even more for the tests that were 
checking counts. Thirdly, we have more confidence in our system now that we are 
using tests based upon rigorous statistical theory.  What became an intractable
problem with our former version of UrbanSim seems quite tractable and pleasant now.

We plan to continue to refine our use of this testing framework, which is part 
of our upcoming Opus release.  Providing it as part of the Opus system, makes 
it available to anyone using Opus for their modeling systems. We look forward 
to seeing the types of design patterns that emerge as we expand convert the 
rest of our stochastic tests to using this system, and add more stochastic 
tests for the UrbanSim models and model in other packages that we and our 
colleagues are developing.


% LocalWords:  UnitTests LandPriceModel UnitTest

%%% Local Variables: 
%%% mode: latex
%%% TeX-master: "main"
%%% End: 
