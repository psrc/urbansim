% $Id: unit-tests.tex,v 1.1 2006/01/26 06:29:25 borning Exp $

*** long unit tests related work section (currently mostly unused)

\subsection{Unit Tests}
\label{related-work-unit-tests}

To help ensure the code quality of UrbanSim, we make use of unit tests 
\cite{beckgamma,hunt:2003,Noonan:2002} to test all classes automatically. For 
each of the program's units a test is written --- typical units under test are 
classes and their operations.  For each class a corresponding test case exists 
that makes sure that every method is called and tested.  By implementing the 
test, not only is the class tested initially,
but also a test case will be available and run automatically as the system
evolves.  To automate testing, a unit test framework and a testrunner is 
employed.
Each test case follows a specific implementation pattern, so that the test
runner 
is able to collect all test cases, and to identify and invoke
all the test methods that they contain.

The availability of test cases allows all programmers of the development
team to make use of all tests during ongoing development.  This is important
for agile development processes \cite{cockburn:2001,highsmith:2002},
especially test driven development \cite{beck:2003} as they heavily
rely on short cycles and frequent use of refactorings \cite{fowler:2001}.
After implementing a new feature or refactoring existing code, all unit
tests are run to make sure all parts of the software still run without
errors. In order to support short development cycles (and make it easy for
programmers to actually run the tests), these tests must run in a fairly
short time.  Nevertheless, unit tests may have errors as well.  To minimize
this, unit tests should be kept as simple and as straightforward as
possible.  In general, unit tests should be 
responsible for testing only a single
unit; if a class makes use of complex other classes these 
should be replaced by mock objects.

With unit tests the quality of the software product increases significantly 
because unit tests help find errors: we can make sure that each unit is tested 
(coverage), unit tests are simple, they are automated (test runner), they run 
quickly, and 
they identify problems in remote parts of the software that have not 
been touched by the actual programming activity.

% unit test philosophy
This philosophy of unit tests is to provide a set of tests for the programmer 
that enables him/her at any point in time during the development process to 
check whether new or changed parts in the program affect existing code. 
Typically one runs all unit tests before development starts and ensures that 
all tests run without a problem. Then, after development, one runs the tests 
again and expects them to still run successfully. If new parts are introduced 
in the system, corresponding tests have to be written. If an error was to be 
fixed in the system, first a test was written that identifies this error and 
then runs successfully after the error has been removed. A stronger approach, 
called test-driven-development \cite{beck:2003}, incorporates these tests more 
tightly into the process of development by requiring the programmer to write a 
failing test for each new function and then fix this test by implementing the 
function correctly. While the more relaxed
approach requires the tests to run in a 
moderate time, the stricter one
requires that tests be run every five to ten 
minutes, leaving only little time for the tests to run completely.

The aim of this way of testing is to find errors early in the development 
process. An additional benefit of using unit tests is to find problems in 
remote parts induced by programming activity. The underlying assumption of 
existing unit test frameworks is that the unit under test (in most cases a 
class) has a deterministic behavior. Therefore, tests can assume an initial 
state when creating an instance of that particular class. Consequently, each 
operation modifies the instance's state in a deterministic way. Probing the 
instance's state will result in a repeatable constant value. By applying 
classic unit tests, 
we implicitly deal with deterministic finite state machines.

The vast majority of the testing literature deals with such deterministic
finite state machines \cite{lee:1996,yannakakis:1991,sidhu:1989}, in which
test cases are considered to be action sequences, and the tester is assumed
to be in full control of the state of the unit under test.
