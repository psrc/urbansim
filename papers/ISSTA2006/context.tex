% $Id: context.tex,v 1.20 2006/05/10 03:08:24 hana Exp $

\section{Project Context}

In many urban regions, there is increasing concern 
about pollution, traffic jams,
resource consumption, loss of open space, loss of coherent community, lack of
sustainability, and unchecked sprawl.  Elected officials, planners, and
citizens in urban areas grapple with these difficult issues as they develop
and evaluate alternatives for such decisions as building a new rail line or
freeway, establishing an urban growth boundary, or changing incentives or
taxes.  These decisions interact in complex ways.  There are both legal and
common sense reasons to try to understand the long-term consequences of these
interactions and decisions.

Unfortunately, the need for this understanding far outstrips
the capability of the analytic tools used in current practice.  In response
to this need, we have been developing UrbanSim, a sophisticated, reusable
simulation package for predicting patterns of urban development for periods
of twenty years or more, under different possible scenarios, each a package
of possible policies and investments 
\cite{waddell-japa-2002,waddell-nse-2003}.  Its
primary purpose is to provide urban planners and other stakeholders with
tools to aid in more informed decision-making.  When provided with
different scenarios,
UrbanSim models the resulting patterns of urban growth and redevelopment,
of transportation usage, and of resource consumption and other
environmental impacts.  To date, UrbanSim has been applied in the
metropolitan regions in the U.S. around Eugene,
Honolulu, Houston, Phoenix, Salt Lake City, and Seattle.
Internationally, it has
been applied in Paris, Tel Aviv, and in the Netherlands.

Having reliable, credible software is essential, since
the domain is politically charged, with many regions having sharp and
long-standing disagreements over such issues as the balance of
automobile-oriented transportation facilities, public transportation, and
bicycles, regarding housing affordability, environmental impacts, and
others.  

For UrbanSim, an important unit of credibility is to determine whether each
of the UrbanSim components works correctly.  UrbanSim is implemented as a
set of interacting component models that represent major actors and
processes in the urban system \cite{noth-ceus-2003}.  For example, the
\emph{Residential Location Choice model} simulates the choice process of a
household selecting a new place to live, while the \emph{Developer model}
simulates the actions of a real estate developer deciding whether to
renovate existing buildings or construct new houses, apartments, offices,
or the like.  UrbanSim takes a highly disaggregated approach, modeling
individual households, jobs, and real estate development and location
choices using grid cells of $150\times 150$ meters in size.  The model
system microsimulates the annual evolution in locations of individual
households and jobs, and the evolution of the real estate within each
individual grid cell as the result of actions by real estate developers.
The Puget Sound application of UrbanSim, for instance, includes 1.3 million
simulated households --- each making choices involving randomness, every
year.  The question addressed in this paper is how to write robust and
useful unit tests of these models, given the stochastic nature of the code.

The most recent version of the system, UrbanSim~4, is implemented using
Opus (the Open Platform for Urban Simulation), a new object-oriented
architecture and platform developed by our group and others
\cite{waddell-opus-2005}.  Opus and UrbanSim~4 are implemented in Python,
making heavy use of highly optimized array and matrix manipulation
packages, written in C++, to handle all of the inner loop computations.
(Previous versions were in Java.)  Opus and UrbanSim are open source, under
the GNU Public License.  For more information please see the project
website \url{www.urbansim.org}.

Simulation and modeling is used extensively in other politically-charged,
economically, socially, and environmentally significant domains as well,
and the testing methodology described in this paper is applicable to
stochastic models of many sorts.  In the remainder of this paper, we first
provide a brief discussion of related work in software testing.  We then
describe our earlier experience with using automated testing for UrbanSim,
and our initial experience with \emph{ad hoc} nondeterministic tests.  This
experience motivates the need for a more rigorous statistical analysis of
probabilistic tests for stochastic algorithms, which we present in
Section~\ref{sec:statform}.  In Section~\ref{sec:power} we provide an
evaluation of these tests.  Section~\ref{sec:guidelines} provides
guidelines and practical suggestions for implementing unit tests for
stochastic algorithms in other applications.  We conclude with an
assessment of the current state of the work and directions for future
research.

% LocalWords:  borning UrbanSim microsimulates socha hana

%%% Local Variables: 
%%% mode: latex
%%% TeX-master: "main"
%%% End: 
