\documentclass[12pt]{article}

\usepackage{txfonts}
\usepackage[T1]{fontenc}
%\usepackage[latin1]{inputenc}

\usepackage{array}
%\usepackage{lscape}
\usepackage{rotating}
\usepackage{longtable}
%
% Change coordinates to edge of page
\voffset=-1in \hoffset=-1in
%
% Set for US Letter
\textheight=9.0in \textwidth=6.5in \topmargin=0.5in
\oddsidemargin=1in \evensidemargin=1in
%
\frenchspacing \raggedbottom
%\renewcommand{\thesection}{\Alph{section}}
%\renewcommand{\theenumii}{\arabic{enumii}}

%\newcommand{\ve}[1]{\mathbf{#1}}
\newcommand{\vk}[1]{\mbox{\boldmath $#1$}}

\begin{document}

\title{Application of UrbanSim to the Wasatch Front Region, Utah}

\author{Paul Waddell$^{\dagger}$ \and Gudmundur F. Ulfarsson$^{\ddagger}$
\and\\ \\
$^{\dagger}$ Evans School of Public Affairs, University of
Washington, \\ Box 353055, Seattle, WA 98195 \\ (206) 221-4161,
pwaddell@u.washington.edu \\
$^{\ddagger}$ Department of Civil Engineering, Washington
University in St. Louis, \\ Campus Box 1130, One Brookings Drive,
St. Louis, MO 63130 \\ (314) 935-9354, gfu@wustl.edu}

\maketitle

\abstract{TBD}

\section{Introduction}

The Greater Wasatch Front Region of Utah, home of Salt Lake City
and host of the 2002 winter Olympics, has been experiencing rapid
economic and population growth, with predictable consequences for
spatial expansion and travel demand. Concerns over the effects of
such rapid growth precipitated the launching of a unique regional
visioning process known as Envision Utah, an initiative that
convened community leaders and residents to attempt to forge
consensus about vision for the region over the next 20 to 30
years.

Having adopted a preferred strategy that attempts to reconcile the
range of values and perspectives within the community, the task
now at hand is to determine the most effective ways to bring the
vision to reality. To assist in doing this, the Governor's Office
of Planning and Budget, in coordination with the Wasatch Front
Regional Council (WFRC) and the Mountainlands Association of
Governments (MAG), chose to support the analysis of alternative
land use and transportation policy strategies for achieving the
vision by using the UrbanSim (Waddell, 2002) model system in
conjunction with the regional travel model system operated by WFRC
and MAG.

The empirical setting for the analysis is centered on Salt Lake
City, and runs between the edge of the Wasatch Front mountain
range and the shores of the Great Salt Lake and Lake Utah, from
Ogden in the north to Provo-Orem in the south, as shown in
Fig.~\ref{fi:resloc1}. It is a place in which urban development is
on the one hand constrained by natural boundaries, but on the
other hand is developing at relatively low density.

This paper presents the design and specification of the
residential location choice model, the multi-sectoral employment
location model, the real estate development, and land price model
components of UrbanSim, and the results of estimation of these
models and their application for the Greater Wasatch Front region.

A description of earlier estimation results from the Greater
Wasatch Front application of the residential location were
reported by Waddell and Nourzad (2002). Earlier estimation results
for employment location, real estate development and land price
components of UrbanSim were presented by Waddell and Ulfarsson
(2003a, 2003b).

Further references provide an overview of the UrbanSim model and
validation results of its application in Eugene-Springfield
(Waddell, 2002), empirical results from the original specification
of UrbanSim (Waddell, 2000), description of the data development
process (Waddell, 1998), a detailed description of the current
model system implementation (Waddell et al., 2003), analysis of
its relationship to land supply monitoring (Waddell, 2000), its
theoretical foundations (Waddell, 2001b; Waddell and Moore, 2001),
its application as a decision support system (Waddell, 2001a), and
the underlying software infrastructure (Noth et al., 2003).

In the following sections, we briefly review the theory and
approach used in designing the models, and we describe the
methodology and results for the Greater Wasatch Front Region.

\section{Overview of UrbanSim} The land use models are
collectively described as the UrbanSim model system, which is
linked to a recently updated travel demand model system in Utah
that integrates the models of WFRC and MAG, and adds non-motorized
modes to the model.  Fig.~\ref{fi:resloc2} depicts the model
components and their relationships. The UrbanSim model system and
software architecture is described in detail elsewhere (Waddell,
2002; Waddell et al., 2003; Noth et al., 2003), and is available
as Open Source software at http://www.urbansim.org.

The data used by UrbanSim to construct the model database includes
parcel data from tax assessor offices, business establishment
files from the state unemployment insurance database or from
commercial sources, census data, GIS overlays representing
environmental, political, and planning boundaries, and a location
grid. These data are diagnosed and analyzed for missing,
erroneous, or inconsistent values. Each household in the
metropolitan area is represented as an individual entity, with the
primary characteristics relevant to modeling location and travel
behavior: household income, size, age of head, presence of
children, and number of workers. Employment is represented as
individual records for each job and its employment sector.
Locations are represented using grid cells of 150 by 150 meters
(just over 5.5 acres), whose size can be modified. This location
grid allows explicit cross-referencing of spatial features such as
planning and political boundaries, including city, county, traffic
zones, urban growth boundaries; and environmental features such as
wetlands, floodways, stream buffers, steep slopes, or other
environmentally sensitive areas. The interchange between UrbanSim
and an external transportation model system is loosely coupled.
Outputs from UrbanSim are used as inputs to the trip generation
models, and outputs from the mode choice model are inputs to
UrbanSim.

Regional accessibility is important for residential and employment
location choice, as well as real-estate development. We use travel
time to central business district and travel time to airport as
two of our regional accessibility measures. The other two measures
are accessibility to population and accessibility to employment.
These latter two measures are calculated by UrbanSim in a separate
accessibility model component.

The accessibility model calculates the regional accessibility for
a given location as the distribution of travel opportunities
weighted by the composite utility of all modes of travel to those
destinations, defined as the logsum from the mode choice model for
each origin-destination pair for a given auto-ownership category
$a$. The resulting access measure $A_{ai}$ for each location $i$
becomes:
\begin{equation}
    A_{ai} = \sum_j^J D_j\mathrm{e}^{L_{aij}},
    \label{eq:regacc}
\end{equation}
where $D_j$ is the quantity of activity in location $j$ (it is
either the population or employment depending on if we seek the
accessibility to population or employment), $L_{aij}$ is composite
utility, or logsum, for vehicle ownership category $a$, from
location $i$ to $j$, scaled to a maximum value of 0 for the
highest utility interchange. The accessibility model component
reads the composite utility from the travel model and this is one
of the ways that UrbanSim links to a regional travel model.

There are numerous definitions of local accessibility. It
fundamentally is used to measure non-motorized access to local
activities on a neighborhood scale (Crane, 2000; Ewing and
Cervero, 2001). The maximum distance that people in general will
walk for daily activities (e.g. grocery shopping, restaurant) is
not well defined and no consensus in the literature, but the range
is from about one quarter to one half mile. This clearly depends
on local conditions such as weather, terrain, street and sidewalk
configuration, and safety, in addition to personal characteristics
such as age and health status (Waddell and Nourzad, 2002). In this
study, we define the neighborhood scale as a radius of 600 meters,
which is roughly one third of a mile. This radius is used in the
spatial queries of the area surrounding grid cells, and we explore
land use and quantity of employment by sector using this radius.

UrbanSim implements the residential location choice, employment
location choice, and real-estate development choice models as
discrete choice models. In particular we assume that each
alternative $i$ has associated with it a utility $U_i$ that can be
separated into a systematic part and a random part:
\begin{equation}
    U_i = u_i + \epsilon_i,
    \label{eq:utility}
\end{equation}
where $u_i = \vk{\beta}\cdot\vk{x}_i$ is a linear-in-parameters
function, $\vk{\beta}$ is a vector of $k$ estimable coefficients,
$\vk{x}_i$ is a vector of observed, exogenous, independent
alternative-specific variables that may be interacted with the
characteristics of the agent making the choice (e.g. household
characteristics in the residential location choice model), and
$\epsilon_i$ is an unobserved random term. Assuming the unobserved
term in (\ref{eq:utility}) to be distributed with a Gumbel
distribution (Type I extreme value distribution) leads to the
familiar multinomial logit model (McFadden 1974, 1981):
\begin{equation}
    P_i = \frac{\mathrm{e}^{u_i}}{\sum_j \mathrm{e}^{u_j}},
    \label{eq:mnl}
\end{equation}
where $j$ is an index over all possible alternatives. The
estimable coefficients of (\ref{eq:mnl}), $\vk{\beta}$, are
estimated with the method of maximum likelihood (see for example
Greene, 2002).

%\subsection{Simulating Residential Location Choice}

In every UrbanSim simulation year, the population is changed using
an exogenous economic and population growth forecast. This
typically adds new households to the region. Also, UrbanSim uses
an estimate of the fraction of the population that moves, derived
from a survey, to randomly pick a set of households that will move
in the year.

Given a set of new and moving households the residential location
choice model takes each household and calculates the probability
of that household moving to every cell with vacant housing. A
random draw from that probability distribution determines the
chosen location, and the number of units occupied on that cell is
updated.

%\subsection{Simulating Employment Location Choice}

During every simulation year, jobs are created and jobs are
predicted to move by UrbanSim. The employment location choice
model is used to simulate the choice of a location for each job
that has no location. For each such job, a sample of locations
with vacant non-residential space, or space in housing units for
home-based jobs, is randomly selected from the set of all possible
alternatives. The estimated logit model for the job's industry
sector is then used to calculate the probabilities of each
alternative location being chosen. Once the probabilities of all
allowed location choice alternatives are calculated for a job, the
location choice is simulated using a Monte Carlo sampling process.
After cells have been selected as the chosen locations for all
jobs, the cells' characteristics are updated, i.e. the previously
vacant job spaces becomes occupied, the number of jobs in the
cells is updated, etc.

The database links individual jobs to job spaces. The job spaces
can be either nonresidential square footage, or a residential
housing unit to account for home-based employment. When jobs are
predicted to move, the space they occupy becomes vacant, and when
jobs are assigned to a particular job space, that space is
reclassified as occupied.

%\subsubsection{Simulating Real Estate Development}

UrbanSim simulates real estate development using the presented
model on a yearly basis. Each year, the model iterates over all
grid cells on which development is allowed and creates a list of
possible development alternatives for each cell. Development
constraints may reduce the number of alternatives from the
estimation stage.

Constraints on development outcomes are included through a
combination of user-specified spatial overlays and decision rules
about specific types of development allowed in different
situations. Each cell is assigned a series of overlays through
spatial preprocessing using GIS overlay techniques. These overlays
can be used to assign user-specified constraints on the type of
development that is allowed to occur within each of these overlay
designations. The constraints are indicated as allowed conversions
between each land use plan designation and each development type.
Currently, if users wish to examine the impact of these
constraints, they would need to relax a particular constraint and
compare the results to the results for more restrictive policy.
For example, the plan designation of 'agricultural' may not allow
conversion to any developed urban category under restrictive
interpretation of the land use plan, or may allow conversion to
rural density single-family residential under a less restrictive
interpretation. The overlays used in the Greater Wasatch Front
Region of Utah model application include the following features:
a) water cover, b) flood plane, c) steep slope, d) open space, e)
public space, f) roads, g) land use plan designation. Constraints
are implemented by eliminating the constrained development
alternatives from the choice set for any cell affected by the
constraint. These constraints are therefore interpreted as binding
constraints, and not subject to market pressure.

The estimated logit model is used to calculate the probabilities
of each allowed alternative, i.e. the no development alternative,
and one or more development alternatives. Development is then
simulated using a Monte Carlo sampling process. Actual
implementation of development takes place by using a development
template, which gives the most likely characteristics of the
resulting development project within the cell. The development
template has defined probability distributions for development
changes, including the number of housing units, square feet of
commercial, industrial and government space, improvement value,
and construction schedule. These development events are then added
to the 'development event' queue in UrbanSim, to be built as
scheduled.

Each year, after all other model components have executed, the
land price model simulates end-of-year prices of land, based on
the updated cell characteristics. These become the land prices
that influence location choice and developer behavior in the
subsequent year.

\section{Theory}

\subsection{Residential Location Choice}

The residential location choice model predicts the probability
that a household will select a location defined by a grid cell of
150 by 150 meters. Each grid cell can include zero or more housing
units and households can only select cells with vacant housing.
All housing units on a cell are assumed to be identical and we
therefore do not assign the household to a particular unit.

The model is therefore a disaggregate choice model with over
500,000 housing units. The number of alternatives that households
can select form is the total number of available housing units,
though noting that we only specify the location to a grid cell.
This creates a very large choice set and in this implementation of
the model we reduce that set by simply selecting a random sample
of nine alternatives from the universe of vacant housing, in
addition to the observed choice for each household, yielding 10
alternatives in the model. An estimation based on a random sample
of alternatives does lead to consistent estimates of the model
coefficients (Ben-Akiva and Lerman, 1987).


The model is specified as a multinomial logit model (\ref{eq:mnl})
with a systematic utility for a particular location (we drop the
index $i$ for convenience) on the form:
\begin{equation}
    u =  \alpha
     + \vk{\beta}_H \vk{x}_H
     + \vk{\beta}_R \vk{x}_R
     + \vk{\beta}_N \vk{x}_N,
    \label{eq:uresloc}
\end{equation}
where each utility term is a linear combination of variables that
have been grouped in to categories: $H$ indicates housing
characteristics (e.g. prices, density, age), $R$ indicates
regional accessibility, and $N$ reflects neighborhood-scale
effects (socioeconomic composition, land use mix, density, local
accessibility).

The principal data used in the analysis is based on a travel
survey conducted in the Wasatch Front region in 1997 of about
4,000 households. This data is supplemented with housing and
spatial information by linking the survey coordinates to the
UrbanSim grid cells and retrieving grid cell values for the
characteristics of housing, neighborhood, and regional access
based on the traffic analysis zone containing the cell. The
variables are drawn from the literature in urban economics, urban
geography, and urban sociology. We refer to Waddell and Nourzad
(2002) for more detailed discussion of the residential location
choice model and its context in the literature.

From Waddell and Nourzad (2002) we do point out that the model
generalizes the classical urban economic trade-off between
transportation and land cost by including regional and local
access measures, such as those of travel time to the classic
monocentric CBD, travel time to airport, distance to highway,
multi-modal access to employment opportunities, local shopping. As
it turns out, not all of these factors are important in the model.
A more detailed discussion of the accessibility measures is given
in Section~\ref{se:accessibility}.

The independent variables are organized into the three categories
of housing characteristics, regional accessibility, and
neighborhood-scale effects as shown below.  All independent
variables are endogenous to the model system---that is, they are
predicted by other parts of the model system shown in
Fig.~\ref{fi:resloc2}, and therefore, predicted values are
provided in future years for the application of the model system
over periods of 20--30 years.


\subsection{Employment Location Choice}

Theoretical models of employment location date at least to the
seminal work of von Th�nen (1826), which described a negatively
sloped agricultural land rent gradient in which land prices fall
with distance from a central market, to offset transportation
costs to the market.  This early work on bid-rent later stimulated
the development of the monocentric model of urban structure (Muth,
1969; Mills, 1967; Alonso, 1964).

Early applications of spatial theory of urban firm location can be
traced to Christaller's (1933) work on central place theory and
the hierarchy of cities, and that of Losch (1944), who derived an
idealized hexagonal representation of market areas based on
spatial competition between firms.

While these early contributions provided conceptual foundations
for understanding the competitive bidding for sites with higher
accessibility, which produces declining land rent gradients from
high access locations, and the spatial separation of firms
competing for market share, the framework would be insufficient to
explain the rise of central business districts in the 19th
century, and the rapid rise of secondary suburban centers in the
latter third of the 20th century.

A third major theoretical contribution is the concept of
agglomeration economies, which help to explain the existence of
employment clusters on the basis of externalities associated with
spatial proximity. These agglomeration economies have been
described as arising from information spillovers, local non-traded
inputs, and a local skilled labor pool (Marshall, 1920).

An important theoretical problem in urban economic models is that
neo-classical economic assumptions include constant returns to
scale, but the essence of agglomeration economies is the idea of
increasing returns to scale for firms that cluster with other
firms in their own or related industrial sectors (Krugman, 1991).
There are offsetting forces that neutralize the agglomeration
advantages of clustering as centers become large, producing
opportunities for the creation and growth of suburban centers.
Other relevant work on employment location has focused on
transportation costs (Chinitz, 1960), the influence of amenities
and governmental services and taxes (Bartik, 1991; Waddell and
Shukla, 1993).

The model developed in this paper draws on these antecedents,
bringing together the concepts of bid-rent theory, agglomeration
economies, and the effects of transportation and local government
policy in a discrete choice model.  The following section develops
the model specification, and is followed by a discussion of the
data and the estimation results from application of the model to
the metropolitan region of Salt Lake City, Utah.

The employment location choice model component of UrbanSim
simulates location choices for new jobs created as a byproduct of
economic expansion, predicted by an external macroeconomic model,
and for jobs that have been predicted to move by the employment
relocation model component of UrbanSim.

The employment location model is a discrete choice model from the
vantage point of an employer locating a job anywhere in the city.
All jobs that do not have a space in a current year, i.e. all new
jobs and all jobs that are moving, are faced with alternative
locations to choose from. The choice set of location alternatives
is defined by \emph{job spaces} (quantities of vacant
nonresidential square footage of sufficient size to accommodate a
job, and some fraction of the housing units, to accommodate
home-based employment) within 150 by 150 meter grid cells.

The probability of each alternative (the different job spaces open
to a particular job) being chosen is calculated using a discrete
choice model. We draw on discrete choice theory and random utility
maximizing models, following the work of McFadden (1974, 1981), to
specify a multinomial logit model.

To arrive at a choice model for employment location we assume that
1) each job belongs to a firm (whose characteristics other than
industry sector remain latent) which is faced with a choice
between alternative locations for the job, 2) that each location,
indexed by $i$, has attached to it some utility, $U_i$, for the
firm, and 3) that the location with the highest utility has been
chosen (maximization of utility).

We refer to the more general concept of utility maximization
rather than profit maximization, since the utility may be based
largely or exclusively on expectations of profit for some sectors,
but profit may represent a small or nonexistent part of the
utility for other sectors, such as governmental and educational
establishments.

We only observe the current location of jobs, and do not observe
the alternative locations open to the employer before locating the
job, nor do we observe the utility. We proceed with the
multinomial logit assumptions for the utility (\ref{eq:utility})
leading to (\ref{eq:mnl}).

The systematic component of the utility of a particular location
(dropping the index $i$ for simplicity) is specified as a function
of an array of characteristics at the \emph{site} ($\vk{x}_S$),
including the real estate characteristics (land value, residential
units, commercial sq. ft., land use) and proximity of the site to
freeways and arterials; characteristics of the land use mix and
value (quantity of residential units, average land values, average
improvement values) in the immediate \emph{neighborhood}
surrounding the site ($\vk{x}_N$); agglomeration economies from
geographic \emph{clustering} (employment by sector within 600 m)
of firms of the same and each of the other sectors ($\vk{x}_C$);
and multi-modal \emph{accessibility} to labor, consumers, the
Central Business District (CBD), and the regional airport
($\vk{x}_A$):
\begin{equation}
    u = \vk{\beta}_S \vk{x}_S
     + \vk{\beta}_N \vk{x}_N
     + \vk{\beta}_C \vk{x}_C
     + \vk{\beta}_A \vk{x}_A.
     \label{eq:uemploc}
\end{equation}

The probability, $P_i$, represents the probability of the firm
choosing location alternative $i$ for a particular job. We
estimate one choice model (i.e. one set of coefficients) for each
of the 14 industry sectors shown in Table\ref{ta:sectors} on a
random sample of 5000 observed jobs in each sector.

To estimate the model coefficients we use data for business
establishments in 1997, geo-coded to a grid of 150 by 150 meter
cells. The database links individual jobs to job spaces. The job
spaces can be either nonresidential square footage, or a
residential housing unit to account for home-based employment.

To arrive at a set of alternatives we allow each job in a sector
the choice of all locations in the universe open to that industry
sector. This generates a very large choice set. We use a uniform
distribution to randomly sample a set of nine alternatives in
addition to the chosen location and estimate a model using this
random sample of alternatives. This makes it impossible to
estimate alternative-specific coefficients or alternative-specific
constants. However, it can be proven that the coefficients of a
choice model estimated from a random sample of alternatives,
selected with a uniform distribution, are consistent, as explained
by McFadden (1978) in his paper on residential location choice,
which faces a similar issue.



\subsection{Real-Estate Development and Land Price}

The theoretical foundations of the model components described here
draw on Random Utility Maximization (RUM) models pioneered by
McFadden (1974, 1981), on bid-rent theory of land markets (Alonso,
1964; Wheaton, 1977), and on hedonic price theory (Rosen, 1974).
The work represents an ongoing effort to integrate these strains
of theory and methodological developments into an operational
urban simulation framework.

Our approach in modeling real estate prices assumes that
individual consumers and suppliers are too small in scale to
manipulate prices directly, making them exogenous to these
individual actors. Whereas this assumption could be argued in the
event of oligopolistic behavior by large-scale developers or large
corporations seeking sites, it is a relatively weak assumption to
impose and avoids complications arising from modeling prices as
endogenous to the interaction between consumers and sellers, such
as having to simulate search and auction processes, imperfect
information, and oligopolistic market behavior. A second
assumption is that the advantages of location, such as
neighborhood amenities and accessibility, are capitalized into
land values. This assumption follows from a wide consensus of
theoretical and empirical work in urban economics that has
consistently found that in competitive land markets, the
quasi-unique characteristic of land (they aren't producing any
more of it, every location is unique, and housing or commercial
buildings are tied to their location) implies that consumers bid
for location based on their willingness to pay for locational
attributes, and the highest bidder wins the use of the site and
sets the market price for it (Alonso, 1964; Mills, 1967).

Rosen (1974) developed the approach of hedonic price analysis,
which attempts to disentangle the implicit prices for the
components of the bundle of services provided by housing (the same
theory applies to nonresidential space). By regressing the sale
price of housing on characteristics of the housing structure and
location, we obtain estimates of the implicit prices of individual
characteristics-holding other characteristics constant-despite us
observing only the single price of the bundle for any individual
property. These implicit prices do not, strictly speaking,
represent either demand functions (willingness to pay) or supply
functions (reservation prices), but rather, the composite of all
of the willingness to pay and reservation price functions of all
consumers and sellers in the market. Given our assumption that
prices are exogenous to individual consumers or sellers, this
provides a reasonable way to estimate the land price function
within a given market.

Following DiPasquale and Wheaton (1996), we interpret market
prices of land within a metropolitan market as consisting of two
parts. The first component is a mean price level, which fluctuates
around long-term trends that are driven by short-term imbalances
between supply and demand of real estate, by interest rates and
other development costs, and in the longer-term by overall
expansion and contraction of the metropolitan economy, population,
and changes in income. The second component is the relative price
of land across sites within the metropolitan market. These
relative prices are based on relative advantage and abundance of
sites with characteristics that are valued or avoided by
consumers. As these underlying characteristics and the resulting
relative advantage change, so to do relative prices, as these
advantages are capitalized into land values. This paper focuses
principally on the characteristics influencing relative prices,
since these will have the greatest influence on intra-metropolitan
variation in real estate development and consumer location
choices.

Real estate development is a collection of choices made by
individual developers on individual sites, about whether, when,
and how to develop or redevelop those sites. We assume their
behavior is motivated by profit (they attempt to maximize their
profits), within constraints imposed by their resources, the
physical environment, and by public land use regulations. The main
influences on their choices will then be factors influencing
prices of different types of real estate at different locations,
the costs of producing those development projects, and the
constraints relevant at those sites. There are two general
approaches that developers consider in making development choices.
The first is known as the site looking for a use, and corresponds
to a specialized developer who has a specific project in mind, and
attempts to find the most profitable site for the project. The
second general approach is known as the use looking for a site,
and corresponds more closely to the landowner's problem of sorting
out which type of developer to sell the property to, that will
generate the highest return. In the real world, both approaches
occur. We have structured the current model as a discrete choice
model from the perspective of the site looking for a use-the
landowner's perspective. This approach lends itself to formulation
as a standard multinomial logit model, where an individual
landowner considers alternative uses, or developments, for a
particular site.

Taking these elements together, we propose the modeling of land
prices as a hedonic regression, and the real estate development
model as a multinomial logit model of development of a site into
alternative uses over a specific time frame.

The parcel data are collapsed into the 150 by 150 meter cells to
generate composite representations of the mix and density of real
estate at each location, labeled development types. These
development types are somewhat analogous to the development
typology developed by Calthorpe (1993), in that they represent at
a local neighborhood scale the land use mix and density of
development. Table~\ref{ta:devtypes} provides the rules for
classifying grid cells into development types, based on the
combination of housing units, nonresidential square footage, and
the principal land use of the development. Cells containing some
housing and almost no nonresidential square footage are considered
residential in character. Those containing a diverse mixture of
housing and nonresidential floor-space are considered mixed-use,
and those cells containing principally nonresidential square
footage are further classified into commercial, industrial or
governmental types. For a more detailed description, see (Waddell,
2002; Waddell et al., 2003).

The purpose of the real estate development model is to simulate
discrete developer choices about whether to develop particular
sites within a given year, what type of construction to undertake,
and the quantity of construction. The construction of real estate
can be either new development (sometimes referred to as Greenfield
development) or the intensification or conversion of existing
development (referred to as infill and redevelopment,
respectively). The model takes a bottom-up view, i.e. from the
vantage point of a developer or a land-owner at a single location
(grid cell) making choices about whether to develop, and into what
type of real estate. This bottom-up view is tempered by market
information that reflects the state of the market as a whole, such
as vacancy rates.

The model is designed in terms of discrete alternatives that
represent development events, including the base case of no
development on a particular site within a given year. In addition,
there are development alternatives that represent transitions
between the different development types defined in
Table~\ref{ta:devtypes}, including the alternative of increasing
the density of the current cell without changing its development
type (where this is feasible).

The probability of each alternative (the no development, the
increasing density of current cell within its development type,
and transitions to other development types) being chosen is
calculated using a discrete choice model. We draw on discrete
choice theory and random utility maximizing models, following the
work of McFadden (1974, 1981), to design a multinomial logit
model. Similar approaches have been developed to model land cover
change (Turner and Gardner, 1991) and land use change (Landis and
Zhang, 1998), although none of these models interact with
disaggregate demand-side models of residential and employment
location choice as is done in UrbanSim.

To arrive at a choice model for development we assume that 1) each
cell has a developer, 2) that each development alternative,
indexed by $i$, has attached to it some utility, $U_i$, for the
developer, based principally on profit expectations, and 3) that
the development event with the highest utility has occurred
(maximization of utility). We only observe events that have
occurred and do not observe the utility.

To form the estimation data we take all the development event
cells, i.e. cells with a known development event and look up the
values for a set of independent variables from the grid cell
database. The independent variables in the real estate development
model include characteristics of the \emph{site} ($\vk{x}_S$),
including current development, land use plan, environmental
constraints, policy constraints, land and improvement value,
proximity to highways, arterials, existing development, and recent
development; characteristics of the land use mix, property values,
and local accessibility measures in the \emph{neighborhood}
surrounding the site ($\vk{x}_N$); and multi-modal
\emph{accessibility} ($\vk{x}_A$), including access to population
and employment and travel time to the central business district
and airport:
%
\begin{equation}
    u = \alpha
     + \vk{\beta}_S \vk{x}_S
     + \vk{\beta}_N \vk{x}_N
     + \vk{\beta}_A \vk{x}_A.
     \label{eq:uemploc}
\end{equation}
%
We proceed with the multinomial logit assumptions for the utility
(\ref{eq:utility}) leading to (\ref{eq:mnl}). The probability,
$P_i$, represents the probability of a particular cell developer
choosing development alternative $i$.

We now need to take into account the much larger set of cells that
didn't experience a development event. We take a random sample of
these cells to generate a set of similar size as the development
event set. This gives us a choice-based sample of cells.
Choice-based sampling only biases the alternative-specific
constants but other coefficients remain consistent (Manski and
McFadden, 1981). We adjust the alternative-specific constants
after estimation to account for this bias.

We estimate one choice model (i.e. one set of coefficients) for
each development type, since the types are very different and the
development alternatives open to each development type vary. To
estimate the model coefficients we need data for cells
experiencing no development and for cells with development events
of all types.

\subsubsection{Data and Estimation Process for Real Estate Development
Model}

The estimation data are derived from the parcel and grid data for
a base year of 1997. The year-built values of the existing
development in the assessor's records are the foundations of the
process. Year-built values are imputed for records for which they
are missing by examining the surrounding cells of the same type
and drawing from the distribution of observed values. Historical
development 'events' are identified in the data for a
user-specified period of time. Events, within this framework, are
any changes in the real estate development within a cell that is
identified by examining the year built values within the data.

The procedure is capable of identifying any new construction that
has a year-built occurring within the specified time frame.
However, the procedure does not identify events that involve the
demolition of buildings at some time in the past, since normally
there is no record of demolitions within the current assessor
database. This procedure could be augmented with data derived from
building demolition and permit records.

The result is a set of cells experiencing development events that
represent all observed transitions between any pairs of
development types, including increases in density that didn't
result in a development type change, within each year of the
specified historical time frame. The time slice for determining
the existence of an event is annual, since this is the limit of
the information on the vintage of real estate. For further
explanations of this process, see (Waddell et al., 2003).







\subsubsection{Data and Estimation Process for Land Price
Model}

The land value for each cell, taken as the aggregation of the land
value of the parcel fragments that lie within the cell, and
originating from the tax assessor's estimates of the land value of
each parcel, is used as the basis for the dependent variable of
the land price model. The independent variables used as
predictors-essentially the same as for the real estate development
model-are the characteristics of the cell, its surrounding
environment, and its accessibility. A semi-log specification is
used, with the log of land price as the dependent variable, as is
common in hedonic price studies since it generally provides a more
robust specification.

The model is a linear multiple-regression of the log of land
prices, $\ln(v_i)$, for each cell $i$, on an array of housing
structural ($\vk{x}_{Si}$), neighborhood ($\vk{x}_{Ni}$), and
accessibility ($\vk{x}_{Ai}$) characteristics:
\begin{equation}
    \ln(v_i) = \alpha
        + \vk{\beta_S}\vk{x_{Si}}
        + \vk{\beta_N}\vk{x_{Ni}}
        + \vk{\beta_A}\vk{x_{Ai}}
        + \epsilon_i,
\end{equation}
where $\alpha$ is the estimable intercept term; $\vk{\beta_S}$,
$\vk{\beta_N}$, and $\vk{\beta_A}$ are the estimable coefficient
vectors on the housing structural, neighborhood, and accessibility
characteristics, respectively; $\epsilon_i$ is an unobserved error
term, assumed to be normally distributed with mean zero and
variance $\sigma^2$.

The full set of grid cells in the study area is used in model
estimation, using base year (1997) characteristics and values. As
such, this is a cross-sectional estimation of the market hedonic
price function, rather than an estimation of a dynamic price
function. Dynamics are introduced through the process of annual
changes in the characteristics of grid cells due to simulated
results from the real estate development, residential location and
employment location models, and the external transportation model
system, all of which combine to change the characteristics of grid
cells on an annual basis.

\section{Results}

\subsection{Residential Location Choice}
The variables used in the residential model estimation, and the
results of the model estimation, are presented in
Table~\ref{ta:resloc1}.

The first variable captures the effect of the relationship between
local housing price in the cell to the income of the household.

Household income interacted with the percent residential in the
grid cell.

Distance to nearest highway.

Regional accessibility to employment.  We use access to
employment, stratified by household vehicle ownership category, as
defined earlier in the description of regional accessibility.
These are expected to be positive effects on the probability of
residential location.

Variable description? "When the quantity (household income minus
one tenth of the grid cell's average price per residential unit)
is negative, this is a very low negative number, otherwise it is
the log of that quantity"

Household income interacted with regional access to employment for
single-vehicle households.

Percent residential within 600 m.

Number of residential units on the cell.

Quantity of retail within 600 m.

Percent high-income households within 600 m if the decision maker
household is high-income, and percent low-income households within
600 m if the household is low-income represent neighborhood income
segregation.

Number of residential units in the cell for households with
children. We expect, a priori, that households prefer lower
housing density, all else being equal, since urban economic theory
and substantial empirical evidence suggest that as incomes rise,
land consumption increases (O'Sullivan, 2000). But we explore
variation in the responsiveness to density based on the presence
of children. Households with children may have stronger
preferences for low density housing and neighborhoods. There is
reason to expect this coefficient to be negative.

We also test the possibility that younger households (head of
household under 40) differ from other households in their taste
for density, all else being equal, by interacting a dummy for
young households with dummies for development types 6-8,
representing higher density residential housing.


\subsection{Employment Location Choice}
Estimation results for the employment location choice logit models
for all industry sectors in the Greater Wasatch Front region are
presented in Tables~\ref{ta:emploc2}--\ref{ta:emploc3}.  Since the
coefficients are based on random sampling of alternatives, there
are no alternative-specific constants, and no base alternative.
The coefficients are therefore interpretable in terms of the
direction of the influence of a variable on the utility and the
probability of a location choice.  In addition, coefficients can
be compared across industry sectors, since the same specification
is used for all sectors, with the exception of insignificant
variables, which were restricted to zero.  Interpreting the
coefficients is complex, however, due to the interaction between
correlated variables. This is particularly true of the access to
population and access to employment variables, and the travel time
to the CBD and to the regional airport, since these are fairly
close to each other within the broader region.  Nevertheless,
including these correlated variables improved the goodness of fit
of the model.

Indicators for development types and groups

Indicators for employment sectors

Log of the average land value per acre within walking distance

Log of commercial sq.ft. in the grid cell

Log of the distance to nearest highway

Log of improvement value per residential unit within walking
distance

Log of the number of residential units in the grid cell

Log of the number of residential units within walking distance

Log of total value of the cell

Log of work accessibility to employment for one vehicle households in the cell's TAZ

Log of work accessibility to population for one vehicle households
in the cell's TAZ

AM peak hour travel time by single-occupancy vehicle from the
cell's TAZ to the CBD's TAZ (or a representative TAZ for the CBD)

AM peak hour travel time by single-occupancy vehicle from the
cell's TAZ to the airport TAZ


\subsection{Real-Estate Development and Land Price}

\subsubsection{Real-Estate Development}

The real estate development model is estimated separately for
cells of each development type, representing a total of 24 models.
Due to space limitations, only the results of one of the 24
models, representing cells initially classified as vacant land,
are reported in Table~\ref{ta:devlp2}. Although this model
represents the predominant mode of new real estate construction,
conversion of vacant land to urban uses (greenfield development),
it is important to recognize that infill and redevelopment do
contribute substantially to the real estate inventory, and
moreover, change the spatial structure of urban areas in quite
significant ways. The dynamics of infill and redevelopment are
represented in the other 23 real estate development models not
reported here.

The results shown in Table~\ref{ta:devlp2} provide the estimation
results for conversion of cells classified initially as vacant
land to one of the alternative 23 development types. The
alternatives not reported as outcomes (8, 10-16, 18, 19, 22, 23)
are omitted because there were too few historical transitions of
this type observed in the 10 years over which these events were
compiled. The first seven alternatives are residential
development, ranging from very low density (one residential unit
in a cell of just over 5.5 acres), to relatively high density (31
to 75 units in the same area). Alternative 9 is a low-density
mixed use development type, alternative 17 is low-density
commercial development, and alternatives 20 and 21 are low- and
moderate-density industrial development, respectively.
Coefficients in many cases are constrained to be the same across
related alternatives.

Indicator for cells near an arterial

Indicator for cells near a highway

Log of accessibility to population for one-vehicle households in
the cell's TAZ

Log of the total land value in the grid cell

Percent of development type group commercial within walking
distance

Percent of development type group industrial within walking
distance

Percent of development type group residential within walking
distance

A measure of proximity to development



\subsubsection{Land Price}

Results of the land price model show that the model explains
approximately 75\% of the variation in the log of land value of
cells. In these results, the coefficients reported are all
significant at the 95\% level, and the coefficients are directly
interpretable. The coefficients on the continuous independent
variables that are nominal show the percentage effect on land
value in a cell associated with a one-unit change in the
independent variable (multiply the coefficient with 100 to arrive
at the percentage change). Coefficients on variables that are
log-transformed are directly interpretable as elasticities. For
dummy, or categorical variables, the coefficients indicate the
percentage change in the land value of a cell associated with a
change from a value of 0 to a value of 1 on the dummy variable
(multiply the coefficient with 100 to arrive at the percentage
change). Each coefficient must be interpreted holding all other
variables constant.

Indicators for development types.

Indicator for cells near a highway

Log of the average total value per residential unit within walking
distance

Log of commercial sq.ft. in the grid cell

Log of commercial sq.ft within walking distance

Log of the distance to nearest highway

Log of accessibility to employment for one-vehicle households in
the cell's TAZ

Log of accessibility to population for one-vehicle households in
the cell's TAZ

Log of the percent of development type groups within walking
distance

Log of the number of residential units in the grid cell

Log of the number of residential units within walking distance

Log of total employment within walking distance

Log of total improvement value in the cell

Percent of cell covered by overlays

Indicators for plan types



\section{Conclusion}

\section*{Acknowledgements}

This material is based upon work supported by the National Science
Foundation under Grants CMS-9818378, EIA-0090832, BCS-0120024, and
EIA-0121326, and by the Governor's Office of Planning and Budget
(GOPB), the Wasatch Front Regional Council (WFRC), the
Mountainlands Association of Governments (MAG). In particular, we
wish to acknowledge the assistance of John Britting at WFRC, Mick
Crandall at WFRC, Carl Johnson at MAG, Peter Donner  at GOPB,
Natalie Gochnour (formerly at GOPB), Stuart Challender (formerly
with Utah Automated Geographic Reference Center), and for their
assistance.

Bree Jones at WFRC

UrbanSim crew members


\section*{References}

\newlength{\lengthstorage}
\lengthstorage=\parindent
\parindent=0pt
\parskip=2mm

Alonso W, 1964 \textit{Location and Land Use} Harvard University
Press, Cambridge

Bartik T, 1991 \textit{Who Benefits from State and Local Economic
Development Policies?} W.E. Upjohn Institute, Kalamazoo, Mich

Ben-Akiva M, and Lerman S, 1987 \textit{Discrete Choice Analysis:
Theory and Application to Travel Demand}  The MIT Press,
Cambridge, MA

Calthorpe P, 1993 \textit{The Next American Metropolis: Ecology,
Community and the American Dream} Princeton Architectural Press,
New York, NY

Cervero R, and Kockelman K, 1997 "Travel Demand and the Three Ds:
Density, Diversity, and Design" \textit{Transportation Research,
Part D} \textbf{2} (2) pp 199--219

Chinitz B, 1960 "The Effect of Transportation Forms on Regional
Economic Growth" \textit{Traffic Quarterly} \textbf{14} pp
129--142

Christaller, W. 1933 \textit{Die Zentralen Orte in S�ddeutschland}
(In \textit{Central Places in Southern Germany}) Fischer J, ed.,
Prentice-Hall, Englewood-Cliffs, NJ

Crane R, 2000 "The Influence of Urban Form on Travel: An
Interpretative Review" \textit{Journal of Planning Literature}
\textbf{15} (1) pp 3--23

DiPasquale D, and Wheaton W, 1996 \textit{Urban economics and real
estate markets} Prentice Hall, Englewood Cliffs, NJ

Greene W, 2002 \textit{Econometric Analysis} 5th Ed. Pearson
Education

Ewing R, and Cervero R, 2001 "Travel and the Built Environment: A
Synthesis (with Discussion)" \textit{Transportation Research
Record 1780} pp 87--114

Handy S, 1993 "Regional Versus Local Accessibility: Implications
for Nonwork Travel" \textit{Transportation Research Record 1400}
pp 58--66

Krugman P, 1991 "Increasing Returns and Economic Geography"
\textit{Journal of Political Economy} \textbf{99} pp 483--499

Landis J, and Zhang M, 1998 "The Second Generation of the
California Urban Futures Model Part I: Model Logic and Theory"
\textit{Environment and Planning B: Planning and Design}
\textbf{25} pp 657--666

Losch A, 1944 \textit{Die Raumliche Ordnung der Wirtschaft} (In
\textit{The Economics of Location}) Fischer J, ed., Yale
University Press, New Haven

Manski C F, and McFadden D, 1981 "Alternative Estimators and
Sample Designs for Discrete Choice Analysis" In \textit{Structural
Analysis of Discrete Data with Econometric Applications} Manski C
F, and McFadden D, eds, MIT Press, Cambridge, MA, pp 2--50

Marshall A, 1920 \textit{Principles of Economics} 8th Ed,
Macmillan, London

McFadden D, 1974 "Conditional logit analysis of qualitative choice
behavior" In \textit{Frontiers in Econometrics} Zarembka P, ed.,
Academic Press, New York, NY

McFadden D, 1978 "Modeling the choice of residential location in
spatial interaction theory and planning models" In \textit{Spatial
Interaction Theory and Planning Models} Karlqvist A, Lundqvist L,
Snickars F, and Wiebull J W, eds, North Holland, Amsterdam, pp
75--96

McFadden D, 1981 "Econometric Models of Probabilistic Choice" In
\textit{Structural Analysis of Discrete Data with Econometric
Applications} Manski C F, and McFadden D, eds, MIT Press,
Cambridge, MA, pp 198--272

Mills E S, 1967 "An Aggregative Model of Resource Allocation in a
Metropolitan Area" \textit{American Econometric Review}
\textbf{57} pp 197--210

Muth R F, 1969 \textit{Cities and Housing} University of Chicago
Press, Chicago

Noth M, Borning A, and Waddell P, 2003 "An Extensible, Modular
Architecture for Simulating Urban Development, Transportation, and
Environmental Impacts" \textit{Computers, Environment and Urban
Systems} \textbf{27} (2) pp 181--203

O'Sullivan A, 2000 \textit{Urban Economics} McGraw-Hill, New York,
NY

Rosen S, 1974 "Hedonic prices and implicit markets: product
differentiation in pure competition" \textit{Journal of Political
Economy} \textbf{82} pp 34--55

Turner M, and Gardner R, 1991 \textit{Quantitative Methods in
Landscape Ecology} Springer-Verlag, New York, NY

von Th�nen J H, 1826 "Der Isolierte Staat" In \textit{Beziehung
auf Landwirtshaft und Nationalekonomie} Hamburg, Germany

Waddell P, 2000 "A behavioral simulation model for metropolitan
policy analysis and planning: residential location and housing
market components of UrbanSim" \textit{Environment and Planning B:
Planning and Design} \textbf{27} (2) pp 247--263

Waddell P, 1998 "Exploiting parcel level GIS in land use models"
In \textit{ASCE Conference on Land Use, Transportation and Air
Quality: Making the Connection} American Society of Civil
Engineers Portland, OR

Waddell P, 2000 "Monitoring and Simulating Land Capacity at the
Parcel Level" In \textit{Monitoring Land Supply with Geographic
Information Systems: Theory, Practice and Parcel-Based Approaches}
Vernez-Moudon A, and Hubner M, eds, John Wiley \& Sons, Inc, New
York, NY pp 201--217

Waddell P, 2001a "Between Politics and Planning: UrbanSim as a
Decision Support System" In \textit{Planning Support Systems:
Integrating Geographic Information Systems, Models, and
Visualization Tools} Brail R K, and Klosterman R E, eds, ESRI
Press and Center for Urban Policy Research Redlands, CA, pp
201--228

Waddell P, 2001b "Towards a Behavioral Integration of Land Use and
Transportation Modeling" In \textit{The Leading Edge in Travel
Behavior Research} D. Hensher, ed., Pergamon Press

Waddell P, 2002 "UrbanSim: Modeling Urban Development for Land
Use, Transportation and Environmental Planning" \textit{Journal of
the American Planning Association} \textbf{68} 3 pp 297--314

Waddell P, and Moore T, 2001 "Forecasting Demand for Urban Land"
In \textit{Land Market Monitoring for Smart Urban Growth} Knaap G,
ed., Lincoln Institute for Land Policy, Cambridge, MA, pp 187--217

Waddell P, and Nourzad F, 2002 "Incorporating Non-Motorized Mode
and Neighborhood Accessibility in an Integrated Land Use and
Transportation Model System" \textit{Transportation Research
Record 1805} pp 119--127

Waddell P, and Shukla V, 1993 "Employment Dynamics, Spatial
Restructuring, and the Business Cycle" \textit{Geographical
Analysis} \textbf{25} (1) pp 35--52

Waddell P, and Ulfarsson G F, 2003a "Accessibility and
Agglomeration: Discrete-Choice Models of Employment Location by
Industry Sector" \textit{The 82nd Annual Meeting of the
Transportation Research Board} January 12--16, Washington, DC,
Preprint CD-ROM

Waddell P, and Ulfarsson G F, 2003b "Dynamic Simulation of Real
Estate Development and Land Prices within an Integrated Land Use
and Transportation Model System" \textit{The 82nd Annual Meeting
of the Transportation Research Board} January 12--16, Washington,
DC, Preprint CD-ROM

Waddell P, Borning A, Noth M, Freier N, Becke M and Ulfarsson G F,
2003 "Microsimulation of Urban Development and Location Choices:
Design and Implementation of UrbanSim" \textit{Networks and
Spatial Economics} \textbf{3} (1) pp 43--67

Wheaton W C, 1977 "A bid rent approach to housing demand"
\textit{Journal of Urban Economics} \textbf{4} (2) pp 200--217

\section*{Not Yet Cited References}

Franklin J, Waddell P, and Britting J, 2002 "Sensitivity Analysis
Approach for an Integrated Land Development \& Travel Demand
Modeling System" \textit{The Association of Collegiate Schools of
Planning 44th Annual Conference} November 21--24, Baltimore, MD,
Paper available at http://www.urbansim.org/papers

Waddell P, and Borning A, 2004 "A Case Study in Digital
Government: Developing and Applying UrbanSim, a System for
Simulating Urban Land Use, Transportation, and Environmental
Impacts" \textit{Social Science Computer Review} \textbf{22} (1)
pp 37--51

Waddell P, and Ulfarsson G F, forthcoming "Introduction to Urban
Simulation: Design and Development of Operational Models" In
\textit{Handbook in Transport, Volume 5: Transport Geography and
Spatial Systems} Stopher, Button, Kingsley, Hensher eds. Pergamon
Press

Waddell P, Outwater M, Bhat C, and Blain L, 2002 "Design of an
Integrated Land Use and Activity-Based Travel Model System for the
Puget Sound Region" \textit{Transportation Research Record 1805}
pp 108--118

\parindent=\lengthstorage
\parskip=0pt

\appendix
\section{Informative Tables}

\begin{table}[!htbp]
\caption{Industry sectors.} \label{ta:sectors}
\begin{center}
\begin{tabular}{cll}
\hline\hline
 Sector number &  Sector description & Sector type \\\hline
  1 & Resource Extraction & Basic \\
  2 & Construction  &  Basic \\
  3 & Manufacturing & Basic \\
  4 & Transport, Communications and Utilities & Basic \\
  5 & Trucking and Warehousing, Wholesale Trade &  Basic \\
  6 & General Retail & Retail \\
  7 & Restaurants and Food Stores & Retail \\
  8 & Auto Sales and Services & Retail \\
  9 & Finance & Service \\
 10 & Insurance and Real Estate &  Service \\
 11 & Business and Professional Services & Service \\
 12 & Health Services & Service \\
 13 & General Services & Service \\
 14 & Government and Education & Service \\\hline\hline
\end{tabular}
\end{center}
\end{table}
\clearpage

\begin{table}[!htbp]
\caption{Development type classification}\label{ta:devtypes}
\begin{center}
\begin{tabular}{llrrrrl}
 \hline\hline
 Dev. Type   &   Name    &   Min. Units  &   Max. Units  &   Min. Sqft.  &   Max. Sqft.  &   Primary Use
 \\\hline
 1   &   R1  &   1   &   1   &   0   &   999 &   Residential \\
 2   &   R2  &   2   &   4   &   0   &   999 &   Residential \\
 3   &   R3  &   5   &   9   &   0   &   999 &   Residential \\
 4   &   R4  &   10  &   14  &   0   &   2,499   &   Residential \\
 5   &   R5  &   15  &   21  &   0   &   2,499   &   Residential \\
 6   &   R6  &   22  &   30  &   0   &   2,499   &   Residential \\
 7   &   R7  &   31  &   75  &   0   &   4,999   &   Residential \\
 8   &   R8  &   76  &   65,000  &   0   &   4,999   &   Residential \\
 9   &   M1  &   0   &   9   &   1,000   &   4,999   &   Mixed Use   \\
 10  &   M2  &   10  &   30  &   2,500   &   4,999   &   Mixed Use   \\
 11  &   M3  &   10  &   30  &   5,000   &   24,999  &   Mixed Use   \\
 12  &   M4  &   10  &   30  &   25,000  &   49,999  &   Mixed Use   \\
 13  &   M5  &   10  &   30  &   50,000  &   9,999,999   &   Mixed Use   \\
 14  &   M6  &   31  &   65,000  &   5,000   &   24,999  &   Mixed Use   \\
 15  &   M7  &   31  &   65,000  &   25,000  &   49,999  &   Mixed Use   \\
 16  &   M8  &   31  &   65,000  &   50,000  &   9,999,999   &   Mixed Use   \\
 17  &   C1  &   0   &   9   &   5,000   &   24,999  &   Commercial  \\
 18  &   C2  &   0   &   9   &   25,000  &   49,999  &   Commercial  \\
 19  &   C3  &   0   &   9   &   50,000  &   9,999,999   &   Commercial  \\
 20  &   I1  &   0   &   9   &   5,000   &   24,999  &   Industrial  \\
 21  &   I2  &   0   &   9   &   25,000  &   49,999  &   Industrial  \\
 22  &   I3  &   0   &   9   &   50,000  &   9,999,999   &   Industrial  \\
 23  &   GV  &   0   &   99,999  &   0   &   9,999,999   &   Government  \\
 24  &   Vacant  &   0   &   0   &   0   &   0   &   Vacant  \\
     &   Developable &       &       &       &       &   Developable \\
 25  &   Undevelopable   &   0   &   0   &   0   &   0   &   Undevelopable   \\
 \hline\hline
\end{tabular}
\end{center}
\end{table}
\clearpage

\section{Result Tables}
\input result_tables

\end{document}
