% $Id: conclusion.tex,v 1.7 2006/09/08 15:17:10 borning Exp $

\section{Conclusion}
\label{sec:conclusion}

This paper presented the Indicator Browser, a web-based application for
generating visualizations of UrbanSim simulation results. We showed how a
diverse set of techniques (Value Sensitive Design, iterative design, paper
prototyping, agile software development) can be used in concert
in the design and implementation of an interactive digital government
application, in particular, how consideration of human values can be
integrated with user-centered design techniques.

The evaluation described in Section \ref{sec:evaluation} had only 6
participants, and so these results are only preliminary
regarding the usability of the system in practice.  However, these results
were mostly encouraging as far as they went.  Users were especially
interested in using the Indicator Browser to visualize the batch of
indicators that are generated by the Python script, and to create new
indicators to further investigate interesting issues that arise from the
visualization of the pre-generated indicators.  There were some usability
issues that still need to be resolved before deploying the interface but in
general, users seemed very positive about the outcome.

Some of the disadvantages of this system are that it doesn't allow as much
flexibility as some other indicator generation mechanisms, such as the
Python script. UrbanSim developers who are comfortable writing code cannot
modify the indicator generation code through the Indicator Browser, but
they can through the Python script. In addition, the Indicator Browser only
displays a subset of the available indicators which do not include the
difference, aggregate and disaggregate indicators, among others.

The main advantages that the Indicator Browser has over other indicator
generation mechanisms is that it provides all the functionality from within
one application, a web-browser, which all of our users were comfortable
with. The users can also access both custom and pre-generated indicators
directly through the web-browser instead of having to access some remote
file server or having to remote desktop to their work machine. The users
can also email the URL of the indicator's visualization page
directly from the interface, as opposed to
having to compose an email and attach the indicator visualization.  Most
importantly, urban modelers and planners are comfortable with this
interface since it doesn't require any programming.

The evaluations showed that the less comfortable the users were with
respect to manipulating code and the more indicators they generate on a
regular basis, the more appealing they found the Indicator Browser
interface. It is better for single indicator generation than the Python
script since one would need to edit and rerun the script. However, the
Python script provides more flexibility and functionality than the Indicator
Browser.

Users found the Indicator Browser to be a good source of ready-to-hand
information. They were able to see information about available scenario
runs, their particular configurations, indicator scripts and documentation,
available geographies and indicator visualizations all within the same
interface. Usually, our users would need to access the code, several
databases and documentation in order to retrieve this information.

Browsing already-computed indicator results is much faster (and more
satisfying) than submitting new requests.  So in future work, we will
investigate additional graphical interface support for configuring and
editing scripts that produce batches of indicators, which can then be
browsed using the Indicator Browser or a successor tool.  We also plan to
expand our target users to include engaged citizens as well as urban
planners.  
(During the evaluation of the interactive system some users mentioned they
would recommend this interface to engaged citizens and activist groups.)
In addition, we plan to construct a web-based tool (U-Build-It)
for configuring alternate scenarios to be simulated by selecting different
transportation infrastructure options, zoning changes, and the like, from a
palette of alternatives.  The new alternative would then be simulated using
UrbanSim and then visualized.  
With this system, engaged citizens will have
tools that we hope will aid them in making more informed decisions and
participating more fully in the urban planning process.

% include this command to even out the final columns:
% \pagebreak[4]

% LocalWords:  borning UrbanSim pre  disaggregate
