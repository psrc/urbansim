% $Id: implementation.tex,v 1.7 2006/09/08 23:57:16 borning Exp $

\subsection{System Implementation}
\label{sec:implementation}

\subsubsection{Agile Test-first Software Development}

The Indicator Browser was developed using Agile Test-first software
development \cite{cockburn-book-2002}, which is a technique that focuses on
producing software systems in small steps driven by unit tests. The
idea is to deliver working software early and continuously to enhance
customer satisfaction. A developer should first write a test for a new piece
of functionality before it is implemented (test-driven development).  
Both independent modules and interactions between modules
are tested.  UrbanSim developers are encouraged to commit their code as
often as practical.  Fireman, a program that runs all the project's tests
every time the code is committed
to the source code repository, also aids in maintaining code quality.
Agile development encourages constant communication amongst the people
involved in the development process.  Developers write ``Today
messages'' \cite{brush-hicss-2005} every day to let the rest of the
development team know about their daily progress. In addition, software
engineers and project leaders meet with customers on a frequent basis to
communicate about the project's progress and redefine requirements.

\subsubsection{System Reliability and Robustness}

Robustness refers to the ability of the software system to perform
correctly even under extreme circumstances or when there are external or
internal changes that affect system's performance. Reliability refers to
the ability of the software system to perform as intended for a long
period of time.  Opus is an evolving system that constantly redefines its
requirements and whose code base is unstable.  However, the Indicator
Browser needs to be robust, reliable and flexible with respect to the
changes in the Opus code. Therefore, there were certain measures taken in
order to improve such reliability and robustness.

In order for the Indicator Browser to be reliable, all displayed indicators
should be able to be generated and visualized. To achieve this goal, we
created a set of Python unit tests that generate all possible indicator
visualizations under all possible visualization types for a small test
scenario run. The only indicators that are displayed on the interface are
those that passed the aforementioned test, therefore considerably improving
the chances that all the indicators displayed on the Indicator Browser can
in fact be successfully generated and visualized. Currently, these tests
and the set of available indicators are run and updated manually.  However,
these tests will be eventually added to the Opus nightly build and the
available indicator list will be updated automatically.

% LocalWords:  borning UrbanSim
