% $Id: prevwork.tex,v 1.13 2006/09/07 14:37:46 yaels Exp $

\section{Value Sensitive Design of Indicators for UrbanSim}
\label{sec:prevwork}

In this section, we summarize prior work by Borning, Friedman, Davis, and
Lin \cite{borning-ecscw-2005} that underlies much of the research presented
here.  The domain of urban planning is both value laden and rife with
long-standing disagreements, including in particular the choice of
indicators to present, and how they are presented and described.  To
approach these value issues in a principled fashion, we rely on the Value
Sensitive Design theory and methodology \cite{friedman-amis-2006}.

Value Sensitive Design is an approach to the design of information systems
that seeks to account for human values in a principled and comprehensive
way throughout the design process.  Key features are its interactional
perspective, tripartite methodology, and emphasis on indirect as well as
direct stakeholders.  Value Sensitive Design is an interactional theory:
people and social systems affect technological development, and
technologies --- such as urban simulation systems --- shape (but do not
rigidly determine) individual behavior and social systems.  Value Sensitive
Design employs a tripartite methodology, consisting of conceptual,
empirical, and technical investigations.  Conceptual investigations
comprise philosophically informed analyses of the central constructs and
issues under investigation.  Empirical investigations focus on the human
response to the technical artifact, and on the larger social context in
which the technology is situated, using quantitative and qualitative
methods from social science research.  Technical investigations focus on
the design and performance of the technology itself.  A third key aspect of
Value Sensitive Design is its focus on both direct and indirect
stakeholders.  The direct stakeholders are the users of the system; the
indirect stakeholders are those who don't use the system directly, but who
are affected by it --- a group often overlooked in other design
methodologies.

For UrbanSim in its current form, the direct stakeholders are the urban
modelers and planners who use UrbanSim and manipulate its
results.  The indirect stakeholders are those who do not use the system
directly, but who are affected by it.  They include for example elected
officials, members of advocacy and business groups, and more generally all
the residents of the region being modeled, as well as residents of nearby
regions.  One of our goals in the UrbanSim project is to open the planning
process and use of models to wider participation and direct use --- in
other words, to move more people from the indirect to the direct
stakeholder category.

Early in our conceptual investigations, we made a sharp distinction between
explicitly supported values (i.e., ones that we explicitly want to support
in the model system) and stakeholder values (i.e., ones that are important
to some but not necessarily all of the stakeholders).  Next, we committed
to several key moral values to support explicitly: fairness and more
specifically freedom from bias \cite{friedman-tois-1996},
representativeness, accountability, and support for a democratic society.
In turn, as part of supporting a democratic society, we decided that the
system should not a priori favor or rule out any given set of stakeholder
values, but instead should allow different stakeholders to articulate the
values that are most important to them, and evaluate the alternatives in
light of these values.  For example, for one stakeholder economic values
might be paramount, while for another environmental values, or for a third,
issues of equity.  These value concerns might lead to particular attention
to different indicators (e.g.\ acres of vacant land available for
development, greenhouse gas emissions, or distributions of wealth and
poverty).  In addition to the explicitly supported values listed above, we
also identified comprehensibility, and subsequently legitimation and
transparency, as key instrumental values to be supported explicitly.
Legitimation in particular developed as a key value --- if some
stakeholders don't perceive the use of UrbanSim as legitimate, they may
never accept its use in the decision-making process, and may disengage from
discussions involving it, reducing the diversity of stakeholders present at
the table and undermining democratic participation.  If stakeholders who do
not see UrbanSim as legitimate nevertheless choose to stay at the table, their
constant questioning of simulation results may detract from discourse about
what really matters in the outcome of adopting a course of action.

We were concerned with usability issues as well, in particular with making
information ``ready-to-hand'' \cite{winograd-flores-book-1986}, that is,
easy to access in the course of interacting with UrbanSim, both as an end
in itself and also in support of transparency.

Building on this analysis, we designed and built a series of prototype
tools for browsing through the results from UrbanSim simulations.  The key
design problem we addressed was: how can we create an interaction design
around indicators for UrbanSim that will provide improved functionality,
support stakeholder values, enhance the transparency of the system, and
contribute to the system's legitimation?  The final version presented in
reference \cite{borning-ecscw-2005} included careful technical
documentation of each indicator, a web-based browser for looking through
the available indicators, and a separate Indicator Perspectives framework
that supports different organizations in presenting their perspectives on
which are the important indicators and how they should be interpreted from
a policy perspective.  The Technical Documentation for each indicator, with
an eye toward providing useful, ready-to-hand, comprehensible information
about each indicator, as well as minimizing perceptions of bias, includes
sections such as a concise definition, a more formal specification, a
discussion of how to interpret indicator results, and known limitations.

At the time this work was done, all UrbanSim output was stored in a SQL
database, using the open-source MySQL database system, and indicator
results were defined as SQL queries.  The documentation included the
SQL code to compute the results of the indicator, as well as tests that
could be evaluated to check whether the SQL code was correct.  These tests
were automatically evaluated each time the indicator code was checked into
the source code
repository.  The Technical Documentation was ``live'' in that the SQL
code and tests were extracted directly from the code base each time they
were displayed, guaranteeing that what the user read in the Technical
Documentation was current.  

In our empirical evaluation of the Technical Documentation, we found that
users were able to perform information extraction tasks much more quickly using it than with their
current work practices \cite{borning-ecscw-2005}, which involved using a
disjoint collection of reference material and tools.
Further, users positively evaluated the inclusion of
the SQL code and tests as part of the indicator documentation.  These
features support the values of transparency and legitimation: readers of
the documentation could see exactly what is being computed, and further, could
see the tests that were used for that code.

One significant gap in this work, however, was a graphical interface for
browsing through the results of computing indicator values, as well as for
requesting that additional values of indicators be computed.  Instead, one
had to evaluate a Python script to do this.  It is this gap that the
present work addresses (Sections \ref{sec:interface} -- \ref{sec:evaluation}).

% LocalWords:  borning UrbanSim  interactional SQL MySQL yaels analyses priori
% LocalWords:  UrbanSim's
