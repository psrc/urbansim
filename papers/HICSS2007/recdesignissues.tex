% $Id: recdesignissues.tex,v 1.11 2006/09/08 23:57:16 borning Exp $

\subsection{Recurring Design Issues}

There were some prevalent design issues throughout the interface
evaluation. These issues were discussed extensively and multiple design
alternatives were created in order to address them.

{\bf Providing ready-to-hand information.} This was a key goal,
both to support the value of transparency and also 
to enhance usability.  Its importance
is supported by our earlier empirical work (Section \ref{sec:prevwork}).  In
this context, the Indicator Browser should make all the information easily
available for the user to make an informed selection of the indicator to be
visualized.  In a larger context, interfaces often suffer from lack
of help and documentation in the right place \cite{nielsen-book-1994}.

Through the different stages in the design process, we found the need to
present the user with ready-to-hand information about UrbanSim, the
Indicator Browser, indicators and scenarios. The user wanted to know which
indicators had been precomputed, how long did the indicators take to
compute, the status of their request, what other requests had been
requested and the current visualization's selections. Since it was somewhat
difficult to assess how long were the indicators going to take due to the
complicated interaction amongst models during the simulation, we decided to
give the users a worst case estimate.
While some users were satisfied with this information, some still believed
more precision was necessary.

% *** is this right??? (which was that of three years). 

An example of our efforts to provide ready-to-hand information can be seen
in the results page found in Figure \ref{fig:results}, where the users can
see the status, details, results, computation and error logs for all the
requests.  We also provided links to the scenario run configuration details
and the indicator's Technical Documentation or code (in case the
documentation was not available). All of this information was highly
appreciated by the users at every step of the design process.

{\bf Sequence.} As previously noted,
the Indicator Browser was originally planned as a single-page
interface. However, this was not possible due to the
dependencies between selections. The indicator availability is dependent on
the geography one wants to visualize, since there are some indicators that
make more sense when visualized at certain aggregation levels. For
example, there are special Traffic Analysis Zones to study
transportation-related indicators.  Additionally, the visualization type
depends on which geography is selected, since not all visualizations are
possible for all geographies.  (For example, it is not practical to create
a table for gridcell-level indicators for most applications, since the
table would be too large.  The PSRC application, for instance, has
approximately 800,000 gridcells.)

The order of the choices presented to the user was designed based on a
combination of technical and empirical investigations.  For example, the
tests with paper prototypes indicated that users would prefer to select the
indicator before selecting details about the years, visualization type, and
aggregation levels (geography).  However, once we moved to the
implementation phase, we realized that constructing an Indicator required
knowledge about the selected years and geography, so in the implementation
the Indicator selection page was placed after these two selections.

{\bf Versatility vs.\ simplicity of design.}      
The Python script, which is the current way of generating indicators, is
very flexible and allows the user to create various indicators in addition to
the ones that are available through the existing
Opus attributes. In addition, users
who feel comfortable manipulating code can access UrbanSim's code directly
in order to manipulate the look and feel of the indicator
visualizations. Therefore, the design team had to come up with a design
that allowed for the most flexibility while maintaining its
usability. Consequently, there were several design tradeoffs that limited
the functionality of the interface in order to maximize its usability. One
example being that we only allow users to compare alternative scenarios to
the baseline scenario, rather than allowing them to compare any two
available scenarios.

{\bf Repetition reduction.} The interface is designed as a 
sequence of pages that present users with different 
alternatives in the form of radio buttons or check-boxes, in addition to
giving the users the option to provide a title for the request and an
email where to send the results. This meant that for every request they
had to go through the same set of pages, even if they only wanted to change
one or two alternatives. This proved to be very repetitive and
inefficient, especially since urban planners and modelers usually analyze
indicators starting with general information and then requesting more
specialized information where they see fit. For example, they might
initially want to see population maps at a county level and later request
the actual numbers (or a table) at a district level for a particular year
that looked interesting. Therefore, we decided to create a
feature called \emph{Make request like this}, which, when clicked, opens a new
window with a pre-filled request that the users can modify. This feature
was highly valued by the interface evaluators.

There were other shortcuts provided to reduce the amount of input
repetition, such as an \emph{All years} option in the years page that
automatically checked all the available years, a \emph{Results} page that
allowed users to browse through other users requests and retrieve their
visualizations. Finally, we also created a \emph{Pre-generated Indicators} page
for the users to have immediate access to a list of pre-generated
indicators after the scenario selection.

% LocalWords:  borning VSD UrbanSim gridcell gridcells UrbanSim's pre PSRC
% LocalWords:  tradeoffs
