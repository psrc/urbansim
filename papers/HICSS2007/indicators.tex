% $Id: indicators.tex,v 1.3 2006/09/06 19:09:21 borning Exp $

\section{Indicators}
\label{sec:indicators}

As used in the planning literature, an indicator is a variable that conveys
information on the condition or trend of one or more attributes of the
system considered \cite{gallopin-1997,hart-book-1999}.  The indicator will
then have a specific value at a given time.  Indicators form the principal
means for presenting information from an UrbanSim simulation to the users.
Some examples of UrbanSim indicators are population, number of jobs, land
price per acre, vehicle miles traveled per year, or greenhouse gas
emissions from transportation.  Generally we can produce values for these
indicators for the region as a whole, and also for different subregions (for
example, the population of the region, or of a particular county, city, or
neighborhood).  We can also find values for the indicator for each of the
simulated years, and for different scenarios being investigated.  The
indicator values are then displayed as tables, charts, or maps.  For
example, we might track the greenhouse gas emissions for the region for
each simulated year between 2005 and 2030, under two different scenarios.
The greenhouse gas emissions for the entire region 
would best be displayed as a table
or graph, while a spatially distributed indicator, such as population
density, would typically be displayed as a choropleth map (a map that
portrays different population densities as different shades, e.g.\ darker
blue for higher densities).  Figure \ref{fig:pregenerated} shows a variety
of such indicator visualizations.

One important use of indicators is in policy analysis.  For example,
suppose that we are interested in fostering compact, walkable, more densely
populated neighborhoods within the urban area, and curbing low-density,
auto-oriented development.  In the urban planning literature, population
density is regarded as one of the key indicators of the character of
development, e.g.\ dense urban, low-density suburban, rural, etc.  We can
then use UrbanSim to predict population density in different parts of the
region, under alternate scenarios, and show the results as choropleth maps.
In addition, modelers use UrbanSim indicators diagnostically, to learn
about the system's internal operation, to help assess whether it is
operating correctly, and to debug problems.  In the work reported here, we
are concerned with both evaluative and diagnostic uses.

% LocalWords:  borning UrbanSim PSRC choropleth subregions

