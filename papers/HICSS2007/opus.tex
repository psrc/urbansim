% $Id: opus.tex,v 1.9 2006/09/07 14:37:46 yaels Exp $

\section{Indicators in Opus, the Open Platform for Urban Simulation}
\label{sec:opus}

In collaboration with an international group of urban modelers, the
UrbanSim team recently developed a new open-source platform, named Opus,
the Open Platform for Urban Simulation \cite{waddell-opus-2005}.  Opus is
written in Python, and makes heavy use of efficient C++ array and matrix
manipulation libraries, such as numarray.  The latest version of UrbanSim,
UrbanSim 4, is now written using the Opus framework.  The Opus/UrbanSim 4
work was carried out after the work on
value sensitive design of indicators had been completed, and while it had
no major impact on the results of the conceptual investigations, it did
result in major changes in the way that indicators were computed.  The SQL
queries that were used in UrbanSim 3 greatly enhanced transparency, but at
a cost in efficiency.  In particular, computing a set of indicators could
take longer than the run time for the simulation itself.  In Opus and
UrbanSim 4, indicators can now be defined using Opus variable definitions
in addition to SQL queries.  This is a considerably less transparent
format, but indicator values can now be computed in minutes rather than
hours.  As discussed in reference \cite{friedman-amis-2006}, there is a 
complex relationship between supporting human values and usability. 
Sometimes these goals align, sometimes they are independent, and sometimes
in conflict.  This situation is an example of a conflict: between supporting
the value of transparency and enhancing usability (where the more transparent
SQL indicators in some cases were so inefficient as to be nearly unusable).

Opus represents data using instances of the class DataSet, which in turn
uses efficient numarray storage representations.  Example datasets include
the set of all households in a simulation, and the set of all gridcells
(the basic geographic unit in our simulation).  Households and gridcells in
turn have attributes, such as each household's income, or the number of
people whose residence is in a given gridcell.  There are two types of
attributes: primary and computed.  Primary attributes represent raw data
(e.g.\ the gridcell in which a given household is placed, or the number of
people in a household).  Computed attributes are the result of computations
or aggregations of one or more other attributes (e.g.\ the population in a
gridcell, which is computed by summing the number of people in each
household placed in that gridcell).  Furthermore, attributes that aggregate
multiple gridcells at different geography levels, such as zones or
counties, are considered to be computed as well.  Computed attributes are
generated lazily --- that is, their values are computed only when needed,
and then cached in case they are requested again.  Using this scheme,
indicators can be simply represented as particular kinds of Opus
attributes.  (Note that in general, the running simulation won't compute
these particular attributes --- in other words, the indicator values won't
be available until they are explicitly requested --- unless some other part
of the system happens to need them.)

As with indicators in UrbanSim 3, indicator values in UrbanSim 4 can be
computed and visualized using scripts.  We currently typically use scripts
that generate an entire batch of indicator visualizations based on Opus
attributes at one time.  The script uses a Python dictionary to specify
declaratively which indicators to visualize, the scenario, year(s), level
of geographic aggregation, and visualization type.  The script also allows
the user to aggregate indicators from smaller geographies to larger ones
and vice versa.  For example, one can aggregate the values of an indicator
at a gridcell level onto the entire region, or dis-aggregate an indicator
that gets computed for the counties onto the respective districts within
each county. Finally, the script gives the users the flexibility to alter
the visualization code to, for example, fix the color range displayed on
the maps or determine the desired range on charts.  The currently available
visualization types are charts, graphs, and maps using the matplotlib
library (\url{http://matplotlib.sourceforge.net/}), maps using OpenEV,
latex tables, and comma-separated-value files (which can then be read by
any spreadsheet program).

Even though this interface provides considerable flexibility and power over
the visualization outcome, some modelers and planners are not 
comfortable with writing and running code. They find that this interface 
produces unexpected side effects (such as launching sub-processes from the
script that open unexpected command windows). In addition, the modelers
still need to reformat and rerun the script, synchronize their code with
the source code repository, set the right environment variables, and
utilize various applications in order to create the
visualizations. UrbanSim developers, on the other hand, feel very
comfortable manipulating code, but can only access the visualizations from a
file server located at the office instead of being able to access them
through a web-browser at any location, and they need to edit and rerun
the script even if all they want is a single additional 
indicator visualization.

% LocalWords:  borning UrbanSim DataSets gridcells gridcell SQL EMPAL yaels
% LocalWords:  PSRC FAZ DataSet  refactor numarray datasets declaratively versa
% LocalWords:  matplotlib OpenEV
