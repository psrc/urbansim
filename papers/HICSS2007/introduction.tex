% $Id: introduction.tex,v 1.11 2006/09/08 23:57:16 borning Exp $

\section{Introduction}
\label{sec:introduction}

The process of planning and constructing a new light rail system or
freeway, setting an urban growth boundary, changing tax policy, or
modifying zoning and land use plans is often politically charged.  Our goal
in the UrbanSim project is to provide tools for stakeholders to be able to
consider different scenarios, and then to evaluate these scenarios by
modeling the resulting patterns of urban growth and redevelopment, of
transportation usage, and of environmental impacts, over periods of 20--30
years \cite{waddell-sscr-2004,waddell-ulfarsson-2004}.  

UrbanSim, combined with transportation models and macroeconomic inputs,
performs simulations of the interactions among urban development,
transportation, land use, and environmental impacts. It consists of a set
of interacting component models that simulate different actors or processes
within the urban environment.  For example, one component model, the
Residential Location Choice Model, simulates the decision-making process of
a household seeking a new place to live.  Other component models include
the Land Price Model (which captures key aspects of the real estate
market), the Demographic Transition Model (which simulates changes in the
overall demographics of the region, such as births, deaths, and moves in
and out of the region), and a Real Estate Development Model (which
simulates the actions of real estate developers as they build or renovate
housing, office space, and so forth).  An external travel model models
trips, mode choice (whether the trip is by automobile, bus, bicycle, etc),
and congestion.  Typically, each component model is run on a simulated
annual basis. To apply UrbanSim in a particular area, users 
prepare a ``baseyear database'' that consists of the starting state of the
region in a particular year.  They then prepare alternate scenarios ---
packages of transportation investments, and policy and land use changes ---
and then simulate the effects of these scenarios over periods of 20--30
years.  These scenarios are either realistic alternatives that are
under active consideration, or diagnostic test scenarios to
help evaluate how well UrbanSim performs on a particular region. 

UrbanSim's output is presented using indicators, which are variables that
convey information on significant aspects of the simulation results. This
paper describes the Indicator Browser, a web-based interface that
facilitates indicator generation and visualization. We discuss the
interface design and implementation process, in particular the integration
of Value Sensitive Design \cite{friedman-amis-2006} into user-centered
interface design practices and a discussion of our recurring design issues
and implemented solutions, as an e-government infrastructure case study.

The UrbanSim group has worked with regional government agencies in applying
the system in the urban areas around Detroit, Eugene, Honolulu, Houston,
Puget Sound (Seattle and surrounding cities), and Salt Lake City.  There
have also been research and pilot applications in Amsterdam, Paris,
Phoenix, Tel Aviv, and Zurich.  The Puget Sound application of UrbanSim,
and our work with the regional planning agency, the Puget Sound Regional
Council (PSRC), plays a central role in the system evaluation reported
here.

The rest of the paper is organized as follows.  Section
\ref{sec:indicators} describes how indicators are used to present key
results from UrbanSim simulations, while Section \ref{sec:prevwork}
summarizes prior work on indicators for UrbanSim, informed by the 
Value Sensitive Design theory and methodology, which underlies much of the
research presented here.  Section \ref{sec:opus} discusses some issues
around the latest implementation of UrbanSim and implications for the
Indicator Browser.  Sections \ref{sec:interface}, \ref{sec:system-design},
and \ref{sec:evaluation} describe the interface design, system architecture
and implementation, and evaluation of the Indicator Browser.  Section
\ref{sec:conclusion} concludes.

% LocalWords:  borning UrbanSim PSRC Aviv baseyear UrbanSim's

