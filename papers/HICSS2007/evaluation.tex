% $Id: evaluation.tex,v 1.12 2006/09/08 23:57:16 borning Exp $

\section{Evaluation}
\label{sec:evaluation}

The Indicator Browser was evaluated by four urban planners and two UrbanSim
developers in April 2006. The evaluation consisted of an observation of the
use of the Indicator Browser in a natural setting, followed by an informal
interview guided by five general questions.  The users were asked to
generate and visualize the indicators that they usually work with using our
interface. If they were unclear about what indicators to generate, the
evaluator prompted them to generate the indicators that were discussed on
one of the PSRC-UrbanSim meetings that was held a few days before the
evaluation.  The users were asked how often and how they usually generate
and visualize indicator visualizations. They were asked to compare the
Indicator Browser with the tools they currently use. They were also asked
whether and how they would use the Indicator Browser, and what changes
needed to be made for the interface to be useful for them. Finally,
they were asked to consider whether this interface could be used by a
different set of users and if so, who would that group be. We were
interested in gauging how commonplace it was for the users to generate or
visualize indicators.  Since there are two different types of users: urban
modelers and planners on the one hand, and UrbanSim developers on the other,
each group might have a
different set of needs and uses for the Indicator Browser. Furthermore, we
wanted to prioritize the usability issues found during the evaluation and
to have an understanding of possible future steps.

\subsection{Usability and Value Issues}

% *** COULD SHORTEN OR EVEN SKIP THIS SECTION ***

The evaluations showed that there are still some usability issues to fix
before deploying the interface.  Some of the interface changes made after
the second round of usability testing, mainly due to internal design
discussions, had not been evaluated before and showed the need for
refinement. For example, we added a feature that allowed users to see
pre-generated indicators immediately after the scenario selection (Figure
\ref{fig:pregenerated}).  If the indicator that the user wanted wasn't on
the list there was a \emph{Choose a different indicator} button that allowed
them to continue making their request. This was a very useful feature that
the users liked.  The users showed interest once again in having 
more ready-to-hand information, such as having the scenario's name 
for each of the requests on the results page.

\subsection{Post-task Interview}

After the users interacted with the interface for about half an hour the
evaluator asked them the following five questions:

\begin{itemize}
\item How often and how do you currently generate indicator visualizations?
\item How does the Indicator Browser compare to the tools you currently use
  to generate indicator visualizations?
\item Would you use the Indicator Browser? If so, how?
\item What would need to be fixed for you to use the Indicator Browser?
\item Would you recommend this interface to someone else? If so, who?
\end{itemize}

The interviews revealed that each of the user groups generate indicators
with different frequency and that they have different perspectives on how
useful currently used tools are based on generation frequency and level of
programming expertise.  Their perception of the utility of the currently
used tools had an impact in their perception of utility of the Indicator
Browser.

One of the users is a general urban planner who usually does not generate
indicators himself. The indicators are generated for him and he can access
them through a file server.  He then visualizes them using the
FastStone Image Viewer tool (\url{http://www.faststone.org}), which allows
contiguous visualization of up to four images, as well as synchronized
image zooming and panning. Consequently, he did not find the indicator
generation mechanism extremely helpful for his common work tasks. He
mentioned he usually looks at the same set of indicators, and so he would
use this interface to see the pre-generated indicators and would most
likely not create some of his own.  However, he suggested that the
interface would be a useful way for the general public to access this
information.

Two of our interface evaluators are urban modelers and generate indicators
on a regular basis. Although they know how to create macros and how to
program, they felt much more comfortable generating indicators through a
web application. They found the Python script (described in Section
\ref{sec:opus}) to have a high set-up cost. The Python script requires them
to check out the correct projects from the source code
repository, set environment
variables, modify the Python script, and run it using multiple
applications.  Since several sub-processes are launched within the script,
the users became confused whenever windows popped on their screen which
they didn't know if they could close or what they were meant to do. 
This last user group found the Indicator
Browser very useful for indicator generation due to its low set-up cost,
aggregation of functions into a single web-based application, and the
elimination of the need to program. Their primary concern was with the
pre-generated indicators section of the interface, which they found
confusing, and mentioned they would
use the interface after this issue was resolved.

Two of our evaluators are software engineers and developers with the
UrbanSim project. They are very comfortable using and modifying the Python
script and enjoy the flexibility that it affords. The Python script allows
them to force color ramps on maps and create indicator difference,
aggregations and dis-aggregations of data across multiple levels of
geography. However, they do not have direct access to the file server at
home. One of the evaluators mentioned he had to use Microsoft's Remote
Desktop application to access the generated indicators from the UrbanSim
file servers. Therefore, he found the Indicator Browser to be useful for
accessing the indicators using a web browser from any
location. The second evaluator liked the organization of information. He
liked that he could see which scenarios were available and that he could
access the indicators by selecting the scenario from a list on a webpage
rather than having to find out what the scenario directory was through
database tables and access it through the file server.  He mentioned that
once he analyzes the outcome of the big batch of indicators generated
through the script he would use the Indicator Browser to create additional
indicators instead of editing and rerunning the script. However, the
other software engineer mentioned that he would not use the Indicator
Browser since he enjoys the Python script functionality and does not like
using multiple applications to complete a single task.

Some users mentioned they would recommend this interface to the general
public and activist groups for use in urban planning courses, engaged
citizens workshops, technical advisory committees and general regional
policy inquiries.

All users found emailing the results or accessing the results through a URL
to be very useful. Most of them mentioned they would use the Indicator
Browser to create additional indicators to get more detailed information
after analyzing the pre-generated indicators.


% LocalWords:  borning UrbanSim PSRC  pre FastStone EMPAL FAZ Fortran gauging
% LocalWords:  webpage
