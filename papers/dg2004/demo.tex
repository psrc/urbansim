%% $Id: demo.tex,v 1.1 2004/04/16 00:33:15 borning Exp $
\documentclass[11pt]{article} 
\usepackage{times,url,fullpage}

\newcommand{\tight}{\itemsep 0pt}

\begin{document}

\title{A Technical Modeler's Interface for UrbanSim, a System for
Integrated Land Use, Transportation, and Environmental Modeling}

\author{Alan Borning and Paul Waddell\\
Center for Urban Simulation and Policy Analysis \\
University of Washington, Box 353055 \\
Seattle, Washington 98195 \\
borning@cs.washington.edu, pwaddell@u.washington.edu}

% I should actually put in my CSE address instead, probably, since I don't
% think mail delivery to CUSPA is completely worked out
\date{}

\maketitle

\section{Introduction}
\label{introduction}

Patterns of land use and available transportation systems play a critical
role in determining the economic vitality, livability, and sustainability
of urban areas.  Transportation interacts strongly with land use.  For
example, automobile-oriented development induces demand for more roads and
parking (which in turn induces more automobile-oriented development), while
compact, pedestrian-friendly urban environments can induce more walking and
demand for transit.  Both land use and transportation have strong
environmental effects, in particular on emissions, resource consumption,
and conversion of rural to suburban or urban land.

The process of planning and constructing a new light rail system or freeway,
setting an urban growth boundary, changing tax policy, or modifying zoning
and land use plans is often politically charged.  Strong technical support
can play a critical role in fostering informed civic deliberation and
debate on these issues, as well as on broader issues such as sustainable,
livable cities, economic vitality, social equity, and environmental
preservation.  We want urban planners and stakeholders to be able to
consider different scenarios --- packages of possible policies and
investments --- and then, based on these alternatives, model the resulting
patterns of urban growth and redevelopment, of transportation usage, and of
resource consumption and other environmental impacts, over periods of
twenty or more years.

UrbanSim \cite{waddell-sscr-2004,waddell-ulfarsson-2004}
is a simulation system that is intended to provide such technical support.
It performs simulations of urban development, including transportation,
land use, environmental impacts, and their interactions.  It is written in
Java, and is distributed as Open Source software.  (Source code,
as well as papers and reports, are available from
\mbox{\url{www.urbansim.org}}.)  UrbanSim consists of a set of interacting
component models that simulate different actors or processes within the
urban environment.  For example, the Residential Location Choice model
simulates the process of household location --- of a household deciding
whether to rent or buy a dwelling, what kind (detached house, townhouse,
apartment, etc.), and in what part of the city.  Another model is the
Developer Model, which simulates the activities of real estate developers
as they decide whether to develop new housing or commercial space, or
redevelop existing space.

To date, UrbanSim has been applied in a number of metropolitan regions,
including Eugene/Springfield, Oregon; Honolulu, Hawaii; Houston, Texas; and
Salt Lake City, Utah.  Application to the Puget Sound region (which
includes Seattle, Washington) is substantially completed.  As part of the
process of validating the model, we also performed a historical validation
with the Eugene data, starting UrbanSim with the 1980 data, simulating
until 1994, and then comparing with what actually happened.

\section{The Technical Modeler Interface}

UrbanSim is a large and complex application, with significant input data
requirements, as well as requirements for calibration and fitting to a
particular region.  We plan to provide several different interfaces for the
system.  The one that we currently use, and which we will demonstrate, is
designed for technical modelers: domain experts in modeling urban
development, who typically work at regional planning agencies,
consultancies, or academic research projects.  In the future, we will also
provide a web-based interface, designed to be much more accessible to
decision-makers, members of neighborhood and advocacy groups, and the
general interested citizen.  Finally, we are working on an experimental
\emph{streetscapes visualization tool}, which will produce animated street
scenes, driven by the simulation data, and populated with moving
pedestrians and vehicles.  

When completed, the Technical Modeler Interface will support a variety of
functions, including managing the databases of input data, editing database
tables, running and controlling the simulation, selecting and computing
indicators, and displaying the indicator results using tables, graphs, and
maps.  We will also provide a scripting mechanism that will allow
simulations to be controlled and replayed.  We are building the interface
using a carefully selected set of Open Source components, rather than
writing everything from scratch.  These components include: Eclipse, a
universal tool platform (\url{www.eclipse.org}); MySQL, an efficient
relational database (\url{www.mysql.com}), along with the MyCC database
editor; and JUMP, a unified mapping platform
(\url{www.vividsolutions.com/jump}).

In our demonstration at the conference, we will first give an overview of
the project, and describe current activities (including the outcome of the
lawsuit in Salt Lake City, which was in progress when we talked at the
Digital Government Conference in Boston in 2003).  We will then demonstrate
the system in operation for a representative set of modeler tasks,
including browsing through the results of a simulation, using indicators of
household density, employment density, and the like, in both map-based and
tabular format, and at different spatial resolutions.  We'll also show the
facility for replaying the simulation, so that the indicator results will
be rapidly updated for the different simulated years between 2000 and 2020.

\subsection*{Acknowledgments}

We would like to thank each and every member of the UrbanSim research team
for their work on the project.  This research has been funded in part by
grants from the National Science Foundation (EIA-0090832 and EIA-0121326)
and the Federal Highway Administration, and in part by a partnership with
the Puget Sound Regional Council.

\bibliographystyle{plain}
\bibliography{urbansim}

\end{document}

% LocalWords:  Modeler's UrbanSim Borning Waddell CSE CUSPA sustainability sscr
% LocalWords:  stakeholders ulfarsson Puget consultancies streetscapes MySQL
% LocalWords:  MyCC EIA urbansim
