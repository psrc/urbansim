%% $Id: highlights.tex,v 1.1 2004/04/16 00:33:15 borning Exp $
\documentclass[11pt]{article} 
\usepackage{times,url,fullpage}

\newcommand{\tight}{\itemsep 0pt}

\begin{document}

\title{Integrated Land Use, Transportation, and Environmental Simulation:
UrbanSim Project Highlights}

\author{Alan Borning and Paul Waddell\\
Center for Urban Simulation and Policy Analysis \\
University of Washington, Box 353055 \\
Seattle, Washington 98195 \\
borning@cs.washington.edu, pwaddell@u.washington.edu}

% I should actually put in my CSE address instead, probably, since I don't
% think mail delivery to CUSPA is completely worked out
\date{}

\maketitle

\section{Project Overview and Impacts}

The process of planning and constructing a new light rail system or
freeway, setting an urban growth boundary, changing tax policy, or
modifying zoning and land use plans is often politically charged.  Our goal
in the UrbanSim project is to provide tools for planners and stakeholders
to be able to consider different scenarios --- packages of possible
policies and investments --- and then, based on these alternatives, model
the resulting patterns of urban growth and redevelopment, of transportation
usage, and of resource consumption and other environmental impacts, over
periods of twenty or more years.  UrbanSim
\cite{noth-ceus-2003,waddell-nse-2003,waddell-ulfarsson-2004}
performs simulations of urban development, including transportation, land
use, environmental impacts, and their interactions.  It consists of a set
of interacting component models that simulate different actors or processes
within the urban environment.  The system is written in Java, and is
distributed as Open Source software.  (Source code, as well as papers and
reports, are available from \mbox{\url{www.urbansim.org}}.)

\section{Collaboration Examples and Success Stories}

We have two primary kinds of government collaborators.  The immediate users
of UrbanSim (and systems like it) are Metropolitan Planning Organizations:
regional agencies that are charged with transportation planning in urban
areas.  In the past year, we have been collaborating most closely with two
such organizations: the Wasatch Front Regional Council (the Metropolitan
Planning Organization for the Salt Lake City, Utah, region); and the Puget
Sound Regional Council (Seattle, Washington, and other cities and suburbs
in the region).  In the Salt Lake City region, UrbanSim played a central
role in a lawsuit over a major highway project (the ``Legacy Parkway'').
The suit was settled out of court, with a central provision being an
agreement by all parties to test UrbanSim for operational use in the
region.  An expert Peer Review Panel endorsed continuing the process of
applying UrbanSim, along with providing many thoughtful suggestions for
improvements.  In our own region, we set up a partnership with Puget Sound
Regional Council to apply UrbanSim, with the goal being to use UrbanSim as
the operational model for Puget Sound.  Our first major local use of the
system will be the upcoming update of Vision 2020, the adopted regional
long-range strategy for transportation and growth management.  Working with
our lab, UrbanSim has also been applied in Eugene/Springfield, Oregon;
Honolulu, Hawaii; and Houston, Texas.

Our other primary government collaborators are Federal agencies.  We have
been fortunate to receive significant funding from the NSF Digital
Government and ITR programs.  The case study in Salt Lake City was funded
in major part by a grant from the Federal Highway Administration, which was
designated as matching funds for the Digital Government award.  We have
also had productive discussions with the Environmental Protection Agency
and the Department of Housing and Urban Development, and hope to set up
collaborations with staff in those agencies.

As noted above, UrbanSim is an open source project, and we make our source
code freely available via the web.  As a result, groups in sites as diverse
as Manila, Paris, Taipei, and Torino have downloaded our system and are
seeking to apply it in their regions.

\section{Challenges and Barriers}

Urban regions are incredibly complex systems.  The resulting technical
challenges include difficulties in identifying or developing the necessary
theoretical basis for modeling such regions, significant computational
challenges raised by massive amounts of data and fine-grained simulation,
and the issue of how to make the results accessible and relevant to the
stakeholders.  A second set of challenges arises from aligning academic and
government objectives.  Academia and NSF wants long-term, risky,
potentially high-payoff research.  Planning agencies, on the other hand,
are fundamentally risk-adverse, with limited funds, and must make
politically charged decisions under heavy public scrutiny.  Another
challenge is the tension between research and deliverables.  On one side,
we want to do research on alternative approaches, to explore different
architectures, to make rapid progress.  On the other, our domain requires
that we deliver a robust, highly reliable, credible system.  A third set of
challenges arises from the difficulties of doing academic interdisciplinary
research.  A forthcoming paper \cite{waddell-sscr-2004} discusses these
challenges in more detail.

\subsection*{Acknowledgments}

We would like to thank each and every member of the UrbanSim research team
for their work on the project.  Thanks also to the NSF Digital Government
program, which has provided a solid home for this work within NSF.  In
particular, the Digital Government program has supported the kind of
mixture of basic and applied, and highly interdisciplinary, research that
is needed for this domain.

This research has been funded in part by grants from the National Science
Foundation (EIA-0090832 and EIA-0121326) and the Federal Highway
Administration, and in part by a partnership with the Puget Sound Regional
Council.

\bibliographystyle{plain}
\bibliography{urbansim}

\end{document}

% LocalWords:  Modeler's UrbanSim Borning Waddell CSE CUSPA sustainability sscr
% LocalWords:  stakeholders ulfarsson Puget consultancies streetscapes MySQL
% LocalWords:  MyCC EIA urbansim Wasatch ITR Torino ceus japa waddell nse
