\section{User Scenarios}

In this section we describe several possible entry points to the AZ-SMART system. This is an attempt to identify the different uses of the system and to identify functionality that will need to be developed.  Each of the following subsections provides a brief narrative and also a list of tools that could be utilized.

\subsection{Data Preparation: Diagnostics, Exploration, Refinement}

An AZ-SMART user would be gathering data, exploring data, and if necessary modifying data.  Data could reside in a file structure on disk, or in a database, but it would be expected that data would eventually make its way into a geodatabase.  One could imagine utilizing the following tools in this process (see Appendix A for a description of each tool):

\begin{itemize}
	\item Descriptive Statistics
	\item Variable Distribution
	\item Data Diagnostic Reports
	\item Impute average/most-frequent value
	\item Impute Random Value
	\item Impute via Model
\end{itemize}

\subsection{Model Development: Specification, Estimation, Diagnosis}



\subsection{Scenario Creation}



\subsection{Run Management}



\subsection{Indicators: Production, Visualization, Reporting}

An OPUS indicator could be defined as any data in the OPUS system that a user may want to visualize or examine, whether it be prior to, during, or after a model run, or even in other stages such as model estimation.  A few examples of indicators are population densities (overall or by socioeconomic class), travel times, job densities (overall or by sector), and amount of developable land.  One may also want to examine other indicators such as probability of development.

\begin{itemize}
	\item Generate Spatial Indicator Map
	\item Generate Indicator Graph/Chart
	\item Generate Indicator Report
\end{itemize}

\subsection{Run Refinement}
