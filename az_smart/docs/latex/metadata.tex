\section{Metadata}

In this section we develop an approach to managing metadata throughout the AZ-SMART system.  
We envision three places where one would want to track metadata.  The first place is in a pre-model 
data processing step where the user is bringing raw input data into the system to build a baseyear 
database.  It would be useful to track data on geoprocessing steps, SQL queries that were run, and
perhaps OPUS tools that were used in the creation of the baseyear database.  It is likely that this
would not be done very often.  The second place for metadata tracking would be Recording the details
of the simulation run (e.g. models, specifications, configurations, etc.).  Lastly, we envision
recording metadata for any post-model processing that was done.  For instance, any refinement to
model results or indicators computed would be candidates for recording metadata about.

There are several built-in tools in ArcGIS to create and manage metadata.  We anticipate that AZ-SMART
users will need to define the amount and exact nature of the metadata they wish to document for each of
the input data sources.  The system should be able to preserve geoprocessing metadata and pass forward
any initial metadata concerning inputs.  Similarly, we anticipate a need to generate and manage
metadata for models and scenarios, and to be able to couple these with GIS and geoprocessing
metadata.  A run of the model system should store the metadata that describes the run configuration
and all data inputs to it.  Furthermore, this metadata should be stored in a way that can be queried
later as documentation of a run and its inputs.  By extension, indicators and other postprocessing
should also generate metadata and this should be available to the end user.  In short, we anticipate
three types of metadata:

\begin{itemize}
\item Input Data
\item Model and Run Configuration
\item Indicator and Postprocessing
\end{itemize}


%NOTES TO SELF:  says stuff about the built in geoprocessing metadata, built in FGDC metadata recording in ArcMap/Catalog, a metadata database (include a data model), OPUS run configurations, etc.