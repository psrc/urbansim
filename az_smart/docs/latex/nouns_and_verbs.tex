\section{Nouns and Verbs}

Following are lists of the nouns and verbs identified by reading Appendix G.  (We are still assembling this.)  We expect that each of these nouns will correspond to a class in our object-oriented system, so understanding the nouns and verbs is important for understanding what we are building.  In general, a noun corresponds to a class, and a verb corresponds to a method of that class.  Classes generally have additional methods and properties for internal use in the system.

\subsection{Nouns from Appendix G}

\begin{itemize}
  \item Meta-data.  Every noun has meta-data that covers its who, what, when, where, and why.  In addition, a user may enter arbitrary key, value pairs?
  \item Project.  A project defines the geographical scope, set of issues of concern, time-frame desired, etc. for a specific investigation into some set of issues.
  \item Scenario.  A scenario is a particular configuration of input data, assumptions, and models to run to test a particular alternative future.  Every project will eventually have at least one scenario.  You can only 'run' scenarios.
  \item Scenario run.  Information about the running of a scenario.  This includes meta-data and simulation results.  Simulation results may be viewed by indicators.
  \item Indicator definition.  Specification of how to compute a particular indicator.  
  \begin{itemize}
	  \item Map indicator definition.
	  \item Table indicator definition.
	  \item Chart indicator definition.
	  \item Report indicator definition.  This may consist of meta-data as well as a collection of other indicators (maps, tables, charts, etc.).
  \end{itemize}
  \item Indicator result.  Result of running an indicator or a scenario run.  Synonym: prediction?
  \begin{itemize}
	  \item Map indicator results.
	  \item Table indicator results.
	  \item Chart indicator results.
	  \item Report indicator results.  
  \end{itemize}
  \item Indicator set.  Multiple indicators.  Allows multiple indicators to be operated on as a unit (e.g. create all of these).
  \item Data-flow diagram.  This is a ``model'' in ModelBuilder.  It is a visual representation of a data flow from a set of data sources, through a set of actions, to a set of data outputs.  Includes conditionals and loops.
  \item Scenario subset.  What is this?
  \item Model. 
  \item Development.
  \item Custom procedure.  For instance, a script, or a data-flow diagram.
  \item Tool.  A software component that has a user interface.
  \item Software component.
  \item Data input.
  \item Data output.
  \item \ldots
\end{itemize}

\subsection{Verbs from Appendix G}

\begin{itemize}
  \item New.  Synonym: create.
  \item Save.
  \item Edit.
  \item Open/View.
  \item Delete.
  \item Copy.
  \item Run.  Synonym: project.
  \item Manage.  What does this mean?
  \item Analyze.
  \item \ldots
\end{itemize}

\subsection{Combining Nouns and Verbs}

These nouns and verbs could suggest the following menu items or tools that correspond to features or feature categories:

\begin{itemize}
  \item Project
  \begin{itemize}
	  \item New project.
	  \item Open project.
	  \item Save project.
	  \item Modify project.
	  \item Copy project.
	  \item Delete project.
	\end{itemize}
	
  \item Scenario 
  \begin{itemize}
	  \item New scenario.  Linked to a particular project.
	  \item Open scenario.
	  \item Save scenario.
	  \item Modify scenario.
	  \item Copy scenario.
	  \item Run scenario.  Produces a scenario run.
	  \item Delete scenario.
	\end{itemize}
	
  \item Scenario Run 
  \begin{itemize}
	  \item Open scenario run (read only?).
	  \item Delete all or part of a scenario run.  For instance, to prepare to re-run starting in 2020.
	  \item Copy all or part of a scenario run.  For instance, to send to a colleague.
	\end{itemize}
	
  \item Indicator Definition 
  \begin{itemize}
	  \item New indicator definition.  Defines how to compute an indicator results.
	  \item Open indicator definition.
	  \item Save indicator definition.
	  \item Delete indicator definition.
	\end{itemize}
	
  \item Indicator Results 
  \begin{itemize}
	  \item New indicator results.  Compute an indicator results from an indicator definition.
	  \item Open/View indicator results.
	  \item \ldots
	\end{itemize}
	
  \item Indicator Set 
  \begin{itemize}
	  \item Create an indicator set.
	  \item View/edit/save an indicator set.
  \end{itemize}
  
  \item Data-flow Diagram.  
  \begin{itemize}
    \item New data-flow diagram.
    \item \ldots
    \item Edit component from diagram.  A componet may be a node or an edge.  For instance, right-click on a data-source to set its properties.
    \item Run data-flow diagram.  
    \item Validate data-flow diagram.
    \item Step-over data-flow diagram.  Runs next step in diagram.
    \item Step-into current node.  During simulation.
    \item Set breakpoint.  Execution will stop just before executing the component that has the breakpoint.
    \item Open selected node.  Opens editor for the node, which may be a data-flow diagram itself.
    \item Open selected edge.  Opens property editor for the edge.
    \item Select node.  To edit, open, move, delete, etc.
  \end{itemize}
\end{itemize}
