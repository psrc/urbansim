\section{System Security}
There may be situations in which there is a need to restrict access
to data or functions to different users and/or machines.  We propose
to exploit existing infrastructure wherever possible to address
security requirements.  Several security points are available using
existing system functionality:


\begin{itemize}
\item Database Access: Access to the geodatabase may be protected
at the user or group level, and can be applied to databases or
possibly to the table level, for managing read and write access
to data.  This would provide a lightweight mechanism for managing
user access, for example by restricting which users or groups
have access to a database containing the database containing the
core data for the AZ-SMART application.  More fine-grained control
would also be possibe, for example protecting employment data from
read access outside a set of users within the agency that have
been cleared for this access.

\item Login Access: Each machine that has AZ-SMART installed will
have login access managament, which can be used as a simple but
effective way to prevent unauthorized access to the AZ-SMART system.

\item Machine Access: A machine may be enabled for access by setting
environment variables that can store username and password information,
and this can be loaded by the az-smart system as needed for security
enforcement.  A generic az-smart user could be defined, or a group
account, or an individual user account and a password assigned to
each.  These usernames and passwords could be stored in an
az-smart-security database that only an AZ-SMART Administrator is
enabled to access and maintain.

\end{itemize}
