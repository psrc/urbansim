\subsection{Software Components}

In this section we describe the principal software components and organization of the AZ-SMART system.  AZ-SMART will include three modules (Project Manager, Data Manager, and Tool Manager) based on the AZ-SMART RFQ Appendix G. Each module's organization and implementation are described in turn in the following subsections.

%Projects are defined as the database and model configuration used to support running a variety of scenarios that would share data and model specifications and parameters.  Scenarios are defined as a set of input data, assumptions, and run configuration parameters used for a specific run of the model system.


\subsubsection{Data Manager}

The Data Manager will focus on the management and visualization of data within AZ-SMART.  The Data Manager be implemented within the ArcGIS framework as a dockable window, coded in VB.NET and ArcObjects, that organizes the necessary data management tools in a 'tree-view' framework.  The Data Manager dockable window would be accessible within both ArcMap and ArcCatalog to provide flexibility in use, although in regular production operation it would probably make the most sense to utilize the Data Manager primarily within ArcCatalog.  See below for example screen captures of this type of dockable window implementation in both ArcCatalog and ArcMap. 

\emph{Jesse: please insert the figures, perhaps scaled and clipped to focus on the dockable window so it is easier to read, and change the reference in the preceding sentence to the Latex reference}

The Data Manager will focus on metadata maintenance and management, management of security levels and users, data archiving, and the movement of data between the OPUS cache and the geodatabase.  To the greatest extent possible, existing ArcCatalog functionality will be utilized to manage the geodatabase, including managing and maintaining geographic datasets and creating and tracking relationships.

\emph{There is a need to define requirements for metadata, and data archiving in the AZ-SMART system.  Once these requirements are identified we can proceed with devising suitable designs to address these requirements.}

\subsubsection{Tool Manager}

The Tool Manager will focus on the management and execution of the wide variety of tools to be developed for AZ-SMART.  The ArcToolbox/ModelBuilder framework will be used for the organization, management, indexing, and execution of tools.  Specifically, a new AZ-SMART toolbox will be populated with the required tools, which will then be organized into toolsets based on common functionality.  To the greatest extent possible new functionality for AZ-SMART will be developed as Python geoprocessing tools within these toolsets, while minimizing custom GUI development within the ArcMap and ArcCatalog applications.  If the required functionality cannot be implemented in Python through the existing ESRI geoprocessing object and/or other add-on Python libraries, CUSPA will develop custom geoprocessing tools using VB.NET and ArcObjects.

\emph{Jesse: insert appropriate screenshot and reference here.  Can you flesh out an initial list of the tools you think will be needed in this in order to accomplish what we need?  I believe we need more detail here and you should be able to do this from the information you have on SAM and your understanding of how functionality could be accessed via ArcTools.}

\subsubsection{Land Use Editor}
When SAM-IM was develped in the ArcView environment, there was considerable functionality not present in the underlying GIS platform that needed to be developed in SAM-IM, such as editing of polygons.  In the development of AZ-SMART on the ArcGIS 9.x platform, there is no need to develop such functionality, since editing tools for spatial data are built-in.  The initial plan for AZ-SMART is to maximize the use of built-in functionality and minimize the amount of custom-code development which will need to be maintained and synchronized with evolving ArcGIS functionality and interfaces.  If it becomes clear that there is some need for customization of the built-in functionality for editing after the AZ-SMART system is put into testing and use, specifications for this could be developed and the task prioritized along with other tasks in the project.

%\emph{Note from Jesse:  we may need other customizations within ArcGIS to accomplish what they want in App G.  For instance, they desribe a land use editor that may require some custom GUI development.  Should we attempt to describe those here? }

\subsubsection{Project Manager}

The Project Manager will focus on managing the model components of the system. The Project Manager will leverage and manage modeling infrastructure embeded in the OPUS system, and will take advantage of the run management capabilities of OPUS such as model specification, estimation, execution, and generation of indicators.  The Project Manager user interface would be accessible from within ArcGIS, and will open a windowed application which we refer to as the OPUS GUI.  The OPUS GUI, developed in Python and using the Envisage tools from Enthought, could also be used outside of ArcGIS, providing flexibility for a range of use cases.  The Project Manager will be the central application where simulation runs are configured, managed, started and stopped, and will interoperate seamlessly with the Data Manager and Tol Manager components of AZ-SMART.

Several of the items listed in the Project Manager requirements in AZ-SMART RFQ Appendix G are very focused on control of model operations: controlling model execution, accessing the status of a model while executing, and accessing execution logs and error logs associated with a model run.  Compared to an approach of using only native ArcGIS GUI tools to code the Project Manager, the approach of coding a native OPUS GUI will make it more feasible to implement these requirements.  It would need to be implemented in a way that could inter-operate transparently with other tools in the Tool Manager and Data Manager components, as noted in the first requirement below. Some initial tests of this approach should be developed early in the project to flesh out this aspect of the user interface.

