\documentclass{article}
\title{Using \matsim\ as a travel model plugin to \urbansim}
\author{Kai Nagel}

\usepackage[letterpaper,top=2cm,bottom=2.5cm]{geometry}
\usepackage{color}
\usepackage{soul}
\usepackage{hyperref}
\usepackage[sf]{titlesec}

\def\matsim{MATSim}
\def\urbansim{UrbanSim}

\parskip0.3\baselineskip
\parindent0pt

\begin{document}
\tolerance10000

\maketitle

\section{A thing to note before you start}

There are two OPUS roots:
\begin{itemize}

\item The OPUS \textbf{source code directory root}.  This is where all
the python code is.

\item The OPUS \textbf{data directory root}.  This is where all the
data is.  This one is where \verb#OPUS_HOME# points to.

\end{itemize}
\emph{IMPORTANT: Make sure at every step that you do not confuse these
two.}  Unfortunately, as of now, there are \verb$opus_matsim$
directories starting from either of these two.\footnote{%
%
Might think about renaming one of them.
%
}

Regarding OPUS-\matsim-integration:
\begin{itemize}

\item \verb$<opus-src-root>/opus_matsim$ contains additions on the
OPUS side (i.e.\ python and related)

\item \verb$OPUS_HOME/opus_matsim$ contains \matsim\ material (i.e.\
Java plus \matsim-specific input files)

\end{itemize}

\section{Getting started}

\begin{enumerate}

\item Download \verb#opus_matsim-yyyy-mm-dd-hhmm.tgz# from

\url{http://fgvsp01.vsp.tu-berlin.de/scratch/opus_matsim/}

\item Move it into \verb#OPUS_HOME#.

\item Go into \verb#OPUS_HOME# and type
\begin{verbatim}
  tar zxvf opus_matsim-yyyy-mm-dd-hhmm.tgz
\end{verbatim}

After that, you should have an \verb$opus_matsim$ directory, with
subdirectories
\begin{verbatim}
  data          [[scenario input data]]
  matsim_config [[matsim config files]]
  bin           [[binaries]]
  classes       [[java classes that go beyond standard matsim]]
  jar           [[*.jar files; lib directory with more *.jar files]]
\end{verbatim}

% \item Download \verb$MATSim_libs_rXXXX.zip$ and
% \verb$MATSIM_rYYYY.jar$ from www.matsim.org, on the ``nightly builds''
% www page (currently: \url{http://matsim.org/files/builds}; some
% background info on \url{http://matsim.org/downloads/nightly}), and
% move it into \verb$OPUS_HOME/opus_matsim/jar/$ .
% 
% \item Go into \verb$OPUS_HOME/opus_matsim/jar$ and type
% \begin{verbatim}
%   unzip MATSim_libs_rXXXX.zip
%   ln -s MATSim_rYYYY.jar MATSim.jar
% \end{verbatim}
% Afterwards, you should have
% \begin{verbatim}
%    .../opus_matsim/jar/MATSim.jar  [[as symbolic link]]
%    .../opus_matsim/jar/MATSim_rYYYY.jar
%    .../opus_matsim/jar/libs/...    [[lots of *.jar files]]
% \end{verbatim}

\item Go into the opus source tree, into \verb$opus_matsim/tests$, and
python-run \verb$tests.py$.

That runs a meaningless test scenario, and should run without errors.
(It actually does not use the java stuff; it just tests the opus side.)

\item Go into the opus source tree, into \verb$opus_matsim/configs$.
You should be able to run \verb$seattle_parcel.xml$ by any of the
normal means (from the GUI, from the command line, or by using
\verb$start_run.py$ in the same directory).

If this runs through, you are ready to go!

\end{enumerate}

\section{Visualizer}

This section works only unter macosx and linux-i586.  linux-amd64
should be straightforward (let me know if you can't figure it out).
win also works in many cases, but I don't know where to put the
dynamic library.

There should now be output in \verb$OPUS_HOME/opus_matsim/output$.  Go
into \verb$ITERS/it.0$.  There should be a file \verb$0.otfvis.mvi$.
Type
\begin{verbatim}
  OPUS_HOME/opus_matsim/bin/matsim-vis-XXXX 0.otfvis.mvi
\end{verbatim}
where \verb$XXXX$ refers to the version that is correct for your OS.
This should bring up the visualizer with the usual play and stop
buttons.

\section{Configuration on the OPUS side}

Look at \verb$seattle_parcel.xml$. The relevant section is in
\verb$travel_model_configuration$.

The first part is
{\tiny
\begin{verbatim}
<models type="selectable_list">
 <opus_matsim.models.get_cache_data_into_matsim choices="Run|Skip" type="model_choice">Skip</opus_matsim.models.get_cache_data_into_matsim>
 <opus_matsim.models.run_travel_model choices="Run|Skip" type="model_choice">Run</opus_matsim.models.run_travel_model>
 <opus_matsim.models.get_matsim_data_into_cache choices="Run|Skip" type="model_choice">Run</opus_matsim.models.get_matsim_data_into_cache>
</models>
\end{verbatim}
}
This should be pretty clear.

Next is
\begin{verbatim}
      <sampling_rate type="float">0.01</sampling_rate>
\end{verbatim}
This denotes the sampling rate on which \matsim\ runs.  0.01 means
only one percent of travellers are considered.  This causes some
peculiarities in terms of realism; but I would still recommend to
leave this as is until it is really clear how things work (in
particular the ``warm start'' capability).

Next comes the general matsim config file:
{\footnotesize
\begin{verbatim}
      <matsim_config_filename type="file">matsim_config/seattle_matsim_0.xml</matsim_config_filename>
\end{verbatim}
}
\emph{Note: The root to this is} \verb$OPUS_HOME$.

Next are the years in which the travel model should be run:
{\footnotesize
\begin{verbatim}
      <years_to_run key_name="year" type="category_with_special_keys">
        <run_description type="dictionary">
          <year type="integer">2001</year>
        </run_description>
        <run_description type="dictionary">
          <year type="integer">2002</year>
          <matsim_config_filename type="file">matsim_config/seattle_matsim_2002.xml</matsim_config_filename>
        </run_description>
        <run_description type="dictionary">
          <year type="integer">2003</year>
        </run_description>
      </years_to_run>
    </travel_model_configuration>
\end{verbatim}
}
Note that you can over-ride the general matsim config file in any given
year.  This will read the special matsim config file \emph{instead} of
the general one.

It seems to me that the above configuration commands can be
\emph{either} into \verb$model_manager$ \emph{or} (somewhere) into
\verb$scenario_manager$.  I only tested the first set-up.


\section{Configuring \matsim}

\matsim\ is configured by modifying the \matsim\ config files, which
are referenced as described above.  This is regular \matsim\ set-up
and described elsewhere (seach in \url{matsim.org}).

Noteworthy are the following items:
\begin{itemize}

\item The root of the \matsim\ configuration is
\verb$OPUS_HOME/opus_matsim$.  I do not know if this is good or not,
but it makes life a bit simpler with a couple of aspects.  

(That is, the \verb$../opus_matsim/...$ is the same as \verb$./...$.
There was a reason why that was needed; need to check ...)

\item \verb$inputNetworkFile$ gives the road network, in \matsim\
format.  This should have come with the \verb$*.tgz$-file.  Note that
any \verb$seattle$ scenario still uses the full PSRC network.

\item \verb$inputPlansFile$ gives the ``relaxed'' plans file from
which \matsim\ starts.  If this is commented out, \matsim\ will
construct its initial plans file purely from OPUS input, and take much
longer to relax.

\item \verb$flowCapacityFactor$ and \verb$storageCapacityFactor$
should correspond to the \verb$sampling_rate$ above.

(For the {\tt seattle} scenario, I have set the flow capacity to only
half of this because there is otherwise not congestion.  This is most
probably due to the fact that the cut-out is was debugging purposes
only, and did not correct the resulting boundary effects (e.g.\ trips
to jobs outside the scenario, or from homes outside the scenario, are
simply dropped).)

\item \verb$lastIteration$ gives the last iteration, and in
consequence the number of iterations.

\end{itemize}

\section{Warm start}

The above uses a pre-existing plans file from which to start the
iterations.

\subsection{Generating a warm start plans file}

The generation of a warm start plans file was in principle already
described above:
\begin{enumerate}

\item Remove the \verb$inputPlansFile$ entry (``cold start''), and set
the number of iterations to a plausible number (say ``100'').

\item Then run the \urbansim-\matsim\ set-up until \matsim\ has
finished.

\item
Then search for the resulting plans file in

\verb$opus_matsim/output/ITERS/it.<lastIt>$, 

and move it into a convenient place (e.g.\
\verb$opus_matsim/data/...$).

\item Finally configure the \verb$inputPlansFile$ so that it uses this
newly generated ``relaxed'' plans file.

\end{enumerate}

\subsection{Warm start runs}

Note that in this situation, \matsim\ will always start from that same
initial plans file.  I.e.\ if you keep calling \matsim\ during an
\urbansim\ run, with a changing population, the \matsim\ initial plans
file will be less and less correct, and you will need more and more
iterations to get \matsim\ back into a relaxed state.

Technically, what it will do is keep all persons that have not changed
(i.e.\ are still there, live at the same home, have the same
employment status, and, if applicable, work at the same place).  All
other persons are reset to a ``zero'' state with respect to \matsim.

See ``hot start'' below for more info.


\section{Hot start}

It would be desirable to have the output of one \matsim\ run for one
\urbansim\ year available as input for the next \matsim\ run in
another \urbansim\ year.  

This is meant by ``hot'' start (as opposed to warm start).

If you really know what you are doing, this can be achieved by the
following:
\begin{enumerate}

\item Make a separate \matsim\ config file for every \urbansim\ run in
which you want to run \matsim.

\item Enter these config file names into the OPUS config file for
every year of the run.

\item In the \matsim\ config file, make sure that for every call of
\matsim\ \emph{except the first one}, the following is used:
{\footnotesize
\begin{verbatim}
<param name="inputPlansFile" value="../opus_matsim/output/output_plans.xml.gz" />
\end{verbatim}
}

\item Use consecutive iteration numbers, e.g.

First \matsim\ call:
\begin{verbatim}
<param name="firstIteration" value="0" />
<param name="lastIteration" value="99" />
\end{verbatim}

Second \matsim\ call:
\begin{verbatim}
<param name="firstIteration" value="100" />
<param name="lastIteration" value="199" />
\end{verbatim}

etc.

\end{enumerate}

It will, however, be relatively difficult to see if this really
worked.  So, for the time being, I would rather consider this a
theoretical possibility than a practical option.

\section{Caveats}

Many.  
\begin{itemize}

\item As usual everything was ripped up again in the last week of the
project, because after my project presentation we found really much
better ways of doing things.  But it was ripped up, and not much
used/tested afterwards.  Hard to say if that was a good thing, but it
is considerably faster now.

\item
\emph{Important:} Liming and I put in a way to tag the agents on the
OPUS side.  \emph{However,} I removed that again.  So please do not
assume that this is still there.

(Reason: It became really difficult to have the same tagged persons
sample from one \urbansim\ run to the next, and this made the warm
start capability close to impossible.  What it does now instead: Go
through the \emph{complete} \urbansim\ population and pick those
persons that are in the pre-existing \matsim\ plans file.)

\item There are peculiarities associated with running \matsim\ with a
1\% sample.  I would say that the advantages (much easier computing)
weigh more heavily than the disadvantages.  But the results are
useful for sketch planning only, not for quantitative analysis.  Feel
free to try 10\% or even 100\% if you know what you are doing, and
have the patience, RAM, and disk space.

\item The feedback from \matsim\ to \urbansim\ is really not that well
established.  Right now, it only overwrites the
\verb$single_vehicle_to_work_travel_cost$ value, and it over-writes
this in a unit that is different from what was used previously.
(Now it is in ``seconds''; previously it was ??, and in spite of the
name as of now it does not look at toll.)

This is not so much due to the fact that there is no better solution,
but due to the fact that the \urbansim\ standard seems to be pretty
weak: The documentation talks about \verb$logsum1$, \verb$logsim2$,
and \verb$logsum3$, but I have not seen them anywhere in the
psrc/seattle examples.

In addition, there seems no default way to use these values:
psrc/seattle has specific python functions that eventually use some of
the travel model output.

\end{itemize}

\section{Notes}

Security exception w/ svn.

Download from svn does not untar

really do ONE package

spanning tree not in tgz file




\end{document}


%%% Local Variables: 
%%% mode: latex
%%% TeX-master: t
%%% End: 
