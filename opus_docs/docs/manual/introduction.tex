% $Id: introduction.tex,v 1.23 2007/06/01 23:49:22 borning Exp $

% Copyright (c) 2005-2007 Center for Urban Simulation and Policy Analysis,
% University of Washington.  Permission is granted to copy, distribute and/or
% modify this document under the terms of the GNU Free Documentation License,
% Version 1.2 or any later version published by the Free Software Foundation;
% with no Invariant Sections, no Front-Cover Texts, and no Back-Cover Texts.
% A copy of the license is included in the section entitled "GNU Free
% Documentation License".

\chapter{Introduction}
\label{chapter:introduction}


Opus is a new platform for urban and regional simulation, designed to
support the development and integration of model
components \index{models!components} for simulating the
effects of major transportation investments and policy changes
on transportation, \index{transportation} land use, \index{land use}
and the environment. \index{environmental impacts}
Opus consists of a collection of packages \index{Python!packages}
written in the Python \pythonindex language. The choice of
Python \pythonindex was based on Python's \pythonindex
balance of characteristics, which provide a good combination of a
productive development (even for users who are not expert programmers), and
good performance \index{performance} due to the availability of Python \pythonindex
libraries \index{Python!libraries} for fast
numeric processing \index{numeric processing} that use C \cindex or C++ \cppindex
to implement computationally intensive
aspects.  The basic functionality of Opus is implemented in a package named \package{opus_core}.

UrbanSim Version 4 \index{UrbanSim!version} is implemented as a set of
Opus packages. (Previous versions \index{UrbanSim!previous versions} were in Java.) \javaindex
The decision to migrate from Java \javaindex to Python \pythonindex
was based on the faster development and computation time available
in the Python \pythonindex environment with numeric libraries, and on the objective of
making models \index{models} in UrbanSim more modular than they had been in the previous
Java \javaindex version, allowing users to rapidly configure models, \index{models} estimate their
parameters, and use them. In addition, Python \pythonindex is turning out to be a more
accessible language for modelers, which addresses another key goal: \index{goals} to
create a system where modelers can write their own models, and understand
the models written by other modelers.

\index{User contributions}

This document covers both the base set of Opus packages and the UrbanSim
system implemented in a set of Opus packages.  As descriptions
become available, other packages besides those related to UrbanSim will be
added.  We expect that over time many users and developers will contribute
packages to the system, making their work available to others in the same
way that they themselves have benefited from others' work.

The document includes both a user guide and reference manual.  Part
\ref{part-overview} is currently just this introduction and a
technical overview of the UrbanSim model system, but will soon also
include a technical overview of Opus. Part \ref{part-user-guide}
describes how to install and use Opus and UrbanSim.  Part
\ref{part-reference-guide} contains reference material on the
details of Opus and UrbanSim. Part \ref{part-developers} provides
information for software developers (i.e., people who will be
writing new Opus packages or documentation).  A short Glossary is
provided in an appendix.

This document will continue to evolve --- questions and open issues are in
italics. \index{document conventions} Please feel free to send your comments, criticisms, and
suggestions to
% ** two versions of email address - avoid a real one for the html pages so that
% we aren't such a spam magnet
%begin{latexonly}
\email{info@urbansim.org}. \index{contact information}
%end{latexonly}
\begin{htmlonly}
{\tt info(at)urbansim.org}. \index{contact information}
\end{htmlonly}

\index{latest documentation}
The latest version (as a pdf file, and as html pages) is linked from
\url{http://www.urbansim.org/}.
% LocalWords:  borning UrbanSim pdf
