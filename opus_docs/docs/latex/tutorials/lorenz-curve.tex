% Copyright (c) 2005-2007 Center for Urban Simulation and Policy Analysis,
% University of Washington.  Permission is granted to copy, distribute and/or
% modify this document under the terms of the GNU Free Documentation License,
% Version 1.2 or any later version published by the Free Software Foundation;
% with no Invariant Sections, no Front-Cover Texts, and no Back-Cover Texts.
% A copy of the license is included in the section entitled "GNU Free
% Documentation License".

\documentclass{howto}
\usepackage[T1]{fontenc}

\begin{document}

\newcommand{\package}[1]{\index{opus packages!#1@\textit{#1} package} \textit{#1}}


\title{How to Generate a Lorenz Curve\\and Gini Coefficient}

%\release{0.1}

\author{Center for Urban Simulation and Policy Analysis\\
University of Washington}

\authoraddress{Daniel J. Evans School of Public Affairs \\
University of Washington \\
Seattle, Washington 98195 \\
USA \\
% Put the real email address only in the pdf version (to avoid including
% a spam magnet in the html version).  A mangled version of the address
% (using "info (at) urbansim.org") is included in the html version by
% a parameter to the latex2html command
%begin{latexonly}
Email: \email{info@urbansim.org} \\
Web Site: \url{www.urbansim.org}
%end{latexonly}
}

%\date{\today}
% replace \today with a real date (\date{April 1, 2000} ...} when this
% manual is ready for release so that reformatting doesn't cause a
% new date to be used.  For now, setting the date to \today during
% early editing makes it easier to handle versions.

\maketitle

%\begin{abstract}
%\noindent This document is a quick tutorial on
%generating a Lorenz Curve and Gini Coefficient for given data.
%The intention is to provide a new user or someone interested in learning
%more about UrbanSim a quick way to generate visual output from the model.
%The tutorial uses a set of simple and generally useful graphical user interfaces, 
%or ``tools'', to simplify the interaction
%\end{abstract}

%\tableofcontents
%\newpage

\section*{Introduction}

This tutorial covers creating an indicator with a Lorenz Curve selected as the
visual output.
 
This is intended to assist new and more experienced users in taking
advantage of the functionality specific to generating a Lorenz Curve
and Gini Coefficient.

\section*{Tutorial Prerequisites}

If you have not already done so, install the necessary software and
data for this tutorial by following the directions at
\url{http://www.urbansim.org/wiki/external/index.php/Nightly:Install_Eugene_Tutorial}. 
This tutorial also assumes that the user is familiar with the OPUS framework and 
generating indicators, or has completed the Eugene tutorial 
linked above.

\section*{Formulas used in computing the Lorenz Curve and Gini Coefficient}

The Lorenz curve can be represented by a function $L(F)$, where $F$ is the horizontal axis, 
and $L$ is the vertical axis.

For a population of size n, with a sequence of values $y_{i}$ $i = 1$ to $n$ that are indexed in non-decreasing order $(y_{i} <= y_{i+1})$ the Lorenz curve is the continuous piecewise linear function connecting the points ($ F_i , L_i $), i = 0 to n, where $F_0 = 0, L_0 = 0$, and for i = 1 to n:

    $F_{i} = i/n $ \\
    $S_{i}= \Sigma_{j=1}^i\  y_{j}$ \\
    $L_{i} = S_i / S_n $ \\

The amount of inequality in two societies, or in two scenarios can be compared
based on their Lorenz curves.  If the curve in one case is farther away from
the line of perfect equality for every value along the horizontal axis, then
that case is considered to have less equality than a case with a curve nearer
to the equality line.

The line of perfect equality in the Lorenz curve is $L(F) = F$ and represents a uniform distribution, or equality.  
In cases where the curve does not lie on the line of perfect equality, every point on the curve represents a 
statement like: "the bottom 20\% of households has 10\% of the total income".


The Gini coefficient is defined as a ratio of the areas on the Lorenz curve diagram. 
It is a measure of the inequality of a distribution.  If the area between the line of 
perfect equality and Lorenz curve is A, and the area under the Lorenz curve is B, then the 
Gini coefficient is $A/(A+B)$. Since $A+B = 0.5$, the Gini coefficient, $G = A/(.5) = 2A = 1-2B$. 
If the Lorenz curve is represented by the function $Y = L(X)$, the value of B can be found with integration and:

    $G = 1 - 2\,\int_0^1 L(X) dX$ 

The higher the Gini coefficient, the greater the inequality.


\section*{Meaningful Inputs}

The types of inputs that are commonly used to generate a Lorenz Curve and 
Gini Coefficient are non-aggregate variables. 

The most common input for the Lorenz curve and Gini coefficient is individual or household
income, because income is an important measure of well-being.  Other variables
could also be the basis for distributional comparisons.  These should be
non-categorical quantities, such as the amount of water consumed, carbon emitted, or 
taxes paid.  At a higher level of aggregation it also makes sense to consider
percentages, such as a fraction of a certain area whose residents are ethnic
minorities, voted Independent, or have incomes below poverty.  Finally, it may be
helpful to consider that the normative implication of a Gini coefficient is that perfect 
equality is preferred.  Certain variables which can be analyzed for their distribution, such as miles of light rail track,
may not necessarily be best when equally spread but rather more valuable when
grouped together in certain places.

The typical unit of analysis for a Gini coefficient is the individual.  UrbanSim allows for the use of the Gini
in assessments of equity across units such as gridcells and zones.  Parcel level
analysis will be supported in the future.  It is valid to measure distributions across 
these units, however the user should keep the units in mind when interpreting the results.

\section*{Computing an Indicator}

UrbanSim ``indicators'' are useful representations of dataset attributes.  
How these dataset attributes are displayed is specified by the user.  
In this example a Lorenz Curve is generated.

\begin{enumerate}

\item In your command window, make sure you are in the
\file{eugene\textbackslash tools} directory.  Then type the following command to
open a tool that will allow us to specify and then view an
indicator:

\begin{verbatim}
python create_indicator.py
\end{verbatim}

\item In the ``Cache directory'' field, enter the location of the cache directory
created by the
simulation, e.g. \file{C:\textbackslash urbansim_cache\textbackslash
run_2006_11_15_15_12}.
This will be a directory in the output directory you specified when
starting the simulation.

\item Leave the ``Compare to another cache directory'' unchecked.  This option lets you 
compare results from two different simulations.

\item Select ``Lorenz Curve'' from Type drop down menu. This automatically generates the
Gini Coefficient as well.

\item In the ``Attribute'' field, enter:

\begin{verbatim}
urbansim.gridcell.average_income
\end{verbatim}

This entry contains the ``Opus path'' for a variable that shows the
average income within each gridcell.  More specifically, this Opus path specifies
the location of a Python file defining this gridcell : it is
in the \package{urbansim} Opus package, is defined for the
\verb|gridcell| dataset, and is located in the Python file named
\file{average_income.py}. (Currently,
if you want to see what variables have already been defined in the UrbanSim code,
you need to actually navigate to the installation path for opus
under the Python site-packages directory.)

\item Leave the optional ``Name'' entry empty. If entered, it is used
as the name for the created indicator (and also determines the filename of the
resulting indicator). We'll let the program use the default name
(``average_income'' in this case).

\item Type in ``gridcell'' in the ``Dataset'' entry.  This is consistent with the attribute 
we're going to create indicator for. 

\item Type in 1982 in the ``Year(s)'' entry. Leave the optional Scale entry as it is.

\item Press ``Run Request'' to generate this indicator.

\item The tool will then compute the indicators. When the
``View results'' button becomes active, it means the computation has finished.
Press this button. A web
browser will be launched and page loaded with a link to the requested indicator.

\item Click on the ``1982'' link to see the Lorenz curve. This simple plot was
produced by matplotlib, a Python graphing and mapping package.  


See \url{http://www.urbansim.org/wiki/external/index.php/Indicators} for a
partial list of indicators that are known to work on the Eugene-Springfield
data.  Interpreting the Lorenz Curve and Gini Coefficient is defined for non-aggregate
variables.  

\item Press ``Close'' to exit this tool.

\end{enumerate}

\section*{Closing Comments}

At this point, you have generated a Lorenz Curve and Gini Coefficient 
produced by an indicator from the simulated results. 
For more information, please see the accompanying Reference Manual and User Guide. 

If you have any particular requests for documentation, let us know by emailing
the \url{users@urbansim.org} email distribution list.

\section*{Acknowledgments}

This tutorial was put together by students from the 2007 UrbanSim Capstone class of the Computer
Science and Engineering department at the University of Washington
and from the Daniel J. Evans School of Public Affairs to accompany the integration
of Lorenz Curve output into the OPUS core.


\end{document}

% LocalWords:  urbansim lorenz gini capstone dX UrbanSim gridcells dataset py


%%% Local Variables:
%%% mode: latex
%%% TeX-master: "userguide"
%%% End:
% LocalWords:  eugene gridcell filename matplotlib userguide
