% $Id: run-eugene-model.tex,v 1.3 2007/06/01 23:49:14 borning Exp $

% Copyright (c) 2005-2007 Center for Urban Simulation and Policy Analysis,
% University of Washington.  Permission is granted to copy, distribute and/or
% modify this document under the terms of the GNU Free Documentation License,
% Version 1.2 or any later version published by the Free Software Foundation;
% with no Invariant Sections, no Front-Cover Texts, and no Back-Cover Texts.
% A copy of the license is included in the section entitled "GNU Free
% Documentation License".

\documentclass{howto}
\usepackage[T1]{fontenc}

\begin{document}

\newcommand{\package}[1]{\index{opus packages!#1@\textit{#1} package} \textit{#1}}


\title{How To Run the Eugene-Springfield Model}

%\release{0.1}

\author{Center for Urban Simulation and Policy Analysis \\
University of Washington}

\authoraddress{Daniel J. Evans School of Public Affairs \\
University of Washington \\
Seattle, Washington 98195 \\
USA \\
% Put the real email address only in the pdf version (to avoid including
% a spam magnet in the html version).  A mangled version of the address
% (using "info (at) urbansim.org") is included in the html version by
% a parameter to the latex2html command
%begin{latexonly}
Email: \email{info@urbansim.org} \\
Web Site: \url{www.urbansim.org}
%end{latexonly}
}

%\date{\today}
% replace \today with a real date (\date{April 1, 2000} ...} when this
% manual is ready for release so that reformatting doesn't cause a
% new date to be used.  For now, setting the date to \today during
% early editing makes it easier to handle versions.

\maketitle

%\begin{abstract}
%\noindent This document is a quick tutorial on
%downloading and using the example database for Eugene-Springfield, Oregon.
%The intention is to provide a new user or someone interested in learning
%more about UrbanSim a quick way to install the system and get a
%model running on an existing database.  The tutorial uses a set of simple and
%generally useful graphical user interfaces, or ``tools'', to simplify the
%interaction.  It takes about 20 minutes to install the necessary files, and
%another 20 minutes to run through the tutorial.
%\end{abstract}

%\tableofcontents
%\newpage

\section*{Introduction}

This tutorial covers the essentials of running an UrbanSim simulation.
We do not cover some complications that are not essential to running
UrbanSim, like setting up a database server, creating a base year database, and
estimating the model parameters.  These omitted items are part of the larger
process of preparing to use UrbanSim in a new area, and will be covered
elsewhere.

For the simulation, we use the 1980 base year data for the
Eugene-Springfield, Oregon area.  This is provided in a non-database
format that we refer to as a ``cache'', since it is a copy of the
required data from the associated MySQL database. The base year data
includes all the model parameters.  The entire process of
downloading and installing the system, and getting a simulation
running on a local computer should require no more than 20 minutes.
Since the model runs quickly on this data, one should be able to
compute and visualize indicators from a simulation over 2 years
within another 20 minutes.

We have tested this tutorial on Windows with the Enthought Python Edition
installation, as well as on Linux and Mac with the appropriate Enthought
packages.

This tutorial is the first of what we expect to become a fairly extensive
set of tutorials to assist new and more experienced users in taking
advantage of the functionality in the UrbanSim model system, using
the new Open Platform for Urban Simulation (OPUS).

\section*{Install Software and Download Sample Data}

If you have not already done so, install Opus and UrbanSim, along with
any needed supporting software.  Directions for the current version of
the code are at \url{http://www.urbansim.org/docs/installation/}.  If
you are using a (perhaps older) stable release rather than the current
version fresh out of the subversion repository, the table of releases
at \url{http://www.urbansim.org/download/} includes a link to the installation
instructions corresponding to that particular version.

Next, download the Eugene 1980 baseyear cache zip file by following the 
appropriate link on the download page \url{http://www.urbansim.org/download/}.
If you are using the source code for a stable release, there will be a link
to the Eugene 1980 baseyear cache file that works with this version in the table
of stable release versions.  Alternatively, if you are getting the latest version
of the code from the subversion repository, there is a link in the ``Current Source Code''
section.  (We frequently update the Opus and UrbanSim source code, but the sample database
doesn't change very often --- so the stable release version and latest version will likely 
be the same.)

Unzip the file into a cache
directory of your choice, such as \verb|c:\urbansim_cache| on Windows, or 
\verb|/Users/yourname/urbansim_cache| on the Mac. This example 
Eugene-Springfield database contains all of the data that UrbanSim needs to 
run a simulation of Eugene starting for simulated year 1980.

\section*{Running a Simulation}

In this part of the tutorial, you will execute a two-year simulation run of the
Eugene-Springfield metro area.

\begin{enumerate}

\item First, open a command window (yes, that archaic interface to the
computer where you actually have to \emph{type commands}). For
example, in Windows, go to the \file{Start} menu, choosing
\file{Run...}, typing ``cmd'', and clicking \file{OK}.  On the Mac,
open a terminal window (by picking ''terminal'' from the applications).

\item Next, change directory to the workspace subdirectory containing
the Eugene tools, e.g.\
\file{C:\textbackslash workspace\textbackslash eugene\textbackslash tools}
(modify this appropriately if your workspace has a different name).
For example, on Windows type:

\begin{verbatim}
cd C:\workspace\eugene\tools
\end{verbatim}

On the Mac, type:
\begin{verbatim}
cd /Users/yourname/workspace/eugene/tools
\end{verbatim}
replacing {\tt yourname} with your login name on your machine.

\item Now enter the following command at the command line:

\begin{verbatim}
python run_simulation_on_baseyear_cache.py
\end{verbatim}

This will launch a very simple graphical user interface that will allow you to
run a test simulation.  \emph{Note that the first time you launch one of these
graphical user interfaces, it may take up to a minute to load.  Subsequent
launches are much faster.}

If you don't see the graphical user interface, check your task bar, as the
application may be hidden behind another window.

\item In the ``Get baseyear data from this cache directory'' field , enter the
path to the location where you unzipped the
baseyear cache, above, e.g. 
\file{C:\textbackslash urbansim_cache\textbackslash eugene_1980_baseyear_cache\textbackslash}
on Windows, or
\file{/Users/yourname/urbansim_cache/eugene_1980_baseyear_cache/} on the Mac.

The ``Use this configuration'' entry is pre-set to use the
default baseline configuration in the eugene Opus package
(\verb|eugene.configs.baseline|), so you don't have to modify that
now.  If you had multiple scenarios configured, this is where you
could set which one to run.  Note that the name is in parts,
separated by periods.  These parts correspond with the system path
to the actual module that has the configuration.  Don't believe me?
Look in the \file{eugene\textbackslash configs} directory.

\item In the ``Create the output cache in this directory'' field, set the
output directory to a directory into which you want the simulation results to be
written, e.g. \file{C:\textbackslash urbansim_cache}. UrbanSim will
create a new subdirectory at this location, named with an embedded date/time, such as
\file{run_2006_11_15_15_12}, so that you can know when it was
started and to reduce the chances of accidentally over-writing
simulation results from earlier runs.

\item Leave the ``Number of years to run'' at 2.

\item To start the simulation, press the ``OK'' button.  The tool will
close, and UrbanSim will start running. While the simulation runs, a
series of messages are sent to the command window to provide
feedback on what is happening, but this will generally only need to
be consulted if you are diagnosing a problem. The simulation takes
about 3 1/2 minutes per year on a computer using a 3.2 GHz
Intel\textregistered Pentium\textregistered 4 processor, and about
the same on a 2Gh Pentium\textregistered M.  On a 3Ghz
Xeon\textregistered 64-bit processor running Linux (Fedora Core 6)
it runs in 35 seconds per year.

If you want to stop the simulation while it is running, open the
command prompt window and type Ctrl-C several times. When the
simulation is finished, it will say something like:

\begin{verbatim}
Done running simulation for years 1981 thru 1982
\end{verbatim}

\end{enumerate}

That's it!  You have now installed Opus and UrbanSim, a full
application for Eugene-Springfield, Oregon, and run a simulation
using a default scenario.

The results are all stored on your computer in a subdirectory of the
output cache, with a name that indicates when the output was
created. If you run this multiple times, it will create a new
directory for each run.  The contents of these directories are the
complete set of ``primary'' variables and values predicted for all objects in
the model, from which UrbanSim can recompute any ''computed'' variables defined
by an Opus .  However, these results are in binary files (arrays) that
are not easy to read directly.  Now we move on to how to examine the results by
creating indicators from them.

\section*{Computing an Indicator}

We have now run the simulation, so let's explore the results.
UrbanSim supports the concept of ``indicators'' that are useful
representations of dataset attributes.  Conceptually, an indicator
is a dataset attribute that is presented in a useful manner, such as
a map, comma-separated-value file, or a tab-delimited file for use by
another program.

First, let's produce a simple map (don't get too excited, it is only
an image map -- we're working on fancier maps with ArcGIS and other
systems too, but this is a quick visualizer that doesn't have all
the overhead of a GIS).

\begin{enumerate}

\item In your command window, make sure you are in the
\file{eugene\textbackslash tools} directory.  Then type the following command to
open a tool that will allow us to specify and then view an
indicator:

\begin{verbatim}
python create_indicator.py
\end{verbatim}

\item In the ``Cache directory'' field, enter the location of the cache directory
created by the
simulation, e.g. \file{C:\textbackslash urbansim_cache\textbackslash
run_2006_11_15_15_12}.
This will be a directory in the output directory you specified when
starting the simulation.

\item Leave the ``Compare to another cache directory'' unchecked.  This option lets you 
compare results from two different simulations.

\item Select ``Matplotlib map'' from Type drop down menu. You can see there are currently 4 other
indicator types available: Comma-separated table, Tab-separated table, Chart, and Dbf export. We'll try 
tab-delimited table in a moment.

\item In the ``Attribute'' field, enter:

\begin{verbatim}
urbansim.gridcell.population
\end{verbatim}

This entry contains the ``Opus path'' for a variable that shows the
population within each gridcell.  More specifically, this Opus path specifies
the location of a Python file defining this gridcell : it is
in the \package{urbansim} Opus package, is defined for the
\verb|gridcell| dataset, and is located in the Python file named
\file{population.py}. You can find the directory that holds the code for
the \package{urbansim} package in your workspace directory.

\item Leave the optional ``Name'' entry empty. If entered, it is used
as the
name for the created indicator (and also determines the filename of the
resulting indicator). We'll let the program use the default name
(``population'' in this case).

\item Type in ``gridcell'' in the ``Dataset'' entry.  This is consistent with the attribute 
we're going to create indicator for. 

\item Type in 1982 in the ``Year(s)'' entry. Leave the optional Scale entry as it is.

\item Press ``Run Request'' to generate this indicator.

\item The tool will then compute the indicators. When the
``View results'' button becomes active, it means the computation has finished.
Press this button. A web
browser will be launched and page loaded with a link to the requested indicator.

\item Click on the ``1982'' link to see the map. This simple map was
produced by matplotlib, a Python graphing and mapping package.  More
sophisticated maps can be generated by importing the data into other
tools, such as ArcMap.

\item Now let's produce a tab-delimited file for a different
attribute. Switch back to the indictors interface, change the ``Type'' 
to ``Tab-delimited table'', and change the
``Attribute'' field to be:

\begin{verbatim}
urbansim.zone.residential_units
\end{verbatim}

(Note that \verb|zone| refers to a traffic analysis zone.)

\item Type in ``zone'' for the ``Dataset'' field and a single year for 
which you want to generate the data (e.g. 1981 or 1982) in the ``Year(s)'' 
entry,  and press ``Run request'' to generate the data.

\item When the ``View results'' button is active again, press it to
take you to the resulting web page which will now also contain a link
to the tab file.  In Windows, the default viewer for a tab file is
often Excel, which will open your file automatically when you click on
the link in this web page. Alternatively, you can of course load it
into any software package that can read an ASCII, tab-delimited file
--- even a text editor.

\item Just for fun, change the selection in the tool from
``Tab-delimited table (*.tab)'' to ``Matplotlib map (*.png)'', and
run the indicator again.  Note that the map is a gridcell map but
shows the data by traffic analysis zone.  The map shows ``jaggies''
because of the resolution of the gridcells (150 x 150 meters) at
this scale.

See \url{http://www.urbansim.org/wiki/external/index.php/Indicators} for a
partial list of indicators that are known to work on the Eugene-Springfield
data.  Also, note that maps only
work for indicators associated with a geography.  In the case of Eugene, this
includes indicators for \verb|gridcell| and \verb|zone|.
You cannot, for instance, 
create a map for \verb|urbansim.household.is_minority|, since \verb|household|
is not a geography.  You can, however, define and use a variable that links this
information with a geography, such as done by 
\verb|urbansim.gridcell.number_of_minority_households|.  

\item Press ``Close'' to exit this tool.

\end{enumerate}

Note that you can examine any of the data in the urbansim_cache by
using these indicator tools.  You can also export the data for any
year to a MySQL database, or to tab-delimited or comma separated
ASCII files to be able to explore them in other software systems. If
you want to export the cache data, either from the base year
database or from the simulation output, change directory to the
\file{eugene\textbackslash tools} directory and run one of the three
tools we have provided for this purpose to export the output in the
format you prefer:

\begin{verbatim}

python export_cache_to_mysql.py

python export_cache_to_tab.py

python export_cache_to_csv.py

\end{verbatim}

These tools open a simple tool, which at this point you should be
able to figure out from the preceding parts of the tutorial.  If you
are uncertain about the function of an edit window, click on its
label for a bit of help. Enjoy!

\section*{Closing Comments}

At this point, you have run a simulation and generated and
visualized a couple of indicators from the simulated results. You
may have also exported the UrbanSim database for a year and examined
it in another system. This tutorial has just scratched the
surface of what you can do with UrbanSim, of course. For more information,
see the accompanying Reference Manual and User Guide. In addition,
we plan to expand the content in this tutorial and add additional
tutorials as we get time and as the functionality develops.  We
would welcome user contributed tutorials also!

If you have any particular requests for documentation, let us know by emailing
the \url{users@urbansim.org} email distribution list.

Finally, note that while the above tools are currently located in the eugene
Opus package, they will work just as well on any other UrbanSim data set, as
long as you have updated that data and code as noted in the
\file{docs\textbackslash release_notes.html} file.

\section*{Acknowledgements}

This tutorial was put together by the UrbanSim project team in
response to various requests for a sample application that would
allow quickly installing UrbanSim, running an application, and
examining the data, and has been tested by several users.

A special thanks is in order to Bud Reiff, Bob Denouden and others at the
Lane Council of Governments for their generosity in sharing the data we use
in this tutorial.  This research has been funded in part by grants from the
National Science Foundation (EIA-0121326 and IIS-0534094), and in part by a
grant from the Environmental Protection Agency (R831837).  We'd like
to thank several users as well for their financial support of the
development of UrbanSim and OPUS.

\end{document}

% LocalWords:  borning urbansim PYTHONPATH AllTests HouseholdSet mysql xml sql
% LocalWords:  mydatabase Dataset numpyy matplotlib rpy ScenarioDatabase JobSet
% LocalWords:  GridcellSet ZoneSet FazSet NeighborhoodSet RaceSet RateSet logit
% LocalWords:  numpy ChoiceModel rcrr submodels mycoef LocationChoiceModel
% LocalWords:  HLCM AgentLocationChoiceModel gridcells debuglevel config LCMs
% LocalWords:  UrbanSim EmploymentHomeBasedLocationChoiceModelCreator cbd coef
% LocalWords:  EmploymentIndustrialLocationChoiceModelCreator RegressionModel
% LocalWords:  EmploymentCommercialLocationChoiceModelCreator gridcell py un ln
% LocalWords:  DevelopmentProjectLocationChoiceModel InteractionSet indices DDD
% LocalWords:  DevelopmentProjectCommercialLocationChoiceModelCreator hlcm init
% LocalWords:  DevelopmentProjectIndustrialLocationChoiceModelCreator powDDD
% LocalWords:  DevelopmentProjectResidentialLocationChoiceModelCreator
% LocalWords:  Versioning versioning

%%% Local Variables:
%%% mode: latex
%%% TeX-master: "userguide"
%%% End:
