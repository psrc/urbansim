% Copyright (c) 2005-2008 Center for Urban Simulation and Policy Analysis,
% University of Washington.  Permission is granted to copy, distribute and/or
% modify this document under the terms of the GNU Free Documentation License,
% Version 1.2 or any later version published by the Free Software Foundation;
% with no Invariant Sections, no Front-Cover Texts, and no Back-Cover Texts.
% A copy of the license is included in the section entitled "GNU Free
% Documentation License".

\documentclass{howto}
\usepackage[T1]{fontenc}

\begin{document}

\newcommand{\package}[1]{\index{opus packages!#1@\textit{#1} package} \textit{#1}}


\title{How To Run the Eugene-Springfield Model}

%\release{4.2}

\author{Center for Urban Simulation and Policy Analysis \\
University of Washington}

\authoraddress{Daniel J. Evans School of Public Affairs \\
University of Washington \\
Seattle, Washington 98195 \\
USA \\
% Put the real email address only in the pdf version (to avoid including
% a spam magnet in the html version).  A mangled version of the address
% (using "info (at) urbansim.org") is included in the html version by
% a parameter to the latex2html command
%begin{latexonly}
Email: \email{info@urbansim.org} \\
Web Site: \url{www.urbansim.org}
%end{latexonly}
}

%\date{\today}
% replace \today with a real date (\date{April 1, 2000} ...} when this
% manual is ready for release so that reformatting doesn't cause a
% new date to be used.  For now, setting the date to \today during
% early editing makes it easier to handle versions.

\maketitle

%\begin{abstract}
%\noindent This document is a quick tutorial on
%downloading and using the example database for Eugene-Springfield, Oregon.
%The intention is to provide a new user or someone interested in learning
%more about UrbanSim a quick way to install the system and get a
%model running on an existing database.  The tutorial uses a set of simple and
%generally useful graphical user interfaces, or ``tools'', to simplify the
%interaction.  It takes about 20 minutes to install the necessary files, and
%another 20 minutes to run through the tutorial.
%\end{abstract}

%\tableofcontents
%\newpage

\section*{Introduction}

This tutorial covers the essentials of running an UrbanSim simulation.
We do not cover some complications that are not essential to running
UrbanSim, like setting up a database server, creating a base year database, and
estimating the model parameters.  These omitted items are part of the larger
process of preparing to use UrbanSim in a new area, and will be covered
elsewhere.

The tutorial uses a new graphical interface that we are developing.  This
GUI is still in the early stages, and is incomplete in various ways.  We'll
keep updating the tutorial as additional functionality is implemented.

For the simulation, we use the 1980 base year data for the
Eugene-Springfield, Oregon area.  This is provided in a non-database
format that we refer to as a ``cache'', since it is a copy of the
required data from the associated MySQL database. The base year data
includes all the model parameters.  Since the model runs quickly on this data, one should be able to
compute and visualize indicators from a simulation over 2 years
within 20 minutes.

We have tested this tutorial on Windows, Mac and Linux.

\section*{Install Software and Download Sample Data}

If you have not already done so, install Opus and UrbanSim, along with
any needed supporting software, and download the sample data and configurations.  
Directions for the current version of
the code are at \url{http://www.urbansim.org/docs/installation/}.  
This tutorial is for the development version of the code currently in 
the svn repository.    It relies on the PyQt package to produce its 
graphical user interface. Earlier versions (4.1.2 and before) use an
older user interface based on the Enthought Traits packages.  If
you are using a previous release rather than the current
version fresh out of the subversion repository, the table of releases
at \url{http://www.urbansim.org/download/} includes a link to the 
tutorials and installation
instructions corresponding to that particular version.

%% If you are using the source code for a stable release, there will be a link
%% to the Eugene 1980 baseyear cache file that works with this version in the
%% table of stable release versions.  Alternatively, if you are getting the
%% latest version of the code from the subversion repository, there is a link
%% in the ``Current Source Code'' section.  (We frequently update the Opus and
%% UrbanSim source code, but the sample database doesn't change very often ---
%% so the stable release version and latest version will likely be the same.)

\section*{Running a Simulation}

In this part of the tutorial, you will execute a two-year simulation run of the
Eugene-Springfield metro area.

\begin{enumerate}

\item First, run the file \verb|opus.py| in the \verb|opus_gui| directory
in your opus workspace.  You can do this by navigating to the file,
right-clicking on it (on a Mac, control-click), and selecting ``Open with
\dots'' and then ``python'' (Or ``python launcher'' if you are using a
Mac).  Alternatively, you can run the file through a graphical editor like
Eclipse, or from the command line by typing \verb|python opus.py| from the
appropriate directory.

This will launch a graphical user interface that will allow you to, among
other things, run a test simulation.  If you don't see the graphical user
interface, check your task bar, as the application may be hidden behind
another window.

\item In the top menu bar, select ``Project'' and then ``Open Project''.  In the projects directory, select the file ``eugene_gridcell.xml'' and click ``Open.''  This loads the run configuration information for the Eugene simulation.

\item Select the ``Scenario Manager'' tab on the left side of the GUI window.  You should be able to see and navigate the information loaded from ``eugene_gridcell.xml''.  Under the top-level node ``Eugene_baseline'' you will find another called ``years_to_run'' and under this two variables that specify the start and end years for the simulation.  Edit ``endyear'' by double-clicking in the ``Value'' column next to ``endyear''.  Then change the value to ``1982''.  Many values in the configuration can be edited in this way.

\item Still in the ``Scenario Manager'' tab, right click on the top-level node ``Eugene_baseline'' and select ``Run This Model''.  This will cause a new tab with the name ``eugene_gridcell.xml'' to pop up on the right side of the GUI.  This gets the model ready to run.

\item In the ``eugene_gridcell.xml'' tab, click the ``Start Model\ldots''
button.  The model will then begin to run. You can watch its progress in
the status bar of this tab.  
%or maybe watch a log somewhere?  will there 
% be anything else when the GUI is done?  
When the simulation is finished,
the status bar will reach 100\% and underneath it will say something like
\begin{verbatim}Model finished with status = True\end{verbatim}

\end{enumerate}

That's it!  You have now installed Opus and UrbanSim, a full
application for Eugene-Springfield, Oregon, and run a simulation
using a default scenario.

The results are all stored on your computer in the \verb|runs| subdirectory
of the \verb|eugene| directory where you originally unzipped the cache
data.  The folder will have a name that indicates when the output was
created. If you run this multiple times, it will create a new directory for
each run.  The contents of these directories are the complete set of
``primary'' variables and values predicted for all objects in the model,
from which UrbanSim can recompute any ''computed'' variables defined by an
Opus .  However, these results are in binary files (arrays) that are not
easy to read directly.  Now we move on to how to examine the results by
creating indicators from them.  You should keep the GUI open for this next
part.

\section*{Computing an Indicator}

We have now run the simulation, so let's explore the results.
UrbanSim supports the concept of ``indicators'' that are useful
representations of dataset attributes.  Conceptually, an indicator 
 is a variable that conveys information on the condition or trend of
an attribute of the system considered. When we calculate the indicator with a particular set of data,
 we produce an indicator result. We can then view the indicator result with an indicator visualization, such as
a map, comma-separated-value file, or a tab-delimited file for use by
another program.

First, let's produce a simple map (don't get too excited, it is only
an image map --- we're working on fancier maps with ArcGIS and other
systems too, but this is a quick visualizer that doesn't have all
the overhead of a GIS).

\begin{enumerate}

\item On the left side of the GUI window, select the ``Results Manager''
tab.  This tab displays a number of indicators and the results of
computations done on them.  The top-level ``Libraries'' node has a number
of \mbox{``indicator_library''} type subnodes.  These nodes represent
libraries of indicators.  Their subnodes, are the indicators.  The
top-level ``Data_sources'' node contains data about the various runs
available.  It is outside the scope of this tutorial.  The top-level node
``Results'' has a number of subnodes which are ``indicator results'' ---
the results of an indicator computation on particular data.

\item Right-click on the ``population'' indicator under ``Libraries >
model.gridcell'' and select ``Generate results with\ldots''  This will open a
tab on the right side of the GUI which allows you to specify parameters for
generating a particular result.

\item Now we will actually perform the indicator computation.  In the
``Generate results'' tab, choose the value ``population'' from the
``Indicator'' drop-down menu.  From the ``Dataset'' drop-down menu, choose
``gridcell'', and from the ``Source data'' drop-down menu, choose
``eugene_baseline''.  Finally, click the ``Generate results\ldots'' button.
In the log window below the button you will see a small amount of log
output about the indicator calculation.

\item Now we display the results.  In the ``Results Manager'' tab, scroll
down to the ``Results'' node.  Find the subnode called
``population.gridcell.eugene_baseline'' and right-click on it.  Select
``View results as\ldots > Map (Matplotlib)''.  This will cause a tab called
``gridcell_map_1980_gridcell_population'' to pop up in the right side of
the GUI.  This tab will display the resultant map.  \emph{Caution: in this
prototype version of the GUI, there are other possible results displayed
--- these aren't hooked up to anything yet.  Just use the
``population.gridcell.eugene_baseline'' node.}

\item You can also compute the results at different geographies.  In the
  ``Generate results'' tab, as before right-click on the ``population''
  indicator under ``Libraries > model.gridcell'' and select ``Generate
  results with\ldots.'' Again this will open a tab on the right side of the
  GUI which allows you to specify parameters for generating a particular
  result.  This time, choose the value ``population'' as before, but from
  the ``Dataset'' drop-down menu, choose ``zone'' instead of ``gridcell.''
  From the ``Source data'' drop-down menu, choose ``eugene_baseline'' as
  before, and click the ``Generate results\ldots'' button.

\item Now let's display the results as a table of zones.  (You probably
  don't want to have a table of gridcells, since there are so many.)
  Right-click on the ``population.zone.eugene_baseline'' node under
  ``Results''.  Now select ``Table (one per selected indicator)''.  A table
  will pop up on the right side of the GUI.  \emph{Caution: right now the
  map visualization won't work for zones, just gridcells --- so for zones
  just view these as a table.}

\end{enumerate}

%See \url{http://www.urbansim.org/wiki/external/index.php/Eugene_indicators} for a
%partial list of indicators that are known to work on the Eugene-Springfield
%data.  Also, note that maps only
%work for indicators associated with a geography.  In the case of Eugene, this
%includes indicators for \verb|gridcell| and \verb|zone|.
%You cannot, for instance, 
%create a map for \verb|urbansim.household.is_minority|, since \verb|household|
%is not a geography.  You can, however, define and use a variable that links this
%information with a geography, such as done by 
%\verb|urbansim.gridcell.number_of_minority_households|.  If you have a problem
%with generating a map at the gridcell level, try generating it at the zone
%level instead.  (Sometimes matplotlib seems to get overwhelmed by the amount of 
%data at the gridcell level.)

%Note that you can examine any of the data in the urbansim_cache by
%using these indicator tools.  You can also export the data for any
%year to a MySQL database, or to tab-delimited or comma separated
%ASCII files to be able to explore them in other software systems. If
%you want to export the cache data, either from the base year
%database or from the simulation output, change directory to the
%\file{eugene\textbackslash tools} directory and run one of the three
%tools we have provided for this purpose to export the output in the
%format you prefer: % do these still work?

%\begin{verbatim}

%python export_cache_to_mysql.py

%python export_cache_to_tab.py

%python export_cache_to_csv.py

%\end{verbatim}

%These tools open a simple tool, which at this point you should be
%able to figure out from the preceding parts of the tutorial.  If you
%are uncertain about the function of an edit window, click on its
%label for a bit of help. Enjoy!

\section*{Closing Comments}

At this point, you have run a simulation and generated and
visualized a couple of indicators from the simulated results. You
may have also exported the UrbanSim database for a year and examined
it in another system. This tutorial has just scratched the
surface of what you can do with UrbanSim, of course. For more information,
see the accompanying Reference Manual and User Guide. In addition,
we plan to expand the content in this tutorial and add additional
tutorials as we get time and as the functionality develops.  We
would welcome user contributed tutorials also!

If you have any particular requests for documentation, let us know by emailing
the \url{users@urbansim.org} email distribution list.

\section*{Acknowledgements}

This tutorial was put together by the UrbanSim project team in
response to various requests for a sample application that would
allow quickly installing UrbanSim, running an application, and
examining the data, and has been tested by several users.

A special thanks is in order to Bud Reiff, Bob Denouden and others at the
Lane Council of Governments for their generosity in sharing the data we use
in this tutorial.  This research has been funded in part by grants from the
National Science Foundation (IIS-0534094 and IIS-0705898), and in part by a
grant from the Environmental Protection Agency (R831837).  We'd like
to thank several users as well for their financial support of the
development of UrbanSim and OPUS.

\end{document}

% LocalWords:  borning urbansim PYTHONPATH AllTests HouseholdSet mysql xml sql
% LocalWords:  mydatabase Dataset numpyy matplotlib rpy ScenarioDatabase JobSet
% LocalWords:  GridcellSet ZoneSet FazSet NeighborhoodSet RaceSet RateSet logit
% LocalWords:  numpy ChoiceModel rcrr submodels mycoef LocationChoiceModel
% LocalWords:  HLCM AgentLocationChoiceModel gridcells debuglevel config LCMs
% LocalWords:  UrbanSim EmploymentHomeBasedLocationChoiceModelCreator cbd coef
% LocalWords:  EmploymentIndustrialLocationChoiceModelCreator RegressionModel
% LocalWords:  EmploymentCommercialLocationChoiceModelCreator gridcell py un ln
% LocalWords:  DevelopmentProjectLocationChoiceModel InteractionSet indices DDD
% LocalWords:  DevelopmentProjectCommercialLocationChoiceModelCreator hlcm init
% LocalWords:  DevelopmentProjectIndustrialLocationChoiceModelCreator powDDD
% LocalWords:  DevelopmentProjectResidentialLocationChoiceModelCreator
% LocalWords:  Versioning versioning

%%% Local Variables:
%%% mode: latex
%%% TeX-master: "userguide"
%%% End:
