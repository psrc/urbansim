
\chapter{The Results Manager}

The Results Manager, corresponding to the Results tab of the GUI, has
two main responsibilities: to manage simulation runs for this project
(Section~\ref{sec:managing-simulation-runs}) and to allow the
interrogation of these simulation runs through the use of
\emph{indicators}
(Section~\ref{sec:interrogating-results-with-indicators}). We explore
both of these in this chapter.

\section{Managing simulation runs}
\label{sec:managing-simulation-runs}

\begin{figure}[tp]
\begin{center}
%\includegraphics[scale=0.4]{part-gui/images/result-manager-.png}
\end{center}
\caption{Importing a run from disk that is not showing up as a
``Simulation run''.}
\label{fig:results-manager-import-run}
\end{figure}

The ``Simulation runs'' node captures all
available simulation runs for this project. For example, when a
simulation is run from the ``Scenarios-manager'' (see
chapter~\ref{chap:scenarios-manager}), an entry is made under
``Simulation runs''. If for some reason a run that you know exists
for this project is not listed, right-click on 
``Simulation runs'' and select ``Import run from disk'' (See
Figure~\ref{fig:results-manager-import-run}). The GUI will try to load
the missing run and make it available.

A couple operations can be performed on a simulation run. To view
information about a given run, right-click on the run and select
''Show details''. To remove all traces of a simulation run,
including the data on disk, right-click on the run and select
''Remove run and delete from harddrive''.

\section{Interrogating results with Indicators}
\label{sec:interrogating-results-with-indicators}

Indicators are variables defined explicitly for use as a meaningful
measure (see Section~\ref{sec:opus-indicators}). Like model variables,
they can be defined using the domain-specific programming language via
the ``Variable Library'' accessible through the Tools menu (see
Chapter~\ref{chap:variable-library}). An indicator can then be
visualized as either a map or as a table in a variety of
formats. The GUI provides two ways to use indicators to understand what
has happened in a simulation run: interactive result exploration
(Section~\ref{sec:interactive-result-exploration}) and Batch
indicator configuration and execution
(Section~\ref{sec:batch-indicator-configuration}).

\subsection{Interactive result exploration}
\label{sec:interactive-result-exploration}
Often, it is desirable to explore simulation results in a lightweight
fashion in order to get a basic idea of what happened. You don't
necessarily want to go through the process of exporting results to a
GIS mapping tool in order to gain some basic intuitions into spatial
patterns.

\begin{figure}[tp]
\begin{center}
%\includegraphics[scale=0.4]{part-gui/images/result-manager-.png}
\end{center}
\caption{Using the ``Result browser'' for interactive result
exploration.}
\label{fig:results-manager-result-browser}
\end{figure}


The Opus GUI's ``Result Browser'', available from the ``tools''
menu, allows interactive exploration of simulation results. The Result
Browser presents a selectable list of available simulation runs, years
over which those simulations were run, and available indicators. You
can then configure an indicator visualization by selecting a simulation
run, a year, and an indicator. To compute and visualize the configured
indicator, simply press the ``generate results'' button (See
Figure~\ref{fig:results-manager-result-browser}). The
indicator will then be computed for the year of the selected simulation
run. After it is computed, a tab should appear at the bottom of the
window with the name of the indicator. Subtabs allow you to see the
results as a table or map (using the Matplotlib Python module). 

\fbox{
\begin{minipage}{.5\linewidth}
\begin{enumerate}
  \item Open the Results Browser from the Tools menu. Use the
  Results Browser to answer the following questions.
  \item Just from visual inspection, is there more than one cluster of
  gridcells with high land value in the Eugene region in 1980 in the
  baseyear data?
  \item Is this cluster(s) in the same general area as the
  greatest number of jobs in Eugene for the same year of the
  baseyear data?
\end{enumerate}
\end{minipage}
}

Two additional aspects of the Result Browser should be mentioned:
\begin{enumerate}
  \item If the checkbox ``Automatically view indicator'' is
  clicked, everytime you change the indicator configuration (i.e.
  select a different simulation run, year, or indicator), the
  indicator will be automatically visualized (as if you pressed the
  ``Generate results'' button). 
  \item The ``Export results'' button will export the table data
  of the currently configured indicator to a database. This feature
  is not yet implemented. 
\end{enumerate}

\subsection{Batch indicator configuration and execution}
\label{sec:batch-indicator-configuration}

The ``Result Browser'' is good for poking around in the
data. But often you'll want to generate the same
set of indicators for each of many runs and you don't want to
redefine them every time. Instead, you'd like to configure and save a
group of them that can be executed on demand on an arbitrary
simulation run. In the Opus GUI, this functionality is supported with
\emph{indicator batches}. 

\begin{figure}[tp]
\begin{center}
%\includegraphics[scale=0.4]{part-gui/images/result-manager-.png}
\end{center}
\caption{Creating a new indicator batch}
\label{fig:results-manager-new-batch}
\end{figure}

To create a new indicator batch, right-click on the
 ``Indicator\_batches'' node in the ``Results tab'' and select
 ``Add new indicator batch...'' (See
Figure~\ref{fig:results-manager-new-batch}). A new batch will be
created under the Indicator\_batches node. If desired, you can rename the new
batch by double-clicking its name and typing in a new one.

A batch is a collection of ``Indicator visualization''
definitions. Each indicator visualization is a configuration of
the indicator variable to be used, a visualization style (e.g. map or
table), and some format options. To add a new indicator visualization
to the batch, right-click on the respective batch and select
``Add new indicator visualization...''. A dialog box will
appear where you can define the visualization. The visualization
options for an indicator visualzation are discussed in depth later.

You can add as many indicator visualizations to a batch as you want. In
order to execute an indicator batch on a simulation run, right-click on
the indicator batch and hover over
 ``Run indicator batch on...''. A list of all the available simulations
 runs will
appear as a submenu. You can then select the appropriate simulation.
The indicator visualizations in the batch will be executed over all the
years of that simulation run. If the resulting indicators are tables or
maps stored in a file, they can then be found on disk in your
``OPUSHOME/data/PROJECTNAME/runs/RUNNAME/indicators'' directory, where
``PROJECTNAME'' is the name of your project (e.g.
 ``eugene\_gridcell'') and ``RUNNAME'' is the name of the
simulation run that you selected to run the batch on. The indicator
visualizations configured to write to a database will have produced
tables in the specified database with the name of the respective
indicator visualization.

\fbox{
\begin{minipage}{.5\linewidth}
Create, configure, and execute a new indicator batch:
\begin{enumerate}
\item Create a
new indicator batch by right-clicking on the  ``Indicator\_batches'' node
in the Results tab and selecting the appropriate option. 
\item Optional: Rename the new
indicator batch.
\item Add an indicator visualization configuration to
that batch. Right-click on your new indicator batch and select  ``Add new
indicator visualization''.
\item Configure a Map visualization that contains the 
{\sf zone\_job\_density} and {\sf zone\_population\_density} indicators for the
zone dataset.
\item Close the batch visualization configure dialog.
\item Right-click on the batch and
execute your indicator batch on the results of a simulation run.
\end{enumerate}
\end{minipage}
}


\subsubsection{Indicator visualization configuration options}
\label{sect:indicator-visualization-options}


\begin{figure}[tp]
\begin{center}
%\includegraphics[scale=0.4]{part-gui/images/result-manager-.png}
\end{center}
\caption{The batch visualization creation dialog.}
\label{fig:results-manager-batch-viz}
\end{figure}

Opus provides a variety of ways to visualize indicators and this
functionality is exposed in the ``Indicator visualization'' dialog
box options (e.g. multi-year indicators, exporting to
databases). This section describes the range
of available options in the Batch indicator visualization dialog
box, which is separated into three components:  ``indicator
selection'',  ``output options'', and  ``format options'' (See
Figure~\ref{fig:results-manager-batch-viz}). 

\heading{Indicator selection}

The bottom of the dialog box has two list boxes,  ``available
indicators'' and  ``indicators in current visualization''. The
indicators here are those variables defined in the 
``Variable Library'' (Chapter~\ref{chap:variable-library}) whose
\emph{use} has been set to be \emph{indicator} or \emph{both}. Note that the set of
indicators available is filtered by the currently selected dataset in
the  ``output options'' (described later in this section).

By moving an indicator from the  ``available indicators'' box to the
 ``indicators in current visualization'' box via the  ``+'' button, you
include that indicator in this indicator visualization. Likewise, to
remove an indicator from the visualization, select the indicator in
the  ``indicators in current visualization'' box and press the  ``-''
button.


\heading{Output options}

\emph{Visualization Name}. The base name of any produced
visualizations. Because you might be producing this visualization for
different years and different simulation data, more information will
be appended to this name to ensure uniqueness of the resulting file
or database table when the visualization is run on some data. 

\emph{Type}. There are two different types of indicator
visualizations that can be produced: maps and tables. Tables are just
raw data organized into rows and columns, while maps are
spatial projections of this data. The available format options
(described later) are fully dependent on the visualization type. 

\emph{Dataset name}. The dataset that this visualization corresponds
to. When the selected indicator(s) are run, they will be computed over
this dataset. Most commonly you are choosing a geographic granularity
(e.g. gridcell, zone) that you want to see the results at. Note that
when you change the dataset, the set of available indicators changes
because a given indicator is valid only for a single dataset.

\heading{Format options for maps}

Map visualizations will produce a map file for every selected
indicator for every available year when it is executed on a
simulation run. 

Currently, the only availabe map format is Matplotlib, a Python mapping
module that quickly produces low-quality images. Note that Matplotlib
map is not intended to replace GIS-based mapping, which allows far more
control and the overlay of other features for visual reference.  It is
merely a quick tool to visualize data to get a sense of the spatial
patterns in it. In order to support visualization in a GIS environment
such as ArcGIS or QGIS, the results may be exported to a database or
geodatabase environment, and the GIS software used to create a more
interactive and flexible display of the data. See the following section
for a description of how to export indicator results to a SQL database
or a DBF file for use in external GIS tools.

\heading{Format options for tables}

There are four different available formats for tables. Each has its
own parameters that need to be set. Note that the id column for the
dataset will automatically be included in all outputted tables
regardless of format.

\emph{Tab-delimited (.tab)}. This will output a file (or multiple
files) where the values are separated by tabs. There are three
different modes to choose from that affects how data is split across
files when the visualization is executed on a simulation run.
``Output in a single file'' will create single tab file that has a
column for each selected indicator for each year of the simulation
run. ``Output in a table for every year'' will create a tab file
for each year of the simulation run, with each column
corresponding to a selected indicator. Finally, 
``Output a table for each indicator'' will create a tab file for
each selected indicator, where each column
corresponds to the indicator values of a year of the
simulation run.

\emph{Fixed-field}. The fixed field format will output a single file
whose fields are written with fixed width and no delimiters. The file
contains a column for each selected indicator for each year of the
simulation run for which the visualization is being created. Format
info for each column needs to be specified. To specify the format of
the dataset id column, fill in the ``id\_col'' input field. To
specify the format of each selected indicator, enter the format in the
respective row of the ``field format'' column in the  ``indicators in
current visualization'' box. The field format has two parts, the length
of the field and the type. Available types are float ( ``f'') and
integer ( ``i''). Specified field formats follow the pattern  ``10i'' and
 ``5.2f'', where the latter specifies a float with five leading digits
with floating point precision carried out to two decimal places.

\emph{SQL database}. This format option can be used to export to an
arbitrary SQL database. The database server used is that specified in
the ``Database server connections'' under  ``indicators\_database''
(see Section~\ref{sec:database-server-connections}). The exported data
will take the form of a newly created in the specified database (if the database doesn't exist, it will be created
first). The SQL table will contain a column for every selected
indicator for every year of the simulation run that it is being
executed against. The name of the table is a combination of the name of
the visualization and the name of the simulation run. Additionally,
if you are exporting to a PostGRES database and have an existing
spatial table corresponding to the dataset of the visualization, a view
defining a join over the spatial table and the indicator table will
automatically be created. This allows you to instantly view the
indicator results in QGIS. 

\emph{ESRI database}. This option exports the
indicator data to an ESRI database that can be loaded into ArcMap.
Simply specify the path to a geodatabase file (.gdb).


\fbox{
\begin{minipage}{.5\linewidth}
Export the results that were found in the previous tutorial inset
to a SQL database. 
\begin{enumerate}
  \item Make sure that you have configured a database
server. From the Tools menu, select  ``Database Server Connections''. Check
to see that the ``indicators\_database\_server" is correctly set up.
If you don't have a remote database server, make sure that it
points to a sqlite connection. Close the
connections dialog box.
  \item Reconfigure the batch to write to a
database. Expand the indicator batch that you defined in the prior
step. Right-click on the visualization and select ``Configure
visualization''. Change the format to ``Export to SQL database'' and then
name a database it should write to. Hit OK and then rerun the batch on the
simulation results from before.
  \item Launch a database browser and check to see if the proper
  tables were created. 
\end{enumerate}
\end{minipage}
}