\chapter{The Menu Bar}

There is one primary menu bar in the OPUS GUI.  The main menu bar has three dropdown menus: File; Tools; and Help.  The File menu, shown in figure ??, allows for simple project manipulation: opening; saving; closing; as well as exiting the OPUS GUI.  Help offers an About option which produces a dialog box with information about OPUS and a set of links to the UrbanSim website, online documentation, and the GNU License.  The Tools dropdown is discussed in greater detail below.

Most of the items in the main menubar are accessible from a secondary menu bar just above the tabs on the left side of the OPUS GUI window.  Hovering over each icon will yield a tooltip with the item's description.

image of file menu

\section{Tools}

The Tools menu, shown in figure ??, enables users to adjust settings and preferences, as well as opening different tabs in the right set of tabs.  The items labeled "Python Console", "Log View", "Editor View", and "Result Browser" will each open new tabs on the right.  The Result Browser is covered in greater detail in section \ref{interactive-result-exploration}.  The items labeled "Variable Library", "Preferences", and "Database Connection Settings" each open a popup when clicked.  The Variable Library is further discussed in section \ref{chap:variable-library}.

image of tool menu

\section{Preferences}

The Preferences dialog box changes some user interface related options in the OPUS UI.  The dialog box is split into two sections, font preferences and previous project preferences.  The font preferences section allows users to adjust font sizes specific to different areas of the GUI.  The previous project preferences section contains a checkbox allowing users to open the most recently opened project each time OPUS GUI is started, this is turned off by default.  Changes to the user preferences take effect as soon as either the "Apply" or "OK" buttons are clicked.

\section{Database Server Connections}\label{sec:database-server-connections}

Database connections can be configured in the Database Server Connections dialog launched from the Tools menu.  The Database Server Connections dialog, pictured in figure ??, holds connection information for four database servers.  Each connection is used for a specific purpose.  While there are four different connections that must be configured, each may be configured to use the same host.  Every connection requires a protocol, host name, user name, and password to be entered.  Editing the protocol field produces a drop down of database protocols that UrbanSim is able to use.  If a server has been setup for UrbanSim's use choose the protocol that corresponds to the type of SQL server being used.  If no SQL server is setup for a particular use, SQLite may be used.  SQLite will create a local flat-file database instead of a remote server.  UrbanSim currently supports MySQL, Microsoft SQL Server, Postgres, and SQLite.

The Database Connection Settings are saved when the Accept Changes button is pressed, ensuring that all future database connections will be made using the new settings.  Database connections that are still in use while the database connection settings are being edited will not be changed until the connection is reestablished, for this reason it may be necessary to reopen any open project after changing the database connection settings.

image of Database Server Connections Dialog Box
