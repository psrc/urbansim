% Copyright (c) 2005-2007 Center for Urban Simulation and Policy Analysis,
% University of Washington.  Permission is granted to copy, distribute and/or
% modify this document under the terms of the GNU Free Documentation License,
% Version 1.2 or any later version published by the Free Software Foundation;
% with no Invariant Sections, no Front-Cover Texts, and no Back-Cover Texts.
% A copy of the license is included in the section entitled "GNU Free
% Documentation License".

\chapter{Using the UrbanSim System}
\label{chapter:using-urbansim}
%
The latest incarnation of UrbanSim, UrbanSim 4, is implemented as a
set of packages in Opus. Package \package{opus_core} provides
general functionality, \package{urbansim} contains everything around
land use models, \modelsindex package \package{opus_emme2} is an
Opus wrapper to the travel model EMME/2.\emmeindex

\section{Running a Simulation}
%
\subsection{Support for Production Runs}
The ``Run Manager'' \runmanagerindex is a set of user functionality
provided by the \file{run_manager} \runmanagerindex directories in
the \package{opus_core} and \package{opus_core.services} packages
provide support for specifying, starting, re-starting, and
processing ``production`` runs. \index{production runs} Such a
system is necessary to deal with the long run times for such a model
\modelsindex system, and the large amount of data each run produces.
A 30-year simulation of the Puget Sound Regional Council's (PSRC)
\psrcindex dataset with their EMME/2 \emmeindex travel model, for
instance, takes about 5 days \index{simulation!run time} to complete
and creates about 14 GB of cache data. \index{cache!size of}
\index{simulation!space used by}  These numbers are for a computer
with dual 3.2 GHz Intel\textregistered{} \intelindex
Xeon\texttrademark{} \xeonindex processors and 4 GB of RAM running
Windows XP Professional. \windowstestindex  The PSCS dataset
\datasetindex contains about 800,000 gridcells, 1.3 million
households, and 1.8 million jobs in the initial year. Each UrbanSim
year takes about 1.5 hours to simulate, and each travel model run,
done every 5 years, takes around 12 hours to simulate.

Here are some of the things Run Manager \runmanagerindex helps us do:
\begin{itemize}
  \item Define a run configuration, including what models \modelsindex to run, what years to
  run, etc.
  \item Start a production run.
  \item Monitor the status of all production runs.
  \item Re-start a production run when something goes wrong.
  \item Compute indicators on the results of a production run, including runs
  that are still simulating.
  \item Inspecting specific dataset values from any year of the simulation,
  as when diagnosing problems.
  \item Manage consistency regarding the availability and status of production runs.
\end{itemize}

\subsection{Run Management}
\label{sec:run-manager}
%
Opus contains a set of scripts that simplify the
process of starting and restarting a set of simulation runs.

At the moment, these scripts are a set of command line applications, or tools, in the
\file{opus_core/tools} directory.  They use a database as the central
repository for coordination and context infomation for run managment, by
default called \verb|services|.  If you don't have this database, create it
(you need to do this from within the \file{opus_core/tools}
directory):
\pythonindex
\begin{verbatim}
python create_services_database.py
\end{verbatim}
This creates the database named \verb|services| on the \verb|localhost|
database server using the user name and password specified in the
\verb|MYSQLUSERNAME| \mysqlusernameindex and \verb|MYSQLPASSWORD| \mysqlpasswordindex environment variables. \environmentvariablesindex To use a
different host for the database server, include the \verb|--hostname|
option. To use a different database name, include the \verb|--database|
option:
\pythonindex
\begin{verbatim}
python create_services_database.py --hostname myhost.mydomain --database
myservices
\end{verbatim}

All scripts described below print a help message when called with the
\verb|-h| or \verb|--help| option.

\subsubsection{Start a simulation using a Python Dictionary Configuration}
\index{start a simulation}

The configuration for a simulation is specified by a configuration object.
Previously, a configuration was specified as a Python dictionary, and we
describe the older option in this section.  We are transitioning toward
specifying configurations using an xml project file for use with the Opus
GUI--- see Section \ref{sec:start-simulation-xml} for information on
starting a simulation using an xml configuration but still from the command
line.  In both cases, the configuration is used to specify different parts
of the simulation, such as the years in which to run UrbanSim, what
UrbanSim models \modelsindex to run each year, what types of development
projects exist, how to configure each type of development project,
etc. See, for instance, the run configuration in
\verb|seattle_parcel/configs/baseline.py| (Python dictionary version) or
\verb|seattle_parcel/configs/seattle_parcel.xml| (xml version).

To use the dictionary version of a configuration, the start_run tool
requires that the referenced configuration Python module either (a) defines
a class (e.g. SubsetConfiguration) in a file whose name is
subset_configuration.py, or (b) defines a run_configuration object that is
the configuration object.  In the first case, the class name is the
CamelCase version of the lowercase_with_underscores file name.  \emph{(The
  preferred method is to define a class.)}

To start a simulation using a dictionary-based configuration, use the
script \verb|start_run| with the desired configuration.  First change to 
the directory \verb|opus_core/tools| and execute:
\begin{verbatim}
python start_run.py -c psrc.configs.subset_configuration
\end{verbatim}

Note that the configuration here is specified as it would be in a Python
``import'' statement.

If the \verb|services| database was created using the \verb|--hostname| and/or
\verb|--database| option, you will need to include these options in
\file{start_run.py} as well.  Use the \verb|--help| option to see
\verb|start_run|'s possible command line parameters.

The \verb|creating_baseyear_cache_configuration| entry in the configuration
contains the information specifying the location of the baseyear inputs.  
If \verb|cache_from_mysql| is \verb|True|, the inputs are taken from
the MySQL database specified in the \verb|input_configuration| entry. 
Otherwise, the inputs are taken from the baseyear cache specified in the
\verb|baseyear_cache| entry.  When the inputs are taken from MySQL, they are
used to create a baseyear cache stored in the directory specified by the
\verb|cache_directory| entry, or, if that entry is missing, from an directory
whose name encodes the date-time of the simulation request and is created in the
directory specified by the \verb|cache_directory_root| entry.  

For large amount of data, copying from MySQL to a baseyear cache takes much
longer than creating a new baseyear cache from an existing baseyear cache.  

Option \verb|--directory-to-cache| \index{cache!getting input from existing cache}
can be used to turn off the caching. It takes as argument the directory name
from which data are copied to the new basyear cache \baseyearcacheindex. With
the option \verb|--years-to-cache| one can control specific years to be
copied. The option takes as argument any python \pythonindex expression that returns a list
of years. These two options overwrite entries in the configuration that control
this behaviour.

\subsubsection{Start a simulation using an XML Configuration}
\label{sec:start-simulation-xml}

To start a simulation using an xml configuration from the command line, use
the script \verb|start_run| with the desired configuration.  Change to the
directory that holds the Opus source code and execute:
\begin{verbatim}
python opus_core/tools/start_run.py -x seattle_parcel/configs/seattle_parcel.xml 
       -s Seattle_baseline
\end{verbatim}
(typed all on one line).

Notice that the \verb|-x| option takes a path to the file with the xml
configuration.  Since xml configurations can hold multiple scenarios, the
scenario name must also be specified using the \verb|-s| option.

This same xml configuration can also be used with the Opus GUI ---
documentation on the GUI is forthcoming.

\subsubsection{Restart a simulation}
\index{restart a simulation}
If you halt a run or it fails, you can restart it at the beginning of any year.
To restart the run with \verb|run_id| 42 at the beginning of year 2005, do:
\pythonindex
\begin{verbatim}
python restart_run.py 42 2005
\end{verbatim}
Again, options \verb|--hostname| and \verb|--database| must be included
if non-default values are used.  Note that the above command will delete any
simulation cache \simulationcacheindex directories for years 2005 onward, since
this information is no longer valid once the simulation is restarted at the
beginning of 2005.

If the 2005 travel model failed and you want to restart in 2005 but not re-run
the UrbanSim models, \modelsindex use the \verb|--skip-urbansim| option:
\pythonindex
\begin{verbatim}
python restart_run.py 42 2005 --skip-urbansim
\end{verbatim}

If the 2005 travel model succeeded, and thus wrote its output to the 2006
simulation cache \simulationcacheindex directory, but you need to restart in year 2006, use the
\verb|--skip-cache-cleanup| option:
\pythonindex
\begin{verbatim}
python restart_run.py 42 2006 --skip-cache-cleanup
\end{verbatim}

\subsubsection{Create Baseyear Cache}
\label{sec:run-manager-baseyearcache}
%
One can explicitely create a baseyear
cache \baseyearcacheindex\index{cache!getting input from existing cache} from
the base year database which can be then used for simulation runs. The script
needs a configuration module passed as an argument. The services database is
not used by the script. If the desired module is located at
\file{psrc/configs/subset_configuration.py} under one of the paths found in the
PYTHONPATH \pythonpathindex environment variable, \environmentvariablesindex the command
\pythonindex
\begin{verbatim}
python create_baseyear_cache.py psrc.configs.subset_configuration
\end{verbatim}
caches the database defined in the \verb|subset_configuration| configuration into a basyear
cache. \baseyearcacheindex The \file{create_baseyear_cache} file is located in
the \file{opus_core/tools/} directory, and the above command should be
run from the same directory. An option \verb|--cache-directory|
can be used to pass the directory to be cached into. Alternatively, this can be
specified in the configuration as an entry \verb|'cache_directory'|. See also
Section~\ref{sec:configuration} for configuration and cache control options.

\subsection{What Happens When Running a Simulation?}
\label{sec:run-manager-tasks}
Here are the steps that occur when you start a run via the \verb|start_run|
script:

\begin{itemize}
  
\item \index{cache!unrolling development_event_history}\index{Unrolling 
development_event_history}\index{development_event_history!unrolling} If 
required, it copies all of the tables from the baseyear database specified by 
the configuration into the baseyear cache \baseyearcacheindex and then creates a
set of pre-baseyear \verb|gridcells| tables by unrolling the
\verb|develoment_event_history| data.

The unrolling processes all records in the \verb|development_event_history| 
table.  It starts with the most current year of records, and moves backward in 
time.  For this description, assume that the baseyear is 2000.  The unrolling 
process first loads the gridcell dataset for year 2000.  For each 
\verb|develoment_event_history| record with \verb|scheduled_year| for the prior 
year (e.g. for 1999) it substracts from the gridcells any development by that 
event (e.g. \verb|residential_units|).  Values that go negative are set to 
zero.  Once all events for this year have been ``undone'', the unrolling 
process writes the modified gridcell dataset to the prior year, e.g. to 1999. 
This process is repreated for each year with data in 
\verb|develoment_event_history|.

If you only want 10 years of data to be unrolled, only put those years of data
into your \verb|develoment_event_history| table.

\item Alternatively, it copies files from a previously used baseyear cache into
the baseyear cache for the current run (including the

\item It adds a row to the \verb|run_activity| table in the \verb|services|
database, using a new \verb|run_id| value unique to this run. In order to help
match runs with their cache directory, the name of the cache directory begins
with the \verb|run_id| value, e.g. \file{run_342.2006_04_25_09_40}.

\item For each year in the set of years specified in the configuration:
  \begin{itemize}
  \item Stores to the cache a pickled version of the configurations for that
  year.
  \item Forks a new process to run the set of UrbanSim models \modelsindex for this year.
  The set of models \modelsindex to be run is specified by the configuration. Using a
  separate process helps reduce memory usage, \index{Memory!Using separate
  process} and helps reduce the impact of problems such as memory leaks.  This
  process writes a log file named, e.g., \file{year_2003_log.txt}.
  \item Run the EMME/2 \emmeindex travel model for this year, if specified by the travel
  model's configuration.  The set of steps to run for the travel model is fully
  specified by the \verb|'models'| \modelsindex section of the
  \verb|'travel_model_configuration'|. Typically, it includes a preparation step
  that prepares travel model inputs from the UrbanSim data, a step that
  actually runs the travel model, and one or more steps to puts the results of
  the travel model into the \verb|travel_data| dataset for the current year, so
  the data is visible to the UrbanSim models \modelsindex in that and following years.  Each
  model step is done in a separate process.  The log for the running of the
  model is written to a file named, e.g., \file{emme2_2005_log.txt}. \emmeindex
  \end{itemize}
  \item The run activity table includes status information about the run.  If
  the simulation succeeds, it will add another row to the run activity
  indicating it is done.  If the simulation fails, it will add a row indicating
  that.  The run activity also contains a copy of the configuration, which is
  used when restarting a run.
  \item Whenever a row is added to the \verb|run_activity| table, a row is
  either added or updated in the \verb|available_runs| table in the
  \verb|services| database. This table has a single row per run and records the
  row's current state and information.
\end{itemize}

\section{Configurations}
\label{sec:configuration}
%
A run configuration is a specification of what parts to use in the simulation
and how to configure each part.  Parts include the list and order of models \modelsindex to
run, where to get the input values, where to store the output data, what years
to simulate, what tables to store into the UrbanSim baseyear and simulation
caches \baseyearcacheindex\simulationcacheindex, etc. Each of these parts in
turn may be configurable.  The configuration for a travel model, for instance,
may specify which values to use from the land use models, \modelsindex and what travel model
results to extract in order to use them in the land use models. \modelsindex

Configurations are used in many places in Opus.  Typically, they are specified
via a Python \pythonindex dictionary that then is used to create an instance of the
\class{Configuration} class.

\subsection{Run Manager Configuration}
\label{sec:run-manager-configuration}
%

UrbanSim can be started via the ``Run Manager'' \runmanagerindex (see
Section~\ref{sec:run-manager}) which is controlled by a user-defined
configuration. The following code contains a fully specified configuration
that influences behaviour of the run manager. \runmanagerindex Mandatory entries and default
values for optional entries are marked in the comments. The actual values for
the listed entries are only examples.

\emph{Note: we are in the process of splitting the following configuration
information into separate parts.  Once we are done with that refactoring,
we will update the documentation to the new arrangement.}

\mysqlhostnameindex\mysqlpasswordindex\baseyearcacheindex
\begin{verbatim}
from opus_core.configurations.database_configuration import DatabaseConfiguration
from opus_core.configurations.baseyear_cache_configuration \
    import BaseyearCacheConfiguration
    
from urbansim.configurations.creating_baseyear_cache_configuration \
    import CreatingBaseyearCacheConfiguration


run_configuration = {
    'model_system':'urbansim.model_coordinators.model_system', # mandatory
    'description':'baseline with travel model',      # default: 'No description'
    'cache_directory':'d:/urbansim_cache/',	         # mandatory
    'creating_baseyear_cache_configuration': CreatingBaseyearCacheConfiguration(
        # default: 'urbansim_tmp'+random string
        cache_directory_root = 'd:/urbansim_cache',

        # mandatory
        cache_scenario_database = 'urbansim.model_coordinators.cache_scenario_database',

        cache_from_mysql = False, # default: True

        # mandatory if 'cache_from_mysql' is False
        baseyear_cache = BaseyearCacheConfiguration(
            # mandatory for this block
            existing_cache_to_copy = 'd:/urbansim_cache/run_397.2006_05_23_18_21',
            
            # default: all years in 'existing_cache_to_copy'
            years_to_cache = range(1996,2001)
            },

        tables_to_cache = [ # default: []
            'gridcells',
            'households',
            'jobs',
            'zones'
            ]

        tables_to_cache_nchunks = { # default: each table defaults to 1
            'gridcells':2,
            },

        tables_to_copy_to_previous_years = { # default: no copied tables
            'development_type_groups':1996, # table name and year to put it in
            'development_types':1996,
            'development_type_group_definitions':1996,
            'urbansim_constants': 1996,
            },
        ),
\end{verbatim}

% long script split into two parts since it won't fit on one page

\begin{verbatim}
    'input_configuration': DatabaseConfiguration(    # mandatory
        host_name = os.environ['MYSQLHOSTNAME']      # mandatory
        user_name = 'urbansim',                      # mandatory
        password = os.environ['MYSQLPASSWORD'],      # mandatory
        database_name = 'PSRC_2000_baseyear',        # mandatory
        )
    'output_configuration': DatabaseConfiguration(   # default: No output
                                                     #     configuration
        host_name = os.environ['MYSQLHOSTNAME']      # mandatory for this block
        user_name = 'urbansim',                      # mandatory for this block
        password = os.environ['MYSQLPASSWORD'],      # mandatory for this block
        database_name = 'PSRC_2000_output',          # mandatory for this block
        },
    'base_year': 2000,                               # default: read from table
                                                     #     'base_year' in
                                                     #     'db_input_database'
    'years': (2001, 2030),                           # mandatory
}
\end{verbatim}
The \verb|'model_system'| entry is the full Opus path to the model system that
will be used by the run manager to run/estimate a set of models.

The \verb|'cache_scenario_database'| entry is the full Opus path to the class to use to
create a baseyear cache from the baseyear data in the MySQL database.  The
\verb|'urbansim.model_coordinators.cache_scenario_database'| version creates  both the
baseyear cache, and unrolls the gridcell data to populate prior years with the
gridcell dataset (see Section~\ref{sec:run-manager-tasks}).

Entry \verb|'creating_baseyear_cache_configuration'| contains the configuration
for creating the baseyear cache.

Entry \verb|'cache_directory_root'| is the root directory where data should be
cached during processing. The actual cache directory is created by adding
the run number and date-time string to this directory.

The \verb|'input_configuration'| is a DatabaseConfiguration object
that determines the MySQL \mysqlindex database with the base year
data.

Entry \verb|'output_configuration'| is a DatabaseConfiguration
object that determines the MySQL \mysqlindex database into to which
to write any database tables related to the results of the
simulation run.  Now that indicators are computed from the attribute
cache, the output database is only needed if you wish to use the SQL
indicators.  Before starting the simulation, the run manager will
remove any tables in the output_database, so be sure it doesn't
contain information you want to keep.

Entry \verb|'years'| determines for what
years the simulation should run as a tuple with first and last year to run.

By default, the run manager \runmanagerindex caches all tables from the input database into the
binary baseyear cache \baseyearcacheindex on which then the simulation runs. If
only selected tables should be cached, they can be put into
\verb|'tables_to_cache'|. Note that the simulation itself then does not use the
database anymore, all data are retreived from baseyear cache \baseyearcacheindex
and written to the simulation cache \simulationcacheindex.  That means, if the
entry \verb|'tables_to_cache'| is used, the user must ensure that it contains
all tables that are used by the simulation.

If a database table is so large that Python \pythonindex runs out of memory when copying it
to cache, you can reduce memory usage (but increase the time it takes to cache
the data) by increasing the number of ``chunks'' in which the dataset's \datasetindex
attributes are read from the table. \index{Memory management!When caching data
from input store}\index{Memory management!tables_to_cache_nchunks@\verb|tables_to_cache_nchunks|} By
default, all attributes of a table are read in a single chunk. Setting the
\verb|'tables_to_cache_nchunks'| configuration for a model will tell the caching code
to use that many chunks.  For instance, if a dataset has 11 attributes, setting
\verb|'tables_to_chunk_nchunks'| to 3 will use three chunks loading 4, 4, and 3
attributes, in each chunk.

For big tables, the caching process can be a very time-consuming task. Often
the baseyear cache \baseyearcacheindex is available from previous runs. Thus,
one can set the entry \verb|'cache_from_mysql'| \mysqlindex to False and define the
\verb|'baseyear_cache'| \baseyearcacheindex block. The directory with the already cached data
should be put into the entry \verb|existing_cache_to_copy|. The run manager \runmanagerindex then
copies data from that directory into the baseyear cache \baseyearcacheindex for this run. If you
want to copy only selected years, they can be specified in the entry
\verb|years_to_cache| as a list of those years; by default all years are
copied. Note that this behaviour can be alternatively controlled directly from
the command line (see description of \verb|start_run| in~\ref{sec:run-manager})
which has priority over entries in this configuration.

The \verb|'tables_to_copy_to_previous_years'| entry is used when a
lag variable needs to compute data for before the base year, and that
computation requires some of the "invariant" data that was copied from the
baseyear database into the baseyear cache.  If this is the case, add those
database tables to the list of tables in the configuration's
\verb|'tables_to_copy_to_previous_years'| entry, and indicate the year to which
to copy the tables.  In general, it is safe to copy the tables to the earliest
year created by the unroll gridcell process.  You can determine what this year
is by examining the year directories created in your baseyear cache.
\emph{(Note: we plan to change this to a better design.)}

There are several run manager \runmanagerindex configurations in Opus. See for example the
directory \file{psrc/configs} for configuration of different PSRC \psrcindex runs.

\subsection{Model System Configuration}
\label{sec:model-system-configuration}
%
If one would pass the above configuration to the run manager, \runmanagerindex it would perform
steps as described in Section~\ref{sec:run-manager-tasks}, but no models \modelsindex would
be run. The configuration should in addition contain entries that control what
models, \modelsindex in what order and with what input and output should be run. It
determines the behaviour of the class \class{ModelSystem}. UrbanSim basic
configuration of the model \modelsindex system can be found in the file
\file{urbansim/configs/general_configuration.py} as an example.

The set of models \modelsindex to run is specified by the entry ``models''. \modelsindex It is a list of
user-defined model names. The order in this list also specifies the order in
which they are run. A production run of UrbanSim consists by default of
following models: \modelsindex
\begin{verbatim}
run_configuration['models'] = [
        "prescheduled_events",
        "events_coordinator",
        "residential_land_share_model",
        "land_price_model",
        "development_project_transition_model",
        "residential_development_project_location_choice_model",
        "commercial_development_project_location_choice_model",
        "industrial_development_project_location_choice_model",
        "development_event_transition_model",
        "events_coordinator",
        "residential_land_share_model",
        "household_transition_model",
        "employment_transition_model",
        "household_relocation_model",
        "household_location_choice_model",
        "employment_relocation_model", 
       {"employment_location_choice_model": {"group_members": "_all_"}},
        "distribute_unplaced_jobs_model"
   ]
\end{verbatim}
Note that the list can contain a particular model multiple times if that model should
run multiple times within one year, such as the ``events_coordinator'' or
``residential_land_share_model'' in the above list.

We can also define a situation when the same model should be run on different subsets of 
a dataset, so called model group\index{model group}. Then we can give the names of the group members to be run, or 
just configure the model group to be run on all subsets. This is the case of ``employment_location_choice_model''
in the list above (see Section~\ref{sec:model-controller-configuration} for further details).
 
By default, the \class{ModelSystem} runs the method
\method{run()} of the listed models. \modelsindex Each entry in this model list can be
alternatively a dictionary containing one entry: The name of the entry is the model
name, the value is a list of model methods to be processed. Thus, one can combine
estimation and simulation of different models. \modelsindex

In addition (or alternatively), the configuration can contain an entry
``models_in_year''. \modelsindex It is a dictionary where keys are years. Each value is
expected to be such list of models \modelsindex as above. In each year, it is checked if
``models_in_year'' \modelsindex (if it is present) contains that year. If it is the case,
its list of models \modelsindex is run, instead of the global set of models. \modelsindex This allows
users to set different set of models \modelsindex for different years, for example an
additional model can be run only in the first year, or last year.

For each entry in the model \modelsindex lists above there must be a corresponding entry in
the ``controller'' configuration which specifies how models \modelsindex are initialized,
what methods to run and what arguments should be passed in. This will be
described in Section~\ref{sec:model-controller-configuration}.

Furthermore, the configuration can contain the following entries (the given
values are defaults set by our system):
\begin{verbatim}
{
    'datasets_to_cache_after_each_model':[],
    'flush_variables': False,
    'seed':0,
    'debuglevel':0
}
\end{verbatim}

The entry \verb|'datasets_to_cache_after_each_model'| specifies names of datasets \datasetindex that
are flushed from memory to simulation cache \simulationcacheindex at the end of each model
run. \index{Memory management!Flushing attributes}\index{Memory management!datasets_to_cache_after_each_model@\verb|datasets_to_cache_after_each_model|} This reduces the memory
usage, but can increase the run time. We recommend to put datasets in this list
that contain huge amount of data. E.g. UrbanSim sets this entry to ['gridcell',
'household', 'job'].

\verb|'flush_variables'| can further decrease the memory usage. \index{Memory management!flush_variables@\verb|flush_variables|} If it is True, after each variable
computation all dependent variables are flushed to simulation
cache \simulationcacheindex, regardless to what dataset \datasetindex the variables belong to.
Nevertheless, it increases the run-time considerably.

Entry \verb|'seed'| specifies the seed of the random number generator that is set at
the beginning of each simulated year. It is passed to the \module{numpy} \numpyindex
function \method{seed()} and therefore it should be a tuple with two integer
values. If both values are 0, the function generates a pseudo-random seed
from the current time.

\verb|'debuglevel'| controls the amount of output information.

Models \modelsindex usually need various datasets \datasetindex to run with. They are specified in the
configuration entry \verb|'datasets_to_preload'|. For example,
\begin{verbatim}
{
'datasets_to_preload': {
        'gridcell':{"id_name": "grid_id"},
        'household':{}
}
\end{verbatim}
It is a dictionary that has dataset \datasetindex names as keys. Each value is again a
dictionary with argument-value pairs that are passed to the corresponding
dataset \datasetindex constructor. \class{ModelSystem} \modelsindex creates those datasets \datasetindex at the
beginning of each simulated year and they are accessible to the models \modelsindex
definition in the controller through their names (see
Section~\ref{sec:model-controller-configuration} for details). One should put
here all datasets \datasetindex that will be passed as arguments to the model constructors
or to the model methods to be processed.

If you do not want to make any output to cache, for example in estimation mode, set
\begin{verbatim}
{
    'low_memory_mode': False
}
\end{verbatim}
This should be only used when you're aware what you're doing.
It suppresses any cache writing during the processing (after each model, as
well as at the end of each year) and will not work for a simulation
over multiple years. Also, the memory usage can increase considerably.


\subsection{Models Configuration}
\label{sec:models-configuration}
%
The run configuration can contain an entry \verb|'models_configuration'| \modelsindex which can
include any information specific to models \modelsindex or common to a set of models. \modelsindex The
value of this entry is a dictionary.  Model specific information would be
included in an entry of the same name as the model name used in the entry
\verb|'models'| (see Section~\ref{sec:model-system-configuration}). The
\class{ModelSystem} \modelsindex class makes this information available to the controller
by creating two local variables: \verb|'models_configuration'| \modelsindex (containing the
value of \verb|'models_configuration'| \modelsindex and available to all models) \modelsindex and
\verb|'model_configuration'| \modelsindex (available to each model at the time of its
processing and containing information for this model). See the variable
\verb|'models_configuration'| \modelsindex in the file \file{urbansim/configs/general_configuration.py} for an
example how UrbanSim configures models. \modelsindex

If a model is running out of memory, you can add a \verb|'chunk_specification'|
to that model's configuration. \index{Memory management!Using multiple chunks per model}
\index{Memory management!chunk_specification@\verb|chunk_specification|} This instructs that model to run in
multiple chunks, each containing a subset of the records (e.g., agents) to
process. This specification can either limit the chunk to contain at most a
given number of records:

\begin{verbatim}
    'chunk_specification':{
        'records_per_chunk':300,     # Put at most 300 records in a chunk
        per chunk },
\end{verbatim}

or specify the number of chunks to use regardless of the number of records:

\begin{verbatim}
    'chunk_specification':{
        'nchunks':10,                # Use 10 chunks
        },
\end{verbatim}

\subsection{Model Controller Configuration}
\label{sec:model-controller-configuration}
%
Each model that is included in the configuration entry \verb|'models'| \modelsindex must have a
controller entry in the \verb|'models_configuration'| \modelsindex entry described in
Section~\ref{sec:models-configuration}.  More specifically, the model specific
section of \verb|'models_configuration'| \modelsindex is expected to contain an entry \verb|'controller'|
for each model. For example, a controller specification for the model
specified by the name \verb|'land_price_model'| would be contained in\\
\verb|run_configuration['models_configuration']['land_price_model']['controller']|.

If a model is specified as a model group \index{model group} it is possible to define a member specific controller, called 
{\em member_name} + '_' + {\em model_name}, e.g. \verb|'home_based_employment_location_choice_model'|.
When choosing the right controller, the \class{ModelSystem} checks for the member specific name. If it is not found,
it uses the group name, in this example \verb|'employment_location_choice_model'|.

The value of this controller entry is a dictionary with a few well-defined entries:
\begin{description}
\item["import"] A dictionary where keys are module names and values are names
  of classes to be imported.
\item["init"] A dictionary with a mandatory entry "name". Its value is the
  name of the class (or class.method) that creates the model. It can be the
  name of the model class itself.  Or, if the model is created via   a
  method e.g. \method{get_model()} of a class \class{MyModelCreator}, it would be
  given as "MyModelCreator().get_model".

  Optional entry "arguments" specifies arguments to be passed into the
  constructor. It is given as a dictionary of argument names and values. All
  values are given as character strings and are later converted by
  \class{ModelSystem} to python \pythonindex objects. If an argument value is suppose to be
  a character string object, it must be given in double quotes, e.g.
  "'my_string'".
\end{description}
If the model in the 'models' entry of the configuration is specified as model group\index{model group}, the controller must contain 
an entry
\begin{description}
\item["group_by_attribute"] Its value is a tuple of a grouping dataset name and grouping attribute (see Sec.~ref{sec:model-group}).
They define the specific kinds of 
subsets of agents on which this model can be run. This dataset must be contained in the 
\verb|datasets_to_preload| entry of the configuration. For example, in the controller of the 
``employment_location_choice_model'' this entry is
\verb|('job_building_type', 'name')|, since the attribute 'name' of the dataset 'job_building_type' contains the various
building types of jobs for which we want to run the model, i.e. 'commercial', 'governmental', 'industrial' and 'home_based'.
If the 'group_members' entry (of the 'models' entry of the configuration) for this model is equal to '_all_', the model runs 
for all values found in this dataset. The  'group_members' can also be a list specifying explicitely for which types the model 
should be run. 
\end{description}

The \class{ModelSystem} class evaluates the given imports and creates an
instance of the model by processing the \verb|'init'| entry. The remaining entries
below are related to specific methods of the created model instance.  As
mentioned in Section~\ref{sec:model-system-configuration}, models \modelsindex
that are listed in the \verb|'models'| entry of the run configuration can be also
specified using a list of methods to be processed. If the list is not given, a
method \method{run()} is assumed to be the only method to be processed. The
\class{ModelSystem} iterates over the set of methods. It first processes a
``preparation'' method (if required) and then the method itself. For this purpose,
the controller should contain the following entries:

\begin{description}
\item['prepare_for_...'], where \verb|...| is the the method
to be processed, e.g. \verb|'prepare_for_run'| is the method to call to prepare
to run. This configuration entry is a dictionary with an
optional entry \verb|'name'| giving the name of the preparation method. If \verb|'name'| is
missing, the method name is assumed to be the same as this entry name. Optional
entry \verb|'arguments'| specifies arguments of this method (see \verb|'arguments'| in \verb|'init'|
above). Optional entry \verb|'output'| defines the name(s) of the output of this
method.  It can be then used as an input to other methods or models. The entry
\verb|'prepare_for...'| is optional and if it's missing, no preparation procedure is
invoked. There can be as many \verb|'prepare_for...'| entries as there are
methods specified.
\item[{\it procedure}] The procedure name must match to the method names given
in \verb|'models'| (there must be one entry per method), or be called \verb|'run'| if no
methods are specified in \verb|'models'|. It is a dictionary with optional arguments
\verb|'arguments'| and \verb|'output'| (see above).
\end{description}

The entry \verb|'arguments'| in the above items can contain any character
strings that are convertable (using python's \verb|eval()|) to python \pythonindex objects,
including python \pythonindex expressions. They must be objects that are known to the
\class{ModelSystem}, for example datasets \datasetindex that are defined in
\verb|'datasets_to_preload'| (described in
Section~\ref{sec:model-system-configuration}), since those are created prior to
the simulation. They can be called either by the dataset \datasetindex name, or using
\code{datasets['{\it name}']}. Also, the \verb|model_configuration| and
\verb|models_configuration| objects described in
Section~\ref{sec:models-configuration} can be used in \verb|'arguments'|. Other
objects that \class{ModelSystem} provides are \verb|sql_storage|
(\class{Storage} object for the input database), \verb|cache_storage|
(\class{Storage} object for the simulation cache \simulationcacheindex storage
in the simulated year), \verb|base_cache_storage| \baseyearcacheindex (\class{Storage} object for
the baseyear cache \baseyearcacheindex storage in the base year),
\verb|model_resources| (all preloaded datasets as an object of
\class{Resources}), \verb|year| (simulated year), \verb|resources|
(Configuration passed into the simulation), \verb|dataset_pool| (object of class \class{DatasetPool} pointing to the current
dataset pool).  If you are using any class names
as arguments, you need to make sure, that those classes are known to the
\class{ModelSystem}, e.g. by putting the appropriate import statement into the
\verb|'import'| section of the controller.

Here is an example of a controller settings for the land price model in UrbanSim:
\modelsindex\datasetindex
\begin{verbatim}
run_configuration['models_configuration']['land_price_model']['controller'] = {
    "import": {"urbansim.models.corrected_land_price_model":
                                                "CorrectedLandPriceModel",
              },
    "init": {"name": "CorrectedLandPriceModel"},
    "prepare_for_run": {
        "arguments": {"specification_storage": "base_cache_storage",
                      "specification_table": "'land_price_model_specification'",
                      "coefficients_storage": "base_cache_storage",
                      "coefficients_table": "'land_price_model_coefficients'"},
        "output": "(specification, coefficients)"
        },
    "run": {
        "arguments": {"n_simulated_years": "year-resources['base_year']",
                      "specification": "specification",
                      "coefficients":"coefficients",
                      "dataset": "gridcell",
                      "data_objects": "datasets" ,
                      "chunk_specification":"{'nchunks':2}"
                     }
           }
   }
\end{verbatim}
Note on an implementation of model group\index{model group}: A constructor of 
a model group, must take as its first argument an object of class \class{ModelGroupMember} (Sec.~\ref{sec:model-group}). 
The controller should though ignore this argument, since the \class{ModelSystem} automatically takes care of creating this object 
and passing it to the model constructor.

\section{Output}
%
\subsection{The Output Database}

The simulation reads and writes all of its data from the simulation cache.  It 
does not directly read or write to any database. 
If you wish to move data from the simulation cache to a MySQL database, use the
\verb|do_export_cache_to_sql_database.py| tool located in
\file{opus_core/tools}. 

\subsection{File-Based
Cache}\label{cache} \baseyearcacheindex\simulationcacheindex

For a variety of reasons, described below, UrbanSim uses two forms of file-based
caches:
\begin{description}
\item[baseyear cache] \baseyearcacheindex
Stores the attribute values read from the baseyear
database.  This includes data for the base year, and data for years before the
base year (unrolled from data in the baseyear).  This data is not modified
during the simulation.

\item[simulation cache] \simulationcacheindex
Stores the attribute values for datasets used during the simulation.
\end{description}
While the distinction between the read-only baseyear cache \baseyearcacheindex and read-write
simulation cache \simulationcacheindex is useful understanding the system, during simulation both
caches form a single cache that UrbanSim datasets \datasetindex and models \modelsindex mine to gather data
for the current and prior years.

\subsubsection{Reasons to Use Cache}
The UrbanSim simulations run on top of a file-based cache.  The initial reason
for the cache was to improve performance, and has turned out to be very useful for
many features:

\begin{itemize}
  \item The cache is much faster to access than the MySQL \mysqlindex database, at least,
  so using the cache dramatically speeds up our processing.
  \item In many cases, a simulation will require more data than can fit in RAM.
  The simulation cache \simulationcacheindex provides a place to store computed
  or non-computed values so that we only need to keep in memory just enough
  data for the computation at hand. \index{Memory management!Caching data}
  \item Lag variables use the baseyear cache \baseyearcacheindex and simulation
  cache \simulationcacheindex to quickly get data from prior years, often
  without having to re-compute it.
  \item When preparing for a simulation, UrbanSim uses the data from lag tables
  (see~\ref{lag-tables}) in the baseyear to create historical data in the
  baseyear cache \baseyearcacheindex for years before the base year.  This
  allows the simulation to run without any distinction between historical and
  predicted data.
  \item Lag variables on gridcell data need gridcell information from prior
  years.  UrbanSim can use the information in the
  \verb|development_event_history| table to create gridcell data for before the
  baseyear by ``unrolling'' the development events from before the base year.
  \item When looking for a primary attribute \primaryattributesindex for a given dataset, \datasetindex Opus
  looks first in the current year.  If the data is not in that year's cache,
  Opus automatically looks backward thru prior years until it finds it.  This
  allows Opus to only store to cache primary attributes \primaryattributesindex in the years in
  which they are modified, while still getting the performance and memory
  improvements by using the cache. \index{Memory management!Caching data}
  \item The information in the simulation cache \simulationcacheindex includes all
  of the intermediate variable values produced for every year.  This allows us
  to inspect inspect this data for diagnostic purposes.
  \item Our indicator framework mines the cache data to produces charts, tables
  and maps that help to diagnose runs as well as providing input to policy
  decisions.
\end{itemize}


\subsubsection{What is Written to Cache}
During a simulation, UrbanSim writes a variety of data to the simulation
cache. \simulationcacheindex These include:
\begin{itemize}
  \item Any dataset \datasetindex attribute read into memory or computed during the year.
  \item A log file for each year, containing anything written to the UrbanSim
  logger.
  \item A log file for the top-level process that runs each year of the
  simulation.
  \item Meta-data about the information in the simulation
  cache, \simulationcacheindex such as the ``version'' of each variable so that
  we only need to recompute variables \variablesindex when their inputs change.
\end{itemize}

You can adjust the frequence with which the simulation flushes dataset \datasetindex
attribute \attributesindex values to the simulation cache: \simulationcacheindex

\begin{itemize}
  \item After computing each year.  This is always done.
  \item After computing each model.  Set
  \verb|'datasets_to_cache_after_each_model'| in the run configuration to
  a list of dataset \datasetindex names you wish to cache. An empty list (default) causes
  no caching.
  \item After computing each variable. \variablesindex Set \verb|'flush_variables'| \variablesindex to \verb|True| in the
  run configuration.
\end{itemize}

\subsubsection{Deleting the File-Based Cache}

The \verb|delete_run| script in \file{opus_core/services/run_manager} directory
provides an easy way to delete cached run data while maintaining the
consistency of the services database.  This is useful, since a simulation can
produce multiple gigabytes of data.

To delete all data for run with \verb|run_id| 42, and remove that run's
information from the \verb|available_runs| table in the \verb|services|
database use:

\pythonindex
\begin{verbatim}
python delete_run.py --run-id 42
\end{verbatim}

To delete a set of years without removing the information from the
\verb|services| database, use the \verb|--years-to-delete| option.  This
option takes an arbitrary Python \pythonindex expression that creates a list of integers.
For instance, to remove the cached data for years 2001 through 2029 use:

\pythonindex
\begin{verbatim}
python delete_run.py --run-id 42 --years-to-delete range(2001,2030)
\end{verbatim}

\section{Converting the Base Year Data}
The base year database for the Opus UrbanSim~4 is almost the same as that used
by UrbanSim~3 (the Java version).  Instructions on converting the database are
available at
\mbox{\url{http://www.urbansim.org/opus/opus_manual/docs/scripts/converting_3_to_4.html}}.

\subsection{Lag Tables}\label{lag-tables}

The baseyear database may contain data from before the base year.  If you want
this data to be available to the simulation, put it into a lag table whose name
is the same as the non-lag table for that dataset except with \verb|_lag| at the
end, e.g. \verb|gridcells_lag|, and that contains a \verb|year| column indicating
the year for each row.  The first step of a simulation run copies this lag
data into the baseyear cache, \baseyearcacheindex so that it can be accessed by lag variables. \variablesindex

