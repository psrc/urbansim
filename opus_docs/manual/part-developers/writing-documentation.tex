% Copyright (c) 2005-2008 Center for Urban Simulation and Policy Analysis,
% University of Washington.  Permission is granted to copy, distribute and/or
% modify this document under the terms of the GNU Free Documentation License,
% Version 1.2 or any later version published by the Free Software Foundation;
% with no Invariant Sections, no Front-Cover Texts, and no Back-Cover Texts.
% A copy of the license is included in the section entitled "GNU Free
% Documentation License".

\chapter{Writing Documentation}

This chapter provides information and guidelines on writing Opus and
UrbanSim documentation.  This guide is written using \LaTeX{}. \latexindex
Other documentation includes a growing number of tutorials, and documentation on individual
indicators.  Tutorials can be written using a variety of formatters, 
including \LaTeX{} and wiki;
guidelines for writing tutorials is on our external wiki at
\url{http://www.urbansim.org/wiki/external/index.php/Contributing_Tutorials}.
Documentation for individual indicators is written using XML files;
guidelines for writing indicator documentation are given in Section
\ref{sec:writing-indicators}.

We use \LaTeX{} for the guide and other major documents,
since it produces elegantly formatted printable document, handles
mathematics well, and includes facilities for cross-referencing and
indexing.  In addition to a pdf version of the documentation written using
\LaTeX{}, \latexindex we use the latex2html \index{latex2html}\latexindex 
converter program to produce an html version.

\section{The Reference Manual and User Guide}

The Opus and UrbanSim 4 documentation in this guide follows the Python \pythonindex
documentation standards (or at least it's supposed to), and is written
using \LaTeX{} \latexindex and the Python \pythonindex documentation macros.
The latest version of this Guide (as a pdf file), and the Opus tutorial, is
automatically posted on the UrbanSim website at
\url{http://www.urbansim.org/} whenever a new version of any of the
manual's components is checked in to the CVS \cvsindex repository.  There is also an
html version produced using latex2html, \index{latex2html}\latexindex also produced by the same build
script.

The Guide includes a table of contents and a (currently rudimentary) index.
The Adobe \index{Adobe} pdf reader also includes a nice full-text-search tool that can,
of course, be used on this manual when reading the pdf version online.

%begin{latexonly}
\begin{seealso}
\seetitle[http://docs.python.org/doc/doc.html]{Documenting Python} {This
  document describes how to document Python using the \LaTeX{} macros
  developed for the purpose.}

\seetitle[http://docs.python.org/download.html]{Python Documentation
  Download page} {The download page provides links for downloading the
  Python documentation in various formats, in particular \LaTeX{} source.
  The macros in the \file{texinputs/} subdirectory there have been copied
  into the \file{docs/latex} directory (and occasionally modified) for use
  in formatting this Guide.  Other manuals there are also useful as
  examples of using the macros.}
\end{seealso}
%end{latexonly}
\begin{htmlonly}
See also:

Documenting Python \pythonindex (http://docs.python.org/doc/doc.html): This document
describes how to document Python \pythonindex using the \LaTeX{} \latexindex macros
developed for the purpose.

Python \pythonindex Documentation Download page (http://docs.python.org/download.html):
The download page provides links for downloading the Python \pythonindex documentation
in various formats, in particular \LaTeX{} \latexindex source.  The macros in the
\file{texinputs/} subdirectory there have been copied into the
\file{docs/latex} directory (and occasionally modified) for use in
formatting this Guide.  Other manuals there are also useful as examples
of using the macros.
\end{htmlonly}

\section{Using Python Macros}
\pythonindex

The Python \pythonindex macros include logical markup for the elements of the language
and related concepts.  Use the \LaTeX{} \latexindex macros defined in python.sty \pythonindex to
format source code, along with function, method, and file names, and so
forth.  This ensures a uniform appearance.  Also, the Python \pythonindex documentation
states that sometime in the future, the Python \pythonindex documentation is likely to
be converted to XML or some other structured markup, and using the logical
markup now will facilitate the transition.  Here are some useful examples
of such macros.  See the ``Inline Markup'' section of ``Documenting Python'' \pythonindex
for the complete list
(\url{http://www.python.org/doc/current/doc/inline-markup.html}).

\begin{verbatim}
\module{spam}
\file{parrot.py}
\function{setup()}
\method{append()}
\member{sproket}
\program{pkgtool}
\url{http://www.conglomocorp.com}
\kbd{Control-D}
\samp{import sys; sys.exit()}.
\code{sys.argv[0]}
\class{Exception}
\character{\e n}     % this is for \n
\samp{\program{python} \programopt{-m} \var{module} [arg] ...},
\keyword{kwd}
\exception{expt}
\class{Giraffe}
\pytype{string}           % built-in Python type
\UNIX    % the system
\Cpp{}
\end{verbatim}

In addition to the standard Python macros, we've redefined the ``package'' macro 
to also add the package to the index, and added another macro ``variable'' for 
formatting Opus variables and expressions.  These are defined in the file 
\file{opus_docs/manual/macros.tex}.


\section{Writing Email Addresses in Documentation}

The Reference Manual and User Guide is published on the UrbanSim website in
both pdf and html form.  To avoid serving as a spam magnet, please don't put
real email addresses in the document -- instead use circumlocutions of
some sort, e.g.\

\begin{verbatim}
info (at) urbansim.org
\end{verbatim}

\section{Indexing}

Terms should be indexed as documentation is written. Use \verb|\index{ ... }| and
related commands to accomplish this. There should be a space on either side of the
index command, and the command should not be placed anywhere a space should not exist
(such as between a word and a subsequent period or comma).

In order to encourage consistency and ease sweeping changes of indexed terms, there
are macros defined in the \file{indexcommands.tex} file. These macros for commonly
used---or complex---index terms should be taken advantage of whenever possible
rather than manually using the \verb|\index{ ... }| command. Care should be taken
that any new macros added to the file maintain the existing alphabetic ordering
to facilitate a manual look-up of index commands.

\section{Some Other Guidelines}

Here are a few other guidelines.  \emph{These are mostly noted as a result
  of noticing issues in the existing documentation \ldots later, we can
  have a more coherent list.}

\begin{itemize}

\item Use the \verb|\begin{verbatim}| environment to format larger blocks of
source code.

\item Use logical rather than physical markup, for example the
\verb|\emph{ ... }| environment for emphasis (italics), rather than
\verb|\em| or \verb|\it|.

\item Use the \verb|\ldots| macro to produce three dots (\ldots) indicating
  ellipses --- if you just write three periods in a row they are spaced too
  closely (for example: ...).

\item The latex2html \index{latex2html}\latexindex program handles the mathematical typesetting from
  \LaTeX{} \latexindex by including the math as a bitmap.  This is reasonable for
  actual equations, but looks a bit odd for simple symbols in the middle of
  a sentence.  There should be a switch to turn these into ordinary text,
  or in the extreme we could write alternate latex \latexindex or html macros.
  However, if this can't be fixed, avoid using the math typesetting
  environments if you don't need them.  For example, to produce a
  backslash, use \verb|\textbackslash| or \verb$\verb|\|$ rather than
  \verb|$\backslash$| (since in latex2html \index{latex2html}\latexindex the first two options produce a
  character, the third a GIF image).

\item Do not use underscores in \verb|.tex| filenames. Use hyphens instead (i.e. use `-' instead of `_').

\end{itemize}


\emph{Normally, a good heuristic is to copy the existing examples in the
  manual when writing new ones.  However, much of the \LaTeX{} here doesn't yet
  take advantage of these macros.  So look at the description linked above
  instead for now.}

\section{Writing Indicator Documentation}
\label{sec:writing-indicators}

Documentation on an individual indicator is contained in an XML file, to
make it convenient to browse to the indicator documentation on the web.

These XML files go in the \module{opus_docs} project, in the directory
\file{opus_docs/docs/indicators}.  The new indicator should also be added
to the \file{predefined_indicators.xml} file in an appropriate category.
(At the moment we don't support having multiple indicator directories,
e.g.\ for specific applications of UrbanSim, but we will probably add this
in the future.)

The indicator documentation is
linked from \url{http://www.urbansim.org/}.  There is also a link
to a web page, ``Reading Indicator Documentation,'' which briefly describes
each section of the indicator documentation.  It is intended for people who
are reading the documentation (as opposed to writing it).
Each section of the documentation for an individual indicator
also includes a link to the corresponding section of the ``Reading
Indicator Documentation'' web page.

The technical documentation for each indicator should be relatively neutral.  There is also
an experimental Indicator Perspectives framework, which allows different organizations
to present their perspectives on which indicators are important and how to interpret
them.  Support for this may be added to UrbanSim 4 in the future.  The mechanism is
outlined in the following paper:
Alan Borning, Batya Friedman, Janet Davis, and Peyina Lin,
``Informing Public Deliberation: Value Sensitive Design of Indicators for a
Large-Scale Urban Simulation,'' \emph{Proceedings of the 9th European
Conference on Computer Supported Cooperative Work}, Paris, September 2005.  (Available
from \url{http://www.urbansim.org/papers/}.)

The technical documentation for each indicator includes a definition,
related indicators, limitations, and other information.  The web display
also includes links to the Opus variable or primary attribute for that
indicator.

Required elements in the XML are marked as \emph{required}; if these
are missing a note in red (such as ``specification missing'') will
appear in the rendered web page. Other elements are optional; if
omitted the heading will still appear in the rendered web page, with
an appropriate note. The DTD file that formally specifies the syntax
of an XML indicator definition is \file{indicator-declaration.dtd};
the file \file{indicators.xsl} specifies how the indicator is
rendered on a web page when browsing.

\begin{description}

\item[Definition] \emph{(required)} A short, accurate definition of this
indicator.  In the rendered web page, this appears directly under the
indicator name (there isn't a separate heading ``definition'').

\item[Principal Uses] What are the principal uses of this indicator?  (This
element should say whether the indicator is useful for policy evaluation,
or mostly diagnostic.  An indicator is useful for policy evaluation if it
is useful in evaluating whether particular policy goals are supported or not by
different scenarios.  An indicator is useful for diagnosis if it has a
clear directionality given different test input scenarios.)

\item[Interpreting Results] Describes how to interpret the results from
this indicator, either from a diagnostic perspective, a policy perspective,
or both.

\item[Display Format]
\emph{(Appears as ``Units of Measurement and Precision'')} This
section provides a machine-readable specification of the default
display for the indicator data.  Since it is machine-readable, it
must use a set of XML elements, rather than being free-form English
text.  The first element should specify the units used, e.g.\
\verb|<units>square feet</units>| or
\verb|<units>persons/acre</units>|.  If the data is unitless then
this element should be \verb|<unitless/>| (this would be the case
for ratios, e.g.\ a vacancy rate).  The next element should specify
how numbers should be displayed, and how many decimal places of
precision should be used.  The options here are
\verb|<numberdigits="N"/>|, \verb|<percentage digits="N"/>|, or
\verb|<scientific digits="N"/>|, where N in each case is the number
of digits.  For percentages, this is the number of digits
\emph{after} the decimal place, so that if N=2 the value 0.08352222
would displayed as 8.35\%.

\item[Specification] \emph{(required)} A human-readable specification of
how the indicator can be computed.  This is not necessarily the same as how
it is actually computed.

\item[Limitations] Known limitations or problems.

\item[Related Indicators] A short discussion of related indicators.  If
this indicator could be computed as the composite of other indicators, list
them here, and describe the relation (see ``Jobs per capita'' for an
example).  Another example of using this category is in the ``Employment
density'', ``Household density'', and ``Population density'' indicators,
which all show different aspects of density.  However, if this indicator is
just related to others in its category, by virtue of being in that
category, then that information can probably be safely omitted.

\item[How Modeled]
This element should describe whether the value of this indicator is
determined by the interactions of UrbanSim's component models, \modelsindex or whether
it is exogenous (that is, is determined outside the operation of UrbanSim).
In some cases, the value of the indicator will be exogenous for the
regional geography, but not for sub-regional geographies.  If this element
is omitted, the default output is ``Not specified.  (In this case,
normally one can assume that the value of this indicator is modeled by
UrbanSim itself, as opposed to being exogenous, that is, coming from an
external source such as a control total.)''

\item[Indicator Source, Evolution, and Examples of Use]
Where did the definition and code for this indicator come from?  How
has its specification evolved?  If known, provide examples of using this
indicator (including citations).

\item[Source] This element gives a link to the source code for the
indicator if it is defined as an Opus variable.  If it is a primary
attribute, the attribute should be listed (with no link), followed by
``Opus primary attribute'' in parentheses.  (See for example the
documentation for ``Residential units''.)

\end{description}
