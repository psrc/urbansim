% Copyright (c) 2005-2008 Center for Urban Simulation and Policy Analysis,
% University of Washington.  Permission is granted to copy, distribute and/or
% modify this document under the terms of the GNU Free Documentation License,
% Version 1.2 or any later version published by the Free Software Foundation;
% with no Invariant Sections, no Front-Cover Texts, and no Back-Cover Texts.
% A copy of the license is included in the section entitled "GNU Free
% Documentation License".

\chapter{Programming in Opus and UrbanSim}
\label{chapter:programming}

This chapter provides some practical details on developing programs in Opus
and UrbanSim, including how to start a new Opus package, design guidelines,
and Python coding standards.

\section{Creating New Opus Packages}
\label{sec:create-opus-package}
\index{opus packages!creating new package}

This section provides information on constructing a new Opus project.

An Opus package is a Python package\index{Python!packages}\index{Python} that conforms to the
Opus conventions.  If you are only creating new child XML configurations and
modifying them with the GUI, you don't need to make a new package.  However, if
you're writing Python code to define new variables as Python classes, to define new 
datasets, or to define new models, then you should construct a new Opus project 
to hold this code.  The new package should be 
placed in your workspace, alongside of the Opus and UrbanSim packages such as 
\package{opus_core} and \package{urbansim}.

For instance, to create an
Opus package for the fictitious ``atlantis'' MPO:

\begin{enumerate}
  \item Use the following Python commands to create a new \file{atlantis} Opus
  package pre-populated with the structure and files expected in an Opus package
  (replace ``atlantis'' with the name for your Opus package):
\begin{verbatim}
>>> from opus_core.opus_package import create_package
>>> create_package('c:/opusworkspace', 'atlantis')
\end{verbatim}
\end{enumerate}

This will create a new directory, e.g. \file{C:/opusworkspace/atlantis},
containing a set of commonly useful files and directories, though some may not
suitable to your application. For instance, you might want to delete the
following four files that are useful in the CUSPA infrastructure,
but probably not in yours:

\begin{description}

\item[\file{.project}] \index{.project@\file{.project} file} 
Eclipse's\index{Integrated Development Environment (IDE)!Eclipse}\index{Eclipse IDE}
container for project-specific information (see \url{http://www.eclipse.org/}).

\item[\file{build.xml}]
\index{build.xml@\file{build.xml} file}
Directives for CUSPA's automated build system.

\end{description}

Each Opus package may have a different set of files and directories, depending
upon what is needed. To illustrate what an Opus package may contain, consider
the \package{psrc} package we created for the Puget Sound Regional
Council's (PSRC)\index{PSRC} application
of UrbanSim.  Here are some of the directories and files it contains.  (This
material is in flux --- some
of these are being superseded by information in the XML configuration file, e.g.\
the run configurations.)


\begin{verbatim}
county/                         <-- Variables for the county dataset
    __init__.py                 <-- Makes this into a Python package
    de_population_DDD.py        <-- An Opus variable
    ...
docs/                           <-- PSRC-specific documents EMME\/2/
<-- Miscellaneous stuff for PSRC's EMME/2 use estimation/ <--
Scripts for estimating PSRC's models
    __init__.py                 <-- Makes this into a Python package
    estimate_dm_psrc.py         <-- Script to estimate the developer model
    estimate_elcm_psrc.py       <-- Script to estimate the employment location
                                    choice model
    ...
faz/                            <-- Additional variables for the FAZ
dataset gridcell/                       <-- Additional variables for
the gridcell dataset household_x_gridcell/             <-- Additional
variables for the
                                    household_x_gridcell dataset
indicators/                     <-- Scripts to generate indicators
large_area/                     <-- Variables for the large_area dataset
run_config/                     <-- Configurations for different simulation runs
    baseline.py                 <-- Configuration for baseline run
    no_ugb.py                   <-- Configuration for baseline without UGB
    ...
tests/                          <-- Automated tests
    __init__.py                 <-- Makes this into a Python package
    all_tests.py                <-- Runs all of this Opus package's tests
    test_estimation_dm_psrc.py  <-- Automated tests for estimate_dm_psrc.py
    ...
utils/                          <-- Miscellaneous utilities
zone/                           <-- Additional variables for the zone dataset
__init__.py                     <-- Makes this into a Python package
opus_package_info.py            <-- Information about this Opus package
\end{verbatim}

\section{Unit Tests}
\label{sec:unit-tests}

This section describes how Opus developers write unit tests for Opus code. One
of the goals of the Opus project is to practice test-driven development,
wherein we write tests before we write \emph{any} code.  This practice is
becoming increasingly common, since it has many advantages including
dramatically increasing the ability to change code without breaking code.

In general, each Opus module includes just two classes: a class defining the
functionality of this module, e.g. MyClass, and a test class containing unit
tests of MyClass.  The contents of this module are structured like
this:

\begin{verbatim}
#
# UrbanSim software. Copyright (C) 1998-2007 University of Washington
#
# You can redistribute this program and/or modify it under the terms of the
# GNU General Public License as published by the Free Software Foundation
# (http://www.gnu.org/copyleft/gpl.html).
#
# This program is distributed in the hope that it will be useful, but WITHOUT
# ANY WARRANTY; without even the implied warranty of MERCHANTABILITY or
# FITNESS FOR A PARTICULAR PURPOSE. See the file LICENSE.html for copyright
# and licensing information, and the file ACKNOWLEDGMENTS.html for funding and
# other acknowledgments.
#

... imports ...

class MyClass:
    .... methods for this class ...

from opus_core.tests import opus_unittest
class MyClassTests(opus_unittest.OpusTestCase):
    ... test methods ...

if __name__ == '__main__':
    opus_unittest.main()
\end{verbatim}

The file starts with the required GPL header.  Then some imports.  Then the
MyClass class, followed by the test class whose name is the name of the primary
class followed by ``Tests''.  Finally, the last two lines will
cause the unit tests to be run if this module is invoked directly by Python.

Test classes should derive from \verb|opus_unittest.OpusTestCase| in order to 
ensure that the test runs in a ``clean'' environment.  In particular, 
\verb|OpusTestCase| removes all singletons before and after each test, so
that a singleton created by one test will not accidentally be used by another
test. 

\verb|OpusTestCase| also provides the following helper methods:

\begin{description}
  \item[assertDictsEqual(first, second, *args, **kwargs)] This is like the
  normal \verb|assertEqual|, except that it also works for Python dictionaries
  that may contain numpy data. 
  \item[assertDictsNotEqual(first, second, *args, **kwargs)] This is like the
  normal \verb|assertNotEqual|, except that it also works for Python
  dictionaries that may contain numpy data. 
\end{description}

Tests that are defined in this manner will be automatically found and run by the
\module{all_tests} module when it is executed. Please ensure that the test case
class (e.g. MyClassTests) is found outside of the
\verb|if __name__ == '__main__':| check, otherwise the tests will not be found.

For more examples, check out the files in the \package{opus_core} package.

\subsection{Writing Unit Tests for a Variable}
%
By convention, there are a set of ``unit tests'' for every Opus variable
, and for every Opus model.  These tests check that the variable
or model is doing the right thing, and were useful to uncover bugs when we wrote
the tests.  In addition, the innards of the Opus system also include
many unit tests to check that the framework works.

Here are some guidelines to writing unit tests:
\begin{itemize}
  \item Create \emph{just enough} data for the test.  For instance, create
  three gridcells with two attributes on each gridcell.  This
  makes it easier to write, understand and modify tests.
  \item Use data that has interesting differences; the test data does not have
  to be realistic.
  \item Use different test data for each test.  Otherwise, changing the data to
  better serve one test can cause problems with a different test.
\end{itemize}

The \package{urbansim} package contains many examples of variables and their
tests.  For instance, see the Python modules in \module{urbansim.gridcell}.

When writing a variable, be careful about forcing the values of a variable's
computation to be within a specified range.  This can hide
errors.  For instance, when we were developing UrbanSim, we initially used
numpy's \verb|clip| function to force the
vacant_industrial_sqft variable values to be at least zero.
However, it turned out that there was a bug in our code that led to more
industrial jobs being allocated to a gridcell than the gridcell's industrial
sqft could support.  This sometimes led to a negative value of
vacant_industrial_sqft for that gridcell.  Since these negative values were
being forced to zero, however, we saw an un-intuitive result where the number
of industrial jobs was going up at the same time as the vacant_industrial_sqft,
even though no new industrial sqft was being built. Removing the clip made it
easier to diagnose the problem.

\subsection{Stochastic Unit Tests}

\index{Unit tests!stochastic} \index{Stochastic unit tests}
Testing stochastic systems presents challenges for the traditional unit
testing framework, since in general either the tests are trivial, or else
they will sometimes fail.  A research project by our group developed a
methodology to put stochastic unit tests on a firm statistical basis, and
this is now used in writing such tests in Opus and UrbanSim.  See
Hana {\v{S}}ev\v{c}\'{\i}kov\'{a}, Alan Borning, David Socha, and
Wolf-Gideon Bleek, ``Automated Testing of Stochastic Systems: A
Statistically Grounded Approach,'' in \emph{Proceedings of the
International Symposium on Software Testing and Analysis}, ACM,
July 2006 (available from
\url{http://www.urbansim.org/papers/sevcikova-issta-2006.pdf}).

A result of this is that, indeed, such stochastic unit tests may fail, even
though the underlying code is correct.  This formerly resulted in a 
periodic failure, which would go away when we ran the test again.  To automate
this process, the \method{run_stochastic_test} method in the
\class{StochasticTestCase} class now includes a keyword parameter \code{number_of_tries}
with a default value of 5, which simply tries again that number of times.  Since
introducing the \code{number_of_tries} parameter this issue has not come up

\section{Programming by Contract}
\label{sec:programming-by-contract}
%
Several of the types of Opus objects (variables, models) have
optional \method{pre_check} and \method{post_check} methods that allow you to
use a ``programming by contract'' model of software development. In this
technique a piece of code, such as a variable definition, can define a
``contract'' that specifies that if a given set of pre-conditions are met, the
code is guaranteed to meet a given set of post-conditions. This ``contract''
can help (a) reduce debugging time by checking for invalid inputs or outputs,
and (b) document the assumptions and behavior of the code.

Opus implements this programming by contract mechanism by automatically
invoking the \method{pre_check} and \method{post_check} methods of variables,
models, and simulations.  This is done only if the Resources being used
contain the ``check_variables'' key.

\begin{description}

\index{Opus!Variable!pre_check@\method{pre_check}}
\index{Opus!Model!pre_check@\method{pre_check}}
\index{pre_check@\method{pre_check}}
\item[\method{pre_check}] This method for an Opus
variable is called before the variable's \method{compute} method.  This is a
place to put optional tests of ``pre-conditions'' that this variable assumes in
order to operate properly.

\item[\method{post_check}] \index{post_check@\method{post_check}} This method for an Opus
variable is called after the variable's \method{compute} method.  This is a
place to put optional tests of ``post-conditions'' of what this Opus variable
``guarantees'' it will produce given that the ``pre-conditions'' tested in the
\method{pre_check} passed.

\end{description}

\section{Design Guidelines}

These design guidelines are key rules that, if followed, will help you
create code that is easy to develop and to maintain.

\begin{itemize}

\item Practice test-driven development.  Before you write
{\em any} code, write an automated test that tests if that code works.
Use this for new features as well as bug fixes.

\item
Include automated unit tests.  A unit test tests an internal part
of the code, such as a method. A method is a contract: given
certain assumptions about the data used by the method, the method
is guaranteed to produce certain data.  A good rule of thumb is to
have at least one unit test for every contract of every method. As
always, use your judgment as to what tests to write.

\item Include automated acceptance tests.

% This sentence seemed incomplete:
% An acceptance test runs the application in the same

\item Never duplicate code.  Avoiding duplication results in lots of good, such
as decoupled designs.

\item Favor composition over inheritance.

\item If you can think of a reason you want to use different
versions of a method, convert the method to an class and use
composition.

\item Relentlessly keep your abstractions current with your
understanding of the system.

\item Have an abstraction for every ``real world'' object that the modeler
interacts with, e.g. ``a run request,'' ``a queue of run requests,'' etc.

\item Create decoupled designs.  A class C should only know about another
class B when C's abstraction requires knowledge of B.

\end{itemize}

\section{Coding Guidelines}
\label{coding-guidelines}

These draft guidelines reflect what we believe makes for readable,
understandable code.  (Note that not all of the code in the current Opus
code base follows these conventions at this point.)

The Enthought coding standards
\url{https://svn.enthought.com/enthought/browser/trunk/docs/coding_standard.py}
is a good start.  In addition, we have a number of other suggestions particular
to Opus or our preferences.

\subsection{Indentation}

Use spaces rather than tabs.  (Python interprets a single space as the same
level of indentation
as a single tab, yet 8 spaces can look the same as 1 tab --- which can lead
to some very confusing debugging sessions otherwise.)

If for some reason you need a tab inside a
string, use \verb|'\t'|.  This makes it easy to automatically check for and
complain about tabs in files.

Indent 4 spaces for each successive level of indentation.

Indent nested functions.

Use nested functions only for very minor, short ones.

\subsection{Naming Conventions}

Use the following naming conventions for class, method, and variable names:

\begin{verbatim}
ClassNamesLikeThis
CONSTANTS_LIKE_THIS
instance_variables_like_this
method_variables_like_this
_semi_private_method_variables_like_this
__private_method_variables_like_this
public_method_names_like_this
_semi_private_method_names_like_this
__private_method_names_like_this
\end{verbatim}

Exceptions are the class names for Opus variable classes (e.g.\
\mbox{\tt total_land_value}), for model component classes (e.g.\
\mbox{\tt linear_utilities}) and for pluggable model building
blocks (e.g.\ \verb|mnl_probabilities|).  These all are lower case
with underscores, since users specify these classes by strings
and our experience is that it is much less confusing if strings are
case-insensitive.

If a variable name won't be used but is needed, such as when a method returns
a pair of values of which only one is used, start the variable name with 
\verb|dummy|, as in:

\begin{verbatim}
coef, dummy = cm.estimate(specification, ... 
\end{verbatim} 

It is important for a class to clearly communicate its intended usage
patterns.  Part of having a well defined interface is being clear about which
methods and which instance variables may be used from where.  This helps others
understand how to use the class, which helps prevent usage errors.  Some
methods, for instance, should only be called from within that class; perhaps
they can only safely be called with data that is private to that object.

Python provides three types of method names that help span the public to
private spectrum.  Opus' convention is to use these three types of method
names, as follows:

\begin{description}
\item[\code{def foo(self):}] This is a public method callable by anyone.  Others
objects can always call this and expect it to do the right thing.  This
corresponds to Java's \verb|public| methods.
\item[\code{def _foo(self):}] This is a semi-private method.  It has a single leading
underscore.  It should only be called by another object that is closely related
to this class, either by inheritance or by being in the same Python package.
This corresponds to Java's \verb|protected| and \verb|package| methods,
respectively.
\item[\code{def __foo(self):}] This is a private method.  It has two leading
underscores.  It should only be called from within this class.  This
corresponds to Java's \verb|private| methods.
\end{description}

Always use setters and getters to access another object's instance variables.
Accessing instance variables directly ties the outside object to the particular
implementation of the instance variable. For instance, the object can no
longer decide to compute the variable on each access.  This leads to code that
is harder to extend and maintain.

Class names should not start with an underscore.

Python has no special syntax or semantics for abstract classes, but good practice is to
explicitly name a class as abstract, e.g.\ \verb|AbstractModel|, if it is
abstract.

Use lower case names with underscores for file names, e.g.
\verb|my_stuff.py|, for improved readability, and because some
file systems are case sensitive while others are case insensitive.
Beware that some file systems have limits on the length of file
names, so don't make them too long.

Variable names should be at most 64 characters long, since
MySQL \index{MySQL!column name length} does not allow column names with more
than 64 characters.

Historical naming issues:
\begin{itemize}
\item Method names \verb|thatStartWithALowerCaseLetter| are deprecated, although
it is not urgent to change them.
\item Method names \verb|ThatStartWithACapitalLetter| should be renamed, since this
style is used for class names instead.
\end{itemize}

\subsection{Methods and Side-effects}

A common type of bug that can be very difficult to find is when a method
changes the value of a variable that is defined outside that method.  The next
two guidelines help prevent these bugs.

Methods that return meaningful values should not have any side effects, except
for benign side effects. In programming language jargon, they should act as
functions.  Benign side effects that don't change the values produced, such as
caching a value to speed up processing, are fine though be careful that the
lifetime of the values in the cache is what you want.

Methods that have no side effects should return any values.  The one exception
is some cases where a status code is returned.

\subsection{Documentation and Comments}

Have a doc string for every public class and method.  Doc strings for
internal methods are optional -- use your good judgment.

Include informative comments, of course.  Comments should be at the
right level --- comment entire methods and potentially obscure bits of
code; don't comment obvious things.

\subsection{Source Files and Packages}

Python source files may contain more than one class, if the classes are
closely related.  If a file is getting too long, break it into several
files however.  Exception: files containing an Opus variable definition
must contain just the one class defining the variable.

Use Python's name space mechanisms, such as modules and packages, to handle
name space issues.  Use \verb|import myfile| or
\verb|from myfile import MyClass|.  Don't use \verb|import *| in code in
the repository.  (It's convenient to use at the console for interactive debugging
however.)

Avoid \verb|import as|, since it makes it harder to figure out what a name
refers to.  If there is a name clash, use a qualified name
instead.  For example, numpy and numpy.ma both have a log
function.  Rather than
using \verb|from numpy import log as nlog|,
use \verb|import numpy|, and then within your code qualify the names, e.g.
\verb|numpy.log|.

Do not use multiple-line imports, such as:

\begin{verbatim}
from numpy import array, repeat, transpose, ndarray, reshape, \
        indices, zeros, float32, asarray, arange, compress, \
        logical_not, cumsum, log, where, ones, strings
\end{verbatim}

These multiple line imports make it hard to search for imports of specific functions,
since the ``import'' is separate from some of the function names.  This in turn
increases the chance of missing some instances of things you might want to change.

\subsection{Typing Issues}

Python \index{Python!types} is type-safe but dynamically typed --- no type declarations are
needed.  This gives flexibility and good support for rapid development, but
it's also easy to write code that is difficult to understand and maintain.
Here is a general rule of thumb: it should be possible to write the type
signature of any Python variable or method, in a way that lets you
statically check that there aren't any type violations.  (Python doesn't
require or even support such declarations, of course; this is a rule.)
There are times you will want to violate this guideline, but make sure you
have a good reason to do so.

For non-computer-science types, here are some practical implications of
this heuristic:

\begin{itemize}

\index{Python!True and False versus 1 and 0}
\item Use an actual boolean (\verb|True| or \verb|False|) when you need a
boolean value, not 0, 1, the empty string, or the other things that Python
allows for booleans.

\item If you have a method that takes a sequence of some kind, let it take
sequences only.  Don't generalize the method to also allow single values
that get coerced into a sequence --- this is confusing, and also leads to
bugs.

\item If you are writing code that uses \verb|type(x)| or \verb|is|,
there is very likely a better way to write it (using objects and
inheritance).

\item Don't use the AND hack from \emph{Dive Into Python}
(\url{http://diveintopython.org}).

\end{itemize}

The ubiquitous ``resources'' variable is a violation of this rule, of
course (which is why there is something a bit off about it); but for now,
we just let this be an exception.

There are also a few typing style recommendations.

Tuples should in general just be used locally, if you are counting on
elements being at specific positions --- if you need to pass around a tuple
of this kind, define a small class instead.  For example, this is a
reasonable use of a tuple (for formatted text):

\begin{verbatim}
'Unexpected value -- squid was %s but %s was expected' % (a,b)
\end{verbatim}

Here the tuple is used quite locally, and is clear.

However, if you find yourself passing around a tuple and then writing
things like \verb|squid[4][2][8]|, see if there is a better way to do it.
Similarly, lists used to package up a fixed number of elements, with the
positions being significant, are also suspect.  (An exception is picking
apart the results from SQL queries --- here you're stuck with picking it
apart.)


\subsection{Strings}

When constructing strings by combining string parts and values, use the
\verb|%| operator instead of the \verb|+| operator, e.g. use

\begin{verbatim}
'Could not find %s in file %s' % (value, file_name)
\end{verbatim}

not \verb|'Could not find ' + value + ' in file ' + file_name|.

When quoting strings, favor single quotes (\verb|'|) over double-quotes, except
where that doesn't make sense.


\subsection{Functional Programming Style where Appropriate}

Use list comprehensions when feasible.  Use functional programming style in
lambda functions.


\subsection{Classes and Methods}

Always use ``new-style'' Python classes (see
\url{http://www.python.org/doc/newstyle.html}), as they have some subtle and
important benefits over the old-style Python classes.

Always have an \verb|__init__| method for each class.  Make it the first
method of the class.  (If there isn't anything to do, just put pass in the
body.)  Name, and initialize, all instance variables in the
\verb|__init__| method.

Use keyword parameters in method calls, as it is more readable.
In method definitions, use defaults as appropriate, but keep the design
clear and understandable. If the parameter is required, just don't put a
default.

Ideally, a default for a given parameter should occur in just one
place, and not be repeated in multiple method or function definitions.  (In
the other definitions, use a required parameter instead and pass along the default
from the other method or function.)  For example, suppose some method has a
parameter that includes a default value: \verb|replace_nulls_with=0.0|.  Other
methods that have the \verb|replace_nulls_with| parameter should require it,
and use the value passed along from the method in which it was given the default.
Otherwise, if you want to change the default, there would be lots of places to change.
If it turns out to be awkward to follow this rule, and you really do need to give a default
in lots of places, give it a name and use that name in the parameter list, e.g.
\verb|replace_nulls_with=self.default_null_replacement_value|.

\emph{We'll delete the following example in a bit --- but there are
quite a few places in the existing code that use keyword parameters in more
complex ways than need be, e.g.\ by setting the default to {\tt None}, and
then having an {\tt if} in the code that tests for {\tt None} and if so sets
the parameter to a default.  So I put in the discussion for the moment.}

For example, consider the following.  (Any resemblance to actual Opus
code is purely coincidental.)
\begin{verbatim}
class _Database :
    def __init__(self, host=None, user=None, password=None, database=None, \
        environment=None, silo_path=None, show_output=True) :
        """ Returns a connection to this database.
        If database does not exist, it first creates it.  If the
        environment is given, then take host, user and pass assignments
        from it, or partially override the environment values if specific
        others are given.  'database' is actually a required parameter; if
        left as None, failure is certain, but it is named nonetheless so
        that the signature is flexible. """
        self.show_output = show_output
        if environment != None:
            self.host = environment['host_name']
            self.user = environment['user_name']
            self.password = environment['password']
        if host != None:
            self.host = host
        if user != None:
            self.user = user
        if password != None:
            self.password = password
        self.database_name = database
        .....
\end{verbatim}

This has various useful defaults for the parameters, but could be improved.

\begin{verbatim}
class MySQLDatabase :
    """This class provides uniform access in Opus to a MySQL database ..."""
    def __init__(self, database,
            host='localhost',
            user=os.environ['MYSQLUSERNAME'],
            password=os.environ['MYSQLPASSWORD'],
            silo_path=None,
            show_output=True) :
        self.show_output = show_output
        self.database = database
        self.user = user
        self.password = password
        .....
\end{verbatim}

Note that the treatment of defaults is simplified --- having optional
parameters that set the values of other optional parameters should be
avoided.  Also, putting the defaults in the method header makes it clearer
what is going on.  Naming the parameter \verb|database| and then naming
the corresponding instance variable \verb|database_name| is confusing.
Finally, the comment in the old code about the \verb|database| parameter
is incorrect -- if a parameter required, just don't put in a default value.
Python still allows using a keyword for it in the method call, and the
keyword parameters can be in any order.  You'll get a failure if the required
parameter is missing, which is the desired behavior.

(There are other issues as well, for example, why is there something about
silage in \package{opus_core}, but we'll ignore that for the moment.)

If a variable is the name of an object, include \verb|name| in the object's
name, e.g. variable_name, to help distinguish it from a variable that contains
the object itself, e.g. variable.

\subsection{Continuation Lines}

For readability, only use Python's backslash (``\verb|\|'') continuation
character when necessary.  This character is used to split a statement or
expression across more than one line.  It is not necessary when the line
break is inside a (), [], or {} pair, or in triple-quoted strings.

\subsection{Open Issues}

\begin{itemize}

\item Need to specify naming convention for items in resources; data model.
\emph{Note that the resources concept is being replaced by
Traits-based configuration classes.  This provides an explicit API
for each configurable class, and avoids many of the problems
associated with the prior use of resources.}

\item Conventions for throwing exceptions.  (We should make much more use of
Opus-specific exceptions, rather than generic exceptions like StandardError.)

\item Put all SQL strings into a single or small number of
isolated files, so that it is easier to localize these strings to different
databases.

\item Follow the conventions in Python's PEP 290: Code
Migration and Modernization PEP 8
(\url{http://www.python.org/peps/pep-0290.html}) For instance, when
searching for a substring, where you don't care about the position of the
substring in the original string, using the in operator makes the meaning
clear, e.g. use \verb|string2 in string1| instead of
\verb|string1.find(string2) >= 0| or \verb|string1.find(string2) != -1|.  Also,
when testing dictionary membership, use the 'in' keyword instead of the
'has_key()' method. The result is shorter, more readable and runs faster.\item

\item Follow the conventions in PEP 8: Style Guide for Python Code
PEP 257 (\url{http://www.python.org/peps/pep-0008.html})

\item See Python's PEP 257: Docstring Conventions
(\url{http://www.python.org/peps/pep-0257.html}).

\end{itemize}


%%% Local Variables:
%%% mode: latex
%%% TeX-master: "userguide"
%%% End:

% LocalWords:  terhorst UrbanSim Un Uninstall uninstalling py PYTHONPATH CVS un
% LocalWords:  urbansim init xml dir MyClass GPL TestVariables gridcells sqft
% LocalWords:  gridcell numpy's gridcell's pre mnl ok AbstractModel MySQL de
% LocalWords:  instanceVariablesAreWrittenLikeThis ThatStartWithACapitalLetter
% LocalWords:  thatStartWithALowerCaseLetter myfile numpy nlog malog cumsum
% LocalWords:  SQL comprehensions distutils setuptools Docstring userguide