% Copyright (c) 2005-2008 Center for Urban Simulation and Policy Analysis,
% University of Washington.  Permission is granted to copy, distribute and/or
% modify this document under the terms of the GNU Free Documentation License,
% Version 1.2 or any later version published by the Free Software Foundation;
% with no Invariant Sections, no Front-Cover Texts, and no Back-Cover Texts.
% A copy of the license is included in the section entitled "GNU Free
% Documentation License".

\chapter{UrbanSim Database Tables}
\label{chapter:urbansim-database-tables}

\section{General Database Information}
\subsection{Baseyear Database}
\subsection{Scenario Database}
\subsubsection{Scenario Linking}
\subsection{Output Database}
\section{Tables Used by Models}
\section{Table Definitions}
\subsection{The {\tt some_table} table}
\subsubsection{Short Description of Table}
\subsubsection{Actual Table Definition}
\subsubsection{Notes, Pitfalls, Restrictions}
\subsection{The {\tt next_some_table} table}
\subsubsection{Short Description of Table}
\subsubsection{Actual Table Definition}
\subsubsection{Notes, Pitfalls, Restrictions}

This chapter describes what UrbanSim needs in its baseyear database, ways in
which the baseyear may be structured, how to create a set of scenario databases
that share much of their data, and the use of output databases.

UrbanSim currently uses three types of databases, each of which is described in
more details in following sections:

\begin{description}
\item[baseyear database] -- defining the initial state of a simulation in a
particular base year.
\item[scenario database] -- defining changes to a baseyear (or another scenario)
database.
\item[output database] -- optional repository for simulation results.
\end{description}

Most database tables are optional. The required set of tables is determined by
the variables \variablesindex used by the models.  This can be found by looking at the models'
specification tables.






\section{Input Database Design: Baseyear and Scenario Databases}
\label{urbansim-database-tables-baseyear-scenario-db}

UrbanSim gets its input data from either a \emph{baseyear} database or a
\emph{scenario} database.  The UrbanSim simulator treats \emph{baseyear} and
\emph{scenario} databases as read-only databases, however other data
preparation applications such as the estimators or the household synthesizer
may write to them.  In fact, when the run manager starts a new simulation, the
first step is to copy the baseyear data into the baseyear cache. \baseyearcacheindex The
simulation then reads all of its baseyear information from the baseyear cache \baseyearcacheindex
and writes all results to the simulation cache. \simulationcacheindex Data is only written to the
output database when specified (currently done manually).

At the moment, UrbanSim only supports the MySQL \mysqlindex database server (we will be
adding support for other database servers).

A \emph{baseyear} database contains a snapshot of the base
information defining the initial state before the UrbanSim
simulation. Most of the data typically is about a particular year,
e.g., geographic information, initial household and job information,
etc., for a given year.

A \emph{scenario} database contains additional and augmenting information to
alter the base year data when simulating a particular scenario e.g., new
transportation links, an expanded urban growth boundary, etc. Any of the tables
may be placed in either of the databases, although typically most are placed in
the baseyear database.  The scenario databases typically only contains tables
specifying different possible futures, e.g. tables of exogenous events
scheduled for future years.

The way that the scenario can modify the information in the baseyear database or
another scenario database is determined by the scenario linking.

\subsection{Scenario Linking}
\subsubsection{blah}

The scenario databases are linked to each other and, eventually, to a baseyear
database via a tree structure: each scenario can refer to exactly one parent
database; that parent can be either another scenario or a baseyear database.
The baseyear database is the root of the tree.  In this way, multiple scenarios
may share the same baseyear.

When UrbanSim looks for a particular input database table, it traverses this
chain looking for that table.  The first table matching that name is used.  In
this way, any tables contained in the scenario database ``shadow'' or ``hide''
the same-named tables in the scenario's parent database(s).

Consider these example of how to create derivative scenarios:\\

\begin{center}
\includegraphics*{scenarios}
\end{center}

\begin{itemize}
\item Scenario 1 is the base year plus a larger urban growth boundary, thus
scenario 1's parent is the base year database. The UrbanSim scenario file will
specify the scenario 1 database as the ``scenario-data''.
\item Scenario 2 is the base year plus a major employer leaving the
municipality. Scenario 2's parent is also the base year database. The UrbanSim
scenario file will specify the scenario 2 database as the ``scenario-data''.
\item Scenario 3 is the same as scenario 1, but with additional changes in the
zoning laws to compensate for the larger UGB. Scenario 3's parent is scenario 1
and the UrbanSim scenario file will specify the scenario 3 database as the
``scenario-data''.
\item Scenario 4 is the same as scenario 1, but with changes in the population
demographics as a result of the larger UGB. Scenario 4's parent is scenario 1
and the UrbanSim scenario file will specify the scenario 4 database as the
``scenario-data''.
\end{itemize}

Any table in a scenario database ``hides'' the same-named table in the
scenario's parent database.

\subsection{Scenario Database Design}

The only required table in the scenario database is the
\verb|scenario_information| table (see Sec.~\ref{urbansim-database-tables-scenario-inforamtion}).
This table points to a baseyear database, or
to another scenario database.

In addition, the scenario database may include any other tables for data that
is different in this scenario. For example, if the scenario is simulating a
large retail development in the suburbs, the \verb|development_events_exogenous| table
would be included in the scenario database.  In this way, a scenario database
may change any of the information contained in the baseyear.

\section{Output Database}

An output database may contain the results of an UrbanSim simulation.  This
database is optional; the urbansim cache is the primary storage location for
simulation inputs and results.

\section{General Database Design}

\subsection{Guidelines}

Here are some guidelines on database design:

\begin{itemize}
\item Use only lower-case letters, digits, and underscores for the names of
databases, database tables, and database columns. This avoids problems when
moving database between different operating systems (e.g. between Windows \windowsindex and
Linux). \linuxindex
\item The \verb|tables_to_cache| entry of the \verb|run_configuration| in
\verb|urbansim.configs.base_config| lists the tables required for the models \modelsindex used in
the Puget Sound region.  (This information \emph{should} be in the
\package{psrc} \psrcindex package, not the \package{urbansim} package; we will fix this in
the future.)
\item Avoid overly abbreviated names.  While very short names were required by
some other systems, UrbanSim itself has no limit on the length of names. Most
database system allow database names, table names and column names to be 32
characters, 64 characters (e.g. MySQL), \mysqlindex or more.
\end{itemize}

\subsection{Data Types}

\index{Python!data types} \index{Opus!data types} \index{MySQL!data
types} When Opus reads data from a database table, it stores the
data in a Python type that is close to the type of the corresponding
column in the database.  The particular mapping between database
types and Python types currently is defined for MySQL \mysqlindex
and should be re-defined for each additional type of database.  The
conversion for MySQL is:

\begin{tabular}{ll}
MySQLdb FIELD_TYPE & Python/numpy type \\
\hline
tinyint(1) & bool8 \\
short & int16 \\
int24 & int32 \\
long & int32 \\
longlong & int64 \\
float & float32 \\
double & float32 \emph{(this should be float64)} \\
decimal & float64 \\
\end{tabular}

Similarly, when writing from Python to a database, Opus converts from Python
types to database-specific data types.  The conversion for MySQL is:

\begin{tabular}{ll}
Python/numpy type & MySQL type \\
\hline
bool8 & int(1) \\
int8 & int(8) \\
int16 & int(16) \\
int32 & int(32) \\
int64 & int(64) \\
float32 & double \\
float64 & decimal \\
\end{tabular}

Note that these conversions are not symmetrical, since multiple
database types map onto a single Python type.  The result is that
when written back to the database, the column types may change from
that of the input database table.

\section{What Tables are Used by a Model?}

\section{Miscellaneous Database Tables}

This section provides some information about some of the input database tables.
For each table, it defines the table's schema followed by a list of rules that
should be true for the entries in the table.

\subsection{Coefficient and Specification tables}

The UrbanSim models \modelsindex are configured through user specified
variables and coefficients. \coefficientsindex The coefficients should be
estimated separately for each region to be modeled by UrbanSim. The art and
science of estimating the coefficients is a matter for a series of college
courses so this description assumes that appropriate variables for each model
have been chosen, and the appropriate coefficients have been estimated for
those variables.

Each of the regression models and Logit models have two associated
tables: a table to store the specification of what variables to use
for that model, and a table to store the estimated coefficients to
use for those variables.  The names of these tables are composed by
appending either \verb|_coefficients| or \verb|_specification| to
the model name.  The tables for the land price model, for instance,
are \verb|land_price_model_coefficients| and
\verb|land_price_model_specification|.

All coefficient tables share the same schema, as do all
specification tables. The schemas are:

\subsubsection{Coefficient table}

\begin{tabular}{|l|l|p{4.5in}|}

\hline
\textbf{Column Name} & \textbf{Data Type} & \textbf{Description} \\

\hline
sub_model_id & integer &  Identifier for a submodel, if used by the model, or
-2 if not used by this model. \\

\hline
coefficient_name & varchar & Unique name of a coefficient  \\

\hline
estimate & double & The estimated value of this coefficient \\

\hline
standard_error & double & \emph{(optional) }
The standard error of this estimated value. This is for reference only and is
not used by UrbanSim.  \\

\hline
t_statistic & double & \emph{(optional) }
The t-statistic of this coefficient for the test of significance from 0. This
is for reference only and is not used by UrbanSim.  \\

\hline p_value & double & \emph{(optional) } The p-value of this t-statistic,
gives the Prob(|x|\textgreater{}|estimated coefficient|) when x is drawn from a
t-distribution with mean 0. This is for reference only and is not used by
UrbanSim.   \\

\hline

\end{tabular}

\begin{itemize} \tight
\item sub_model_id must be in the appropriate cross-reference table for each
model.
\item Each combination of (sub_model_id, coefficient_name) must be unique.
\end{itemize}

\subsubsection{Specification table}

\begin{tabular}{|l|l|p{4.5in}|}

\hline
\textbf{Column Name} & \textbf{Data Type} & \textbf{Description} \\

\hline sub_model_id & integer & Defines the submodel, if the model
contains submodels. The attribute defining the submodel is
determined by the \verb|submodel_string| parameter of the model. The
values in this field are the id values of that
\verb|submodel_string| attribute of the model's dataset. The
employment location choice models, for instance, define submodels by
employment sectors, so the values of this field are the
\verb|sector_id| values
of the jobs dataset. \\

\hline
equation_id & integer & If a submodel has multiple equations, this field
contains an id identifying which equation this row applies to.  If a model does
not have multiple equations, use ``-2'' for this field. \\

\hline
coefficient_name \coefficientsindex & varchar & Each sub-model does not have
multiple equations, so use '-2' in this column  \\

\hline
variable_name \variablesindex & varchar & \\

\hline

\end{tabular}

\begin{itemize} \tight
\item The sub_model_id values must exist in the \verb|submodel_string| attribute
of the model's dataset.
\item Each combination of (sub_model_id, coefficient_name) \coefficientsindex
must exist in the model's coefficients table.
\item variable_name must be a legitimate specification for an Opus variable.
\item Each combination of (sub_model_id, equation_id, variable_name) \variablesindex must be unique.
\item Specific models \modelsindex have additional requirements: see Developer Model \modelsindex and Non-home-based Employment Location Choice Model. \modelsindex

\end{itemize}

A legitimate specification for an Opus variable may be one of the following:

\begin{itemize}

\item The word \verb|constant| indicating a value specific to this combination
of (sub_model_id, equation_id).

\item The name of a primary attribute of a dataset, specified as a period-separated
tuple of dataset name, attribute name, .e.g.
\verb|parcel.percent_slope|.

\item The name of a dataset attribute, specified as a period-separated triple of
Opus package name, dataset name, attribute name, .e.g. \verb|urbansim.parcel.population|.

\end{itemize}

\subsection{The {\tt development_constraints} table}

This table defines rules that restrict the possible development types a
developer can create on a particular parcel.  Each row defines one rule.
Development is not allowed on any parcel that matches any of these rules.  A
parcel matches a rule if the attribute values for the parcel match all of
the values in the rule (rule columns with the value ``-1'' are ignored when
determining a match).

\begin{tabular}{|l|l|p{3.5in}|}

\hline
\textbf{Column Name} & \textbf{Data Type} & \textbf{Description} \\

\hline constraint_id & integer & Unique rule identification number  \\

\hline \emph{name-of-a-parcel-attribute} & integer or float &  Value for this
attribute, or ``-1'' if this attribute is not part of the constraint (e.g.
\emph{don't care}) \\

\hline min_units & integer & Minimum number of residential units for a parcel.
A development project may only be placed on this parcel if it will result in
this parcel containing at least this number of residential units. \\

\hline max_units & integer & Maximum number of residential units for a parcel.
A development project may only be placed on this parcel if it will result in
this parcel containing at most this number of residential units. \\

\hline min_commercial_sqft & integer & Minimum number of commercial
sqft. for a parcel.  A development project may only be placed on
this parcel if it will
result in this parcel containing at least this number of commercial sqft. \\

\hline max_commercial_sqft & integer & Maximum number of commercial
sqft. for a parcel. A development project may only be placed on
this parcel if it will
result in this parcel containing at most this number of commercial sqft. \\

\hline min_industrial_sqft & integer & Minimum number of industrial
sqft. for a parcel.  A development project may only be placed on
this parcel if it will
result in this parcel containing at least this number of industrial sqft. \\

\hline max_industrial_sqft & integer & Maximum number of industrial
sqft. for a parcel. A development project may only be placed on
this parcel if it will
result in this parcel containing at most this number of industrial sqft. \\

\hline
\end{tabular}

\begin{itemize}
\tight
\item constraint_id must be a unique positive integer.

\item \emph{name-of-a-parcel-attribute} is the name of a parcel attribute,
e.g. \verb|city_id| or \verb|is_in_wetland|.  The set of available attribute
names is determined by the set of numeric column names on the parcel table.

\item Within each \emph{min_}/\emph{max_} pair, the max must be greater than or
equal to the min, e.g., \verb|min_units| $<=$ \verb|max_units|.

\end{itemize}

\subsection{The {\tt target_vacancies} table}
\label{sec:table-target-vacancies}

The \verb|target_vacancies| table is used by the developer project transition
model. It gives the model information about acceptable vacancy rates. The table
has one row for each year the simulation runs. Each row gives target values for
the residential and nonresidential vacancies for that year, which are defined
below.  Only data after the base year is used.

\begin{tabular}{|l|l|l|}

\hline
\textbf{Column Name} & \textbf{Data Type} & \textbf{Description} \\

\hline year & integer & Year of the simulation for which the vacancy
targets
apply  \\

\hline target_total_residential_vacancy & float & Ratio of unused residential
units to total residential units  \\

\hline target_total_non_residential_vacancy & float & Ratio of unused
nonresidential sqft to total nonresidential sqft  \\

\hline
\end{tabular}

\begin{itemize}
\tight
\item There must be exactly one row for each year to be simulated.
\item target_total_residential_vacancy must be between 0 and 1, inclusive.
\item target_total_non_residential_vacancy must be between 0 and 1, inclusive.
\end{itemize}

%%% Local Variables:
%%% mode: latex
%%% TeX-master: "userguide"
%%% End:
