% Copyright (c) 2005-2009 Center for Urban Simulation and Policy Analysis,
% University of Washington.  Permission is granted to copy, distribute and/or
% modify this document under the terms of the GNU Free Documentation License,
% Version 1.2 or any later version published by the Free Software Foundation;
% with no Invariant Sections, no Front-Cover Texts, and no Back-Cover Texts.
% A copy of the license is included in the section entitled "GNU Free
% Documentation License".

{\huge Preface}
\label{preface}

Opus is a new platform for urban and regional simulation, designed to
support the development and integration of model
components \index{models!components} for simulating the
effects of major transportation investments and policy changes
on transportation, \index{transportation} land use, \index{land use}
and the environment. \index{environmental impacts}
Opus consists of a collection of packages\index{Python!packages}\index{packages}
written in the Python\index{Python} language. The choice of
Python was based on Python's
balance of characteristics, which provide a good combination of a
productive development (even for users who are not expert programmers), and
good performance due to the availability of Python
libraries for fast
numeric processing \index{numeric processing} that use C or C++
to implement computationally intensive
aspects.  The basic functionality of Opus is implemented in a package named \package{opus_core}.

UrbanSim Version 4 \index{UrbanSim!Version 4} is implemented as a set of
Opus packages. (Previous versions \index{UrbanSim!previous versions} were in Java.) 
The decision to migrate from Java to Python 
was based on the faster development and computation time available
in the Python environment with numeric libraries, and on the objective of
making models \index{models} in UrbanSim more modular than they had been in the previous
Java version, allowing users to rapidly configure models, \index{models} estimate their
parameters, and use them. In addition, Python is turning out to be a more
accessible language for modelers, which addresses another key goal: \index{goals} to
create a system where modelers can write their own models, and understand
the models written by other modelers.

\index{User contributions}

This document covers both the base set of Opus packages and the UrbanSim
system implemented in a set of Opus packages.  As descriptions
become available, other packages besides those related to UrbanSim will be
added.  We expect that over time many users and developers will contribute
packages to the system, making their work available to others in the same
way that they themselves have benefited from others' work.

The document includes both a user guide and reference manual.  Part
\ref{part-overview} is currently just this introduction and a
technical overview of the UrbanSim model system, but will soon also
include a technical overview of Opus. Part \ref{part-user-guide}
describes how to install and use Opus and UrbanSim.  Part
\ref{part-reference-guide} contains reference material on the
details of Opus and UrbanSim. Part \ref{part-developers} provides
information for software developers (i.e., people who will be
writing new Opus packages or documentation).  A short Glossary is
provided in an appendix.

This document will continue to evolve --- questions and open issues are in
italics. \index{document conventions} Please feel free to send your comments, 
criticisms, and
suggestions to
% ** two versions of email address - avoid a real one for the html pages so that
% we aren't such a spam magnet
%begin{latexonly}
\email{info@urbansim.org}. \index{contact information}
%end{latexonly}
\begin{htmlonly}
{\tt info(at)urbansim.org}. \index{contact information}
\end{htmlonly}

\index{latest documentation}
The latest version (as a pdf file, and as html pages) is linked from
\url{http://www.urbansim.org/}.

\newpage
{\huge Acknowledgments}

This research has been funded in part by the National Science Foundation, 
grants CMS-9818378, EIA-0121326 and EIA-0090832, and by the Puget
Sound Regional Council, the Maricopa Association of Governments, and the 
Environmental Protection Agency.  

The software implementation of UrbanSim and the Open Platform for Urban 
Simulation has been developed by contributions of numerous people at the 
Center for Urban Simulation.  Alan Borning, Co-director of the Center 
for Urban Simulation and Policy Analysis, has been instrumental in the 
development of the software implementation of UrbanSim and the Open 
Platform for Urban Simulation, most recently by implementing the Opus
 expression language and by implementing
a design for using XML to store and manipulate configurations in a new 
Graphical User Interface.  Hana Sevcikova has been the principal software 
engineer
on the new Open Platform for Urban Simulation, and has done much of the 
implementation of the conversion of UrbanSim from Java to the new Opus 
platform in Python.  She has also implemented new Bayesian Melding methods 
to calibrate uncertainty in the model system.  Liming Wang has made invaluable 
contributions to the development of UrbanSim, especially in the new Opus 
platform.  He has developed the sampling toolbox used as a core component
of the choice models, has implemented most of the new real estate development
 model, and the new home based job choice and workplace choice models.
Travis Kriplean has also contributed immensely to the project, creating 
an installer for the Windows platform, taking the lead on the development 
of the
indicator framework used in the results manager of the new Graphical User
 Interface, and generally contributing to the project.  Jesse Ayers has 
contributed
extensively in the creation of the Graphical User Interface and in the 
implementation of connections between UrbanSim and ArcGIS.  Aaron Racocort 
has been essential in the new GUI development.  David Socha contributed
substantially to the previous version of UrbanSim in Java, and to the 
conversion to the new Python-based platform.  In fact, too many other 
people have contributed in too many ways to be adequately acknowledged.  
This work is an ongoing collaboration.


% LocalWords:  borning UrbanSim pdf
