\chapter{Defining Variables using Expressions}
\label{chapter:expressions}
\index{expressions}

As described in Chapter \ref{chap:data-in-opus}, datasets contain primary
and computed attributes.  Opus variables are used to represent the required
computation for a computed attribute.  Underneath, Opus variables are
defined as Python classes.  However, writing Python code is a significant
barrier to using the system, and in many cases Opus variables can be
defined instead as simple expressions in a custom domain-specific
programming language (called Tekoa).  When this can be done, it is much
simpler than defining them in Python itself.\footnote{See
  \cite{borning-dgo-tekoa-2008} for a discussion of the impact on code size
  reduction and other benefits of using Tekoa.}

\section{Informal Introduction}

An expression consists of the name of an attribute, or a function or
operation applied to other expressions.  Thus, a more complex expression
can be built up from simpler expressions.  The functions and operations are
ones from the Numpy \index{numpy} numerical Python package, and operate on
arrays.  Here are some examples.

\begin{itemize}

\item \code{gridcell.distance_to_highway<300} generates a dummy
  variable whose value is 1 for gridcells that have an attribute value of
  distance to highway less than 300 meters, and otherwise 0.  In this case,
  we are operating on a primary attribute of gridcells, namely
  \code{gridcell.distance\_to\_highway<300}.  This means that this
  attribute is part of the data that we initially load into the model, as
  opposed to data we compute within the model (computed attributes).  

\item \code{ln(urbansim.gridcell.population)} computes the log of
  an existing variable, \code{population}, which is in the
  \package{urbansim} package and applies to the gridcell dataset.

\item \code{urbansim.gridcell.population / urbansim.gridcell.number_of_jobs}
  computes the ratio of population to employment, using variables stored 
  in the urbansim package, associated with the gridcell dataset.

\end{itemize}

Two useful methods in the language are \code{aggregate} and
\code{disaggregate}.  Aggregation allows an expression to compute a result
on one dataset and aggregate the results to assign to another dataset, such
as summing the population in households that live in a gridcell.  For
example, this variable defines the population in each zone, by aggregating
the number of people in each gridcell contained in that zone:

\code{zone.aggregate(urbansim.gridcell.population)}

Sometimes multiple levels of geography are required in the aggregation.
For example, the following variable computes the population for each
parcel:

\code{parcel.aggregate(household.persons, intermediates = [building])}

In the parcel-based models (e.g. the Seattle\_parcel sample model and
data), households are assigned to buildings, and buildings are assigned to
parcels --- so to find the number of people living on a parcel we first
aggregate to the building level, then to the parcel level.

Working in the opposite direction, the \code{disaggregate} method assigns
values from one dataset to another, in a one to many relationship.  (In
contrast the aggregate method is many to one.)  An example is to assign the
zonal population density to all households living the zone.  In this
example, we use the \code{population_density} variable in the zone dataset
of the package{urbansim\_parcel} package, and assign it to households,
which are connected to zones indirectly through buildings and then parcels:
household $\rightarrow$ building $\rightarrow$ parcel $\rightarrow$ zone.
So the expression is

\code{household.disaggregate(urbansim\_parcel.zone.population\_density, intermediates = [building, parcel])}

Note that for both aggregate and disaggregate methods, the first element,
preceding the method name, is the name of the dataset to which the result
should be assigned.  \code{zone.aggregate(...)} should generate some result
and assign it to zone, whereas \code{household.disaggregate(...)} should
assign some value from a larger geography to the household dataset.

The Tekoa expression language has been added to Opus fairly recently, and
there are many variables in the existing \package{urbansim} package that
could be rewritten as expressions rather than using Python classes.  (We
plan to replace these with expressions in specifications in the future; but
in case you run across some Python variable classes that look like they
might as well be simple expressions, that's the reason.)

\section{Syntax}

The syntax of Tekoa is a subset of Python syntax, but the semantics are
somewhat different --- for example, a name like
\code{gridcell.distance_to_cbd} is a reference to the value of that
variable.  (If you just evaluated this in a Python shell you'd get an
error, saying that the \package{gridcell} package didn't have a
\code{distance_to_cbd} attribute.)  Further, expressions are evaluated
lazily (in other words, values are computed only when they are needed).

An expression consists of an attribute name, or a function or operation
applied to other expressions.  Thus, a more complex expression can be built
up from simpler expressions.  For example, here are some legal expressions:

\begin{itemize}

\item \code{gridcell.population}
\item \code{ln(gridcell.population)+1}

\end{itemize}

\section{Variable Names in Expressions}

The attribute names used in an expression can be:

\begin{itemize}
\item a fully-qualified variable name (for a variable defined as a Python
  class).  Example: \code{urbansim.gridcell.population}.

\item a dataset-qualified variable name (for variables in the XML
  expression library) or primary attribute.  Example:
  \code{household.income}.

\item an attribute of a constant dataset.  (Constant datasets are
  identified syntactically --- their names end in the string \verb|_constant|.)
  Example: \code{urbansim_constant.near_highway_threshold}.

\end{itemize}

The variable names can include arguments (Section
\ref{sec:tutorial-numbersinvariables}).\index{Variable!arguments}\index{arguments
  to variables} For example, \class{is_near_highway_if_threshold_is_2}
matches the variable definition for
\class{is_near_SSS_if_threshold_is_DDD}.\footnote{This syntax for arguments
  to variables is not ideal --- we hope to replace it with a more standard
  syntax in the future.}

\section{Unary Functions for Opus Expressions}
\label{sec:functions-for-opus-expressions}

There is a set of unary functions defined to use in expressions, as
follows.  These all operate on numpy arrays.

\begin{description}

\item[\code{clip_to_zero}]\index{clip_to_zero@\code{clip_to_zero} function}
  Returns the input values with all negative values clipped to 0.

\item[\code{exp}]\index{exp@\code{exp} function} Returns an array
  consisting of $e$ raised to the input values.

\item[\code{ln}]\index{ln@\code{ln} function} Returns an array of natural
  logarithms.  Input values of 0 result in 0.  (The intent is that this
  function be used on arrays of values, where 0 denotes a missing value.
  However, be cautious --- as you approach 0 from the positive side, the
  result value becomes more and more negative, and then suddenly returns to
  0 at 0.)

\item[\code{ln_bounded}]\index{ln_bounded@\code{ln_bounded} function}
  Returns an array of natural logarithms. Values less than 1 result in 0.

\item[\code{ln_shifted}]\index{ln_shifted@\code{ln_shifted} function}
  Returns an array of natural logarithms.  The input values are shifted by
  the second argument before taking the log.  (The default shift value is 1.)

\item[\code{ln_shifted_auto}]\index{ln_shifted_auto@\code{ln_shifted_auto}
  function} If the input values includes values that are less than or equal
  to 0, they are shifted so that the minimum of the shifted values is 1
  before taking the log.  Otherwise the log is taken on the original
  values.

\item[\code{sqrt}]\index{sqrt@\code{sqrt} function} Returns an array of
  square roots.  Values less than 0 result in 0.

\item[\code{safe_array_divide}]\index{safe_array_divide@\code{safe_array_divide}
  function} The first three arguments are the nominator, denominator and a
  constant. The function returns an array of numerator / denominator for
  all values where denominator is not 0, otherwise the constant (default
  value is 0). An optional fourth argument controls the type of the
  resulting array. The default value is `float32'.
\end{description}

In addition, all of the functions in Numpy are available.  To avoid name
collisions, the function name in an expression must include the package
name \code{numpy}.  For example, this expression gives you the reciprocals
of all the values in a variable \code{v}:

\code{numpy.reciprocal(v)}

\section{Operators for Opus Expressions}
\index{numpy!operators}

All of the numpy operators can be used in Opus expressions, including
\verb|+ - * / ** | (\verb|**| denotes exponentiation).  Note the
numpy semantics for these --- for example, \verb|*| does elementwise
multiplication of two numpy arrays, or with a scalar argument, scales all
 the elements in an array, e.g.\ \code{1.2*household.income}.

\section{Casting the Value of an Expression}
\index{casting}

Normally the value of a variable defined by an expressions will be a
\code{float64} (this is a numpy type).  For large datasets this may use too
much space.  You can cast the result of any expression to a different type
using the numpy \code{astype} method.  For example, the type of
this expression is \code{float32}:

\begin{verbatim}
ln(urbansim.gridcell.population).astype(float32)
\end{verbatim}

The following numpy types can be used as the argument to the \code{astype}
method: \code{bool8}, \code{int8}, \code{uint8}, \code{int16},
\code{uint16}, \code{int32}, \code{uint32}, \code{int64}, \code{uint64}
\code{float32}, \code{float64}, \code{complex64}, \code{complex128}, and
\code{longlong}.
 
\section{Interaction Sets}
\index{interaction sets}

If you access an attribute of a component of an interaction set in the
context of that interaction set, the result is converted into a 2-d array
and returned.  These 2-d arrays can then be multiplied, divided, compared,
and so forth, using the numpy functions and operators.  For example,
suppose we have an interaction set \code{household_x_gridcell}.  The
component \code{household} set has an attribute \code{income}, and
the \code{gridcell}
component has an attribute \verb|cost|.  Then
\begin{verbatim}
urbansim.gridcell.cost*urbansim.household.income
\end{verbatim}

evalutes to a 2-d array of cost times income.  See 
Section~\ref{sec:urbansim-tutorial-interaction-sets} for sample data illustrating this.

Both the arguments to the operation and the result can be used in more
complex expressions.  For example, if we wanted to give everyone
a \$5000 income boost, and also scale the result, this could be done using
\verb|(household.income+5000)*gridcell.cost * 1.2|.

\section{Aggregation and Disaggregation}
\label{sec:aggregation}
\index{Variable!aggregation} \index{Variable!disaggregation}

The methods \class{aggregate} and \class{disaggregate} are used to
aggregate and disaggregate variable values over two or more datasets.  
\index{aggregate variable} \index{disaggregate variable} The \class{aggregate} method associates
information from one dataset to another along a many-to-one relationship, while
the \class{disaggregate} method does the same along a one-to-many relationship. Some
examples are:

\begin{itemize}
\item \code{zone.aggregate(gridcell.population)}

\item \code{zone.aggregate(2.5*gridcell.population)}

\item \code{zone.aggregate(urbansim.gridcell.population)}

\item \code{neighborhood.aggregate(gridcell.population, 
  intermediates=[zone,faz])}

\item \code{neighborhood.aggregate(gridcell.population, 
  intermediates=[zone, faz], function=mean)}

\item \code{zone.aggregate(gridcell.population, function=mean)}

\item \code{region.aggregate_all(zone.my_variable)}

\item \code{region.aggregate_all(zone.my_variable, function=mean)}

\item \code{faz.disaggregate(neighborhood.population)}

\item \code{gridcell.disaggregate(neighborhood.population, 
      intermediates=[zone, faz])}

\end{itemize}

Here is a description of the syntax for these two methods.  Also see 
Section~\ref{sec:urbansim-tutorial-aggregation} for sample data and results
for the example computations below.

\subsubsection{Aggregation}

Suppose we have three different geographical units: gridcells, zones and
neighborhoods.  We have information available on the gridcell level and
would like to aggregate this information for zones and neighborhoods. We
know the assignments of gridcells to zones and of zones to neighborhoods.

An aggregation over one geographical level for the \verb|locations|
attribute `capacity' can be done by: 
\begin{verbatim}
zone.aggregate(gridcell.capacity)
\end{verbatim}
By default, the aggregation function applied to the aggregated data is the
`sum' function. This can be changed by passing the desired function as second
argument:
\begin{verbatim}
zone.aggregate(urbansim.gridcell.is_near_cbd_if_threshold_is_2, function=maximum)
\end{verbatim}

The \method{aggregate} method accepts the following aggregation functions:
sum, mean, variance, standard_deviation, minimum, maximum,
center_of_mass. These are functions of the scipy package
\module{ndimage}.

An aggregation over two or more levels of geography is done by passing a
third argument into the \class{aggregate} method. It is a list of dataset
  names over which it is aggregated, excluding datasets
  for the lowest and highest level. For example, aggregating
the gridcell attribute `capacity' for the neighborhood set
can be done by: 
\begin{verbatim}
neighborhood.aggregate(gridcell.capacity, function=sum, intermediates=[zone])
\end{verbatim}

\subsubsection{Disaggregation}

Disaggregation is done analogously. The \class{disaggregate} method takes
information from a coarse set of entities and allocates it to a finer set of
entities, in the manner of a one-to-many relationship. By default, the function
for allocating data is to simply replicate the data on the parent entity for
each inheriting entity. The method takes one required argument, an
attribute/variable
name, and one optional argument, a list of
dataset names. Here we distribute an attribute
``is_cbd'' from the neighborhood set across gridcells:
\begin{verbatim}
gridcell.disaggregate(neighborhood.is_cbd, intermediates=[zone])
\end{verbatim}

To aggregate over all members of one dataset, one can use the
built-in method \method{aggregate_all}. It must be used with a dataset
  that has one element which is the case of the
\package{opus_core} dataset \class{AlldataDataset}
implemented in the directory \file{datasets}. For example, the total
capacity for all gridcells can be determined by:
\begin{verbatim}
alldata.aggregate_all(gridcell.capacity, function=sum)
\end{verbatim}
In addition to \method{sum}, the \class{aggregate_all} class accepts all
functions that are accepted by the \class{aggregate} class;
the default is \method{sum}.

If the attribute being aggregated or disaggregated is a simple variable, it
should be either dataset-qualified or fully-qualified, i.e.\ always
including the dataset name and optionally including the package name.  The
attribute being aggregated can also be an expression.  (In this case,
behind the scenes the system generates a new variable for that expression,
and then uses the new variable in the aggregation or disaggregation
operations.  However, this isn't visible to the user.)  The result of an
aggregation or disaggregation can also be used in more complex expressions,
e.g. \code{ln(2*aggregate(gridcell.population))}.

\section{Number of Agents}
\index{number_of_agents}

A common task in modeling is to determine a number of agents of one dataset
  that are assigned to another dataset. For this
purpose, Opus contains a built-in method \class{number_of_agents}, which
takes as an argument the name of the agent dataset.  For example,
the number of households in each location can be determined by:

\begin{verbatim}
gridcell.number_of_agents(household)
\end{verbatim}

Similarly, the number of zones in neighborhoods is computed by
\begin{verbatim}
neighborhood.number_of_agents(zone)
\end{verbatim}

As in the case of \method{aggregate} and \method{disaggregate}, the
\method{number_of_agents} method must be preceded by the `owner' dataset
name, e.g. \verb|neighborhood.number_of_agents| for computing on the
\verb|neighborhood| dataset.

\section{Principal Dataset of an Expression}
\index{expressions!principal dataset}
\index{principal dataset}

Every expression has a single \emph{principal dataset} to which it applies.
For example, the principal dataset for \code{ln(gridcell.population)} is
\code{gridcell}.  Generally, it wouldn't make sense for an expression to
have more than one principal dataset --- for example, the expression
\code{gridcell.population+zone.population} doesn't make sense semantically.

There are a few additional aspects of this rule:

\begin{itemize}

\item A constant dataset doesn't count as a principal dataset --- for example,
\code{gridcell.distance_to_highway<urbansim_constant.near_highway_threshold}
is a meaningful expression, whose principal dataset is \code{gridcell}.
(\code{urbansim_constant.near_highway_threshold} returns an array of
length 1, and numpy replicates that value for each gridcell, so that the
value of the whole expression is an
array of booleans of the same length as the \code{gridcell} dataset.)

\item For an \code{aggregate} or \code{disaggregate} expression, the
  principal dataset is the dataset being aggregated or disaggregated to,
  i.e.\ the dataset on which the aggregate or disaggregate method is being
  called.  For example, \code{zone} is the principal dataset for
  \code{zone.aggregate(gridcell.population)}.

\item For expressions involving interaction sets, different subexpressions
  can reference one or the other of the datasets being interacted.
  For example, the principal dataset for
  \code{urbansim.gridcell.cost*urbansim.household.income} is
  \code{household_x_gridcell}.

\end{itemize}

\section{Equality of Expressions}
\label{sec:expression-equality}
\index{expressions!equality}

Two expressions are equal if their defining strings are identical, ignoring
spaces.  Thus these two expressions are equivalent:

\begin{verbatim}
urbansim.gridcell.population+1
urbansim.gridcell.population + 1
\end{verbatim}

However, two textually different expressions are \emph{not} equivalent,
even if they are algebraically equal, and the system will treat them as
two different variables.  For example,
\verb|1+urbansim.gridcell.population| is different from the previous
expressions.  In many cases this doesn't matter.  Reasons it may matter
are: (1) if the resulting value uses a lot of memory or takes a long time
to compute, having a second copy of the value will waste memory or
computation time; and (2) if the variable defined by the expression is used
in a specification, you could inadvertently end up with two variables.  For
this reason, good practice is to put each expression that you'll need for a
given package and dataset in the expression library.  Elsewhere use its
name.  (Also see Section \ref{sec:expressions-aliasing}.)

\section{Names and Aliasing}
\label{sec:expressions-aliasing}
\index{aliasing}

The expression library includes a name field for each expression.  This is
used to bind the name to the corresponding expression.  When using such
expressions intermixed with Python code, another mechanism is needed to
give a name (or alias) to an expression -- this is done with something much
like an assignment statement, for example:

\code{lnpop = ln(urbansim.gridcell.population)}

Since this functionality is only relevant when writing expressions as part
of Python code, rather than using the GUI and the expression library,
further discussion of this aspect of the Tekoa language is deferred to
Section~\ref{sec:urbansim-tutorial-aliasing}.
