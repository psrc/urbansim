\chapter{Model Types in Opus}
\label{chap:creating-models}
\section{Overview}

Opus provides infrastructure to develop, specify, estimate,
diagnose and predict with a variety of model types.  The
Opus GUI currently supports the creation of several types of
models by providing templates that can be copied and
configured. More will be added in the future.  These initial
types are:

\squishlist
\item Simple Models
\item Sampling Models
\item Allocation Models
\item Regression Models
\item Choice Models
\item Location Choice Models
\squishend 

\section{Simple Models}
%
\label{sec:components-simple-model}
\index{models!opus_core models!simple model}
%
The simplest form of a model in Opus is called, for lack of
imagination, Simple Model.  It is about as simple as a model
can get: compute a variable and write the results to a
dataset.  Here are some examples of what could be done with
a simple model:

\squishlist
\item Aging Model: add one to the age of each person, each
  year
\item Walkability Model: write the result of an expression
  that evaluates the amount of retail activity within
  walking distance
\item Redevelopment Potential Model: compute the ratio of
  improvement to total value for each parcel and write this
  to the parcel dataset \squishend

  A Simple Model template is available in the Model Manager,
  and can be copied and configured in order to create a new
  Simple Model like the examples above. It takes only three
  arguments in its configuration:

\squishlist
\item Dataset: the Opus Dataset that the result will be
  written to
\item Outcome Attribute: the name of the attribute on the
  Dataset that will contain the predicted values
\item Expression: the Opus Expression that computes the
  result to be assigned to the outcome attribute \squishend

\section{Sampling Models}
The second type of model template is a Sampling Model.  This
generic model takes a probability (a rate), compares it to a
random number, and if the random number is larger than the
given probability (rate), it assigns the outcome as being
chosen.  Some examples will make the use of this model
clearer. Say we want to build a household evolution model.
We need to deal with aging, which we can do with a Simple
Model.  We also models that predict:

\squishlist
\item Births
\item Deaths
\item Children leaving home as they age
\item Divorces
\item Entering the labor market
\item Retiring
\squishend

For all of these examples -- assuming that we want to base
our predictions on expected rates that vary by person or
household attributes -- we need a more sophisticated model
that we shall call a Sampling Model.  Since we need to
assign a tangible outcome rather than a probability, we use
a sampling method to assign the outcome in proportion to the
probability.  This method is also occasionally referred to
as a Monte Carlo Sampling algorithm.

The algorithm is simple.  Let's say we have a probability of
a coin toss, heads or tails each having a probability of
0.5.  A sampling model to predict an outcome attribute of
Heads, would take the expected probability of a fair coint
toss resulting in an outcome of Heads as being 0.5.  We then
draw a random number from a univariate distribution between
0 and 1, and compare it to the expected probability. If the
random draw is greater than the expected probability, then
we set the choice outcome to Heads.  If it is less than 0.5,
then we set the choice outcome to Tails.  Since we are
drawing from a univariate random distribution between 0 and
1, we would expect that around half of the draws would be
less than 0.5 and half would be greater than this value.
Larger numbers of draws will tend to converge towards the
expected probability by the law of large numbers.  A very
large number of draws should match the expected probability
to a very high degree of precision.

To make the model useful for practical applications, we can
add a means to apply different probabilities to different
subsets of the data.  For example, death rates or birth
rates vary by gender, age, and race/ethnicity, and to some
extent by income.  We might want to stratify our
probabilities by one or more of these attributes, and then
use the sampling model to sample outcomes using the expected
probabilities for each subgroup.

The Sampling Model takes the following arguments:

\squishlist
\item Outcome Dataset: the name of the dataset to receive the predicted values
\item Outcome Attribute: the name of the attribute to contain the predicted outcomes
\item Probability Dataset: the name of the dataset containing the probabilities
\item Probability Attribute: the name of the attribute
  containing the probability values (or rates)
\item List of Classification Attributes: attributes of
  Outcome Dataset that will be used to index different
  Probabilities (e.g. age and income in household
  relocation) \squishend

\section{Allocation Models}
%
\label{sec:components-allocation-model}
\index{models!opus_core models!allocation model}
%
Another simple generic model supported in Opus is the
Allocation Model, which does not require estimating model
parameters.  This model proportionately allocates some
aggregate quantity to a smaller unit of analysis using a
weight.  This model could be configured, for example, to
allocate visitor population estimates, military population,
nursing home population, and other quantities to traffic
analysis zones for use in the travel model system.  Or it
could be used to build up a simplistic incremental land use
allocation model (though this would not contain much
behavioral content).

The algorithm for this type of model is quite simple.  To
create an Allocation Model, we need to specify six
arguments:

\squishlist
\item Dataset to contain the new computed variable
\item Name of the new computed variable $Y$, which will be
  indexed by the ids of the dataset it is being allocated
  to, $Y_i$.
\item Dataset containing the total quantity to be allocated
  (this can contain a geographic identifier, and will
  include a year column).
\item Variable containing the control total to be allocated,
  $T$
\item Variable containing the (optional) capacity value $C$,
  indexed as $C_i$
\item Variable containing the weight to use in the
  allocation $w$, indexed as $w_i$, with a sum across all
  $i$ as $W$ \squishend

The algorithm is then just:

\begin{equation}
Y_i = min(round(T\frac{w_i}{W}),C_i)
\end{equation}

If a capacity variable is specified, we add an iterative
loop, from $m$ to $M$, to allocate any excess above the
capacity to other destinations that still have remaining
capacity:

\begin{equation}
T^m = sum(round(T\frac{w_i}{W}) - C_i)
\end{equation}

In each iteration, we exclude alternatives where $Y>=C$, and
repeat the allocation with the remaining unallocated total:

\begin{equation}
Y^m_i = Y^{m-1}_i + (T_m\frac{w_i}{W})
\end{equation}

We then iterate over $m$ until $T^m = 0$ 

This simple algorithm is fairly versatile, and can be used
in two modes: as incremental growth or as total values. If
used in incremental mode, it adds the allocated quantity to
the existing quantities.  The alternative, total, mode for
this model replaces the quantities with the new predicted
values.

\section{Regression Models}
%
\label{sec:components-regression-model}
\index{models!opus_core models!regression model}


Regression models are available to address problems in which
the dependent variable is continuous, and a linear equation
can be specified to predict it.  The primary use of this
model in a core model in UrbanSim is the prediction of
property values.  In the context of predicting property
values, the model is referred to as a hedonic regression
\cite{waddell-hedonic-1993}, but the Opus regression model
is general enough to address any standard multiple
regression specification.  Other examples of applications
for this basic class of models would be to predict water or
energy consumption per household, or parking prices.

The basic form is:

\begin{equation}
Y_i = \alpha + \beta X_i + \epsilon_i
\end{equation}

where $X_i$ is a matrix of explanatory, or independent,
variables, and $\beta$ is a vector of estimated parameters.
Opus provides an estimation method using Ordinary Least
Squares, and additional estimation methods are available by
interfacing Opus with the R statistical package.  For the
current discussion, we focus on working with the built-in
estimation capacity.

\section{Choice Models}
%
\label{sec:components-choice-model}
\index{models!opus_core models!choice model}
%

Many modeling problems do not have a continuous outcome, or
dependent variable.  It is common to have modeling problems
in which the outcome is the selection of one of a set of
possible discrete outcomes, like which mode to take to work,
or whether to buy or rent a property.  This class of problem
we will refer to as discrete choice situations, and we
develop choice models to address them.

Recall from Section \ref{sec:discrete-choice} that the
standard multinomial logit model
\cite{mcfadden-1974,mcfadden-1981} can be specified as:

\begin{equation}
    P_i = \frac{\mathrm{e}^{V_i}}{\sum_j \mathrm{e}^{V_j}},
\end{equation}
where $j$ is an index over all possible alternatives,
$V_i = \beta\cdot {x}_i$ is a linear-in-parameters
function, $x_i$ is a vector of observed, exogenous, independent
alternative-specific variables that may be interacted with the
characteristics of the agent making the choice,
and $\beta$ is a vector of $k$ coefficients
estimated with the method of maximum likelihood \cite{greene-2002}.

The multinomial logit model is a very robust and widely used
model in practical applications in transportation planning,
marketing, and many other fields.  It is easy to compute and
is therefore fast enough to use on large-scale computational
problems such as residential location choice.  For
explanatory purposes, we will focus initially on choice
problems with small numbers of alternatives, such as the
choice to rent or own a house, or the number of vehicles a
household will choose to own.

Note that there are limitations to the MNL model, and
assumptions a user should be aware of.  The most important
of these is the Independence of Irrelevant Alternatives
(IIA) property, which implies that adding or eliminating an
alternative from a choice set will affect all of the
remaining alternatives proportionately.  Stated another way,
the relative probabilities of any two alternatives will be
unaffected by adding or removing another alternative.  See
\cite{train-book-2003} for a thorough introduction to
discrete choice modeling using MNL and other choice model
specifications.

We now turn to a tutorial for creating models of some of
these types using the Opus GUI.  In the following sections,
we provide a worked example of creating a new model of each
type.  The other model types follow the same pattern in the
Opus GUI.


