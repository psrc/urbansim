% Copyright (c) 2005-2008 Center for Urban Simulation and Policy Analysis,
% University of Washington.  Permission is granted to copy, distribute and/or
% modify this document under the terms of the GNU Free Documentation License,
% Version 1.2 or any later version published by the Free Software Foundation;
% with no Invariant Sections, no Front-Cover Texts, and no Back-Cover Texts.
% A copy of the license is included in the section entitled "GNU Free
% Documentation License".

\chapter{Running a Simulation or Estimation}
\label{part:running-simulation}
%
\section{Running a Simulation}

An alternative way of running a simulation from the GUI (as described
in Chapter~\ref{chap:scenarios-manager}) is to launch it from a command
line. Opus contains a set of scripts that simplify the process of
creating a base year cache, starting, and restarting a set of
simulation runs. These scripts are a set of command line applications,
or tools, in the \file{opus_core/tools} directory. This section
describes how to run a simulation using these tools. 

All scripts described below print a help message when called with the
\verb|-h| or \verb|--help| option.

% 
\subsection{Run Management}
\label{sec:run-manager}

\subsubsection{The Services Database}

Before we begin, it is important to note that the Opus framework uses a
database to store information about the simulations that have been
run. This database is by default called \verb|services|. 

The default database engine, hostname and other necessary information
for the services database is set in
\file{[OPUS_HOME]/settings/database_server_configurations.xm}. By default,
it will use sqlite (a database management system that just uses local
files). You can, however, update the appropriate settings in the XML
to configure the services database to use a mysql, postgres, or mssql
server. The services database is created automatically on the
specified database server by Opus if it does not yet exist when it is needed.

\subsubsection{Create Baseyear Cache}
\label{sec:run-manager-baseyearcache}
%
Opus simulations usually run on data stored in a base year cache.
If the raw data are stored in a database, 
one can use an Opus tool to create such base year cache.\baseyearcacheindex 
The script needs a configuration module passed as an argument. 
If the module is for example located at
\file{psrc/configs/subset_configuration.py} under one of the paths found in the
PYTHONPATH environment variable, the command
\begin{verbatim}
python create_baseyear_cache.py psrc.configs.subset_configuration
\end{verbatim}
caches the database defined in the \verb|subset_configuration| module into a baseyear
cache. \baseyearcacheindex Run this command from 
the \file{opus_core/tools/} directory. An option \verb|--cache-directory|
can be used to pass the directory to be cached into. Alternatively, this can be
specified in the configuration as an entry \verb|'cache_directory'|. See also
Section~\ref{sec:configuration} for configuration and cache control options.

\subsubsection{Start a simulation using a Python Dictionary Configuration}
\index{start a simulation}

The configuration for a simulation is specified by a configuration object.
Opus supports a configuration defined either as a Python dictionary, or as an xml project file
(see Section \ref{sec:start-simulation-xml}). This section describes the former.
In both cases, the configuration is used to specify different parts
of the simulation, such as the years in which to run UrbanSim, what
UrbanSim models \modelsindex to run each year, what types of development
projects exist, how to configure each type of development project,
etc. See, for instance, the run configuration in
\verb|seattle_parcel/configs/baseline.py| (Python dictionary version) or
\verb|seattle_parcel/configs/seattle_parcel.xml| (xml version).

To use the dictionary version of a configuration, it is required
that the referenced configuration Python module defines
a class (e.g. SubsetConfiguration) in a file whose name is
subset_configuration.py. The class name should be the
CamelCase version of the lowercase_with_underscores file name.

To start a simulation using a dictionary-based configuration, use the
script \verb|start_run| with the desired configuration. First, change to 
the directory \verb|opus_core/tools| and (using for example the configuration from the previous section) execute:
\begin{verbatim}
python start_run.py -c psrc.configs.subset_configuration
\end{verbatim}

Use the \verb|--help| option to see the
possible command line parameters for \verb|start_run|.
Also see Section~\ref{sec:run-manager-configuration} for 
configuring a simulation run.


\subsubsection{Start a simulation using an XML Configuration}
\label{sec:start-simulation-xml}

To start a simulation using an xml configuration from the command line, use
the script \verb|start_run| with the desired configuration.  Change to the
directory that holds the Opus source code and execute:
\begin{verbatim}
python opus_core/tools/start_run.py -x seattle_parcel/configs/seattle_parcel.xml 
       -s Seattle_baseline
\end{verbatim}
(typed all on one line).

Notice that the \verb|-x| option takes a path to the file with the xml
configuration.  Since xml configurations can hold multiple scenarios, the
scenario name must also be specified using the \verb|-s| option.

\subsubsection{What Happens When Running a Simulation?}
\label{sec:run-manager-tasks}
Here are the steps that occur when you start a run via the \verb|start_run|
script:

\begin{itemize}
\item It copies files from the baseyear cache into
a cache for the current run.
\item It adds a row to the \verb|run_activity| table in the \verb|services|
database, using a new \verb|run_id| value unique to this run. In order to help
match runs with their cache directory, the name of the cache directory begins
with the \verb|run_id| value, e.g. \file{run_342.2006_04_25_09_40}.
\item For each simulated year:
  \begin{itemize}
  \item Forks a new process to run the set of UrbanSim models \modelsindex for this year.
  The set of models \modelsindex to be run is specified by the configuration. Using a
  separate process helps reduce memory usage, \index{Memory!Using separate
  process} and helps reduce the impact of problems such as memory leaks.  This
  process writes a log file named, e.g., \file{year_2003_log.txt}.
  \end{itemize}
 \item The run activity table includes status information about the run.  If
  the simulation succeeds, it will add another row to the run activity
  indicating it is done.  If the simulation fails, it will add a row indicating
  that.  The run activity also contains a copy of the configuration, which is
  used when restarting a run.
  \item Whenever a row is added to the \verb|run_activity| table, a row is
  either added or updated in the \verb|available_runs| table in the
  \verb|services| database. This table has a single row per run and records the
  row's current state and information.
\end{itemize}

\subsubsection{Restart a simulation}
\index{restart a simulation}
If you halt a run or it fails, you can restart it at the beginning of any year.
To restart the run with \verb|run_id| 42 at the beginning of year 2005, do:
\pythonindex
\begin{verbatim}
python restart_run.py 42 2005
\end{verbatim}
from the \verb|opus_core/tools| directory. Note that the above command will delete any
simulation cache \simulationcacheindex directories for years 2005 onward, since
this information is no longer valid once the simulation is restarted at the
beginning of 2005. One can suppress this behavior using the option \verb|--skip-cache-cleanup|.

\section{Running an Estimation}
\label{sec:running-estimation}


\section{Configurations}
\label{sec:configuration}
%
A configuration is a specification of what parts to use in the simulation or estimation
and how to configure each part.  Parts include the list and order of models \modelsindex to
run, where to get the input values, where to store the output data, what years
to simulate, what tables to store into the UrbanSim baseyear and simulation
caches \baseyearcacheindex\simulationcacheindex, etc. Each of these parts in
turn may be configurable. Configurations are used in many places in Opus.  Typically, they are specified
via a Python \pythonindex dictionary that then is used to create an instance of the
\class{Configuration} class.

\subsection{Run Manager Configuration}
\label{sec:run-manager-configuration}
\index{Run Manager configuration (command line version)}

As described in Sections~\ref{sec:run-manager} and~\ref{sec:running-estimation}, 
running UrbanSim simulation or estimation is controlled by a user-defined
configuration. The following code, \file{baseline.py}, contains a fully specified configuration
that influences the run management. Mandatory entries and default
values for optional entries are marked in the comments. The actual values for
the listed entries are only examples.

\baseyearcacheindex
\begin{verbatim}
from opus_core.configuration import Configuration
from opus_core.database_management.configurations.scenario_database_configuration \
    import ScenarioDatabaseConfiguration
from opus_core.database_management.configurations.estimation_database_configuration \
    import EstimationDatabaseConfiguration
from opus_core.configurations.baseyear_cache_configuration \
    import BaseyearCacheConfiguration
from urbansim.configurations.creating_baseyear_cache_configuration \
    import CreatingBaseyearCacheConfiguration
    
class Baseline(Configuration):
    def __init__(self):
        config = {
          'project_name':'urbansim_gridcell',
          'description':'baseline',  
          'model_system':'urbansim.model_coordinators.model_system', # mandatory
          'base_year': 2000,   # default: read from table 'base_year' in cache
          'years': (2001, 2030),                                     # mandatory
          'cache_directory':'d:/urbansim_cache/',                    # mandatory
          
          'scenario_database_configuration': ScenarioDatabaseConfiguration(
              database_name = 'my_baseyear_database'), # mandatory for simulation
              
          'estimation_database_configuration': EstimationDatabaseConfiguration(
              database_name = 'my_estimation_database'), # mandatory for estim.
              
          'creating_baseyear_cache_configuration': CreatingBaseyearCacheConfiguration(
              default: 'opus_tmp'+random string
              cache_directory_root = 'd:/urbansim_cache',
              
              cache_from_database = False, # default: True
              
              # mandatory if 'cache_from_database' is False
              baseyear_cache = BaseyearCacheConfiguration(
                 # mandatory for this block
                 existing_cache_to_copy = 'd:/urbansim_cache/run_397.2006_05_23_18_21',
            
                 # default: all years in 'existing_cache_to_copy'
                 years_to_cache = range(1996,2001)
              ),
              tables_to_cache = [ # default: []
                      'gridcells', 'households', 'jobs', 'zones'
                                 ]
              tables_to_cache_nchunks = { # default: each table defaults to 1
                      'gridcells':2,
                   },
              tables_to_copy_to_previous_years = { # default: no copied tables
                 'development_type_groups':1996, # table name and year to put it in
                 'development_types':1996,
                 'development_type_group_definitions':1996,
                 'urbansim_constants': 1996,
                   },
              unroll_gridcells = True  # default: True
           ),
          Configuration.__init__(self, config)
\end{verbatim}

The \verb|'model_system'| entry is the full Opus path to the model system that
will be used by the run manager to run/estimate a set of models.

Entry \verb|'years'| determines for what
years the simulation should run as a tuple with first and last year to run.

Entry \verb|'cache_directory'| is used by the scripts \verb|create_baseyear_cache.py| and 
\verb|start_estimation.py| only. It is not used for simulations.

Entry \verb|'creating_baseyear_cache_configuration'| contains a configuration
for creating the simulation cache. Entry \verb|'cache_directory_root'| is the root 
directory where data should be
cached during processing. The actual cache directory is created as a subdirectory
of this location. 

The entry \verb|'tables_to_cache'| is used by the script \verb|create_baseyear_cache.py|. 
Only tables listed here are cached from database into the baseyear cache.

If a database table is so large that Python \pythonindex runs out of memory when copying it
to cache, you can reduce memory usage (but increase the time it takes to cache
the data) by increasing the number of ``chunks'' in which the dataset's \datasetindex
attributes are read from the table. \index{Memory management!When caching data
from input store}\index{Memory management!tables_to_cache_nchunks@\texttt{tables_to_cache_nchunks}} By
default, all attributes of a table are read in a single chunk. Setting the
\verb|'tables_to_cache_nchunks'| configuration  will tell the caching code
to use that many chunks.  For instance, if a dataset has 11 attributes, setting
\verb|'tables_to_chunk_nchunks'| to 3 will use three chunks loading 4, 4, and 3
attributes, in each chunk.

In the \verb|'baseyear_cache'| \baseyearcacheindex block, the directory with the already cached data
should be put into the entry \verb|existing_cache_to_copy|. The run manager then
copies data from that directory into the simulation cache \baseyearcacheindex for this run. If you
want to copy only selected years, they can be specified in the entry
\verb|years_to_cache| as a list of those years; by default all years are
copied. Note that this behaviour can be alternatively controlled directly from
the command line (see \verb|start_run.py -h|)
which has priority over entries in this configuration.

The \verb|'tables_to_copy_to_previous_years'| entry is used when a
lag variable needs to compute data for before the base year.  
If this is the case, add those
tables to this list, and indicate the year to which
to copy the tables.  In general, it is safe to copy the tables to the earliest
year created by the unroll gridcell process.  You can determine what this year
is by examining the year directories created in your baseyear cache.

The entry \verb|unroll_gridcells| is specific to the urbansim gridcell project. 
It controls if gridcells are unrolled into years before the base year. 

There are several run manager configurations in Opus. See for example the
directory \file{psrc/configs} for configuration of different PSRC \psrcindex runs.

\subsection{Model System Configuration}
\label{sec:model-system-configuration}
%
If one would pass the above configuration to the run manager, it would perform
steps as described in Section~\ref{sec:run-manager-tasks}, but no model \modelsindex would
be run. The configuration should in addition contain entries that control what
models, \modelsindex in what order and with what input and output should be run. It
determines the behaviour of the class \class{ModelSystem}. UrbanSim basic
configuration of the model \modelsindex system can be found in the file
\file{urbansim/configs/general_configuration.py} as an example.

The set of models \modelsindex to run is specified by the entry ``models''. \modelsindex It is a list of
user-defined model names. The order in this list also specifies the order in
which they are processed. For example, the UrbanSim gridcell project consists by default of
following models: \modelsindex
\begin{verbatim}
   'models': [
        "prescheduled_events",
        "events_coordinator",
        "residential_land_share_model",
        "land_price_model",
        "development_project_transition_model",
        "residential_development_project_location_choice_model",
        "commercial_development_project_location_choice_model",
        "industrial_development_project_location_choice_model",
        "development_event_transition_model",
        "events_coordinator",
        "residential_land_share_model",
        "household_transition_model",
        "employment_transition_model",
        "household_relocation_model",
        "household_location_choice_model",
        "employment_relocation_model", 
       {"employment_location_choice_model": {"group_members": "_all_"}},
        "distribute_unplaced_jobs_model"
       ]
\end{verbatim}\label{pg:config-models}
Note that the list can contain a particular model multiple times if that model should
run multiple times within one year, such as the ``events_coordinator'' or
``residential_land_share_model'' in the list above.

We can also define a situation when the same model should be run on different subsets of 
a dataset, so called model group\index{model group}. Then we can give the names of the group members to be run, or 
just configure the model group to be run on all subsets. This is the case of ``employment_location_choice_model''
in the list above (see Section~\ref{sec:model-controller-configuration} for further details).
 
By default, the \class{ModelSystem} runs the method
\method{run()} of the listed models. \modelsindex Each entry in this model list can be
alternatively a dictionary containing one entry: The name of the entry is the model
name, the value is a list of model methods to be processed. Thus, one can combine
estimation and simulation of different models. \modelsindex

In addition (or alternatively), the configuration can contain an entry
``models_in_year''. \modelsindex It is a dictionary where keys are years. Each value is
expected to be such list of models \modelsindex as above. In each year, it is checked if
``models_in_year'' \modelsindex (if it is present) contains that year. If it is the case,
its list of models \modelsindex is run, instead of the global set of models. \modelsindex This allows
users to set different set of models \modelsindex for different years, for example an
additional model can be run only in the first year, or last year.

For each entry in the model \modelsindex list there must be a corresponding entry in
the ``controller'' configuration which specifies how models \modelsindex are initialized,
what methods to run and what arguments should be passed in. This will be
described in Section~\ref{sec:model-controller-configuration}.

Furthermore, the configuration can contain the following entries (the given
values are defaults set by our system):
\begin{verbatim}
    'datasets_to_cache_after_each_model':[],
    'flush_variables': False,
    'seed':0,
    'debuglevel':0
\end{verbatim}

The entry \verb|'datasets_to_cache_after_each_model'| specifies names of datasets \datasetindex that
are flushed from memory to simulation cache \simulationcacheindex at the end of each model
run. \index{Memory management!Flushing attributes}\index{Memory management!datasets_to_cache_after_each_model@\texttt{datasets_to_cache_after_each_model}} This reduces the memory
usage, but can increase the run time. We recommend to put datasets in this list
that contain huge amount of data, e.g.  ['gridcell', 'household', 'job'].

\verb|'flush_variables'| can further decrease the memory usage. \index{Memory management!flush_variables@\texttt{flush_variables}} If it is True, after each variable
computation all dependent variables are flushed to simulation
cache \simulationcacheindex, regardless to what dataset \datasetindex the variables belong to.
Nevertheless, it increases the run-time considerably.

Entry \verb|'seed'| specifies the seed of the random number generator that is set at
the beginning of each simulated year. It is passed to the \module{numpy} \numpyindex
function \method{seed()}. If it is 0, the function generates a pseudo-random seed
from the current time.

\verb|'debuglevel'| controls the amount of output information.

Models \modelsindex usually need various datasets \datasetindex to run with. They are specified in the
configuration entry \verb|'datasets_to_preload'|. For example,
\begin{verbatim}
    'datasets_to_preload': {
        'gridcell':{"id_name": "grid_id"},
        'household':{}
\end{verbatim}
It is a dictionary that has dataset \datasetindex names as keys. Each value is again a
dictionary with argument-value pairs that are passed to the corresponding
dataset \datasetindex constructor. \class{ModelSystem} \modelsindex creates those datasets \datasetindex at the
beginning of each simulated year and they are accessible to the models \modelsindex
definition in the controller through their names (see
Section~\ref{sec:model-controller-configuration} for details). One should put
here all datasets \datasetindex that will be passed as arguments to the model constructors
or to the model methods to be processed.

From the preloaded datasets, \class{ModelSystem} creates a dataset pool (accessible
to the models as \verb|dataset_pool|). Creating this pool is controlled by
the entry
\begin{verbatim}
    'dataset_pool_configuration': DatasetPoolConfiguration(
                package_order=['urbansim', 'opus_core'],
                )
\end{verbatim}


\subsection{Model Controller Configuration}
\label{sec:models-configuration}
\label{sec:model-controller-configuration}
%
The run configuration can contain an entry \verb|'models_configuration'| \modelsindex which can
include any information specific to models \modelsindex or common to a set of models. \modelsindex The
value of this entry is a dictionary.  Model specific information would be
included in an entry of the same name as the model name used in the entry
\verb|'models'| (see Section~\ref{sec:model-system-configuration}). The
\class{ModelSystem} \modelsindex class makes this information available to the controller
by creating two local variables: \verb|'models_configuration'| \modelsindex (containing the
value of \verb|'models_configuration'| \modelsindex and available to all models) \modelsindex and
\verb|'model_configuration'| \modelsindex (available to each model at the time of its
processing and containing information for this model). See the variable
\verb|'models_configuration'| \modelsindex in the file \file{urbansim/configs/general_configuration.py} for an
example how UrbanSim configures models. \modelsindex

Each model that is included in the configuration entry \verb|'models'| \modelsindex must have a
controller entry in the \verb|'models_configuration'| \modelsindex entry.  More specifically, the model specific
section of \verb|'models_configuration'| \modelsindex is expected to contain an entry \verb|'controller'|
for each model. For example, a controller specification for the model
specified by the name \verb|'land_price_model'| would be contained in\\
\verb|config['models_configuration']['land_price_model']['controller']|.

If a model is specified as a model group \index{model group} it is possible to define a member specific controller, called 
{\em member_name} + '_' + {\em model_name}, e.g. \verb|'home_based_employment_location_choice_model'|.
When choosing the right controller, the \class{ModelSystem} checks for the member specific name. If it is not found,
it uses the group name, in this example \verb|'employment_location_choice_model'|.

The value of this controller entry is a dictionary with a few well-defined entries:
\begin{description}
\item["import"] A dictionary where keys are module names and values are names
  of classes to be imported.
\item["init"] A dictionary with a mandatory entry "name". Its value is the
  name of the class (or class.method) that creates the model. It can be the
  name of the model class itself.  Or, if the model is created via   a
  method e.g. \method{get_model()} of a class \class{MyModelCreator}, it would be
  given as "MyModelCreator().get_model".

  Optional entry "arguments" specifies arguments to be passed into the
  constructor. It is given as a dictionary of argument names and values. All
  values are given as character strings and are later converted by
  \class{ModelSystem} to python \pythonindex objects. If an argument value is suppose to be
  a character string object, it must be given in double quotes, e.g.
  "'my_string'".
\end{description}
If the model in the 'models' entry of the configuration is specified as model group\index{model group}, the controller must contain 
an entry
\begin{description}
\item["group_by_attribute"] Its value is a tuple of a grouping dataset name and grouping attribute (see Sec.~\ref{sec:model-group}).
They define the specific kinds of 
subsets of agents on which this model can be run. This dataset must be contained in the 
\verb|datasets_to_preload| entry of the configuration. For example, in the controller of the 
``employment_location_choice_model'' this entry is
\verb|('job_building_type', 'name')|, since the attribute 'name' of the dataset 'job_building_type' contains the various
building types of jobs for which we want to run the model, i.e. 'commercial', 'governmental', 'industrial' and 'home_based'.
If the 'group_members' entry (of the 'models' entry of the configuration) for this model is equal to '_all_', the model runs 
for all values found in this dataset. The  'group_members' can also be a list specifying explicitely for which types the model 
should be run. 
\end{description}

The \class{ModelSystem} class evaluates the given imports and creates an
instance of the model by processing the \verb|'init'| entry. The remaining entries
below are related to specific methods of the created model instance.  As
mentioned in Section~\ref{sec:model-system-configuration}, models \modelsindex
that are listed in the \verb|'models'| entry of the run configuration can be also
specified using a list of methods to be processed. If the list is not given, a
method \method{run()} is assumed to be the only method to be processed. The
\class{ModelSystem} iterates over the set of methods. It first processes a
``preparation'' method (if required) and then the method itself. For this purpose,
the controller should contain the following entries:

\begin{description}
\item['prepare_for_...'], where \verb|...| is the the method
to be processed, e.g. \verb|'prepare_for_run'| is the method to call to prepare
to run. This configuration entry is a dictionary with an
optional entry \verb|'name'| giving the name of the preparation method. If \verb|'name'| is
missing, the method name is assumed to be the same as this entry name. Optional
entry \verb|'arguments'| specifies arguments of this method (see \verb|'arguments'| in \verb|'init'|
above). Optional entry \verb|'output'| defines the name(s) of the output of this
method.  It can be then used as an input to other methods or models. The entry
\verb|'prepare_for...'| is optional and if it's missing, no preparation procedure is
invoked. There can be as many \verb|'prepare_for...'| entries as there are
methods specified.
\item[{\it procedure}] The procedure name must match to the method names given
in \verb|'models'| (there must be one entry per method), or be called \verb|'run'| if no
methods are specified in \verb|'models'|. It is a dictionary with optional arguments
\verb|'arguments'| and \verb|'output'| (see above).
\end{description}

The entry \verb|'arguments'| in the above items can contain any character
strings that are convertable (using python's \verb|eval()|) to python \pythonindex objects,
including python \pythonindex expressions. They must be objects that are known to the
\class{ModelSystem}, for example datasets \datasetindex that are defined in
\verb|'datasets_to_preload'| (described in
Section~\ref{sec:model-system-configuration}), since those are created prior to
the simulation. They can be called either by the dataset \datasetindex name, or using
\code{datasets['{\it name}']}. Also, the \verb|model_configuration| and
\verb|models_configuration| objects described in
Section~\ref{sec:models-configuration} can be used in \verb|'arguments'|. Other
objects that \class{ModelSystem} provides are  \verb|cache_storage|
(\class{Storage} object for the simulation cache \simulationcacheindex storage
in the simulated year), \verb|base_cache_storage| \baseyearcacheindex (\class{Storage} object for
the baseyear cache \baseyearcacheindex storage in the base year),
\verb|model_resources| (all preloaded datasets as an object of
\class{Resources}),  \verb|resources|
(Configuration passed into the simulation), \verb|datasets| (all preloaded datasets as a dictionary),
\verb|year| (simulated year), \verb|base_year| (the base year),
\verb|dataset_pool| (object of class \class{DatasetPool} pointing to the current
dataset pool), \verb|debuglevel| (integer controlling the amount of output messages).  
If you are using any class names
as arguments, you need to make sure, that those classes are known to the
\class{ModelSystem}, e.g. by putting the appropriate import statement into the
\verb|'import'| section of the controller.

Here is an example of a controller settings for the land price model in UrbanSim:
\modelsindex\datasetindex
\begin{verbatim}
run_configuration['models_configuration']['land_price_model']['controller'] = {
    "import": {"urbansim.models.corrected_land_price_model":
                                                "CorrectedLandPriceModel",
              },
    "init": {"name": "CorrectedLandPriceModel"},
    "prepare_for_run": {
        "arguments": {"specification_storage": "base_cache_storage",
                      "specification_table": "'land_price_model_specification'",
                      "coefficients_storage": "base_cache_storage",
                      "coefficients_table": "'land_price_model_coefficients'"},
        "output": "(specification, coefficients)"
        },
    "run": {
        "arguments": {"n_simulated_years": "year - base_year",
                      "specification": "specification",
                      "coefficients":"coefficients",
                      "dataset": "gridcell",
                      "data_objects": "datasets" ,
                      "chunk_specification":"{'nchunks':2}",
                      "debuglevel": "debuglevel"
                     }
           }
   }
\end{verbatim}
Note on an implementation of model group\index{model group}: A constructor of 
a model group, must take as its first argument an object of class \class{ModelGroupMember} (Sec.~\ref{sec:model-group}). 
The controller should though ignore this argument, since the \class{ModelSystem} automatically takes care of creating this object 
and passing it to the model constructor.

\section{Output}
%
\subsection{Exporting the Output Data}

The simulation reads and writes all of its data from the simulation cache.  It 
does not directly read or write to any database. 
If you wish to move data from the simulation cache to a SQL database, use the
\verb|do_export_cache_to_sql_database.py| tool located in
\file{opus_core/tools}. 


\subsubsection{Deleting the File-Based Cache}

The \verb|delete_run| script in \file{opus_core/tools} directory
provides an easy way to delete cached run data while maintaining the
consistency of the services database. 

To delete all data for run with \verb|run_id| 42, and remove that run's
information from the \verb|services|
database use:

\pythonindex
\begin{verbatim}
python delete_run.py --run-id 42
\end{verbatim}

To delete a set of years without removing the information from the
\verb|services| database, use the \verb|--years-to-delete| option.  This
option takes an arbitrary Python \pythonindex expression that creates a list of integers.
For instance, to remove the cached data for years 2001 through 2029 use:

\pythonindex
\begin{verbatim}
python delete_run.py --run-id 42 --years-to-delete range(2001,2030)
\end{verbatim}

