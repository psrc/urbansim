% $Id: indicators.tex,v 1.61 2007/06/01 23:49:22 borning Exp $

% Copyright (c) 2005-2007 Center for Urban Simulation and Policy Analysis,
% University of Washington.  Permission is granted to copy, distribute and/or
% modify this document under the terms of the GNU Free Documentation License,
% Version 1.2 or any later version published by the Free Software Foundation;
% with no Invariant Sections, no Front-Cover Texts, and no Back-Cover Texts.
% A copy of the license is included in the section entitled "GNU Free
% Documentation License".

\chapter{Generating and Visualizing Indicators}
\indicatorsindex

As used in the planning literature, an indicator is a
variable \variablesindex that conveys information on the condition or
trend of an attribute \attributesindex of the system considered.  The
indicator will then have a specific value at a given time.
For UrbanSim, indicators provide the principal mechanism
for presenting simulation results to modelers and other stakeholders so
that they can be assessed and compared.  In addition, modelers use
indicators diagnostically to help assess whether the
system is operating in a reasonable fashion and to help debug problems.

We will often be interested in the value of an indicator at different levels of
aggregation, for example, 
population in each grid cell, in different political divisions of
the region, and for the region as a whole.  We will often also be
interested in the change in the value of an indicator
in successive years, or from each year of the
simulation to the baseyear, or between two different scenarios.
Indicator values should be displayed in an
appropriate way, for example, using graphs,
\index{graphs} tables, \index{tables} or choropleth maps.
\index{maps} Some key indicators for both policy
evaluation and model diagnosis include population, residential
units, land value, employment, and square feet of commercial,
industrial, and governmental space, all at various levels of
aggregation, from the grid cell up.

Requests for indicator visualizations can be made using either a graphical
interface (Section \ref{sec:indicator-configuration-gui}), or a Python
script (Section \ref{sec:indicator-configuration-script}).  The GUI is more
user-friendly, but only allows a single indicator request to be made at a
time.  Scripts are Python code, but one script can specify an entire
batch of indicators to be computed. Some of the more advanced functionality is
also available only through the script. 

There is online documentation for some of the indicators, 
linked from \url{http://www.urbansim.org/docs/indicators/index.xml}.  
See Section \ref{sec:writing-indicators} for information on writing indicator
\indicatorsindex documentation.  (Formerly we computed the values for
indicators using SQL queries, but this proved too slow in many cases, so we
switched to using Opus attributes exclusively.  There was much more
extensive documentation for the SQL versions of the indicators; if there is
demand for this and as time allows, we will also provide documentation for
other indicators represented as Opus attributes.)

\section{Using Opus Attributes in Indicators}

The basic class for dealing with data in Opus is the class \class{Dataset}
\datasetindex (Section \ref{sec:opus-core-datasets}).  A dataset
\datasetindex is a collection of attributes \attributesindex for a
particular type of entity, such as a set of grid cells, or a set of
households.  Each member in this set has the same set of characteristics,
such as income of household.  In Opus, these characteristics are called
attributes. \attributesindex Attributes \attributesindex can be either read
from a data store (primary attributes) \primaryattributesindex or computed
using an Opus variable definition (computed
attributes). \computedattributesindex

Any Opus attribute \attributesindex (primary or computed)
\computedattributesindex\primaryattributesindex can be used as an
indicator, \indicatorsindex although of course only some attributes
\attributesindex will be particularly \emph{useful} as
indicators. \indicatorsindex The primary attributes \primaryattributesindex
of interest are commonly in the database tables for the given Opus
application.  For UrbanSim, these database tables and their attributes
\attributesindex are documented in Chapter
\ref{chapter:urbansim-database-tables}, ``UrbanSim Database Tables.''  
% (A
% fine point: models or other Opus code can also create other primary
% attributes, \primaryattributesindex even on the fly --- so the database
% tables don't provide a comprehensive list of primary
% attributes. \primaryattributesindex However, probably all of the primary
% attributes \primaryattributesindex of interest for indicators
% \indicatorsindex will be in the database tables.)  
Each computed attribute
\computedattributesindex is defined by an Opus variable \variablesindex
definition.

Indicators can be computed and visualized, via a
GUI\ref{sec:indicator-configuration-gui} or Python script
\ref{sec:indicator-configuration-script}. Before describing these in detail, 
a number of different ways that attributes can be specified needs to be
described; specifically, attributes can be specified through its
fully-qualified name, as an aggregation of other attributes, or as an expression
involving multiple attributes. 

\subsection{Definition through Fully-qualified Name}

For both primary \primaryattributesindex and computed attributes,
\computedattributesindex the attribute \attributesindex to be used as an
indicator \indicatorsindex can be identified by its fully-qualified name,
for example:

\begin{itemize}
\tight
\item \module{urbansim.gridcell.residential_units}
\item \module{urbansim.gridcell.population}
\end{itemize}

Of these, \module{urbansim.gridcell.residential_units} is a
primary \primaryattributesindex attribute --- the number of residential
units is part of the data stored for each gridcell. 
\computedattributesindex Population is computed, even for
gridcells; for a gridcell, it is computed by summing the number of
persons in each household located in that grid cell. 

Attributes \attributesindex can also be in project-specific packages in
addition to ones in the \package{urbansim} package.  For example, in our
PSRC \psrcindex application of UrbanSim, one of the indicators
\indicatorsindex is
\module{psrc.zone.travel_time_hbw_am_drive_alone_to_cbd}, for the
\verb|zone| geography defined for this application.

As with other Opus variables, \variablesindex the variable \variablesindex
name for variables \variablesindex used as indicators \indicatorsindex can
be a template that matches a family of related variables, \variablesindex
such as \module{psrc.houeshold.has_DDD_cars}.  This variable
\variablesindex can then be instantiated with a particular number of cars,
e.g.\ \module{psrc.houeshold.has_2_cars}.
\subsection{Definition through Aggregation}
Expressions may also be used in the place of simple variables 
(see Section~\ref{sec:opus-core-expressions}). For example, the following are
all valid names of an indicator: 

\subsection{Definition through Aggregation}

Attributes can also be specified as aggregations of primary or computed
attributes. For example:

\begin{itemize}
\tight
\item \module{zone.aggregate(gridcell.residential_units, function=sum)}
\item \module{alldata.aggregate_all(urbansim.zone.number_of_home_based_jobs)}
\end{itemize}

In these examples, an attribute defined at a particular geographic dataset is
aggregated to a larger dataset. For example,
\module{zone.aggregate(gridcell.residential_units, function=sum)} will, for
every zone, sum the residential_units of all gridcells that comprise that zone. 
The second example,
\module{alldata.aggregate_all(urbansim.zone.number_of_home_based_jobs, function=sum)}, will
result in a single value because the alldata dataset represents a single
regional value. The number of home-based jobs in all zones will be added
together to produce a single region-level total. 

\subsection{Definition through Expressions}

Variable expressions may also be used in the place of an attribute 
(see Section~\ref{sec:opus-core-expressions}). For example, the following are
all valid attribute declarations for an indicator: 

\begin{itemize}
\tight
\item \module{2 * urbansim.gridcell.population + 1}
\item \module{urbansim.gridcell.residential_units + urbansim.gridcell.commercial_units}
\end{itemize}

\section{The Indicator GUI}
\label{sec:indicator-configuration-gui}

The Indicator GUI is implemented using the Enthought Traits packages
(\url{http://www.enthought.com}).  It provides a graphical editor for
specifying and generating an indicator.  The GUI can be started by running
the indicator_gui.py located in the indicators subdirectory of each
project. For example, for the PSRC application, run
\file{psrc/indicators/indicator_gui.py}.  This will open the indicator GUI.

Currently, there are two parts of the GUI: one for specifying scenario 
information, and another for specifying the indicator. 

\subsection{Specifying Scenario Information}
The scenario specification pane is where you input information 
about which scenario results should be used for computing the indicators. 
The fields are:

\begin{itemize}

\tight

\item \code{Cache directory} is the directory holding the cache with the
simulation results from which indicators are to be produced.

\item \code{Comparison cache directory} is an optional field. It can be set to
point to another cache directory of simulation results.  If it is
specified, the computed indicators will be compared between the two
scenarios. See section \ref{sec:indicator-cross-scenario} for more details.

\item \code{Compare to another cache directory} is a checkbox that toggles
on and off the ability to specify a comparison cache directory.

\end{itemize}

\subsection{Specifying Indicators}

The indicator specification pane is where you specify the indicator you
wish to compute. There are a number of possible output types.  Three of
these (Tab-deliminated table, Comma-separated table, and Dbf table) are all
variants of a basic ``table'' output type, only varying by the particular
format.  ``Chart'' and ``map'' produce \file{.png} files with an
appropriate chart or map, produced by the \package{matplotlib} package.
See Section \ref{sec:indicator-configuration-script}
for more information on these output formats.

The fields are:

\begin{itemize}
\tight
\item \code{Type} is the output type of the indicator.

\item \code{Attribute} is the fully qualified path of an opus variable, as 
described at the beginning of this section. 

\item \code{Name} is the desired name of the indicator. This field is optional.
It defaults to the attribute field.

\item \code{Dataset} is the dataset for which the indicator should be 
computed from (e.g. grid cell, zone).

\item \code{Years} lets you select the years for
which these indicator values should be computed. Examples include
``2001,2003'' and ``2010-2020'', the latter of which results in 
the indicator being computed for all years between 2010 and 2020
inclusive. 

\item \code{Comparison cache directory} is an optional field. It can 
be set to point to another cache directory of simulation results.
If it is specified, the computed indicators will be compared 
between the two scenarios. See section \ref{sec:indicator-cross-scenario} 
for more details. 

\item \code{Compare to another cache directory} is a checkbox that toggles
on and off the ability to specify a comparison cache directory.

\end{itemize}

Some of the visualization types have additional fields. For example, the
\verb|map| visualization type has a \code{Scale} field which specifies 
the minimum and maximum values for the map coloring. 

There is a ``run all requests'' button at the bottom of the GUI that starts the 
indicator computations.  There is also a ``view results'' button that launches 
a static HTML page with the results of the indicator computations in a web 
browser (see section \ref{sec:indicator-results}). In addition, the ``file'' 
menu includes items for saving a configuration, and opening a previously saved 
one.  (The ``run'' action is also available via the run menu.) When the 
indicators are being computed, both the ``run all requests'' and ``view 
results'' buttons are grayed out. It is safe, however, to change the fields in 
the GUI to different values in preparation for the next batch.


\section{Constructing a Batch of Indicators using a Python Script}
\label{sec:indicator-configuration-script}

It is also possible to set up a indicator request configuration
programatically, i.e.\ using a Python script. A full example can be found at 
\module{opus_core.indicator_framework.make_indicators_example}. In this 
section, the elements of setting up an indicator request configuration 
is described. There are three parts: setting up the \verb|SourceData| 
object, generating a list of indicator objects, and running the 
indicators through the \verb|IndicatorFactory|.

\subsection{Constructing the SourceData object}

The first step is to construct a \verb|SourceData| object.
The SourceData specifies 
the location of the simulation results that should be 
used for computing the indicators, the years for which the indicators should be
computed, and, optionally, a second data directory for which the indicators will
be compared against. Each indicator requires a source data object to be passed
to it. Every SourceData object accepts the following arguments:
\begin{itemize}
\tight
\item \code{dataset_pool_configuration} is an object that handles which 
datasets get loaded and in what order. 

\item \code{cache_directory} is a path to the directory containing the 
simulation results that the indicators should be computed from.

\item \code{comparison_cache_directory} is an optional field. Set this field 
to the path to a second cache directory in order to run a 
cross-scenario indicator comparison. See \ref{sec:indicator-cross-scenario}
for more information.

\item \code{run_description} is a description of this indicator batch. 
This field is optional. 

\item \code{years} are the default years that all the indicators will be computed for.
This field is optional, although all indicators will then need to have a 
years field if not specified here.

\end{itemize}

Here is an example:

\begin{verbatim}
source_data = SourceData(
   cache_directory = r'D:\urbansim_cache\run_1090.2006_11_14_12_12',
   comparison_cache_directory = r'D:\urbansim_cache\run_1091.2006_11_14_12_12',
   years = [2010],
   dataset_pool_configuration = DatasetPoolConfiguration(
         package_order=['urbansim','opus_core'],
         package_order_exceptions={},
         ),                  
)
\end{verbatim}

\subsection{Creating Indicator Objects}

The next step is to create a list of indicators to generate. There are a 
number of different visualization types available. 

\begin{itemize}
\tight
\item \code{table} \index{tables} produce a table \index{tables} of indicator \indicatorsindex values
\item \code{chart} \index{charts} produce a chart \index{charts} or graph \index{graphs} using matplotlib \matplotlibindex
\item \code{map} \index{maps} produce a choropleth map \index{maps} using matplotlib \matplotlibindex
\item \code{dataset_table} produces a table for every specified year 
with the values of each of the specified indicators.
\end{itemize}

First, the fields 
that are common to each visualization type are described, and then examples and
specific fields are described for each visualization type. Every indicator 
object takes the following parameters:

\begin{itemize}
\tight
\item \code{source_data} references the desired \verb|SourceData| object (see above). 

\item \code{dataset_name} is the name of the dataset that this indicator will be 
computed for.

\item \code{years} are the years that the indicator will be computed for.
This field is optional if the \verb|SourceData| object also 
has a years field. The indicator years field overrides
the \verb|SourceData| years field.

\item \code{name} is the desired name of the indicator. This field is optional. 
The default name is the indicator attribute, although 
some indicators overload the default name. Name replaces 
the old 'as' syntax.

\end{itemize}

\subsection{Table}
A table is a simple output file that can be read into a spreadsheet application. 
A Table object accepts the following parameters:

\begin{itemize}
\tight
\item \code{attribute} is the the fully qualified opus path of the indicator. 
\item \code{output_type} specifies whether the results should be separated 
by tabs or commas, or output as a dbf file (``tab'', ``csv'', or ``dbf'', respectively).
\end{itemize}

An example:
\begin{verbatim}
Table(
    source_data = source_data,
    dataset_name = 'zone',
    attribute = 'urbansim.zone.industrial_sqft',
    output_type = 'tab'
) 
\end{verbatim}

\subsection{Map}

Matplotlib maps can be constructed through the Map object. 
Matplotlib automatically creates a color
ramp using a continual color range.  The continual range of colors can be
quite useful for distinguishing different values. 
A Map object accepts the following parameters:

\begin{itemize}
\tight
\item \code{attribute} is the the fully qualified opus path of the indicator. 
\item \code{scale} is a two element list that specifies the min and max values 
for the scale of the values on the map. It is optional and defaults to the
min/max values of the indicator. 
\end{itemize}
  
An example:
\begin{verbatim}
Map( 
    source_data = source_data,
    dataset_name = 'zone',
    name = 'my_population_at_zone_level',
    attribute = 'urbansim.zone.population',
    years = [2010], 
    scale = [-5000, 250000]
)
\end{verbatim}

\subsection{Chart}

Matplotlib charts can be constructed through the Chart object. 
Charts should only be used when there are a small number of different 
entities (e.g. at higher levels of geographic aggregation).
A Chart object accepts the following parameters:

\begin{itemize}
\tight
\item \code{attribute} is the the fully qualified opus path of the indicator. 
\end{itemize}

An example:
\begin{verbatim}
Chart(
    source_data = source_data,
    dataset_name = 'gridcell',
    name = 'my_population_at_gridcell',
    attribute = 'urbansim.gridcell.population',
)
\end{verbatim}

\subsection{Dataset Table}
Dataset tables can be constructed through the DatasetTable object. 
This indicator is useful for examining the values of different 
variables for the values of a particular geography. 
A DatasetTable object accepts the following parameters:

\begin{itemize}
\tight
\item \code{attributes} is the the fully qualified opus path of the indicator.
\item \code{exclude_condition} determines the condition under which certain rows 
are not included in the result. For example, if all the values in the row 
are zero, set this to "==0". This field is optional.
\end{itemize}

An example:
\begin{verbatim}
DatasetTable(
    source_data = source_data,
    dataset_name = 'zone',
    name = 'pop_and_ind_sqft',
    attributes = [ 
      'urbansim.zone.population',
      'urbansim.zone.industrial_sqft',                     
    ],
    exclude_condition = '==0' 
)
\end{verbatim}

\subsection{Operations}

A small set of indicator operations are also supported. 
Every indicator object except DatasetTable 
accepts an optional parameter \emph{operation}. This parameter can take on the
following values:

\begin{itemize}
\tight
  \item \code{change} returns the difference of the specified
  attribute for the current year with respect to the baseyear (currently fixed
  at 2000).
  \item \code{percent_change} returns the percent difference of the specified
  attribute for the current year with respect to the baseyear (currently fixed
  at 2000). 
  \item \code{size} returns the size of the data set used.
\end{itemize}

An example using operations:
\begin{verbatim}
Table(
    source_data = source_data,
    dataset_name = 'large_area',
    name = 'de_population_change',
    operation = 'change',
	years = [2015]
)
\end{verbatim}

\subsection{Creating the Indicators}
After the indicator list has been created, an
\verb|IndicatorFactory| can process the list of indicators you have specified. 
For example:

\begin{verbatim}
from opus_core.indicator_framework.indicator_factory import IndicatorFactory
IndicatorFactory().create_indicators(indicators = indicators)
\end{verbatim}

\subsection{Cross-scenario Indicators}
\label{sec:indicator-cross-scenario}

Cross-scenario comparisons are currently generated by specifying two 
cache directories that each contain the simulation results for the 
scenarios that you wish to compare. The comparison is 
accomplished with a subtract. To be more specific, for each specified 
indicator, the indicator is computed for both cache directories. Then
the indicator values from the comparison cache directory are subtracted
from the values of the indicator from the cache directory. 
In the near future, a percent different comparison will also be provided.

\subsection{Indicator Results}
\label{sec:indicator-results}

A static HTML page that describes the computed indicators for a given
cache directory is automatically updated everytime an indicator from 
that cache directory is computed. For every indicator request, the 
details of the request are displayed, as well as a link to the generated 
indicator. The HTML file is located in the indicators subdirectory of the 
cache directory and is named \verb|indicator_results.html|.

\section{When to Compute Indicator Values}
\indicatorsindex\attributesindex

Both the GUI and script-based indicator configuration requests invoke a
class \class{IndicatorFactory}.  The indicator factory fires up separate
processes for each indicator.  So the indicators \indicatorsindex for a
given year (or up to a given year) can be run as soon as the flt files for
that year have been written out from the simulation, even if the simulation
is still running.  Doing this in fact can be quite useful for long
simulation runs spanning several days --- run some indicators
\indicatorsindex on early years in the output to see if things look OK, and
if not, stop the simulation.  Of course it is also fine to run any or all
of the indicators \indicatorsindex after the simulation has completed.

\section{Defining New Indicators}
\indicatorsindex\attributesindex

To add a new indicator \indicatorsindex that uses Opus attributes,
\attributesindex define an appropriate variable. \variablesindex (As usual,
a good way to proceed is to find an existing definition that is similar to
what you want, and copy and modify it.)  If the indicator \indicatorsindex
is a specialized one, it should be defined in a package specific to that
application, rather than in the \package{urbansim} package.  (And even if
the indicator \indicatorsindex is of general utility, it should probably be
in a separate package unless you are part of the core development team ---
that way, if you upgrade to a version of the \package{urbansim} package,
your indicator \indicatorsindex definition won't be lost.)

For example, here is the definition for the ``population'' variable
\variablesindex in \module{psrc.large_area.population}.

\variablesindex\attributesindex\numpyindex
\begin{verbatim}
class population(Variable):
    """Number of people in each area"""
    _return_type="int32"

    def dependencies(self):
        return [attribute_label("faz", "large_area_id"), 
                attribute_label("faz", "population"), 
                my_attribute_label("large_area_id")]

    def compute(self, dataset_pool):
        faz = dataset_pool.get_dataset('faz')
        return self.get_dataset().sum_over_ids(faz.get_attribute("large_area_id"), 
                                   faz.get_attribute("population"))

    def post_check(self, values, dataset_pool):
        size = dataset_pool.get_dataset('faz').get_attribute("population").sum()
        self.do_check("x >= 0 and x <= " + str(size), values)
    

from opus_core.tests import opus_unittest
from urbansim.variable_test_toolbox import VariableTestToolbox
from numpy import array
from numpy import ma

class Tests(opus_unittest.OpusTestCase):
    variable_name = "psrc.large_area.population"
 
    def test_my_inputs(self):
        population = array([21,22,27,42]) 
        faz_large_area_ids = array([1,2,1,3]) 
        faz_id = array([1,2,3,4])
            
        values = VariableTestToolbox().compute_variable(self.variable_name, \
                {"large_area":{
                "large_area_id":array([1,2, 3])}, \
            "faz":{ \
                "population":population,\
                "large_area_id":faz_large_area_ids, \
                "faz_id":faz_id}}, \
            dataset = "large_area")

        should_be = array([48, 22, 42])
        
        self.assertEqual(ma.allclose(values, should_be, rtol=1e-2), \
                         True, msg = "Error in " + self.variable_name)


if __name__=='__main__':
    opus_unittest.main()
\end{verbatim}

Again, this is just an ordinary Opus variable. \variablesindex Here
is a brief explanation of the different parts of the definition (see
also Section~\ref{sec:opus-variable}).  We define a new class
\class{population}, which must be a subclass of \class{Variable}.
\variablesindex (The convention is that the names of Opus variables
\variablesindex are lower case, even though usually Python
\pythonindex class names are capitalized.)  Its values depend on the
values of certain other variables, \variablesindex which are listed
in the \method{dependencies} method.  The \method{compute} method is
where the action is: this defines how the values of this variable
\variablesindex are computed.  In this case, we know that each
\verb|large_area| contains a number of FAZ's (Forecast Analysis
Zones). \fazindex The \verb|sum_over_ids| method iterates through
the FAZ's, \fazindex adding up populations depending on which
\verb|large_area| the FAZ \fazindex is in, and returns an array of
the population in each \verb|large_area|.  If invoked, the
\method{post_check} method performs a sanity check on the results:
is the population in each \verb|large_area| between 0 and the total
population in all the grid cells?  (See Section
\ref{sec:programming-by-contract}.)  Finally, a unit test is defined
for this variable. \variablesindex This unit test, as with any other
Opus unit test, sets up just enough data to test whether the
variable's \variablesindex value is being computed correctly for a
small example (Section \ref{sec:unit-tests}). The
\verb|if__name__=='__main__':| part at the end means that the test
will be run if the file is run from the command line, but not if it
is merely imported.

Aside: Why not iterate over grid cells instead, you ask, since it's
ultimately the grid cells that keep track of households and hence
population?  One reason is that in this implementation grid cells
know about the \verb|faz_id| \fazindex that contains the grid cell,
but not the \verb|large_area_id|\@.  So it wouldn't work.  We could
add the \verb|large_area_id| to each grid cell to make this work,
but that would use extra space (there are a lot of grid cells).
Further, if the populations of the FAZ's \fazindex are needed for
any other purposes, these values will be computed on demand and
cached
--- and there are way fewer FAZ's \fazindex than grid cells, making
the \verb|large_area| computation more efficient.

%%% Local Variables:
%%% mode: latex
%%% TeX-master: "userguide"
%%% End:
