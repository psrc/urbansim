
\emph{This part under construction...}

These are the possible items that need to be covered in this part:
\begin{enumerate}
    \item anatomy of opus_home
    \item xml-based approach to project management (adapting an existing project vs. barebones)
    \item start with an empty/minimal (ur) project (urproject) and creating a minimal database (structure this section as a tutorial?); need opus package where skeletal xml project can live
    \item "Model system architectures" (zone, gridcell, parcels)
    \item "Building on existing project templates" decide which of zone, gridcell, parcel works best, then, when "new project..." selected inherit from correct project
    \item (advanced) inheritance mechanism
\end{enumerate}

Also proposed: 
This part is focused on setting up a parcel-based or gridcell-based
model. Chapter one might be called "Applying Opus/UrbanSim" and
describe the necessary pieces to have a minimally working parcel or
gridcell based system.  A second chapter, say, "Regionalizing your
UrbanSim", might describe working from an existing implementation
(seattle_parcel, eugene_gridcell) vs. creating a barebones model
application. A last chapter, "integration with external tools", would
discuss database management, in particular, setting up the database
server configuration xml. It would also discuss interfacing with GIS
and automatic view creation for indicators exported to postgres.
