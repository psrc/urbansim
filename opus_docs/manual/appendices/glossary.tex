% Copyright (c) 2005-2009 Center for Urban Simulation and Policy Analysis,
% University of Washington.  Permission is granted to copy, distribute and/or
% modify this document under the terms of the GNU Free Documentation License,
% Version 1.2 or any later version published by the Free Software Foundation;
% with no Invariant Sections, no Front-Cover Texts, and no Back-Cover Texts.
% A copy of the license is included in the section entitled "GNU Free
% Documentation License".

\chapter{Glossary}

This section defines important terms used in describing the Opus system,
from a modeler's viewpoint.  In addition, there are a few
technology-oriented terms that might otherwise be confused with these
domain-specific terms.

\begin{description}

\item[Agent] \index{agent!definition}
An object, such as a job or a household, that takes direct action 
on its environment, such as making choices among a set of alternatives.  
Typically, agents represent ``real-world'' objects from the model
domain.  Agents have state. 
Agents can engage in behavior, communicate, change state, etc. 
An agent's behavior usually depends upon its context.  

\item[Agent set] \index{agent set!definition}
An object that contains a set of agents.

\item[Alternative] \index{alternative!definition}
One of the objects that an agent may choose.

\item[Attribute]\index{attributes!definition}
A value that can be attributed to a particular object.  An
inherent characteristic of an object.  For instance, the UrbanSim
job object may have attributes such as ``id'', ``gridcell_id'', 
and ``is_in_scalable_sector_group''.  Attributes may be 
"primary''\index{attributes!primary}\index{primary attributes}, in that the jobs in the input database 
have values for that attribute.  Or attributes may be computed
by a variable definition.  There are no ``constant'' attributes. 
Models may modify any attributes, including ``primary'' attributes.

\item[baseyear cache]\index{baseyear cache!definition}
The file-based storage for the attribute values read from the baseyear
database.  Access to a file-based cache is much faster than to a database. 
Data cached during model simulation is written to a simulation cache. 

\item[Characteristic] \index{characteristics!definition}
Synonym for attribute.

\item[Choice] \index{choice!definition}
We avoid using this term because it is ambiguous: it can mean either
a member of a choice set, or it can mean the single chosen
alternative.

\item[Choice set] \index{choice set!definition}
An object that contains a set of alternatives.  By convention, we use
this term instead of ``alternative set''.

\item[Category] \index{category!definition}
A single state of a nominal variable, such as a development type.
Often used to group objects based upon their characteristics. \index{characteristics}

\item[Category set] \index{category set!definition}
A set of categories.

\item[Dataset]\index{dataset!definition}
A set of objects, such as gridcells or jobs, of the same type.
You can think of a dataset as a table, with one column
for each attribute, and one row for each object.  
Each object in the dataset has a unique identifier (an integer) stored in
an attribute whose name is referenced as ``id_name''.  

In practice, a dataset often is constructed from a single
table read from a Storage object, though some datasets 
are constructed by joining information from multiple tables.

\item[Logit Equation] \index{Logit Equation!definition}
Also known as a Logit Transformation, this transforms the
collection of an agent's computed direct utilities for all
alternatives in the choice set into an estimated probability for
each alternative:
\begin{equation}
    P_{ij}=\frac{e^{\lambda U_{ij}}}{\sum_{j' \in J}e^{\lambda
    U_{ij'}}},
\end{equation}

\item[Estimated Utility Function] \index{Estimated Utility Function!definition}
A set of paired variables and coefficients of the form $\sum
c_{i}*v_{i}$.

\item[Graphical User Interface]\index{Graphical User Interface (GUI)!definition}
\index{GUI|see {Graphical User Interface}}
A general term for an interactive, graphical user interface to a computer
application.  The Opus/UrbanSim GUI provides access to much of the system's
functionality without needing to use Python scripts.

\item[Integrated Development Environment (IDE)]\index{Integrated Development Environment (IDE)!definition}
\index{IDE|see {Integrated Development Environment}}
An application to assist software engineers in writing programs.  An IDE might 
include support for code browsing and searching, debugging, and so forth.
The Opus/UrbanSim developers usually use either the 
Eclipse\index{Integrated Development Environment (IDE)!Eclipse}\index{Eclipse IDE} or 
Wing\index{Integrated Development Environment (IDE)!Wing}\index{Wing IDE}\@.

\item[Interaction Set] \index{interaction sets!definition}
A dataset of variables describing the interaction between two
different datasets.  See ``Interaction Variable''.

\item[Interaction Variable] \index{interaction variable!definition}
A variable that computes the interaction between two different
datasets, such as that gridcell ``number_of_households'' 
variable that computes the number of households residing in
each gridcell.

\item[Logit Model] \index{Logit Model!definition} \index{model!Logit}
A Logit model is defined using an indirect utility equation, which
includes the error term and specifies how that error is
distributed. For example:
\begin{equation}
    U_{ij}={\theta}{X}_{ij}+\epsilon_{ij}
    =\sum_{k \in K}{\theta}_{k}X^{k}_{ij}+\epsilon_{ij},
\end{equation}
\indent where $\epsilon_{ij}$ is i.i.d. distributed Gumbel Type I.

\item[Model] \index{model!definition}
An object that defines behavior and that has a \verb|run()|
method.  Models have no state. They get their data from data
objects, and store their results in data objects.  A model can be
a formula that calculates on the data provided to it. A model can
be a sequence of actions to perform.  A model also can act as an
action.  A model must have a \verb|run()| method.

By convention, every model
module contains a set of tests that test that model.

\item[Model implementation] \index{model!implementation}
A model whose specification, or both specification and
coefficients\index{coefficients}, have been estimated to a particular data set.

\item[Model object] \index{model!object}
A model, or a model part.  Model objects act on data objects.

\item[Model component] \index{model!component}
An object that performs a logically distinct task that is part of
how the model runs.  Every model is formed by a sequence, and
perhaps cycle, of model steps.  Different models may 
combine the same model components in different ways to define
different models.

Note that models can become components in other higher-level models, such as a
land-price model and a neighborhood choice model that are parts of a
household-location-choice model, which is part of an UrbanSim model.

\item[Model specification] \index{model!specification}
A definition of what variables are included in an econometric
model.  Coefficient values\index{coefficient values} may, or may not, also be part of the
specification.

\item[Opus] \index{Opus!definition} The Open Platform for Urban Simulation, a new
Python-based framework for writing urban and regional models.

\item[Opus package] \index{Opus package!definition} The term ``Opus package'' refers to 
any Python package\index{Python package} built using the Opus framework and that follows
the Opus package guidelines.  The packages in the Opus base distribution
will all be Opus packages in this sense, as will contributed Opus packages. See
Section~\ref{sec:create-opus-package} for directions for creating your own Opus
package. 

\item[Puget Sound Regional Council (PSRC)]\index{PSRC}\index{Puget Sound Regional Council|see {PSRC}}
\url{www.psrc.org}
The Metropolitan Planning Organization for the Central Puget Sound Region in Washington
State, which includes Seattle, Bellevue, Tacoma, and other cities, towns, and unincorporated 
areas.  PSRC is one of the users of UrbanSim. 

\item[Python package]\index{Python!packages}
A way of structuring Python's module namespace by using ``dotted module
names''.  See the Python documentation at \url{http://www.python.org/doc}.

\item[Resources] \index{Resources!definition} 
A Python dictionary object.  It is used to contain parameters passed
between model steps.

\item[simulation cache]\index{simulation cache!definition}
The file-based storage for the attribute values for the datasets used by the
models. Access to a file-based cache is much faster than to a database.  The
baseyear attributes are stored in the baseyear cache\index{baseyear cache}.

\item[]\item[Simulation cycle] \index{simulation cycle!definition}
A set of sub-models that run in sequence before repeating.
Typically, this refers to the outermost loop.

\item[Submodel] \index{submodel!definition}
A \ldots \emph{need to supply this definition}

\item[Time step] \index{time step!definition}
The unit of simulated time between each simulation cycle. For many
simulations, this will be a year.

\item[Variable]\index{variables!definition}
An attribute that is computed by a variable definition.  Each
variable definition resides in its own Python module.  The 
Python module for a variable, e.g. ``is_industrial'', is the
same as the variable name except with a ``.py'' extension, 
e.g. ``is_industrial.py''.  

\item[Work request system] \index{work request system!definition}
A system for filing and tracking work requests, including bug
reports.

\end{description}
