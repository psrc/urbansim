% Copyright (c) 2005-2008 Center for Urban Simulation and Policy Analysis,
% University of Washington.  Permission is granted to copy, distribute and/or
% modify this document under the terms of the GNU Free Documentation License,
% Version 1.2 or any later version published by the Free Software Foundation;
% with no Invariant Sections, no Front-Cover Texts, and no Back-Cover Texts.
% A copy of the license is included in the section entitled "GNU Free
% Documentation License".

\chapter{Version Numbers}
\label{appendix:version-numbers}
\index{version numbers}

Opus and UrbanSim use version numbers for the source code, for the XML
project configuration files, and for the sample data.  You should generally
use consistent versions of these three things, since the source code will
expect that the XML files are laid out in a particular way and that
particular tables are in the sample data.  A given version number
corresponds to a specific version of the source code and the exact contents
of the sample data.  For the XML configuration files, the version number
identifies the expected layout of the XML, not the exact content of the
configuration --- if we were using DTDs or other XML schema descriptions,
it would be the version of the DTD\@.  (We hope to support such a schema in
the future.)

\section{Version Numbers for Stable Releases}

For stable releases, the version is something like `4.2.1', where 4 is the
major release number (i.e. UrbanSim 4), 2 is the minor number, and 1 is the
bugfix number.  The same version number is used for the source code, XML
configurations, and the sample data, with the goal of making it easy to
check that these are consistent.

The version number of the source code is associated with the
\package{opus_core} package, and can be found in the usual Python fashion
by evaluating the following expression:

\code{opus_core.__version__}

(In the future we may add the \code{__version__} attribute to all
Opus/UrbanSim modules.)

The version number for an XML configuration is given in an element with the
\code{xml_version} tag, for example:

\code{<xml_version>4.2.0</xml_version>}

Currently this isn't visible in the GUI itself --- you need to view the
contents of the XML file.

For the sample data, the version number is built into the zip file name.
For example \file{opus-4-1-2.zip} is the sample data file for version
4.1.2.  This file can be downloaded from the UrbanSim download page at
\url{http://www.urbansim.org/download/}.

\section{Version Numbers for Development Versions}

For development versions, the version for the source code is something like
`4.2-dev3216' where 3216 is the svn revision number for this version of the
module. When this development version is first released as a stable
release, it will become `4.2.0'. The 4.2 series then enters bugfix and
minor enhancement mode.  Further stable releases in the 4.2 series will be
bugfix versions (perhaps also with some minor enhancements), e.g. 4.2.1,
4.2.2, etc.  When the code in the main trunk is ready to move to a major
change in functionality, the version number for the main trunk will then
move to the 4.3 series, for example 4.3-dev3377.

Version numbers for the XML and sample data will track the source code
numbers, except that since these change much less frequently, we use
manually incremented numbers.  For example, 4.2.1-beta1 would be a beta
version of the XML and sample data for upcoming stable release 4.2.1.

The system has a minimum and maximum XML version number for which the code
is known to work, and checks that XML configuration version numbers are
within this range when they are loaded.  (If not, it raises an exception.)
You can find the minimum and maximum XML version numbers that your code
expects by evaluating
\begin{verbatim}
import opus_core.version_numbers
opus_core.version_numbers.minimum_xml_version
opus_core.version_numbers.maximum_xml_version
\end{verbatim}

{\bf Information for Developers.}  Please see the trac wiki for information
on updating version numbers:
\url{http://trondheim.cs.washington.edu/cgi-bin/trac.cgi/wiki/VersionNumbersAndReleases}.

