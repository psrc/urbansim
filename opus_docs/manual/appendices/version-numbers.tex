% Copyright (c) 2005-2009 Center for Urban Simulation and Policy Analysis,
% University of Washington.  Permission is granted to copy, distribute and/or
% modify this document under the terms of the GNU Free Documentation License,
% Version 1.2 or any later version published by the Free Software Foundation;
% with no Invariant Sections, no Front-Cover Texts, and no Back-Cover Texts.
% A copy of the license is included in the section entitled "GNU Free
% Documentation License".

\chapter{Version Numbers}
\label{appendix:version-numbers}
\index{version numbers}

Opus and UrbanSim use version numbers for the source code and for the XML
project configuration files.  You should use consistent versions of these,
since the source code will expect that the XML files are laid out in a
particular way.

A given source code version number corresponds to a specific version of the
source code.  For the XML configuration files, the version number
identifies the expected layout of the XML, not the exact content of the
configuration --- if we were using DTDs or other XML schema descriptions,
it would be the version of the DTD\@.  (We plan to support such a schema in
the near future.)

\section{Source Code Version Numbers for Stable Releases}

For stable releases, the version is of the form `4.2.1', where 4 is the
major release number (i.e.\ UrbanSim 4), 2 is the minor number, and 1 is
the bugfix number.  The version number of the source code is associated
with the \package{opus_core} package, and can be found in the usual Python
fashion by evaluating the following expression:

\code{opus_core.__version__}

(In the future we may add the \code{__version__} attribute to all
Opus/UrbanSim modules.)

\section{Source Code Version Numbers for Development Versions}

For development versions, the version for the source code is something like
`4.2-dev5216' where 5216 is the svn (subversion code repository) revision
number for this version of the code.  When this development version is
first released as a stable release, it becomes `4.2.0'. The 4.2 series
then enters bugfix and minor enhancement mode.  Further stable releases in
the 4.2 series will be bugfix versions (perhaps also with some minor
enhancements), e.g. 4.2.1, 4.2.2, etc.  When the code in the main trunk is
ready to move to a major change in functionality, the version number for
the main trunk will then move to the 4.3 series, for example 4.3-dev3377.

\section{XML Version Numbers}

XML version numbers are of the form `1.0', where 1 is the major number and
0 is the minor number.  These are independent of the source code version
numbers.\footnote{We had a brief period of trying to keep them the same,
  but soon decided this wasn't a good idea, since the code version numbers
  change more rapidly than the XML version numbers, and we didn't want to
  require that users change the XML version numbers on all their
  configurations if there weren't any other changes.}  The version number
for an XML configuration is given in an element with the \code{xml_version}
tag, for example:

\code{<xml_version>1.0</xml_version>}

Currently this isn't visible in the GUI itself --- you need to view the
contents of the XML file.

The system has a minimum and maximum XML version number for which the code
is known to work, and checks that XML configuration version numbers are
within this range when they are loaded.  (If not, it raises an exception.)
You can find the minimum and maximum XML version numbers that your code
expects by evaluating
\begin{verbatim}
import opus_core.version_numbers
opus_core.version_numbers.minimum_xml_version
opus_core.version_numbers.maximum_xml_version
\end{verbatim}

\section{Sample Data Versions}

For the sample data for a stable release, the version number is built into
the file name.  For example \mbox{\file{opus-4-2-0.zip}} is the sample data
file for version 4.2.0.  This file can be downloaded from the UrbanSim
download page at \url{http://www.urbansim.org/download/}.

The download page will note whether this is still valid for the current
development version; if there are changes to the sample data before a new
stable release, there will be a new zip file with a name like
\file{opus-4-2-dev-20jan2009.zip} (which indicates that this is a sample
data zip file for the 4.2 development trunk, posted on 20 January 2009).

The sample data file also includes starter XML configurations, so if the
XML version number changes a new zip file will be posted.

{\bf Information for Developers.}  Please see the trac wiki for information
on updating version numbers:
\url{http://trondheim.cs.washington.edu/cgi-bin/trac.cgi/wiki/VersionNumbersAndReleases}.

