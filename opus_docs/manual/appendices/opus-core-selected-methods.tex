% Copyright (c) 2005-2008 Center for Urban Simulation and Policy Analysis,
% University of Washington.  Permission is granted to copy, distribute and/or
% modify this document under the terms of the GNU Free Documentation License,
% Version 1.2 or any later version published by the Free Software Foundation;
% with no Invariant Sections, no Front-Cover Texts, and no Back-Cover Texts.
% A copy of the license is included in the section entitled "GNU Free
% Documentation License".

\chapter{Selected Methods and Functions in \package{opus_core}}
\label{app:selected-methods}
%
This chapter lists selected methods and functions implemented in the \package{opus_core} package.
For more information and information about arguments see our Trac site at www.urbansim.org (Browse Source).

%
\section{Dataset}
%
\label{app:selected-methods-dataset}
%
The class \class{Dataset} is implemented in
\verb|opus_core.datasets.dataset.py|. 
It is a child of \class{AbstractDataset} (\verb|opus_core.datasets.abstract_dataset.py|)
and most of its methods is inherited from the parent class. Here we group the methods 
according to their use cases, thinking about a dataset as a table with rows and columns.

%
\subsection{Selected Methods}
%
\paragraph{Adding columns}~\\[1mm]
{\tt add\_attribute (data, name, ...)}\\
{\tt add\_primary\_attribute (data, name)}

\paragraph{Adding rows}~\\[1mm]
{\tt add\_elements (data, ...)}

\paragraph{Obtaining columns}~\\[1mm]
{\tt get\_attribute (name)} \\
{\tt get\_attribute\_by\_id (name, id)} \\
{\tt get\_attribute\_by\_index (name, index)}\\
{\tt get\_id\_attribute()} \\
{\tt get\_attribute\_as\_column (name)}

\paragraph{Obtaining rows}~\\[1mm]
{\tt get\_data\_element (index, ...)} \\
{\tt get\_data\_element\_by\_id (id, ...)}

\paragraph{Obtaining column names}~\\[1mm]
{\tt get_id_name ()} \\
{\tt get_attribute_names ()} \\
{\tt get_known_attribute_names ()} \\
{\tt get_primary_attribute_names ()} \\
{\tt get_computed_attribute_names ()} \\
{\tt get_nonloaded_attribute_names ()} \\
{\tt get_attributes_in_memory ()} \\
{\tt get_cached_attribute_names ()} \\
{\tt get_attribute_long_names ()}

\paragraph{Modifying columns}~\\[1mm]
{\tt modify\_attribute (name, data, index=None)} \\
{\tt set\_value\_of\_attribute\_by\_id (attribute, value, id)}

\paragraph{Deleting columns}~\\[1mm]
{\tt delete\_one\_attribute (name)} \\
{\tt delete\_computed\_attributes ()}

 \paragraph{Deleting rows}~\\[1mm]
 {\tt remove\_elements (index)}\\
 {\tt subset (n, is_random=False)}\\
 {\tt subset\_by\_index (index, ...)}\\
 {\tt subset\_by\_ids (ids, ...)}\\
 {\tt subset\_where\_variable\_larger\_than_threshold (attribute, threshold=0, ...)}
 
\paragraph{Computing variables}~\\[1mm]
 {\tt compute_variables (names, dataset_pool=None, ...)}\\
 {\tt compute_one_variable_with_unknown_package (variable_name, }\\
 \hspace*{3cm} {\tt dataset_pool=None, package_order=None)}
 
\paragraph{I/O methods}~\\[1mm]
{\tt load\_dataset (..., attributes=None, in_storage=None, in_table_name=None, ...)} \\
{\tt load\_dataset_if_not_loaded (..., attributes=None, in_storage=None,} \\
\hspace*{3cm} {\tt in_table_name=None, ...)}\\
{\tt write\_dataset (..., attributes=None, out_storage=None, out_table_name=None, }\\
 \hspace*{3cm} {\tt ...)} \\
{\tt flush\_attribute (name)} \\
{\tt flush\_dataset ()}  \\
{\tt load_and_flush_dataset ()}  \\
{\tt get_cache_directory ()}  \\
{\tt remove_cache_directory ()}

\paragraph{Obtaining row indicies}~\\[1mm]
{\tt get_id_index (id)}  \\
{\tt try_get_id_index (id, return_value_if_not_found=-1)}  \\
{\tt get_index_where_variable_larger_than_threshold (attribute, threshold=0)}  \\
{\tt get_filtered_index (filter, threshold=0, index=None, dataset_pool=None, }\\
\hspace*{3cm} {\tt ...)}

\paragraph{Connecting two datasets}~\\[1mm]
{\tt connect_datasets (dataset)}  \\
{\tt join (dataset, name, join_attribute=None, ...)}  \\
{\tt join_by_rows (dataset, ...)}  \\
{\tt aggregate_dataset_over_ids (dataset, function='sum', attribute_name=None,}\\
\hspace*{3cm} {\tt constant=None)}

\paragraph{Data analysis}~\\[1mm]
{\tt summary (names=[], ...)}  \\
{\tt size ()}  \\
{\tt attribute_sum (name)}  \\
{\tt attribute_average (name)}  \\
{\tt aggregate_all (function='sum', attribute_name=None)}  \\
{\tt correlation_matrix (names)}  \\
{\tt correlation_coefficient (name1, name2)}  \\
{\tt categorize (attribute_name, bins)}  \\
{\tt get_data_type (attribute, ...)}

\paragraph{Plotting methods}~\\[1mm]
{\tt plot_histogram (name, main="", filled_value=0.0, bins=None)}  \\
{\tt r_histogram (name, main="", prob=1, breaks=None)}  \\
{\tt plot_scatter (name_x, name_y, main="", npoints=None, ...)}  \\
{\tt r_scatter (name_x, name_y, main="", npoints=None)}  \\
{\tt r_image (name, main="", xlab="x", ylab="y", min_value=None, max_value=None, }\\
\hspace*{3cm} {\tt file=None, pdf=True)}  \\
{\tt plot_map (name, ...)} \\
{\tt openev_plot (name, ...)}

\paragraph{Memory management}~\\[1mm]
{\tt itemsize_in_memory ()}  \\
{\tt unload_attributes (names)}  \\
{\tt unload_all_attributes ()}  \\
{\tt unload_primary_attributes ()}  \\
{\tt unload_computed_attributes ()}  \\
{\tt unload_one_attribute (name)}

\paragraph{Other methods}~\\[1mm]
{\tt get_dataset_name ()}  \\
{\tt get_attribute_header (name)}  \\
{\tt filled_masked_attribute (name, filled_value=0)}  \\
{\tt get_version (name)}  \\
{\tt has_attribute (attribute_name)}  \\
{\tt get_coordinate_system ()}\\
{\tt create_subset_window_by_ids (ids)}

\subsection{Useful Functions}
%
Examples of creating and using datasets are given in Section~\ref{sec:datasets}. Two functions
that support creating datasets from an flt storage and a tab delimited storage, respectively, are implemented in 
the module \module{opus_core.misc}:

{\tt get_dataset_from_flt_storage (dataset_name, directory, }\\
\hspace*{3cm} {\tt package_order=['opus_core'], dataset_args=None)}\\
{\tt get_dataset_from_tab_storage (dataset_name, directory, }\\
\hspace*{3cm} {\tt package_order=['opus_core'], dataset_args=None)}

If the dataset is defined in a package (for example, dataset 'gridcell' is defined
in the \package{urbansim} package), put the package name into the {\tt package_order} list. 
Otherwise, put all arguments that the {\tt Dataset}'s constructor needs into 
{\tt dataset_args} in form of a dictionary. Both functions return a dataset object.
