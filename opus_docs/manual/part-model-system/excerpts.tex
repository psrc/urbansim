
\begin{table}
\caption{Attributes of Core Datasets Used in Puget Sound Parcel-based UrbanSim Application}
%\addlinespace
\label{tab:parcel-attributes}
\begin{tabular}{p{3cm}p{2.8cm}p{2.7cm}p{2.5cm}p{2.5cm}}
\toprule
Parcels & Buildings & Households    & Persons & Jobs \\
\midrule
parcel\_id  & non\_residential\_sqft    & household\_id & person\_id & job\_id \\
tax\_exempt\_flag & year\_built & income    & household\_id & sector\_id\\
parcel\_sqft\_in\_gis & parcel\_id  & persons    & member\_id & join\_flag\\
x\_coord\_sp &  land\_area  & workers & relate & sqft\\
y\_coord\_sp    & building\_quality\_id & children &    age & taz\_est\\
x\_coord\_utm   & improvement\_value    & building\_size    & sex & building\_type\\
y\_coord\_utm   & stories   & tenure    & edu & building\_id\\
grid\_id    & tax\_exempt   & race\_id & age\_of\_head  \\
zone\_id    & building\_type\_id    & employment\_status & race\_id &\\
census\_block   &  building\_id & building\_id & work\_at\_home &\\
city\_id    & template\_id  & & earning & \\
county\_id   &  sqft\_per\_unit & & job\_id & \\
id\_parcel  & & & &\\
id\_plat & & & &\\
is\_inside\_urban\_growth\_boundary & & & & \\
residential\_units  & & & & \\
plan\_type\_id & & & &\\
plan\_type\_description & & & &\\
num\_building\_records & & & &\\
GenericLandUse1 & & & &\\
land\_use\_type\_id & & & &\\
land\_value & & & &\\
parcel\_sqft & & & &\\
faz\_id & & & &\\
large\_area\_id & & & &\\
zipcode & & & &\\
zip\_id & & & &\\
\bottomrule
\end{tabular}
\end{table}



This data structure includes a representation of each individual person, household, job, building and parcel
in the entire metropolitan area, and their associations, meaning, for example, which building each household occupies, and which parcel each building
is on.  This is a `microsimulation' data structure, and makes it straightforward to model choices or changes in the status of any individual agent or
object.  It also makes it possible to summarize output from the model with a great deal of flexibility.  A spatial hierarchy is also used in the data
model to allow aggregation and disaggregation of information across multiple spatial levels.  Parcels are associated with zones (used in the travel
model), gridcells, Forecast Analysis Zones (FAZ), Large Areas (districts used for analysis of results), census blocks, cities, counties, and zip codes.  This
means that it is straightforward to aggregate or query information such as how many households of each income are in each zone, for example, or what
the employment density is within a FAZ.  We can also use these spatial relationships to assign information about travel conditions, predicted by the travel
model on a zone to zone basis, with households, jobs, and locations by using the zone level of geography.  These relationships are shown in Figure
\ref{fig:geographic-relationships}.

The land use and building type coding systems used by each of the counties were not consistent
with each other, so a more general classification was created that would allow the use of a uniform typology across the region.  These land use
types are shown in Table \ref{tab:landuse}.  Buildings are classified also, using a generic building type that approximates the generic lans use type (they cannot be identical, since there are more land uses to describe vacant land and other uses), along with a somewhat more detailed building type.  A profile
of buildings in the Puget Sound database, and their classification, is shown in Table \ref{tab:buildings}.



\begin{table}
\begin{center}
\caption{Generic Land Use Codes}
%\addlinespace
\label{tab:landuse}
\begin{tabular}{{l}p{3.5cm}}
\toprule
1   & single\_family\_residential\\
2   & multi\_family\_residential\\
3   & office\\
4   & commercial\\
5   & industrial\\
6   & mixed\_use\\
7   & government\\
8   & other\\
9   & no code\\
\bottomrule
\end{tabular}
\end{center}
\end{table}


\begin{table}
\begin{center}
\caption{Building Types and Characteristics in Central Puget Sound}
%\addlinespace
\label{tab:buildings}
\begin{tabular}{p{1cm}p{3cm}p{2cm}{r}{r}{r}}
\toprule
Building Type Id    & Description   & Generic Building Type & Frequency & Residential Units &   Non-res Sqft (000) \\
\midrule
1   & Agriculture   & other & 1,702 & 4 & 6,923\\
2   & Civic and Quasi-Public &  government  & 2,574 & 212   & 26,821\\
3   & Commercial    & commercial    & 19,272    & 2,181 & 210,751\\
4   & Condo Residential & multi-family residential &    9,829   & 132,678   & 0\\
5   & Government    & government    & 847   & 72    & 71,410\\
6   & Group Quarters    & other &   391 & 6,893 & 3,856\\
7   & Hospital / Convalescent Center    & government    & 715   & 118   & 21,407\\
8   & Industrial    & industrial    & 3,783 & 85    & 91,254\\
9   & Military &    government  & 10    & 4 & 32\\
10  & Mixed-Use & mixed-use & 529   & 1,562 & 2,580\\
11  & Mobile Home   & single family residential & 26,691    & 26,691    & 0\\
12  & Multi-Family Residential  & multi-family residential  & 45,215    & 317,612   & 52,784\\
13  & Office    & office    & 10,713    & 1,434 & 193,696\\
14  & Outbuilding   & other  & 37,789   & 519   & 29,312\\
15  & Park and Open Space   & other &   7 & 0   & 564\\
16  & Parking   & other & 1,043 & 150   & 25,606\\
17  & Recreation &  other   & 1,407 & 2,138 & 12,560\\
18  & School    & other &   2,678   & 192   & 57,487\\
19  & Single Family Residential & single family residential & 818,703   & 893,328   & 992\\
20  & Transportation Communication Utilities    & industrial    & 1,319 & 199   & 10,389\\
21  & Warehousing   & industrial    & 10,138    & 756   & 228,374\\
22  & No Code   & other &   13,788  & 3,864 & 8,681\\
\midrule
Total & & & 1,009,143   & 1,390,692 &   1,055,489\\
\bottomrule
\end{tabular}
\end{center}
\end{table}


The parcel and building data used in UrbanSim allow for representation of mixed use development, with multiple buildings per parcel.  Representation
of a single mixed use building can be accommodated by using two building components, each of a single use, and both associated with the same
parcel.