\chapter{Data for Parcel-based Applications}

\section{Database Tables About Parcels}

\subsection{The {\tt parcels} table}
\label{sec:db-tables-parcels}

This table contains attributes about parcels.  In general, there will be an identifier in this table for every other level of geography that you may want to aggregate up to.  In this example, there are attributes for zones, cities, counties, census blocks, etc.  Having these identifiers on the parcel makes it easier to aggregate indicators up to higher level geographies.  Any other attributes that one may want to restrict development by, or update throughout a simulation could be stored here as well.

\begin{tabular}{p{2in}lp{3.5in}}
%\hline
\textbf{Column Name} & \textbf{Data Type} & \textbf{Description} \\
\hline
%\begin{description}
parcel\_id & integer & unique identifier \\ \hline
zone\_id & integer &  id number for the zone that the parcel's centroid falls within \\ \hline
land\_use\_type\_id & integer & identifies the land use of the parcel \\ \hline
city\_id & integer & id number for the city that the parcel's centroid falls within \\ \hline
county\_id & integer & id number for the county that the parcel's centroid falls within\\ \hline
plan\_type\_id & integer & id number that identifies the parcel's plan type\\ \hline
parcel\_sqft & integer & square feet of the parcel as an integer \\ \hline
assessor\_parcel\_id & integer & \emph{(optional)} original tax assessor's id number\\ \hline
tax\_exempt\_flag & integer & \emph{(optional)} identifies parcel as tax exempt or not\\ \hline
land\_value & long & value of the land from the assessor \\ \hline
is\_in\_flood\_plain & integer & \emph{(optional)} indicates whether or not a parcel is in a flood plain\\ \hline
is\_on\_steep\_slope & integer & \emph{(optional)} indicates whether or not a parcel is on a steep slope\\ \hline
is\_in\_fault\_zone & integer & \emph{(optional)} indicates whether or not a parcel is in a fault zone\\ \hline
centroid\_x & long & state plane x coordinate of parcel centroid \\ \hline
centroid\_y & long & state plane y coordinate of parcel centroid\\ \hline
census\_block\_id & integer &  \emph{(optional)} id number for the census block that the parcel's centroid falls within\\ \hline
raz\_id & integer & \emph{(optional)} id number for the raz that the parcel's centroid falls within\\ \hline
\end{tabular}

\section{Database Tables about Buildings}

\subsection{The {\tt buildings} table}

In all recently developed UrbanSim applications, buildings of all types have been represented in their own
table, and linked to the unit of geography used for location choice: gridcell, parcel, or zone.  This configuration
provides a simple and flexible means or organizing the data for UrbanSim.  The buildings table is similar
for each of the types of applications, whether gridcell, parcel or zone -- the only significant difference is the
location identifier.  In the table below, parcel id is included, but for gridcell or zone applications, the user
should substitute gridcell id or zone id.

\begin{tabular}{p{2in}lp{3.5in}}
%\hline
\textbf{Column Name} & \textbf{Data Type} & \textbf{Description} \\
\hline
building\_id & integer & unique identifier \\ \hline
building\_quality\_id & integer & \emph{(optional)} identified for building quality\\ \hline
building\_type\_id & integer & identifier for building type; valid id in the \verb|building_types| table\\ \hline
improvement\_value & long & value of building (replacement cost)\\ \hline
land\_area & long & land area (usually in sqft) associated with building, includes footprint plus associated area such as landscaping and parking.\\ \hline
non\_residential\_sqft & long & non-residential square footage of building\\ \hline 
parcel\_id & integer & identifier of parcel in which building is located\\ \hline
residential\_units & integer & number of residential units in the building\\ \hline
sqft\_per\_unit & integer & number of residential square feet per unit in the building\\ \hline
stories & integer & \emph{(optional)} number of stories in the building\\ \hline
tax\_exempt & integer & \emph{(optional)} indicator for whether building is tax-exempt\\ \hline
year\_built & integer & year of construction of the building\\ \hline
\end{tabular}


\subsection{The {\tt building_types} table}
%
This is a table about available types of buildings.

\begin{tabular}{p{2in}lp{3.5in}}
\textbf{Column Name} & \textbf{Data Type} & \textbf{Description} \\
\hline
building\_type\_id & integer & unique identifier \\ \hline
building\_type\_name & varchar & name of the building type \\\hline
description & varchar & \emph{(optional)} description of the building type \\\hline
generic_building_type_id & integer & \emph{(optional)} identifier for generic building type \\\hline
generic_building_type_description & varchar & \emph{(optional)}\\\hline
is_residential & boolean & 1 if this building type is residential, 0 otherwise \\\hline
unit_name & varchar & name of units for this building type, e.g. 'commercial_sqft' or 'residential_units' \\\hline
\end{tabular}


\section{Database Tables About Development Projects}

\subsection{The {\tt development\_project\_proposals} table}
\label{sec:db-tables-development-project-proposals}

A record in this table, when combined with one or more records in the development\_project\_components table, represents a "known" development project. 
This table would be populated with projects known to be coming in the future. 
This table would also be populated during a simulation run for projects that are not yet complete, in other words, 
projects that are in the middle of developing according to their velocity function. 
It is entirely possible for a simulation run to happen without pre-populating this table with records.

\begin{tabular}{p{2in}lp{3.5in}}
%\hline
\textbf{Column Name} & \textbf{Data Type} & \textbf{Description} \\
\hline
development\_project\_id & integer & unique identifier \\ \hline
development\_template\_id & integer & indicates the development template that represents the project\\ \hline
far & float & floor to area ratio of the project\\ \hline
percent\_open & integer & the percent of the land area of the project accounted for by "overhead" uses such as rights of way or open space\\ \hline
status\_id & integer & this represents active, proposed, or planned developments with the following codes: 
1: in active development, 2: proposed for development, 3: planned and will be developed, 4: tentative, 
5: not available (already developed, 6: refused\\ \hline
parcel\_id & integer & indicates the parcel\_id on which the development occurs\\ \hline
start\_year & integer &  the year in which this project is expected to begin building\\ \hline
built\_sqft\_to\_date & integer & the number of non-residential sqft built in the current simulation year\\ \hline
built\_units\_to\_date & integer & the number of residential units built in the current simulation year \\ \hline
\end{tabular}


\subsection{The {\tt development\_project\_proposal\_components} table}
\label{sec:db-tables-development-project-proposal-components}

A record in this table represents a portion of a development project identified in the development\_project\_proposals table. 
In some sense a single record here is meant to represent a single building, or part of a building. 
Therefore individual records here do not necessarily represent single free-standing buildings, 
although they are mostly treated that way. This table allows for the flexible representation of mixed uses to occur on a parcel. 
Examples include multiple free-standing buildings with different uses, a single building with multiple uses inside of it 
(a single record for each use), or further complex representations of mixed use. This table is not required 
by UrbanSim, but it is created by the developer model and cached every simulation year. 

\begin{tabular}{p{2.1in}lp{3.3in}}
%\hline
\textbf{Column Name} & \textbf{Data Type} & \textbf{Description} \\
\hline
development\_project\_component\_id & integer & unique identifier\\ \hline
development\_project\_id & integer & identifies which development the project belongs to\\ \hline
velocity\_function\_id & integer & identifies the rate or function by which the project develops over time\\ \hline
percent\_of\_building\_sqft & integer & identifies the percentage of the building that this component takes up.
100\% would indicate a free-standing building with a single use, and several records with percent\_of\_building\_sqft 
adding up to 100\% would indicate a multiple use single building. \\ \hline
construction\_cost\_per\_unit & integer & the per unit construction cost for residential uses only\\ \hline
sqft\_per\_unit & integer & the square footage per residential unit\\ \hline
building\_type\_id & integer & indicates the building type of this particular component\\ \hline
land\_area & integer & the land area "claimed" by the building component, including not only the building footprint 
but also additional land used such as yards, parking lots, etc. \\ \hline
residential\_units & integer & the number of residential units in the building component \\ \hline
\end{tabular}

\subsection{The {\tt development\_templates} table}
\label{sec:db-tables-development-templates}

This table, along with corresponding records in the development\_template\_components table, represents development templates that can be used to define virtually any size and configuration of a development project, from a single house on an infill lot to a large subdivision, to a mixed use project with retail on the first floor and condominiums above. The contents of this table are roughly comparable to the development\_projects table, since development templates become proposals once they are determined to fit within a parcel and are allowed by development constraints, and then become projects if they are chosen to be constructed.

\begin{tabular}{p{2in}lp{3.5in}}
%\hline
\textbf{Column Name} & \textbf{Data Type} & \textbf{Description} \\
\hline
development\_template\_id & integer & unique identifier\\ \hline
percent\_open & integer & the percent of the land area of the project accounted for by "overhead" uses such as rights of way or open space\\ \hline
min\_land & integer & minimum amount of land in square feet to be utilized for this development\\ \hline
max\_land & integer & maximum amount of land in square feet to be utilized for this development\\ \hline
density\_type & integer & a readable name that describes the `density' field: units per acre, FAR\\ \hline
density & integer & indicates the density of the development\\ \hline
land\_use\_type\_id & integer & specifies the land use type for the development template\\ \hline
development\_type & integer & a readable name that describes the type of development this record represents (e.g. SFR-parcel, MFR-apartment, MFR-condo, etc.), this field is not used by the model and is there to make the table more readable \\ \hline
\end{tabular}

\subsection{The {\tt development\_template\_components} table}
\label{sec:db-tables-development-template-components}

This table is roughly equivalent to the development\_project_proposal_components table and represents buildings or parts 
of buildings to be included in a particular development template.  By breaking development templates into components, 
development project templates can be configured as hierarchies or combinations of building blocks, providing a very flexible 
means of representing a wide variety of development types.  Note that the templates can be generated using real or hypothetical data, 
since they will be compared to regulatory constraints and the size constraints of parcels.


\begin{tabular}{p{2.3in}lp{3.2in}}
%\hline
\textbf{Column Name} & \textbf{Data Type} & \textbf{Description} \\
\hline
development\_template\_component\_id & integer & unique identifier\\ \hline
development\_template\_id & integer & indicates which development template this component belongs to\\ \hline
velocity\_funtion\_id & integer & indicates the velocity function used by this template\\ \hline
building\_type\_id & integer & indicates the building type of this particular component\\ \hline
percent\_of\_building\_sqft & integer & identifies the percentage of the building that this component takes up\\
construction\_cost\_per\_unit & integer & the per unit construction cost\\ \hline
building\_sqft\_per\_unit & integer & the square footage per residential unit \\ \hline
\end{tabular}

\subsection{The {\tt velocity\_functions} table}
\label{sec:db-tables-velocity-functions}

This table is designed to hold the velocity functions that specify the rate at which development is built out.

\begin{tabular}{p{2.3in}lp{3.2in}}
%\hline
\textbf{Column Name} & \textbf{Data Type} & \textbf{Description} \\
\hline
velocity\_function\_id & integer & unique identifier\\ \hline
annual\_construction\_schedule & string &  this field will contain a numbered list in brackets of this form: `[25, 50, 75, 100]' 
indicating with each entry the percentage complete that the project would be in each year from its initiation.  A particular development\_template\_component or development\_project\_component will have one velocity\_function\_id attached to it.  This could take the form [0, 0, 0, 33, 66, 100] for example, would have no construction in its first 3 years, then in year 4 it would be 33\% complete, and 66\% and 100\% complete in years 5 and 6 respectively.\\ \hline 
\end{tabular}


\subsection{The {\tt demolition\_cost\_per\_sqft} table}
\label{sec:db-tables-demolition-cost-per-sqft}

This table provides information to the developer model about the costs of demolition by building type. These numbers are used to calculate the cost of demolition of existing development so that a more accurate cost of redevelopment can be calculated.

\begin{tabular}{p{2.3in}lp{3.2in}}
%\hline
\textbf{Column Name} & \textbf{Data Type} & \textbf{Description} \\
\hline
building\_type\_id & integer & building type\\ \hline
demolition\_cost\_per\_sqft & integer & cost in dollars per sqft of demolition\\ \hline 
\end{tabular}


\subsection{The {\tt building\_sqft\_per\_job} table}
\label{sec:db-tables-building-sqft-per-job}

This table contains information on the amount of space each job will take in a particular building type, by zone.

\begin{tabular}{p{2.3in}lp{3.2in}}
%\hline
\textbf{Column Name} & \textbf{Data Type} & \textbf{Description} \\
\hline
zone\_id & integer & the zone the record applies to\\ \hline
building\_type\_id & integer & the building type the record applies to\\ \hline
building\_sqft\_per\_job & integer & the sqft per job each job will take in a particular building type in a particular zone \\ \hline
\end{tabular}

\section{Database Tables About Development Constraints}

\subsection{The {\tt development_constraints} table}

Note that this table is substantially different than its counterpart for a gridcell-based
model application.  This is because the real estate development model is fundamentally
different as well.  The model that uses this table in a parcel-based application is the 
development\_project\_proposal\_sampling\_model, which evaluates alternative 
development templates that can be placed on a parcel according to the constraints
specified in this table, and then computes a return on investment on the remaining
viable proposals, before choosing which proposals to build.
This table defines rules that restrict the possible development types a
developer can create on a particular parcel.  Each row defines one rule.
Development is not allowed on any parcel that matches any of these rules. 

\begin{tabular}{llp{3.5in}}

%\hline
\textbf{Column Name} & \textbf{Data Type} & \textbf{Description} \\

\hline constraint_id & integer & Unique rule identification number  \\

\hline constraint\_type & string (14) &  units\_per\_acre or far (floor-area-ratio) \\

\hline generic\_land\_use\_type\_id & integer & Id of a record in the generic\_land\_use\_types table \\

\hline maximum & integer & Maximum value for the allowed development, in terms of the constraint type \\

\hline minimum & integer & Minimum value for the allowed development, in terms of the constraint type \\

\hline plan_type_id & integer & Id of a record in the plan_types table. \\

\hline
\end{tabular}

\begin{itemize}
\tight
\item constraint_id must be a unique positive integer.

\end{itemize}

\section{Database Tables About Target Vacancy Rates}

\subsection{The {\tt target_vacancies} table}
\label{sec:table-target-vacancies-parcel}

The \verb|target_vacancies| table is used by the development proposal choice
model. It gives the model information about acceptable vacancy rates. The table
has one row for each year the simulation runs. Each row gives target values for
the residential and nonresidential vacancies for that year, which are defined
below.  Only data after the base year is used.

\begin{tabular}{lll}

%\hline
\textbf{Column Name} & \textbf{Data Type} & \textbf{Description} \\

\hline year & integer & Year of the simulation for which the vacancy
targets
apply  \\

\hline target_vacancy & float & Ratio of unused 
space to total space, based on residential\_unit or sqft   \\

\hline building_type_id & Integer & Id of a record in the building_types table  \\

\hline
\end{tabular}

\begin{itemize}
\tight
\item There must be exactly one row for each year to be simulated.
\item target_vacancy must be between 0 and 1, inclusive.
\end{itemize}


\section{Database Tables About Refinement of Simulation Results}
\subsection{The {\tt refinements} table}
\label{sec:db-tables-refinements}

This table is used by the refinement model to adjust simulation results to incorporate added information or constraints specified by the user. The entries in this table define refinements to make to an existing simulation run. No fields can be null, if the attribute is not needed put a single quote (') in the field.

\begin{tabular}{p{2.3in}lp{3.2in}}
%\hline
\textbf{Column Name} & \textbf{Data Type} & \textbf{Description} \\
\hline
refinement\_id & integer & unique identifier\\ \hline
agent\_expression & string & string expression defining what agents to add or subtract, e.g. households, jobs\\ \hline
location\_capacity\_attribute & string & defines a capacity attribute such as non\_residential\_sqft\\ \hline
location\_expression & string & expression defining where to add or subtract agents, e.g. `zone = 123'\\ \hline
amount & integer & number of agents to add or subtract\\ \hline
year & integer & year to which this refinement applies\\ \hline
action & string & add, subtract, or target are the valid entries\\ \hline
transaction\_id & integer & if two or more records have matching transaction ids the refinement model will attempt to balance between the refinements \\ \hline
\end{tabular}
