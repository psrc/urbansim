\chapter{Data for Gridcell-based Applications}


\section{Database Tables about Grid Cells}
\label{sec:gridcell-tables}

\subsection{The {\tt gridcells} table}

Geographic information partitioned into a rectangular grid of rectangular cells.

The ``improvement_value'' fields, below, indicate the value (e.g., dollars) of
all buildings of a particular type that are in this grid cell. For instance,
commercial_improvement_value is the total value of all commercial buildings in
this grid cell. The use of ``improvement'' indicates that buildings are
considered ``improvements'' over the grid cell's land value.

Attributes that are marked as optional are only required by specific variables. Thus,
if they are needed or not depends on model specifications.
Attributes that are NOT marked as optional are used by various models. In some situations,
(e.g. by skiping specific models) some of those attributes may not be required. 

\begin{tabular}{p{2in}lp{3.5in}}
%\hline
\textbf{Column Name} & \textbf{Data Type} & \textbf{Description} \\
\hline
grid_id & integer & Unique identifier  \\
\hline
commercial_sqft & integer & The sum of the square footage of buildings that are classified as commercial (generally including retail and office land uses). This is not a measure of land area.  \\
\hline
development_type_id & integer & Index into the Development Types table  \\
\hline
distance_to_arterial & float &  \emph{(optional) } Units: \verb|urbansim_constants|.units \\
\hline
distance_to_highway & float &  \emph{(optional) } Units: \verb|urbansim_constants|.units \\
\hline
governmental_sqft & integer & The sum of the square footage of buildings that are classified as governmental\\
\hline
industrial_sqft & integer & The sum of the square footage of buildings that are classified as industrial\\
\hline
commercial_improvement_value & integer & See description, above  \\
\hline
industrial_improvement_value & integer & See description, above  \\
\hline
governmental_improvement_value & integer & See description, above  \\
\hline
nonresidential_land_value & integer & Units, e.g. dollars  \\
\hline
residential_improvement_value & integer & See description, above  \\
\hline
residential_land_value & integer & Units, e.g. dollars  \\
\hline
residential_units & integer & Number of residential units  \\
\hline
relative_x & integer & X coordinate in grid coordinate system  \\
\hline
relative_y & integer & Y coordinate in grid coordinate system  \\
\hline
year_built & integer & e.g. 2002  \\
\hline
plan_type_id & integer & An id indicating the plan type of the grid cell  \\
\hline
percent_agricultural_protected_land & integer & \emph{(optional) }
\\
\hline
percent_water & integer & \emph{(optional) } Percentage of this cell covered by water  \\
\hline
percent_stream_buffer & integer & \emph{(optional) } Percentage of this cell covered by stream buffer  \\
\hline
percent_floodplain & integer & \emph{(optional) } Percentage of this cell covered by flood plain  \\
\hline
percent_wetland & integer & \emph{(optional) } Percentage of this cell covered by wetland  \\
\hline
percent_slope & integer & \emph{(optional) } Percentage of this cell covered by slope  \\
\hline
percent_open_space & integer & \emph{(optional) } Percentage of this cell covered by open space  \\
\hline
percent_public_space & integer & \emph{(optional) } Percentage of this cell covered by public space  \\
\hline
percent_roads & integer & \emph{(optional) } Percentage of this cell covered by roads  \\
\hline
percent_undevelopable & integer & \emph{(optional) }
\\
\hline
is_outside_urban_growth_boundary & boolean & \emph{(optional) } \\
\hline
is_state_land & boolean & \emph{(optional) }
\\
\hline
is_federal_land & boolean & \emph{(optional) }
\\
\hline
is_inside_military_base & boolean & \emph{(optional) }
\\
\hline
is_inside_national_forest & boolean & \emph{(optional) }
\\
\hline
is_inside_tribal_land & boolean & \emph{(optional) }
\\
\hline
zone_id & integer & Traffic analysis zone that contains this grid cell's centroid  \\
\hline
city_id & integer & \emph{(optional) } City this grid cell belongs to  \\
\hline
county_id & integer & \emph{(optional) } County this grid cell belongs to  \\
\hline
fraction_residential_land & float & Fraction of residential land in this cell  \\
\hline
total_nonres_sqft & integer & \emph{(optional) }
\\
\hline
total_undevelopable_sqft & integer & \emph{(optional) }
\\
\hline

\end{tabular}

\begin{itemize} \tight
\item fraction_residential_land must be between 0 and 1
\item commercial_sqft, governmental_sqft and residential_units must be \textgreater{}= 0
\item development_type_id must be a valid index in the \verb|development_types|
table
\item distance_to_arterial,  distance_to_highway must be \textgreater{}= 0
\item grid_id must be unique and \textgreater{} 0
\item commercial_improvement_value,  industrial_improvement_value, residential_improvement_value, 
and governmental_improvement_value 
must be \textgreater{}= 0 
\item nonresidential_land_value and residential_land_value must be \textgreater{}= 0
\item relative_x,relative_y coordinate pairs must be unique, and
\textgreater{}= 1.
\item The relative_x and relative_y columns are measured in grid cell units.
They are specifically \textbf{not} latitude/longitude or any other universal
measurement system. For example this sparse grid (6 cells in a 3x3 grid; cells
are labeled with grid_id, relative_x, relative_y):

\begin{tabular}{ccc}
(1,1,1)  &(2,2,1)  &-  \\
(3,1,2)  &(4,2,2)  &(5,3,2)  \\
-  &-  &(6,3,3)

\end{tabular}

\item year_built must be less than or equal to the start date of the scenario and larger 
than absolute_min_year which can be defined in the \verb|urbansim_constants| table (default is 1800).
\item plan_type must be a valid index in the \verb|plan_types| table
\item All percent_* attributes must be between 0 and 100
\item zone_id must be a valid id in the \verb|zones| table
\item city_id must be a valid index into the \verb|cities| table or zero if
there is no city
\item county_id must be a valid index into the \verb|counties| table or zero if
there is no county
\item For gridcells with any households on them (i.e., household.grid_id =
gridcell.grid_id), the gridcell.residential_units must be greater than 0.
\end{itemize}

\subsection{The {\tt plan_types} table}

\verb|plan_types| are synonymous with Zoning types: for example
``residential2''. Also synonymous with Planned Land Use (PLU) types. The
distinction is arbitrary and is to be made by the user.

One row per plan type.

\begin{tabular}{lll}
%\hline
\textbf{Column Name} & \textbf{Data Type} & \textbf{Description} \\
\hline
plan_type_id & integer & Unique identifier  \\
\hline
name & varchar & Unique name of the Plan Type  \\
\hline

\end{tabular}

\begin{itemize} \tight
\item plan_type_id must be unique, greater than zero, and less than or equal to 9999.
We recommend that plan_type_id start at 1 and be sequential.
\item name must be unique.

\end{itemize}


\section{Database Tables about Development Types}
\label{sec:development-tables}

%\emph{Note: tables about development types are no longer needed, 
%even for gridcell-based projects, since the
%real estate development model that used them, which was based on modeling
%a transition from one development type to another, has been replaced
%in the gridcell applications with a real estate development project location
%choice model.  The descriptions in this section are retained in the documentation 
%mainly to include this disclaimer, and for readers that need access to the older
%information}.


\emph{Development types} are used to classify a grid cell according to the
``type'' of development currently in the grid cell.  For instance, grid cells
with only a few residential units and no other square footage might be
classified as ``low density residential'' which may be abbreviated as ``R1''.
Other grid cells may be classified as mixed use, commercial, etc.  The set of
development types to use is arbitrary.

Development types are grouped by two nested mechanisms: \emph{groups} and
\emph{non-overlapping-groups}. Each development type may be a member of
multiple groups. Each group may be a member of multiple
non-overlapping-groups. All of the groups in a non-overlapping-group must be
disjoint (i.e., may not share any development types); in other words, each
development type must belong to at most one group in each
non-overlapping-group.

Groups and non-overlapping-groups are used in the computation of the variables \variablesindex
in the models, so to fully understand them requires understanding the model \modelsindex
definitions.

\subsection{The {\tt development_types} table}

This table is used by the Events Coordinator (see Section~\ref{sec:events-coordinator}).
Each row defines one development type.

\begin{tabular}{lll}

%\hline
\textbf{Column Name} & \textbf{Data Type} & \textbf{Description} \\

\hline development_type_id & integer & Unique identifier for this row.  \\

\hline name & varchar & Name of the development type.  \\

\hline min_units & integer & Minimum number of units to be in this development
type.  \\

\hline max_units & integer & Maximum number of units to be in this development
type.  \\

\hline min_sqft & integer & Minimum square feet to be in this development type.
\\

\hline max_sqft & integer & Maximum square feet to be in this development type.
\\

\hline

\end{tabular}

\begin{itemize}
\tight
\item development_type_id must be unique and greater than zero. We recommend that it starts at 1 and is sequential.
\item min_units must be \textgreater{}= 0.
\item max_units must be \textgreater{}= min_units.
\item min_sqft must be \textgreater{}= 0.
\item max_sqft must be \textgreater{}= min_sqft.
\item The development types should not overlap, and should completely cover the
space.  A grid cell should only be able to be in a single development type.
\end{itemize}

\subsection{The {\tt development_type_groups} table}
\label{sec:development-tables-type-groups}

Each row defines one development type group, but not the group's
membership - the memberships are defined in the \verb|development_type_group_definitions| table.

\begin{tabular}{lll}

%\hline
\textbf{Column Name} & \textbf{Data Type} & \textbf{Description} \\

\hline group_id & integer & Unique identifier for this row.  \\

\hline name & varchar & Unique name of the development type group.  \\

\hline non_overlapping_groups & varchar & Name of the non-overlapping-group or
empty for no non-overlapping-group.  \\

\hline
\end{tabular}

\begin{itemize}
\tight
\item group_id must be unique, and greater than zero.
\item name must be unique. The required development type groups must be lower
case with underscores between words e.g. high_density_residential. We recommend
that all names follow this style.
\item names and non_overlapping_groups names must not contain spaces must be lower-case.
\item Development types must not overlap across the groups in the same
non_overlapping_groups.
\end{itemize}

The set of required development type groups and non-overlapping-groups is
determined by the set of variables used by the models being estimated or
simulated.  Thus, there is no way to a-priori specify which development type
groups will be needed for your application of UrbanSim. There are two exceptions:
First, the model Events Coordinator (\ref{sec:events-coordinator}) is internally using groups 'residential', 'mixed_use', 'commercial',
'industrial', and 'governmental'. Second, the Land Price Model (\ref{sec:land-price-model}) is using by default a filter
that requires a group called 'developable'. Therefore, if you do not change this settings, make sure your
table contain these entries.

\subsection{The {\tt development_type_group_definitions} table}

This table defines the set of \verb|development_types| in each
development type group. Each row defines one ``belongs to'' relationship (a
particular development type that ``belongs to'' a particular
development type group).

\begin{tabular}{lll}
%\hline
\textbf{Column Name} & \textbf{Data Type} & \textbf{Description} \\

\hline
development_type_id & integer & Index into the \verb|development_types| table  \\
\hline
group_id & integer & Index into the \verb|development_type_groups| table  \\
\hline

\end{tabular}

\begin{itemize}
\tight
\item development_type_id must be a valid index into the \verb|development_types| table
\item group_id must be a valid index into the \verb|development_type_groups| table
\item The combination of development_type_id and group_id must be unique

\end{itemize}



\section{Database Tables about Development Events}
\label{sec:db-tables-events}

These tables represent events in the real estate development. Events that are scheduled 
to take place in the future are stored in the {\tt development_events_exogenous} table,
events that occured prior to the base year are stored in the {\tt development_event_history} table.

Both tables can contain columns of the pattern ``{\it units}_{\tt change_type}''. Each value determines
a type of change for that type of {\it units}. Possible values are:
\begin{itemize} \tight
\item ``A'' for Add
\item ``R'' for Replace
\item ``D'' for Delete
\end{itemize}
If this column is missing for a certain type of units, the default value is ``A'' for all events.

\subsection{The {\tt development_events_exogenous} table}
\label{sec:db-tables-development-events-exogenous}

These development events are changes to grid cells which are scheduled to take
place
in the future. For any given year, it is possible to schedule any number of
changes to the attributes of any number of gridcells. Each change represents
that addition, subtruction or replacement of the specified number of sqft, residential units, and
improvement values.  For example, if
grid cell 23 is to grow by 200 residential units in 2008 (an apartment building
is built), the table would include a row with scheduled_year = 2008, grid_id =
23, residential_units = 200, and residential_units_change_type = 'A'.

The value in the ``improvement_value'' fields, below, are used to indicate how
to change the associated improvement_value for this grid cell. Each event will
add/subtract/replace (improvement_value * (number
of units [or sqft] being built by this event)) to the current improvement value
in
this grid cell.  The units of the improvement is currency value, e.g. dollars.

The described procedure is implemented in Events Coordinator (see Section~\ref{sec:events-coordinator}).

\begin{tabular}{p{2in}lp{3.5in}}
%\hline
\textbf{Column Name} & \textbf{Data Type} & \textbf{Description} \\
\hline
grid_id & integer & Grid cell where the event takes place  \\
\hline
scheduled_year &short & Year in which the event will be implemented  \\
\hline
residential_units & integer & \\
\hline
commercial_sqft & integer & \\
\hline
industrial_sqft & integer & \\
\hline
governmental_sqft & integer & \\
\hline
residential_units_change_type & char & \emph{(optional) } see \ref{sec:db-tables-events}\\
\hline
commercial_sqft_change_type & char & \emph{(optional) } see \ref{sec:db-tables-events}\\
\hline
industrial_sqft_change_type & char & \emph{(optional) } see \ref{sec:db-tables-events}\\
\hline
governmental_sqft_change_type & char & \emph{(optional) } see \ref{sec:db-tables-events}\\
\hline
residential_improvement_value & integer & See description, above  \\
\hline
commercial_improvement_value & integer & See description, above  \\
\hline
industrial_improvement_value & integer & See description, above  \\
\hline
governmental_improvement_value & integer & See description, above  \\
\hline

\end{tabular}

\begin{itemize} \tight
\item grid_id must be a valid id in the \verb|gridcells| table
\end{itemize}


\subsection{The {\tt development_event_history} table}
\label{sec:table-development-event-history}

The development event history records the development events that occurred
prior to the base year. It is used by the development project transition model~(\ref{sec:development-project-transition-model}),
and for ``unrolling'' the baseyear to create versions of the gridcell data for
prior years.

This table uses a subset of the schema used for
\verb|development_events_exogenous|.  It
can be considered an extension back in time of the \verb|development_events_exogenous|
table, though with additional constraints, specified below.

\begin{tabular}{p{2in}lp{3.5in}}
%\hline
\textbf{Column Name} & \textbf{Data Type} & \textbf{Description} \\
\hline
grid_id & integer & Grid cell where the event takes place  \\
\hline
scheduled_year &short & Year in which the event was implemented  \\
\hline
starting_development_type_id & integer & \emph{(optional) }
This will be the value of the development_type for this gridcell after
``unrolling'' this development event. \\
\hline
residential_units & integer & \\
\hline
commercial_sqft & integer & \\
\hline
industrial_sqft & integer & \\
\hline
governmental_sqft & integer & \\
\hline
residential_units_change_type & char & \emph{(optional) } see \ref{sec:db-tables-events}\\
\hline
commercial_sqft_change_type & char & \emph{(optional) } see \ref{sec:db-tables-events}\\
\hline
industrial_sqft_change_type & char & \emph{(optional) } see \ref{sec:db-tables-events}\\
\hline
governmental_sqft_change_type & char & \emph{(optional) } see \ref{sec:db-tables-events}\\
\hline
residential_improvement_value & integer & See description, above  \\
\hline
commercial_improvement_value & integer & See description, above  \\
\hline
industrial_improvement_value & integer & See description, above  \\
\hline
governmental_improvement_value & integer & See description, above  \\
\hline

\end{tabular}

\begin{itemize} \tight
\item grid_id must be a valid id in the \verb|gridcells| table
\item starting_development_type_id must be a valid index into the \verb|development_types| table
\item starting_development_type_id is only required if the process of unrolling gridcells is activated.
\item Entries for which scheduled_year is greater than or equal to the base year will not be used.
\end{itemize}

\section{Database Tables About Development Constraints}
%
\subsection{The {\tt development_constraints} table}

This table defines rules that restrict the possible development types a
developer can create on a particular grid cell.  Each row defines one rule.
Development is not allowed on any gridcell that matches any of these rules.  A
gridcell matches a rule if the attribute values for the gridcell match all of
the values in the rule (rule columns with the value ``-1'' are ignored when
determining a match).

\begin{tabular}{llp{3.5in}}

%\hline
\textbf{Column Name} & \textbf{Data Type} & \textbf{Description} \\

\hline constraint_id & integer & Unique rule identification number  \\

\hline \emph{name-of-a-gridcell-attribute} & integer or float &  Value for this
attribute, or ``-1'' if this attribute is not part of the constraint (e.g.
\emph{don't care}) \\

\hline min_units & integer & Minimum number of residential units for a gridcell.
A development project may only be placed on this gridcell if it will result in
this gridcell containing at least this number of residential units. \\

\hline max_units & integer & Maximum number of residential units for a gridcell.
A development project may only be placed on this gridcell if it will result in
this gridcell containing at most this number of residential units. \\

\hline min_commercial_sqft & integer & Minimum number of commercial
sqft. for a gridcell.  A development project may only be placed on
this gridcell if it will
result in this gridcell containing at least this number of commercial sqft. \\

\hline max_commercial_sqft & integer & Maximum number of commercial
sqft. for a gridcell. A development project may only be placed on
this gridcell if it will
result in this gridcell containing at most this number of commercial sqft. \\

\hline min_industrial_sqft & integer & Minimum number of industrial
sqft. for a gridcell.  A development project may only be placed on
this gridcell if it will
result in this gridcell containing at least this number of industrial sqft. \\

\hline max_industrial_sqft & integer & Maximum number of industrial
sqft. for a gridcell. A development project may only be placed on
this gridcell if it will
result in this gridcell containing at most this number of industrial sqft. \\

\hline
\end{tabular}

\begin{itemize}
\tight
\item constraint_id must be a unique positive integer.

\item \emph{name-of-a-gridcell-attribute} is the name of a gridcell attribute,
e.g. \verb|city_id| or \verb|is_in_wetland|.  The set of available attribute
names is determined by the set of numeric column names on the gridcell table.

\item Within each \emph{min_}/\emph{max_} pair, the max must be greater than or
equal to the min, e.g., \verb|min_units| $<=$ \verb|max_units|.

\end{itemize}

\section{Database Tables About Target Vacancies}

\subsection{The {\tt target_vacancies} table used in Gridcell-based Applications}
\label{sec:table-target-vacancies-gridcell}

The \verb|target_vacancies| table is used by the development project transition
model~(\ref{sec:development-project-transition-model}). It gives the model information about acceptable vacancy rates. The table
has one row for each year the simulation runs. Each row gives target values for
the residential and nonresidential vacancies for that year, which are defined
below.  Only data after the base year is used.

\begin{tabular}{lll}

%\hline
\textbf{Column Name} & \textbf{Data Type} & \textbf{Description} \\

\hline year & integer & Year of the simulation for which the vacancy
targets
apply  \\

\hline target_total_residential_vacancy & float & Ratio of unused residential
units to total residential units  \\

\hline target_total_non_residential_vacancy & float & Ratio of unused
nonresidential sqft to total nonresidential sqft  \\

\hline
\end{tabular}

\begin{itemize}
\tight
\item There must be exactly one row for each year to be simulated.
\item target_total_residential_vacancy must be between 0 and 1, inclusive.
\item target_total_non_residential_vacancy must be between 0 and 1, inclusive.
\end{itemize}

