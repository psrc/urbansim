\chapter{Gridcell Data}


\section{Database Tables about Grid Cells}
\label{sec:gridcell-tables}

\subsection{The {\tt gridcells} table}

Geographic information partitioned into a rectangular grid of rectangular cells.

The ``improvement_value'' fields, below, indicate the value (e.g., dollars) of
all buildings of a particular type that are in this grid cell. For instance,
commercial_improvement_value is the total value of all commercial buildings in
this grid cell. The use of ``improvement'' indicates that buildings are
considered ``improvements'' over the grid cell's land value.

\begin{tabular}{|p{2in}|l|p{3.5in}|}
\hline
\textbf{Column Name} & \textbf{Data Type} & \textbf{Description} \\
\hline
grid_id & integer & Unique identifier  \\
\hline
commercial_sqft & integer & The sum of the square footage of buildings that are classified as commercial (generally including retail and office land uses). This is not a measure of land area.  \\
\hline
development_type_id & integer & Index into the Development Types table  \\
\hline
distance_to_arterial & float &  Units: \verb|urbansim_constants|.units \\
\hline
distance_to_highway & float &  Units: \verb|urbansim_constants|.units \\
\hline
governmental_sqft & integer & \\
\hline
industrial_sqft & integer & \\
\hline
commercial_improvement_value & integer & See description, above  \\
\hline
industrial_improvement_value & integer & See description, above  \\
\hline
governmental_improvement_value & integer & See description, above  \\
\hline
nonresidential_land_value & integer & Units, e.g. dollars  \\
\hline
residential_improvement_value & integer & See description, above  \\
\hline
residential_land_value & integer & Units, e.g. dollars  \\
\hline
residential_units & integer & Number of residential units  \\
\hline
relative_x & integer & X coordinate in grid coordinate system  \\
\hline
relative_y & integer & Y coordinate in grid coordinate system  \\
\hline
year_built & integer & e.g. 2002  \\
\hline
plan_type_id & integer & An id indicating the plan type of the grid cell  \\
\hline
percent_agricultural_protected_land & integer & \emph{(optional) }
\\
\hline
percent_water & integer & Percentage of this cell covered by water  \\
\hline
percent_stream_buffer & integer & Percentage of this cell covered by stream buffer  \\
\hline
percent_floodplain & integer & Percentage of this cell covered by flood plain  \\
\hline
percent_wetland & integer & Percentage of this cell covered by wetland  \\
\hline
percent_slope & integer & Percentage of this cell covered by slope  \\
\hline
percent_open_space & integer & Percentage of this cell covered by open space  \\
\hline
percent_public_space & integer & Percentage of this cell covered by public space  \\
\hline
percent_roads & integer & Percentage of this cell covered by roads  \\
\hline
percent_undevelopable & integer & \emph{(optional) }
\\
\hline
is_outside_urban_growth_boundary & boolean & \\
\hline
is_state_land & boolean & \emph{(optional) }
\\
\hline
is_federal_land & boolean & \emph{(optional) }
\\
\hline
is_inside_military_base & boolean & \emph{(optional) }
\\
\hline
is_inside_national_forest & boolean & \emph{(optional) }
\\
\hline
is_inside_tribal_land & boolean & \emph{(optional) }
\\
\hline
zone_id & integer & Traffic analysis zone that contains this grid cell's centroid  \\
\hline
city_id & integer & City this Grid Cell belongs to  \\
\hline
county_id & integer & County this Grid Cell belongs to  \\
\hline
fraction_residential_land & float & Fraction of residential land in this cell  \\
\hline
total_nonres_sqft & integer & \emph{(optional) }
\\
\hline
total_undevelopable_sqft & integer & \emph{(optional) }
\\
\hline

\end{tabular}

\begin{itemize} \tight
\item fraction_residential_land must be between 0 and 1
\item commercial_sqft must be \textgreater{}= 0 and \textless{}=
absolute_max_cell_sqft
\item development_type_id must be a valid index in the \verb|development_types|
table
\item distance_to_arterial must be \textgreater{}= 0 and \textless{}=
absolute_max_distance
\item distance_to_highway must be \textgreater{}= 0 and \textless{}=
absolute_max_distance
\item industrial_sqft must be \textgreater{}= 0 and \textless{}=
absolute_max_cell_sqft
\item governmental_sqft must be \textgreater{}= 0 and \textless{}=
absolute_max_cell_sqft
\item grid_id must be unique and \textgreater{} 0
\item industrial_sqft must be \textgreater{}= 0
\item commercial_improvement_value must be \textgreater{}= 0 and \textless{}=
absolute_max_cell_dollars
\item industrial_improvement_value must be \textgreater{}= 0 and \textless{}=
absolute_max_cell_dollars
\item governmental_improvement_value must be \textgreater{}= 0 and \textless{}=
absolute_max_cell_dollars
\item nonresidential_land_value must be \textgreater{}= 0 and \textless{}=
absolute_max_cell_dollars
\item residential_improvement_value must be \textgreater{}= 0 and \textless{}=
absolute_max_cell_dollars
\item residential_land_value must be \textgreater{}= 0 and \textless{}=
absolute_max_cell_dollars
\item residential_units must be must be \textgreater{}= 0 and \textless{}=
absolute_max_cell_residential_units
\item relative_x,relative_y coordinate pairs must be unique, and
\textgreater{}= 1.
\item The relative_x and relative_y columns are measured in grid cell units.
They are specifically \textbf{not} latitude/longitude or any other universal
measurement system. For example this sparse grid (6 cells in a 3x3 grid; cells
are labeled with grid_id, relative_x, relative_y):

\begin{tabular}{ccc}
(1,1,1)  &(2,2,1)  &-  \\
(3,1,2)  &(4,2,2)  &(5,3,2)  \\
-  &-  &(6,3,3)

\end{tabular}

\item year_built must be less than or equal to the start date of the scenario,
and must be between absolute_min_year and absolute_max_year
\item plan_type must be a valid index in the \verb|plan_types| table
\item percent_water must be between 0 and 100
\item percent_stream_buffer must be between 0 and 100
\item percent_floodplain must be between 0 and 100
\item percent_wetland must be between 0 and 100
\item percent_slope must be between 0 and 100
\item percent_open_space must be between 0 and 100
\item percent_public_space must be between 0 and 100
\item percent_roads must be between 0 and 100
\item zone_id must be a valid id in the \verb|zones| table
\item city_id must be a valid index into the \verb|cities| table or zero if
there is no city
\item county_id must be a valid index into the \verb|counties| table or zero if
there is no county
\item gridcells with any households on them (i.e., households.grid_id =
gridcell.grid_id), then the gridcell.residential_units must be greater than 0
\end{itemize}

\subsection{The {\tt plan_types} table}

\verb|plan_types| are synonymous with Zoning types: for example
``residential2''. Also synonymous with Planned Land Use (PLU) types. The
distinction is arbitrary and is to be made by the user.

One row per plan type.

\begin{tabular}{|l|l|l|}
\hline
\textbf{Column Name} & \textbf{Data Type} & \textbf{Description} \\
\hline
plan_type_id & integer & Unique identifier  \\
\hline
name & varchar & Unique name of the Plan Type  \\
\hline

\end{tabular}

\begin{itemize} \tight
\item plan_type_id must be unique, greater than zero, and less than or equal to 9999.
\item We recommend that plan_type_ids start at 1 and be sequential.
\item name must be unique. We recommend that names follow the style guide.

\end{itemize}


\section{Database Tables about Development Types}
\label{sec:development-tables}

\emph{Development types} are used to classify a grid cell according to the
``type'' of development currently in the grid cell.  For instance, grid cells
with only a few residential units and no other square footage might be
classified as ``low density residential'' which may be abbreviated as ``R1''.
Other grid cells may be classified as mixed use, commercial, etc.  The set of
development types to use is arbitrary.

Development types are grouped by two nested mechanisms: \emph{groups} and
\emph{non-overlapping-groups}. Each development type may be a member of
multiple groups. Each group may be a member of multiple
non-overlapping-groups. All of the groups in a non-overlapping-group must be
disjoint (i.e., may not share any development types); in other words, each
development type must belong to at most one group in each
non-overlapping-group.

Groups and non-overlapping-groups are used in the computation of the variables \variablesindex
in the models, so to fully understand them requires understanding the model \modelsindex
definitions.

\subsection{The {\tt development_types} table}

Each row defines one development type.

\begin{tabular}{|l|l|l|}

\hline
\textbf{Column Name} & \textbf{Data Type} & \textbf{Description} \\

\hline development_type_id & integer & Unique identifier for this row.  \\

\hline name & varchar & Name of the development type.  \\

\hline min_units & integer & Minimum number of units to be in this development
type.  \\

\hline max_units & integer & Maximum number of units to be in this development
type.  \\

\hline min_sqft & integer & Minimum square feet to be in this development type.
\\

\hline max_sqft & integer & Maximum square feet to be in this development type.
\\

\hline

\end{tabular}

\begin{itemize}
\tight
\item development_type_id must be unique and greater than zero. We recommend that it starts at 1 and is sequential.
\item min_units must be \textgreater{}= 0.
\item max_units must be \textgreater{}= min_units.
\item min_sqft must be \textgreater{}= 0.
\item max_sqft must be \textgreater{}= min_sqft.
\item The development types should not overlap, and should completely cover the
space.  A grid cell should only be able to be in a single development type.
\end{itemize}

\subsection{The {\tt development_type_groups} table}
\label{sec:development-tables-type-groups}

Each row defines one development type group, but not the group's
membership - the memberships are defined in the \verb|development_type_group_definitions| table.

\begin{tabular}{|l|l|l|}

\hline
\textbf{Column Name} & \textbf{Data Type} & \textbf{Description} \\

\hline group_id & integer & Unique identifier for this row.  \\

\hline name & varchar & Unique name of the development type group.  \\

\hline non_overlapping_groups & varchar & Name of the non-overlapping-group or
empty for no non-overlapping-group.  \\

\hline
\end{tabular}

\begin{itemize}
\tight
\item group_id must be unique, and greater than zero.
\item name must be unique. The required development type groups must be lower
case with underscores between words e.g. high_density_residential. We recommend
that all names follow this style.
\item names and non_overlapping_groups names must not contain spaces.
\item names and non_overlapping_groups names must be lower-case.
\item Development types must not overlap across the groups in the same
non_overlapping_groups.
\end{itemize}

The set of required development type groups and non-overlapping-groups is
determined by the set of variables used by the models being estimated or
simulated.  Thus, there is no way to a-priori specify which development type
groups will be needed for your application of UrbanSim. There are two exceptions:
First, the model Events Coordinator is internally using groups 'residential', 'mixed_use', 'commercial',
'industrial', and 'governmental'. Second, the Land Price Model (\ref{sec:land-price-model}) is using by default a filter
that requires a group called 'developable'. Therefore, if you do not change this settings, make sure your
table contain these entries.

UrbanSim requires some non-overlapping-groups and requires those to have
certain groups within them, though they may have additional groups as well.

\subsection{The {\tt development_type_group_definitions} table}

This table defines the set of \verb|development_types| in each
development type group. Each row defines one ``belongs to'' relationship (a
particular development type that ``belongs to'' a particular
development type group).

\begin{tabular}{|l|l|l|}
\hline
\textbf{Column Name} & \textbf{Data Type} & \textbf{Description} \\

\hline
development_type_id & integer & Index into the development_types table  \\
\hline
group_id & integer & Index into the \verb|development_type_groups| table  \\
\hline

\end{tabular}

\begin{itemize}
\tight
\item development_type_id must be a valid index into the \verb|development_types| table
\item group_id must be a valid index into the \verb|development_type_groups| table
\item The combination of development_type_id and group_id must be unique

\end{itemize}

\subsection{Example}

Let R1-R3 be in the
``residential'' group in the ``dynamic_land_use_variables''
non-overlapping-group, M4-M5 be in the ``mixed_use'' group in the
``dynamic_land_use_variables'' non-overlapping-group, and C1-C9 be in the
``commercial'' group in the ``dynamic_land_use_variables'' \variablesindex
non-overlapping-group. Also let R3 and M5 be in the
``high_density_residential'' group, and ``M5'' and ``C4'' be in the
``noisy_commercial'' group.

This example would involve these two tables: (note: R1=1, R2=2, R3=3,
M4=4, M5=5, C1=6, \ldots, C4=9, etc)

\begin{center}
\begin{tabular}{c}

\begin{tabular}{|l|l|l|}
\multicolumn{3}{c}{\textbf{\tt|development_type_groups|}} \\
\hline
group_id & name & non_overlapping_groups \\

\hline
50 &residential &dynamic_land_use_variables \\

\hline
51 &mixed_user &dynamic_land_use_variables \\

\hline
52 &commercial &dynamic_land_use_variables \\

\hline
100 &high_density_residential & \\

\hline
101 &noisy_commercial & \\

\hline
\end{tabular}

\\ \\

\begin{tabular}{|l|l|}
\multicolumn{2}{c}{\textbf{\tt|development_type_group_definitions|}} \\

\hline
development_type_id & group_id \\

\hline
1 & 50 \\
\hline
2 & 50 \\
\hline
3 & 50 \\
\hline
4 & 51 \\
\hline
5 & 51 \\
\hline
6 & 52 \\
\hline
... &  \\
\hline
3 & 100 \\
\hline
5 & 100 \\
\hline
9 & 101 \\
\hline
5 & 101 \\
\hline
\end{tabular}

\end{tabular}
\end{center}


Note that:
\begin{itemize}
\tight
\item R3 belongs both to groups ``residential'' and
``high_density_residential''.
\item R1 belongs to group ``residential''.
\item M5 belongs to groups ``mixed_use'', ``high_density_residential'', and
``noisy_commericial''.
\end{itemize}


\section{Database Tables about Development Events}
\label{sec:db-tables-events}
These tables represent events in the real estate development. Events that are scheduled 
to take place in the future are stored in the {\tt development_events_exogenous} table,
events that occured prior to the base year are stored in the {\tt development_event_history} table.

Both tables can contain columns of the pattern ``{\it units}_{\tt change_type}''. Each value determines
a type of change for that type of {\it units}. Possible values are:
\begin{itemize} \tight
\item ``A'' for Add
\item ``R'' for Replace
\item ``D'' for Delete
\end{itemize}
If this column is missing for a certain type of units, the default value is ``A'' for all events.

\subsection{The {\tt development_events_exogenous} table}

These development events are changes to grid cells which are scheduled to take
place
in the future. For any given year, it is possible to schedule any number of
changes to the attributes of any number of gridcells. Each change represents
that addition, subtruction or replacement of the specified number of sqft, residential units, and
improvement values.  For example, if
GridCell 23 is to grow by 200 residential units in 2008 (an apartment building
is built), the table would include a row with scheduled_year = 2008, grid_id =
23, residential_units = 200, and residential_units_change_type = 'A'.

The value in the ``improvement_value'' fields, below, are used to indicate how
to change the associated improvement_value for this grid cell. Each event will
add/subtract/replace (improvement_value * (number
of units [or sqft] being built by this event)) to the current improvement value
in
this grid cell.  The units of the improvement is currency value, e.g. dollars.

\begin{tabular}{|p{2in}|l|p{3.5in}|}
\hline
\textbf{Column Name} & \textbf{Data Type} & \textbf{Description} \\
\hline
grid_id & integer & Grid cell where the event takes place  \\
\hline
scheduled_year &short & Year in which the event will be implemented  \\
\hline
starting_development_type_id & integer & \emph{(optional) }
This field is ignored. It is here so that the schema is the same for the
\verb|development_events_exogenous| and \verb|development_event_history| tables.   \\
\hline
ending_development_type_id & integer &  This grid cell's development type
at the ending of the scheduled_year. Index into the \verb|development_types|
table   \\
\hline
residential_units & integer & \\
\hline
commercial_sqft & integer & \\
\hline
industrial_sqft & integer & \\
\hline
governmental_sqft & integer & \\
\hline
residential_units_change_type & char & \emph{(optional) } see \ref{sec:db-tables-events}\\
\hline
commercial_sqft_change_type & char & \emph{(optional) } see \ref{sec:db-tables-events}\\
\hline
industrial_sqft_change_type & char & \emph{(optional) } see \ref{sec:db-tables-events}\\
\hline
governmental_sqft_change_type & char & \emph{(optional) } see \ref{sec:db-tables-events}\\
\hline
residential_improvement_value & integer & See description, above  \\
\hline
commercial_improvement_value & integer & See description, above  \\
\hline
industrial_improvement_value & integer & See description, above  \\
\hline
governmental_improvement_value & integer & See description, above  \\
\hline
fraction_residential_land_value & float & Fraction of residential land in this cell  \\
\hline

\end{tabular}

\begin{itemize} \tight
\item fraction_residential_land_value must be between 0 and 1
\item grid_id must be a valid id in the \verb|gridcells| table or zero
\item development_type_id must be a valid index into the \verb|development_types| table or zero
\end{itemize}


\subsection{The {\tt development_event_history} table}
\label{sec:table-development-event-history}

The development event history records the development events that occurred
prior to the base year. It is used by the development project transition model~(\ref{sec:development-project-transition-model}),
and for ``unrolling'' the baseyear to create versions of the gridcell data for
prior years.

This table uses a subset of the schema used for
\verb|development_events_exogenous|.  It
can be considered an extension back in time of the \verb|development_events_exogenous|
table, though with additional constraints, specified below.

\begin{tabular}{|p{2in}|l|p{3.5in}|}
\hline
\textbf{Column Name} & \textbf{Data Type} & \textbf{Description} \\
\hline
grid_id & integer & Grid cell where the event takes place  \\
\hline
scheduled_year &short & Year in which the event was implemented  \\
\hline
starting_development_type_id & integer & \emph{(optional) }
This will be the value of the development_type for this gridcell after
``unrolling'' this development event. \\
\hline
residential_units & integer & \\
\hline
commercial_sqft & integer & \\
\hline
industrial_sqft & integer & \\
\hline
governmental_sqft & integer & \\
\hline
residential_units_change_type & char & \emph{(optional) } see \ref{sec:db-tables-events}\\
\hline
commercial_sqft_change_type & char & \emph{(optional) } see \ref{sec:db-tables-events}\\
\hline
industrial_sqft_change_type & char & \emph{(optional) } see \ref{sec:db-tables-events}\\
\hline
governmental_sqft_change_type & char & \emph{(optional) } see \ref{sec:db-tables-events}\\
\hline
residential_improvement_value & integer & See description, above  \\
\hline
commercial_improvement_value & integer & See description, above  \\
\hline
industrial_improvement_value & integer & See description, above  \\
\hline
governmental_improvement_value & integer & See description, above  \\
\hline

\end{tabular}

\begin{itemize} \tight
\item grid_id must be a valid id in the \verb|gridcells| table or zero
\item development_type_id must be a valid index into the \verb|development_types| table or zero
\end{itemize}

In addition, the following will produce warnings:
\begin{itemize} \tight
\item Warn if scheduled_year is greater than or equal to the base year. Such
entries will not be used.
\end{itemize}

