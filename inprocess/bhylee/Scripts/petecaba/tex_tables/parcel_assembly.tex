\section{Step 3: Assemble and Standardize Parcel Data}

Initial parcel data for all four central Puget Sound counties was
provided by the PSRC.  The goal of the process was to create a
2000 base year flat parcel table for the central Puget Sound. For
some counties, additional data needed to be obtained to complete
the flat parcel attribute table.  Due to time lags in updating
parcel files, and some systematic gaps in the Snohomish and Pierce
County parcel file that existed in the 2000 database, GIS
coverages of parcel boundaries for the year 2001 were collected
and processed. Attributes of parcels were assembled from county
tax assessor files for each county, and these were supplemented
with information from buildable lands analyses in Pierce and
Snohomish Counties.  For all four counties, the area of the parcel
was obtained from the GIS area measurement for the parcel.

The initial processing procedures taken to construct the flat
parcel files were to extract the compulsory fields from the
attribute tables and export the fields to a MySQL database. Once
the general layout of the table were assembled a series of SQL
queries were used to updated and organize the data.

The parcel data form a foundation for the description of the land
use and real estate inventory within the study area. These
attributes include: Parcel Identification Number, Land Use, Lot
Size, Housing Units, Building Square Feet, Year Built, Zoning
Code, Assessed Land Value, Assessed Improvement Value, and Tax
Exempt Status. These fields become the design frame in which the
flat parcel assembly would be forged into.


GIS Parcel Construction:

For all counties each parcel shapefile was converted to a coverage
file in order to perform spatial operations.

\subsection{King County}

A 2001 parcel shape file and tax assessor's file for King County
was provided by the PSRC.  The coverage only contained a Parcel
Identification Number (PIN) and land use code.  A total of 12
additional tax assessor file extracts from King County were
obtained through the Washington Geospatial Data Archive (WAGDA)
for the following tables: apartment complex extract, condominium
complex extract, residential buildings extract, parcel extract,
real property account extract.

\subsection{Kitsap County}

A 2001 parcel shapefile and tax assessor's file for Kitsap County
was provided by the PSRC.  The assessor's data and parcel
shapefile matched up very well and very little additional work was
needed to join the data sets by PIN.

\subsection{Pierce County}

A 2001 parcel shapefile and attribute file from Metroscan for
Pierce County was provided by the PSRC.  The attribute file had
several problems:  the housing units, square footage, and year
built fields appeared to be corrupted and had unusable data, and
the file did not include assessed land and improvement values. In
order to generate usable data, the 2003 assessor's file was
substituted for the 2001 file, and the attributes appeared to be
reasonably complete in this file, but represented more recent
conditions than the desired base year 2000.  In order to better
reflect the 2000 baseyear, additional data sources were sought
out.  Three buildable lands coverages with 2000 parcel data were
provided by Pierce County via the PSRC.  One coverage was for the
City of Tacoma, one was for several incorporated jurisdictions and
one was for several unincorporated areas.  The buildable lands
data sets included fields for improvement value, land value, and
land use.  The county coverage file was updated with values from
the buildable lands data sets so that the parcel data would better
reflect 2000 totals.

\subsection{Snohomish County}

A 2001 parcel shapefile and tax assessor's file for Snohomish
County was provided by the county.  The tax assessor's file
appeared to have problems with the land use code and the number of
housing units (for example, apartment housing units were not
recorded, and duplexes frequently only recorded 1 unit), based on
analysis of the data and interaction with county staff.  Multiple
shape files for year 2000 buildable lands analysis were obtained
from the PSRC and used to update housing units, land use, land
value and improvement value in the tax assessor's file.


For the most part, these attribute fields in the parcel tables
share the same name and coding standards between counties.
However, for some fields such as 'land use,' there can be
significant variation in the land use categories used by each
county.  An important step in this process is to standardize the
names and coding standards used for these important fields.


Land Use Code Standardization

Each of the four counties contain a unique set of use codes and
land use classification descriptions.  In order for all four
counties to be consistent a generic land use reclassification
table was created. The reclassification table is an extended
version of the Station Land Use Study Areas Draft document
provided by the PSRC. The table categorize the various county land
use codes into standardized and more general land use categories.

The reclassification table creates two levels of land use category
aggregation GENERIC_LAND_USE_1 aggregated land use codes into 26
categories and GENERIC_LAND_USE_2 aggregated land use codes into
six broader categories (commercial, government, industrial,
residential, non-residential, and group quarters).
GENERIC_LAND_USE_1 is utilized later in the job allocation process
and to help analyze data quality indicators. GENERIC_LAND_USE_2 is
used later in Step 7 and Step 8 when parcel data is assigned to
gridcells.  The look up table should map individual land use codes
for each county in the study area to user-defined aggregations.
The mapping of county specific land use codes to the aggregate
land use categories can be made by joining the "county land use
code" and "county identification".

\subsection{Snohomish County}

The Snohomish County land use codes were extracted from the
Buildable Lands Parcel Data (shapefile_name). The attribute table
contained a field labeled "Propclassc", which contained the
3-digit (integer) land use code and a field labeled "Pcc_Desc",
which contained the description of the 3-digit land use code. The
the land use codes and their descriptions were grouped into a
temporary county land use code table summarizing all of the county
use codes that were applied to the parcel records in the Buildable
Lands Parcel set.

With the aid of the reclassification table from the Station Land
Use Study Areas Draft document, the GENERIC_LAND_USE_1 and
GENERIC_LAND_USE_2 fields were created. After an initial draft of
the reclassification table was prepared, the file was put under
review by the PSRC and the county specific staff members.

\subsection{King County}

The King County Land use codes were extracted from the Assessor's
file (EXTR_Parcel.dbf - Parcel Record Description). The table's
"PRESENTUSE" field contained a 3-digit (integer) land use code.
The land use code description is located in an adjoining table
(ExtraLookUp.csv - Look Up Record Description). The table's "Look
Up Description" field is description of the land use code.

\subsection{Pierce County}

The Pierce County land use codes were assembled from the County
Assessor's records (Master_table - Pierce Extract), a supplemental
file from the Pierce County Planning and Land Services Department,
and a table containing suggested reclassifications to the
GENERIC_LAND_USE_1 and GENERIC_LAND_USE_2 categories.

The Assessor's record contained a field labeled, "use_cd", which
contained a 4-digit (integer) land use code and a field labeled,
"use_desc", which was a description of the land use code. These
values were then summarized into a temporary county land use code
table. The preliminary Pierce County land use code table was then
distributed to the PSRC and county staff members for review. The
result of the review led to the supplemental land use code files
in the before mentioned sources.

The supplemental file (shapefile) consisted of parcel id's and
land use codes for records the Assessor's file had missing land
use codes for. The attributes from the shapefile containing the
missing land use values were exported to a MySQL database and were
used to updated the Flat Parcel file through a series of SQL
queries (insert queries here).

The Pierce County reclassification table also received updates
from the suggested reclassifications from the PSRC and Pierce
County Planning and Land Services Department.

\subsection(Kitsap County}

The Kitsap County land use codes were extracted from the County
Assessor's records (shapefile_name). The "USECODE" field contains
a 5-digit (integer) land use code. The "LANDUSE" field provided
the description land use description. A summary table consisting
of the land use codes that were applied to the parcel records was
created.


Parcel Address Standardization

County parcel addresses often follow different naming conventions
and need to be standardized.  A SAS script was used to parse
parcel address fields into five separate fields: street number
(st_no), directional prefix (st_pf), street name (st_name), street
type (st_type), and directional suffix (st_sf).  An example of the
SAS script used to parse Pierce County, Washington addresses is
available here: SAS script for address parsing (add SAS script)

\subsection{King County} -- Chris' input needed

King County's parcel addresses were extracted from an attribute
table associated with a King County Parcel file (.add file). The
format of the raw parcel address file contained all elements of
the parcel's address in one field, "Address". The King County
parcel file was exported to a geodatabase where the "PIN" and
"Address" fields were selected and inserted into a new table
called, "King_address". King_address was exported to a working
folder as a dbase format and processed through the Address Parse
SAS script.

\subsection(Kitsap County}

Kitsap County's parcel addresses were extracted from the Kitsap
parcel coverage file. The file contained the parcel record's
"PIN", "SITENUMBER", AND "SITESTREET". The coverage file was
exported to a geodatabase where the PIN, SITENUMBER, and
SITESTREET were selected and inserted into a new table,
"Kitsap_address". Kitsap_address was exported to a working folder
as a dbase format and processed through the Address Parse SAS
script.

\subsection{Pierce County}

Pierce County's parcel addresses were extracted from a second
parcel shapefile that was received from the PSRC (pieadd.shp)
because our current parcel file did not contain address attribute
information. The parcel file was then exported to a geodatabase
where the "PIN", "SITENUMBER", and "SITESTREET" were selected and
inserted into a new table, "Pierce_address". Pierce_address was
exported to a working folder as a dbase format and processed
through the Address Parse SAS script.

Note - Although Pierce County used a Buildable Lands file and
Assessor's file, there was no address attributes for the Buildable
Lands file.

\subsection{Snohomish County}

Snohomish County's parcel addresses were extracted from a
combination of the Buildable Lands file and Master_Record file
(Assessor data). Because the Snohomish County parcel coverage
contains a 2000 and 2001 parcel coverage two preliminary parcel
address tables needed to be completed.

The first address file contained data from the 2001 Snohomish
County Assessor's address records. The Master_Record table, which
was a supplemental Assessor's file we received from Snohomish
County, contained the necessary address contents and parcel
identification. The "LRSN" and "Address" fields were selected from
the Master_Record table and inserted into a new table,
"Snohomish_master_address". The Snohomish_master_address table was
exported to a working folder as a .dbf and processed through the
Address Parse SAS script.

The merged Snohomish Buildable Lands coverage file was exported to
a geodatabase. The "LRSN" and "ST_ADDRESS" were selected from the
parcel coverage and inserted into a new table,
"Snohomish_buildable_address". The Snohomish_buildable_address
table was exported to a working folder as a .dbf and processed
through the Address Parse SAS script.

The exported merged Snohomish Buildable Lands coverage file was
used as template that would receive the updates from both the 2000
and 2001 address attributes. The table consisted of the parcel
identification number and address attribute place holders (i.e.
prefix, street_number, street_name, etc.). The
Snohomish_master_address and Snohomish_buildable_address were then
used to update the Snohomish Buildable Lands table's address
fields.

(insert sql queries here)

0623089019

Land Value

\subsection{King County}

A 2001 King County Assessor's extract table (Extra_RPAcct.dbf) was
used to extract the land value characteristics of the parcel
record. The table includes a "Major", "Minor", and "ApprLandVal",
which will be used to update the King County flat parcels table.

Because the flat parcels table uses a single field (PARCEL_ID) as
a linking key to all other tables and attributes, the major and
minor fields in the Extra_RPAcct table needed to be concatenated
into one single field. Therefore, an intermediate table was
created that would contain a Parcel Identification Number
(combination of Major and Minor fields) and the corresponding land
values for the parcel records.

In addition to the concatenate operation that was performed on the
Major and Minor fields, a second SQL query needed to be performed
in order to capture the entire land value for the parcel records.
The Extra_RPAcct table includes multiple records for each Parcel
Identification Number (PIN), therefore, each record contained a
different land value. A SQL query was used to sum the land value
per each unique PIN. The result of the SQL operation created a
table containing a unique PIN and a total land value.

This intermediate PIN and Land Value table was then used to update
the flat parcels table via a SQL query (insert SQL query here?)


\subsection(Kitsap County}

The 2001 Kitsap County Assessor's file contains the Parcel
Identification Number and Land Value for each parcel record.
Before the "PIN" and "ASSDLAND" fields can be selected and
inserted into a new table, a SQL query was performed that removed
duplicate PINs that have identical Land Values and null PIN values
(insert SQL query?).

The resulting table was used to update the flat parcels table via
a SQL query (insert SQL query here?)

\subsection{Pierce County}

The 2003 County Assessor's file (Master_Record) and 2000 Buildable
Lands file was used to extracted the Land Value for each Parcel
Identification Number. The 2003 and 2000 files were exported to a
geodatabase, where the "parcel_num" and "land_value" fields from
the 2003 Assessor's file were selected and inserted into a new
table, "Pierce_assessor_land_value" (insert SQL query here?).

The "TAX_PARCEL" and "LAND_VALUE" were selected from the 2000
Buildable Lands file and inserted into a new table,
"Pierce_buildable_land_value" (insert SQL query here?).

The Pierce County's flat parcels table was updated through a set
of queries that assembled the land values from the
Pierce_assessor_land_value and Pierce_buildable_land_value.
Because two tables were being used to update a single field, it
was important to update the the land values containing the
Buildable Lands value last, to overwrite the Assessor's land
values where the Buildable Lands values applied (insert SQL
query?)

\subsection{Snohomish County}

The 2001 County Assessor's file (Master_Record) and 2000 Buildable
Lands file was used to assemble the Land Value for each Parcel
Identification Number. The 2001 and 2000 files were exported to a
geodatabase, where the "LRSN" and "AsdLanVal" selected and
inserted into a new table, "Snohomish_assessor_land_value" (insert
SQL query here?).

The "LRSN" and "MKT_LAND" fields were selected from the 2000
Buildable Lands file and inserted into a new table,
"Snohomish_buildable_land_value" (insert SQL query here?).

Using the two tables described above, perform an update query to
the existing flat parcels table for Snohomish County (insert SQL
query here?). Similarly to Pierce County, update the Buildable
Lands value last to avoid retaining the Assessor's values where
the Buildable Lands' value applies.


Improvement Value

\subsection{King County}

The source for King County's parcel Improvement Value data is from
the Assessor's Extract tables (EXTR_RPAcct.dfb). The table
includes a "Major", "Minor", and "ApprImpVal", which were selected
and inserted into a new table and will be used to update the King
County flat parcels table.

Because the flat parcels table uses a single field (PARCEL_ID) as
a linking key to all other tables and attributes, the major and
minor fields in the Extra_RPAcct table again needed to be
concatenated into one single field. Therefore, an intermediate
table was created that would contain a Parcel Identification
Number (combination of Major and Minor fields) and the
corresponding Improvement value for the parcel records.

This intermediate PIN and Improvement Value table was then used to
update the flat parcels table via a SQL query (insert SQL query
here?)

\subsection(Kitsap County}



\subsection{Pierce County}



\subsection{Snohomish County}



Built Square Feet


\subsection{King County}
\subsection(Kitsap County}
\subsection{Pierce County}
\subsection{Snohomish County}

Year Built


\subsection{King County}
\subsection(Kitsap County}
\subsection{Pierce County}
\subsection{Snohomish County}

Residential Units


\subsection{King County}
\subsection(Kitsap County}
\subsection{Pierce County}
\subsection{Snohomish County}

Imputed Residential Units


\subsection{King County}
\subsection(Kitsap County}
\subsection{Pierce County}
\subsection{Snohomish County}

Residential Units Weight Residential Units


\subsection{King County}
\subsection(Kitsap County}
\subsection{Pierce County}
\subsection{Snohomish County}

Weight Fraction

Imputed Residential Units


\subsection{King County}
\subsection(Kitsap County}
\subsection{Pierce County}
\subsection{Snohomish County}



Binary Tax Exempt Standardization

The county Assessor's records contained supplemental tables
identifying tax exempt parcels. The tax exempt coding system was
relatively consistent throughout all four counties with the
exception of Pierce County. Pierce County contained a unique code
system that required a look up table in order to determine whether
a parcel was tax exempt.

\subsection{King County}
\subsection(Kitsap County}
\subsection{Pierce County}
\subsection{Snohomish County}

Owner Name


\subsection{King County}
\subsection(Kitsap County}
\subsection{Pierce County}
\subsection{Snohomish County}

Nonprofit


\subsection{King County}
\subsection(Kitsap County}
\subsection{Pierce County}
\subsection{Snohomish County}

Undevelopable Parcels


\subsection{King County}
\subsection(Kitsap County}
\subsection{Pierce County}
\subsection{Snohomish County}

Census Block Association


\subsection{King County}
\subsection(Kitsap County}
\subsection{Pierce County}
\subsection{Snohomish County}

Park Undevelopable Parcels


\subsection{King County}
\subsection(Kitsap County}
\subsection{Pierce County}
\subsection{Snohomish County}

Lot Area

\subsection{King County}
\subsection(Kitsap County}
\subsection{Pierce County}
\subsection{Snohomish County}
