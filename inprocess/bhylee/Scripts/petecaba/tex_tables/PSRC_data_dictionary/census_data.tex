
\chapter{U.S. Census Data}
\section{1990 Regional Census Data}
\subsection{Census Block Groups}

{\bf \large Coverage Name:}\\
blgrps90

{\bf \large Coverage Type(s):}\\
Polygon

{\bf \large Documentation Revision:}\\
09-08-04

{\bf \large Status:}\\
Completed

{\bf \large Description}

A census block group (BG) is a cluster of census blocks having the
same first digit of their four-digit identifying numbers within a
census tract. (See also Census Tract.) For example, block group 3
(BG 3) within a census tract includes all blocks numbered from
3000 to 3999. BGs generally contain between 600 and 3,000 people,
with an optimum size of 1,500 people. Most BGs were delineated by
local participants as part of the U.S. Census Bureau's Participant
Statistical Areas Program. The U.S. Census Bureau delineated BGs
only where a local, state, or tribal government declined to
participate or where the U.S. Census Bureau could not identify a
potential local or tribal participant.

BGs never cross the boundaries of states, counties, or
statistically equivalent entities, except for a BG delineated by
American Indian tribal authorities, and then only when tabulated
within the American Indian hierarchy. (See also Tribal Block
Group.) BGs never cross the boundaries of census tracts, but may
cross the boundary of any other geographic entity required as a
census block boundary.

In decennial census data tabulations, a BG may be split for
statistical purposes for every unique combination of American
Indian area, Alaska Native area, Hawaiian home land, congressional
district, county subdivision, place, voting district, or other
tabulation entity. For example, if BG 3 is partly in a city and
partly outside the city, there are separate tabulated records for
each portion of BG 3. BGs are used in tabulating data nationwide,
as was done for the 1990 census, for all block-numbered areas in
the 1980 census, and for selected areas in the 1970 census. For
statistical purposes, BGs are a substitute for the enumeration
districts (EDs) used for reporting data in many parts of the
United States for the 1970 and 1980 censuses and in all areas
before 1970.


{\bf \large Procedures}

The data set was extracted from the 1990, TIGER/Line files. Each
county (King, Kitsap, Pierce, Snohomish) were extracted and using
ArcGis merge operation, a regional census block file was created.

{\bf \large Maintenance}

The data set originated from the Department of Commerce, Census
Bureau, Geography Division. Block Boundaries are released from the
Census Bureau every 10 years.

\begin{landscape}
\begin{longtable}{llrrrrrc}
Attribute Name & Description & Data Type & Width & Decimals &
Precision & Scale & Values unrepresentable domain \\ \hline

FID & Internal feature number & OID & 4 & - & - & Sequential unique whole numbers that are automatically generated.\\
SHAPE & Feature geometry & Geometry & - & - & - & - Coordinates defining features.\\
AREA & Area of feature in internal units & Float & 8 & 5 & - & - & -\\
PERIMETER & Perimeter of feature in internal units & Float & 8 & 5 & - & - & -\\
BLKGRPS90\# & Internal feature number & Binary & 4 & - & - & - & -\\
BLKGRPS90-ID & User-defined number & Binary & 4 & - & - & - & - \\
STATE & State ID & Character & 2 & - & - & - & - \\
COUNTY & County ID & Character & 3  & - & - & - & - \\
TRACT90 & Census tract ID & Float & 8 & - & - & - & - \\
GROUP90 & Census block group & Character & 1 & - & - & - & - \\
GROUP90L & State, county, tract, and block group ID & Character &
12 & - & - & - & - \\
COUNT & N\_A & Binary & 4 & - & - & - & - \\

\end{longtable}
\end{landscape}
\newpage

\subsection{Census Tract}

{\bf \large Coverage Name:}\\
tract90

{\bf \large Coverage Type(s):}\\
Polygon

{\bf \large Documentation Revision:}\\
09-08-04

{\bf \large Status:}\\
Completed

{\bf \large Description}

The coverage file contains only census tracts for King county

{\bf \large Procedures}

The data set was extracted from the 1990, TIGER/Line files.

{\bf \large Maintenance}

The data set originated from the Department of Commerce, Census
Bureau, Geography Division. Block Boundaries are released from the
Census Bureau every 10 years.

\begin{landscape}
\begin{longtable}{llrrrrrc}
Attribute Name & Description & Data Type & Width & Decimals &
Precision & Scale & Values unrepresentable domain \\ \hline

FID & Internal feature number & OID & 4 & - & - & Sequential unique whole numbers that are automatically generated.\\
SHAPE & Feature geometry & Geometry & - & - & - & - Coordinates defining features.\\
AREA & Area of feature in internal units & Float & 8 & 5 & - & - & -\\
PERIMETER & Perimeter of feature in internal units & Float & 8 & 5 & - & - & -\\
TRACT1990\# & Internal feature number & Binary & 4 & - & - & - & -\\
TRACT1990-ID & User-defined number & Binary & 4 & - & - & - & - \\
POLY\# & N\*A & Binary & 4 & - & - & - & - \\
TRACT & Census tract ID & Character & 12 & - & - & - & - \\

\end{longtable}
\end{landscape}
\newpage

\section{2000 Regional Census Data}
\subsection{Census Block}
{\bf \large Coverage Name:}\\
blk00\_4cnty

{\bf \large Coverage Type(s):}\\
Polygon

{\bf \large Documentation Revision:}\\
09-08-04

{\bf \large Status:}\\
Completed

{\bf \large Description}

The Census Blocks data set are the smallest geographic unit used
by the U.S. Census Bureau for reporting census data.

{\bf \large Procedures}

The data set was extracted from the 2000, TIGER/Line files. Each
county (King, Kitsap, Pierce, Snohomish) were extracted and using
ArcGis merge operation, a regional census block file was created.

{\bf \large Maintenance}

The data set originated from the Department of Commerce, Census
Bureau, Geography Division. Block Boundaries are released from the
Census Bureau every 10 years.

\begin{landscape}
\begin{longtable}{llrrrrrc}
Attribute Name & Description & Data Type & Width & Decimals &
Precision & Scale & Values unrepresentable domain \\ \hline

FID & Internal feature number & OID & 4 & - & - & Sequential unique whole numbers that are automatically generated.\\
SHAPE & Feature geometry & Geometry & - & - & - & - Coordinates defining features.\\
AREA & Area of feature in internal units & Float & 8 & 5 & - & - & -\\
PERIMETER & Perimeter of feature in internal units & Float & 8 & 5 & - & - & -\\
BLK00\_4CNTY\# & Internal feature number & Binary & 4 & - & - & - & -\\
BLK00\_4CNTY-ID & User-defined number & Binary & 4 & - & - & - & -\\
STFID & Concatenated state, county, tract, and block & Character & 15 & - & - & - & -\\

\end{longtable}
\end{landscape}
\newpage

\subsection{Census Block Group}
{\bf \large Coverage Name:}\\
blkgrps00

{\bf \large Coverage Type(s):}\\
Polygon

{\bf \large Documentation Revision:}\\
09-08-04

{\bf \large Status:}\\
Completed

{\bf \large Description} A census block group (BG) is a cluster of
census blocks having the same first digit of their four-digit
identifying numbers within a census tract. (See also Census
Tract.) For example, block group 3 (BG 3) within a census tract
includes all blocks numbered from 3000 to 3999. BGs generally
contain between 600 and 3,000 people, with an optimum size of
1,500 people. Most BGs were delineated by local participants as
part of the U.S. Census Bureau's Participant Statistical Areas
Program. The U.S. Census Bureau delineated BGs only where a local,
state, or tribal government declined to participate or where the
U.S. Census Bureau could not identify a potential local or tribal
participant.

BGs never cross the boundaries of states, counties, or
statistically equivalent entities, except for a BG delineated by
American Indian tribal authorities, and then only when tabulated
within the American Indian hierarchy. (See also Tribal Block
Group.) BGs never cross the boundaries of census tracts, but may
cross the boundary of any other geographic entity required as a
census block boundary.

In decennial census data tabulations, a BG may be split for
statistical purposes for every unique combination of American
Indian area, Alaska Native area, Hawaiian home land, congressional
district, county subdivision, place, voting district, or other
tabulation entity. For example, if BG 3 is partly in a city and
partly outside the city, there are separate tabulated records for
each portion of BG 3. BGs are used in tabulating data nationwide,
as was done for the 1990 census, for all block-numbered areas in
the 1980 census, and for selected areas in the 1970 census. For
statistical purposes, BGs are a substitute for the enumeration
districts (EDs) used for reporting data in many parts of the
United States for the 1970 and 1980 censuses and in all areas
before 1970.

{\bf \large Procedures}

The data set was extracted from the 2000, TIGER/Line files. Each
county (King, Kitsap, Pierce, Snohomish) were extracted and using
ArcGis' merge operation, a regional census block group file was
created.

{\bf \large Maintenance}

The data set originated from the Department of Commerce, Census
Bureau, Geography Division. Block Boundaries are released from the
Census Bureau every 10 years.

\begin{landscape}
\begin{longtable}{llrrrrrc}
Attribute Name & Description & Data Type & Width & Decimals &
Precision & Scale & Values unrepresentable domain \\ \hline

FID & Internal feature number & OID & 4 & - & - & Sequential unique whole numbers that are automatically generated.\\
SHAPE & Feature geometry & Geometry & - & - & - & - Coordinates defining features.\\
AREA & Area of feature in internal units & Float & 8 & 5 & - & - & -\\
PERIMETER & Perimeter of feature in internal units & Float & 8 & 5 & - & - & -\\
BLKGRPS00\# & Internal feature number & Binary & 4 & - & - & - & -\\
BLKGRPS00-ID & User-defined number & Binary & 4 & - & - & - & -\\
TRACT & Census Tract ID & Character & 20 & - & - & - & - \\
FIPSSTCO & State and County ID & 7 & - & - & - & - & - \\
STFID & Concatenated state, county, tract, and block & Character & 14 & - & - & - & -\\
GROUP & Census Block Group ID & Character & 2 & - & - & - & - \\
SLIVER & N\_A & Binary & 2 & - & - & - & - \\

\end{longtable}
\end{landscape}
\newpage

\subsection{Census Block FAZ Group}
{\bf \large Coverage Name:}\\
blk00\_fazgp

{\bf \large Coverage Type(s):}\\
Polygon

{\bf \large Documentation Revision:}\\
09-08-04

{\bf \large Status:}\\
Completed

{\bf \large Description}

Coverage file containing 2000 Census blocks associated with 2000
FAZ groups.

{\bf \large Procedures}

The data set was extracted from the 2000, TIGER/Line files. Each
county (King, Kitsap, Pierce, Snohomish) were extracted and using
ArcGis merge operation, a regional census block file was created.
An ArcGis overlay operation was then performed on the file with
2000 FAZ coverage.

{\bf \large Maintenance}

The data set originated from the Department of Commerce, Census
Bureau, Geography Division. Block Boundaries are released from the
Census Bureau every 10 years.

\begin{landscape}
\begin{longtable}{llrrrrrc}
Attribute Name & Description & Data Type & Width & Decimals &
Precision & Scale & Values unrepresentable domain \\ \hline

FID & Internal feature number & OID & 4 & - & - & Sequential unique whole numbers that are automatically generated.\\
SHAPE & Feature geometry & Geometry & - & - & - & - Coordinates defining features.\\
AREA & Area of feature in internal units & Float & 8 & 5 & - & - & -\\
PERIMETER & Perimeter of feature in internal units & Float & 8 & 5 & - & - & -\\
BLK00\_FAZGP\# & Internal feature number & Binary & 4 & - & - & - & -\\
BLK00\_FAZGP-ID & User-defined number & Binary & 4 & - & - & - & - \\
FAZ\_GROUP\# & Internal feature number & Binary & 4 & - & - & - & - \\
FAZ\_GROUP-ID & User-defined number & Binary & 4 & - & - & - & - \\
OBJECTID & N\_A & Binary & 4 & - & - & - & - \\
FAZ2000\_ & FAZ2000 coverage file internal id & Binary & 4 & - & - & - & - \\
FAZ2000\_ID & FAZ2000 User-defined number & Binary & 4 & - & - & - & - \\
FAZ & FAZ ID & Binary & 4 & - & - & - & - \\
COUNT\_ & N\_A & Float & 8 & - & - & - & - \\
FIRST\_COUN & State and County ID & Character & 8 & - & - & - & - \\
SUM\_AREA & Sum of area & Float & 8 & - & - & - & - \\
SUM\_ACRES & Sum of acres & Float & 8 & - & - & - & - \\
FAZ\_GROUP & FAZ Group ID & Binary & 4 & - & - & - & - \\
SHAPE\_LENG & N\_A & Float & 8 & - & - & - & - \\
SHAPE\_AREA & N\_A & Float & 8 & - & - & - & - \\
BLOCK2000\# & Census block internal feature number & Binary & 4 & - & - & - & - \\
BLOCK2000-ID & User-defined number & Binary & 4 & - & - & - & - \\
ID & User-defined number & Binary & 4 & - & - & - & - \\
FIPSSTCO & State and County ID & Character & 5 & - & - & - & - \\
TRACT2000 & Census tract ID & Character & 6 & - & - & - & - \\
BLOCK2000 * Census block ID & Character & 4 & - & - & - & - \\
STFID & State, county, tract, and block ID & Character & 15 & - & - & - & - \\
SOURCETHM & Source of dbf & Character & 16 & - & - & - & - \\
ACRES & Total acres &  Float & 8 & - & - & - & - \\
POPDEN00 & Population density & Float & 8 & - & - & - & - \\
X\_COORD & X coordinate & Float & 8 & - & - & - & - \\
Y\_COORD & Y coordinate & Float & 8 & - & - & - & - \\
TRCT1990C & N\_A & Float & 8 & - & - & - & - \\
FAZ91 & 1991 FAZ ID & Binary & 4 & - & - & - & - \\
FAZ00 & 2000 FAZ ID & Binary & 4 & - & - & - & - \\
POP00CEN & Population & Binary & 4 & - & - & - & - \\

\end{longtable}
\end{landscape}
\newpage

\section{2000 County Census Data}
\subsection{King County Census Blocks}

{\bf \large Coverage Name:}\\
blkkin00

{\bf \large Coverage Type(s):}\\
Polygon

{\bf \large Documentation Revision:}\\
09-08-04

{\bf \large Status:}\\
Completed

{\bf \large Description}

The King County Census Blocks data set are the smallest geographic
unit used by the U.S. Census Bureau for reporting census data.

{\bf \large Procedures}

The data set was extracted from the 2000, TIGER/Line files.

{\bf \large Maintenance}

The data set originated from the Department of Commerce, Census
Bureau, Geography Division. Block Boundaries are released from the
Census Bureau every 10 years.

King County's GIS Division has conflated the blocks.

\begin{landscape}
\begin{longtable}{llrrrrrc}
Attribute Name & Description & Data Type & Width & Decimals &
Precision & Scale & Values unrepresentable domain \\ \hline

FID & Internal feature number & OID & 4 & - & - & Sequential unique whole numbers that are automatically generated.\\
SHAPE & Feature geometry & Geometry & - & - & - & - Coordinates defining features.\\
AREA & Area of feature in internal units & Float & 8 & 5 & - & - & -\\
PERIMETER & Perimeter of feature in internal units & Float & 8 & 5 & - & - & -\\
BLKKIN00\# & Internal feature number & Binary & 4 & - & - & - & -\\
BLKKIN00-ID & User-defined number & Binary & 4 & - & - & - & -\\
CENSUS\_ & Internal feature number & Binary & 4 & - & - & - & -\\
CENSUS\_ID & Internal feature number & Binary & 4 & - & - & - & -\\
STATE & State Code ID & Character & 2 & - & - & - & -\\
COUNTY & State County Code ID & Character & 3 & - & - & - & -\\
TRACT & Census Tract ID & Character & 6 & - & - & - & -\\
BLOCK & Census Block ID & Character & 4 & - & - & - & -\\
NAME & Census Block Name & Character & 1 & - & - & - & -\\
BTFID & 15 Digit Census Block ID number & Character & 15 & - & - & - & -\\
LOGRECNO & Census data files unique key field & Character & 7 & - & - & - & -\\
TOTAL\_POP & Total population per census block & Binary & 4 & - & - & - & -\\
HOUSEHOLDS & Total households per census block & Binary & 4 & - & - & - & -\\
HOUSEUNITS & Total housing units per census block & Binary & 4 & - & - & - & -\\
TAZ & Traffic Analysis Zone ID & Character & 6 & - & - & - & -\\
GTFID & Concatenated state, county, tract, and block & Character & 12 & - & - & - & -\\
STFID & Concatenated state, county, tract, and block & Character & 11 & - & - & - & -\\
TRCTBG & Concatenated tract and block group & Character & 5 & - & - & - & -\\
ACRES & Acres of feature in internal units & Binary & 2 & - & - & - & -\\

\end{longtable}
\end{landscape}
\newpage

%\begin {landscape}
%\begin{tabular}{lrrrrrrc}
%Attribute Name & Description & Data Type & Width & Decimals &
Precision & Scale & Values unrepresentable domain \\ \hline

FID & Internal feature number & OID & 4 & - & - & Sequential unique whole numbers that are automatically generated.\\
SHAPE & Feature geometry & Geometry & - & - & - & - Coordinates defining features.\\
AREA & Area of feature in internal units & Float & 8 & 5 & - & - & -\\
PERIMETER & Perimeter of feature in internal units & Float & 8 & 5 & - & - & -\\
BLKKIN00\# & Internal feature number & Binary & 4 & - & - & - & -\\
BLKKIN00-ID & User-defined number & Binary & 4 & - & - & - & -\\
CENSUS\_ & Internal feature number & Binary & 4 & - & - & - & -\\
CENSUS\_ID & Internal feature number & Binary & 4 & - & - & - & -\\
STATE & State Code ID & Character & 2 & - & - & - & -\\
COUNTY & State County Code ID & Character & 3 & - & - & - & -\\
TRACT & Census Tract ID & Character & 6 & - & - & - & -\\
BLOCK & Census Block ID & Character & 4 & - & - & - & -\\
NAME & Census Block Name & Character & 1 & - & - & - & -\\
BTFID & 15 Digit Census Block ID number & Character & 15 & - & - & - & -\\
LOGRECNO & Census data files unique key field & Character & 7 & - & - & - & -\\
TOTAL\_POP & Total population per census block & Binary & 4 & - & - & - & -\\
HOUSEHOLDS & Total households per census block & Binary & 4 & - & - & - & -\\
HOUSEUNITS & Total housing units per census block & Binary & 4 & - & - & - & -\\
TAZ & Traffic Analysis Zone ID & Character & 6 & - & - & - & -\\
GTFID & Concatenated state, county, tract, and block & Character & 12 & - & - & - & -\\
STFID & Concatenated state, county, tract, and block & Character & 11 & - & - & - & -\\
TRCTBG & Concatenated tract and block group & Character & 5 & - & - & - & -\\
ACRES & Acres of feature in internal units & Binary & 2 & - & - & - & -\\

%\end{tabular}
%\end{landscape}
%\newpage

%\begin{table}[h]
%\centering \vspace{3mm}
%\begin{tabular}{llrrrrrp{2in}}
%Attribute Name & Description & Data Type & Width & Decimals &
Precision & Scale & Values unrepresentable domain \\ \hline

FID & Internal feature number & OID & 4 & - & - & Sequential unique whole numbers that are automatically generated.\\
SHAPE & Feature geometry & Geometry & - & - & - & - Coordinates defining features.\\
AREA & Area of feature in internal units & Float & 8 & 5 & - & - & -\\
PERIMETER & Perimeter of feature in internal units & Float & 8 & 5 & - & - & -\\
BLKKIN00\# & Internal feature number & Binary & 4 & - & - & - & -\\
BLKKIN00-ID & User-defined number & Binary & 4 & - & - & - & -\\
CENSUS\_ & Internal feature number & Binary & 4 & - & - & - & -\\
CENSUS\_ID & Internal feature number & Binary & 4 & - & - & - & -\\
STATE & State Code ID & Character & 2 & - & - & - & -\\
COUNTY & State County Code ID & Character & 3 & - & - & - & -\\
TRACT & Census Tract ID & Character & 6 & - & - & - & -\\
BLOCK & Census Block ID & Character & 4 & - & - & - & -\\
NAME & Census Block Name & Character & 1 & - & - & - & -\\
BTFID & 15 Digit Census Block ID number & Character & 15 & - & - & - & -\\
LOGRECNO & Census data files unique key field & Character & 7 & - & - & - & -\\
TOTAL\_POP & Total population per census block & Binary & 4 & - & - & - & -\\
HOUSEHOLDS & Total households per census block & Binary & 4 & - & - & - & -\\
HOUSEUNITS & Total housing units per census block & Binary & 4 & - & - & - & -\\
TAZ & Traffic Analysis Zone ID & Character & 6 & - & - & - & -\\
GTFID & Concatenated state, county, tract, and block & Character & 12 & - & - & - & -\\
STFID & Concatenated state, county, tract, and block & Character & 11 & - & - & - & -\\
TRCTBG & Concatenated tract and block group & Character & 5 & - & - & - & -\\
ACRES & Acres of feature in internal units & Binary & 2 & - & - & - & -\\

%\end{tabular}
%\end{table}
%\newpage

\subsection{King County Census Block Groups}
{\bf \large Coverage Name:}\\
blkgrpkin00

{\bf \large Coverage Type(s):}\\
Polygon

{\bf \large Documentation Revision:}\\
09-08-04

{\bf \large Status:}\\
Completed

{\bf \large Description}

A census block group (BG) is a cluster of census blocks having the
same first digit of their four-digit identifying numbers within a
census tract. (See also Census Tract.) For example, block group 3
(BG 3) within a census tract includes all blocks numbered from
3000 to 3999. BGs generally contain between 600 and 3,000 people,
with an optimum size of 1,500 people. Most BGs were delineated by
local participants as part of the U.S. Census Bureau's Participant
Statistical Areas Program. The U.S. Census Bureau delineated BGs
only where a local, state, or tribal government declined to
participate or where the U.S. Census Bureau could not identify a
potential local or tribal participant.

BGs never cross the boundaries of states, counties, or
statistically equivalent entities, except for a BG delineated by
American Indian tribal authorities, and then only when tabulated
within the American Indian hierarchy. (See also Tribal Block
Group.) BGs never cross the boundaries of census tracts, but may
cross the boundary of any other geographic entity required as a
census block boundary.

In decennial census data tabulations, a BG may be split for
statistical purposes for every unique combination of American
Indian area, Alaska Native area, Hawaiian home land, congressional
district, county subdivision, place, voting district, or other
tabulation entity. For example, if BG 3 is partly in a city and
partly outside the city, there are separate tabulated records for
each portion of BG 3. BGs are used in tabulating data nationwide,
as was done for the 1990 census, for all block-numbered areas in
the 1980 census, and for selected areas in the 1970 census. For
statistical purposes, BGs are a substitute for the enumeration
districts (EDs) used for reporting data in many parts of the
United States for the 1970 and 1980 censuses and in all areas
before 1970.

{\bf \large Procedures}

The data set was extracted from the 2000, TIGER/Line files.

{\bf \large Maintenance}

The data set originated from the Department of Commerce, Census
Bureau, Geography Division. Tract boundaries are released from the
Census Bureau every 10 years.

\begin{landscape}
\begin{longtable}{llrrrrrc}
\input{tables/king_census_block_group00}
\end{longtable}
\end{landscape}
\newpage

\subsection{Kitsap County Census Blocks}

{\bf \large Coverage Name:}\\
blkkit00

{\bf \large Coverage Type(s):}\\
Polygon

{\bf \large Documentation Revision:}\\
09-08-04

{\bf \large Status:}\\
Completed

{\bf \large Description}

The Kitsap County Census Blocks data set are the smallest
geographic unit used by the U.S. Census Bureau for reporting
census data.

{\bf \large Procedures}

The data set was extracted from the 2000, TIGER/Line files.

{\bf \large Maintenance}

The data set originated from the Department of Commerce, Census
Bureau, Geography Division. Block Boundaries are released from the
Census Bureau every 10 years.

\begin{landscape}
\begin{longtable}{llrrrrrc}
\input{tables/kitsap_census_block00}
\end{longtable}
\end{landscape}
\newpage

\subsection{Kitsap County Census Blocks (Conflated)}

{\bf \large Coverage Name:}\\
blkkit\_conf

{\bf \large Coverage Type(s):}\\
Polygon

{\bf \large Documentation Revision:}\\
09-08-04

{\bf \large Status:}\\
Completed

{\bf \large Description}

The Kitsap County Conflated Census Blocks represents spatially
adjusted blocks that align to the transportation network provided
from the Puget Sound Regional Council. It is only used during the
job allocation algorithm process.

{\bf \large Procedures}

The data set was extracted from the 2000, TIGER/Line files.

{\bf \large Maintenance}

The data set originated from the Department of Commerce, Census
Bureau, Geography Division. Block Boundaries are released from the
Census Bureau every 10 years.

\begin{landscape}
\begin{longtable}{llrrrrrc}
Attribute Name & Description & Data Type & Width & Decimals &
Precision & Scale & Values unrepresentable domain \\ \hline

FID & Internal feature number & OID & 4 & - & - & - & Sequential unique whole numbers that are automatically generated \\
SHAPE & Feature geometry & Geometry & - & - & - & - & Coordinates defining the features. \\
AREA & Area of feature in internal units & Float & 8 & 5 & - & - & - \\
PERIMETER & Perimeter of feature in internal units & Float & 8 & 5 & - & - &  - \\
BLKKIT\_CONF\# & Internal feature number & Binary & 4 & - & - & - & - \\
BLKKIT\_CONF-ID & User-defined feature number & Binary & 4 & - & - & - & - \\
STFID & State, county, tract code, block & Character & 15 & - & - & - & -  \\

\end{longtable}
\end{landscape}
\newpage

\subsection{Kitsap County Census Block Groups}
{\bf \large Coverage Name:}\\
blkgrpkit00

{\bf \large Coverage Type(s):}\\
Polygon

{\bf \large Documentation Revision:}\\
09-08-04

{\bf \large Status:}\\
Completed

{\bf \large Description}

A census block group (BG) is a cluster of census blocks having the
same first digit of their four-digit identifying numbers within a
census tract. (See also Census Tract.) For example, block group 3
(BG 3) within a census tract includes all blocks numbered from
3000 to 3999. BGs generally contain between 600 and 3,000 people,
with an optimum size of 1,500 people. Most BGs were delineated by
local participants as part of the U.S. Census Bureau's Participant
Statistical Areas Program. The U.S. Census Bureau delineated BGs
only where a local, state, or tribal government declined to
participate or where the U.S. Census Bureau could not identify a
potential local or tribal participant.

BGs never cross the boundaries of states, counties, or
statistically equivalent entities, except for a BG delineated by
American Indian tribal authorities, and then only when tabulated
within the American Indian hierarchy. (See also Tribal Block
Group.) BGs never cross the boundaries of census tracts, but may
cross the boundary of any other geographic entity required as a
census block boundary.

In decennial census data tabulations, a BG may be split for
statistical purposes for every unique combination of American
Indian area, Alaska Native area, Hawaiian home land, congressional
district, county subdivision, place, voting district, or other
tabulation entity. For example, if BG 3 is partly in a city and
partly outside the city, there are separate tabulated records for
each portion of BG 3. BGs are used in tabulating data nationwide,
as was done for the 1990 census, for all block-numbered areas in
the 1980 census, and for selected areas in the 1970 census. For
statistical purposes, BGs are a substitute for the enumeration
districts (EDs) used for reporting data in many parts of the
United States for the 1970 and 1980 censuses and in all areas
before 1970.

{\bf \large Procedures}

The data set was extracted from the 2000, TIGER/Line files.

{\bf \large Maintenance}

The data set originated from the Department of Commerce, Census
Bureau, Geography Division. Tract boundaries are released from the
Census Bureau every 10 years.

\begin{landscape}
\begin{longtable}{llrrrrrc}
\input{tables/kitsap_census_block_group00}
\end{longtable}
\end{landscape}
\newpage

\subsection{Pierce County Census Blocks}

{\bf \large Coverage Name:}\\
blkpie00

{\bf \large Coverage Type(s):}\\
Polygon

{\bf \large Documentation Revision:}\\
09-08-04

{\bf \large Status:}\\
Completed

{\bf \large Description}

The Pierce County Census Blocks data set are the smallest
geographic unit used by the U.S. Census Bureau for reporting
census data.

{\bf \large Procedures}

The data set was extracted from the 2000, TIGER/Line files.

{\bf \large Maintenance}

The data set originated from the Department of Commerce, Census
Bureau, Geography Division. Block Boundaries are released from the
Census Bureau every 10 years.

\begin{landscape}
\begin{longtable}{llrrrrrc}
Attribute Name & Description & Data Type & Width & Decimals &
Precision & Scale & Values unrepresentable domain \\ \hline

FID & Internal feature number & OID & 4 & - & - & -\\
SHAPE & Feature geometry & Geometry & - & - & - & -& --\\
AREA & Area of feature in internal units & Float & 8 & 5 & - & - & -\\
PERIMETER & Perimeter of feature in internal units & Float & 8 & 5 & - & - & -\\
BLKPIE00\# & Internal feature number & Binary & 4 & - & - & - & -\\
BLKPIE00-ID & User-defined number & Binary & 4 & - & - & - & -\\
ID & User-defined number & Binary & 4 & - & - & - & - \\
STATE & State Code ID & Character & 2 & - & - & - & -\\
FIPSSTCO & State, tract, county code & Character & 5 & - & - & - &-\\
TRACT2000 & Census tract ID & Character & 6 & - & - & - & -\\
BLOCK2000 & Census block ID & Character & 4 & - & - & - & -\\
STFID & Concatenated state, county, tract, and block & Character & 11 & - & - & - & -\\

\end{longtable}
\end{landscape}
\newpage

\subsection{Pierce County Census Blocks (Conflated)}

{\bf \large Coverage Name:}\\
block2000

{\bf \large Coverage Type(s):}\\
Polygon

{\bf \large Documentation Revision:}\\
09-08-04

{\bf \large Status:}\\
Completed

{\bf \large Description}

The Pierce County Census Blocks were reconfigured by the Pierce
County Assessor's Office to align with the transportation network.

{\bf \large Procedures}

The data set was extracted from the 2000, TIGER/Line files.

{\bf \large Maintenance}

The data set originated from the Department of Commerce, Census
Bureau, Geography Division. Block Boundaries are released from the
Census Bureau every 10 years.

\begin{landscape}
\begin{longtable}{llrrrrrc}
Attribute Name & Description & Data Type & Width & Decimals &
Precision & Scale & Values unrepresentable domain \\ \hline

FID & Internal feature number & OID & 4 & - & - & -\\
SHAPE & Feature geometry & Geometry & - & - & - & -& --\\
AREA & Area of feature in internal units & Float & 8 & 5 & - & - & -\\
PERIMETER & Perimeter of feature in internal units & Float & 8 & 5 & - & - & -\\
BLKPIE00\# & Internal feature number & Binary & 4 & - & - & - & -\\
BLKPIE00-ID & User-defined number & Binary & 4 & - & - & - & -\\
ID & User-defined number & Binary & 4 & - & - & - & - \\
STATE & State Code ID & Character & 2 & - & - & - & -\\
FIPSSTCO & State, tract, county code & Character & 5 & - & - & - &-\\
TRACT2000 & Census tract ID & Character & 6 & - & - & - & -\\
BLOCK2000 & Census block ID & Character & 4 & - & - & - & -\\
STFID & Concatenated state, county, tract, and block & Character & 11 & - & - & - & -\\

\end{longtable}
\end{landscape}
\newpage

\subsection{Pierce County Census Block Groups}
{\bf \large Coverage Name:}\\
blkgrppie00

{\bf \large Coverage Type(s):}\\
Polygon

{\bf \large Documentation Revision:}\\
09-08-04

{\bf \large Status:}\\
Completed

{\bf \large Description}

A census block group (BG) is a cluster of census blocks having the
same first digit of their four-digit identifying numbers within a
census tract. (See also Census Tract.) For example, block group 3
(BG 3) within a census tract includes all blocks numbered from
3000 to 3999. BGs generally contain between 600 and 3,000 people,
with an optimum size of 1,500 people. Most BGs were delineated by
local participants as part of the U.S. Census Bureau's Participant
Statistical Areas Program. The U.S. Census Bureau delineated BGs
only where a local, state, or tribal government declined to
participate or where the U.S. Census Bureau could not identify a
potential local or tribal participant.

BGs never cross the boundaries of states, counties, or
statistically equivalent entities, except for a BG delineated by
American Indian tribal authorities, and then only when tabulated
within the American Indian hierarchy. (See also Tribal Block
Group.) BGs never cross the boundaries of census tracts, but may
cross the boundary of any other geographic entity required as a
census block boundary.

In decennial census data tabulations, a BG may be split for
statistical purposes for every unique combination of American
Indian area, Alaska Native area, Hawaiian home land, congressional
district, county subdivision, place, voting district, or other
tabulation entity. For example, if BG 3 is partly in a city and
partly outside the city, there are separate tabulated records for
each portion of BG 3. BGs are used in tabulating data nationwide,
as was done for the 1990 census, for all block-numbered areas in
the 1980 census, and for selected areas in the 1970 census. For
statistical purposes, BGs are a substitute for the enumeration
districts (EDs) used for reporting data in many parts of the
United States for the 1970 and 1980 censuses and in all areas
before 1970.

{\bf \large Procedures}

The data set was extracted from the 2000, TIGER/Line files.

{\bf \large Maintenance}

The data set originated from the Department of Commerce, Census
Bureau, Geography Division. Tract boundaries are released from the
Census Bureau every 10 years.

\begin{landscape}
\begin{longtable}{llrrrrrc}
Attribute Name & Description & Data Type & Width & Decimals &
Precision & Scale & Values unrepresentable domain \\ \hline

FID & Internal feature number & OID & 4 & - & - & - & Sequential unique whole numbers that are automatically generated \\
SHAPE & Feature geometry & Geometry &  &  &  &  & Coordinates defining the features. \\
AREA & Area of feature in internal units & Float & 8 & 5 & - & - &  \\
PERIMETER & Perimeter of feature in internal units & Float & 8 & 5 & - & - &  \\
BLGRPPIE00\# & Internal feature number & Binary & 4 & - & - & - &  \\
BLGRPPIE00-ID & User-defined feature number & Binary & 4 & - & - & - &  \\
TRACT & Census Tract ID & Character & 20 & - & - & - &  \\
FIPSSTCO & FIPS state and county code & Character & 7 & - & - & - &  \\
STFID & State, county, tract code & Character & 14 & - & - & - &  \\
GROUP & Census block group & Character & 2 & - & - & - &  \\
SLIVER & n/a  & Binary & 2 & - & - & - &  \\

\end{longtable}
\end{landscape}
\newpage

\subsection{Snohomish County Census Blocks}

{\bf \large Coverage Name:}\\
blksno00

{\bf \large Coverage Type(s):}\\
Polygon

{\bf \large Documentation Revision:}\\
09-08-04

{\bf \large Status:}\\
Completed

{\bf \large Description}

The Snohomish County Census Blocks data set are the smallest
geographic unit used by the U.S. Census Bureau for reporting
census data.

{\bf \large Procedures}

The data set was extracted from the 2000, TIGER/Line files.

{\bf \large Maintenance}

The data set originated from the Department of Commerce, Census
Bureau, Geography Division. Block Boundaries are released from the
Census Bureau every 10 years.

\begin{landscape}
\begin{longtable}{llrrrrrc}
Attribute Name & Description & Data Type & Width & Decimals &
Precision & Scale & Values unrepresentable domain \\ \hline

FID & Internal feature number & OID & 4 & - & - & -\\
SHAPE & Feature geometry & Geometry & - & - & - & -& --\\
AREA & Area of feature in internal units & Float & 8 & 5 & - & - & -\\
PERIMETER & Perimeter of feature in internal units & Float & 8 & 5 & - & - & -\\
BLKSNO00\# & Internal feature number & Binary & 4 & - & - & - & -\\
BLKSNO00-ID & User-defined number & Binary & 4 & - & - & - & -\\
ID & User-defined number & Binary & 4 & - & - & - & - \\
FIPSSTCO & State, tract, county code & Character & 5 & - & - & - &-\\
TRACT2000 & Census tract ID & Character & 6 & - & - & - & -\\
BLOCK2000 & Census block ID & Character & 4 & - & - & - & -\\
STFID & Concatenated state, county, tract, and block & Character & 11 & - & - & - & -\\
ACRES & Acres of feature in internal units & Binary & 2 & - & - & - & -\\

\end{longtable}
\end{landscape}
\newpage

\subsection{Snohomish County Census Blocks}

{\bf \large Coverage Name:}\\
blksno\_conf

{\bf \large Coverage Type(s):}\\
Polygon

{\bf \large Documentation Revision:}\\
09-08-04

{\bf \large Status:}\\
Completed

{\bf \large Description}

The Snohomish County Conflated Census Blocks represents spatially
adjusted blocks that align to the transportation network provided
from the Puget Sound Regional Council. It is only used during the
job allocation algorithm process.

{\bf \large Procedures}

The data set was extracted from the 2000, TIGER/Line files.

{\bf \large Maintenance}

The data set originated from the Department of Commerce, Census
Bureau, Geography Division. Block Boundaries are released from the
Census Bureau every 10 years.

\begin{landscape}
\begin{longtable}{llrrrrrc}
\input{tables/snoh_cb00_conf}
\end{longtable}
\end{landscape}
\newpage

\subsection{Snohomish County Census Block Groups}
{\bf \large Coverage Name:}\\
blkgrpsno00

{\bf \large Coverage Type(s):}\\
Polygon

{\bf \large Documentation Revision:}\\
09-08-04

{\bf \large Status:}\\
Completed

{\bf \large Description}

A census block group (BG) is a cluster of census blocks having the
same first digit of their four-digit identifying numbers within a
census tract. (See also Census Tract.) For example, block group 3
(BG 3) within a census tract includes all blocks numbered from
3000 to 3999. BGs generally contain between 600 and 3,000 people,
with an optimum size of 1,500 people. Most BGs were delineated by
local participants as part of the U.S. Census Bureau's Participant
Statistical Areas Program. The U.S. Census Bureau delineated BGs
only where a local, state, or tribal government declined to
participate or where the U.S. Census Bureau could not identify a
potential local or tribal participant.

BGs never cross the boundaries of states, counties, or
statistically equivalent entities, except for a BG delineated by
American Indian tribal authorities, and then only when tabulated
within the American Indian hierarchy. (See also Tribal Block
Group.) BGs never cross the boundaries of census tracts, but may
cross the boundary of any other geographic entity required as a
census block boundary.

In decennial census data tabulations, a BG may be split for
statistical purposes for every unique combination of American
Indian area, Alaska Native area, Hawaiian home land, congressional
district, county subdivision, place, voting district, or other
tabulation entity. For example, if BG 3 is partly in a city and
partly outside the city, there are separate tabulated records for
each portion of BG 3. BGs are used in tabulating data nationwide,
as was done for the 1990 census, for all block-numbered areas in
the 1980 census, and for selected areas in the 1970 census. For
statistical purposes, BGs are a substitute for the enumeration
districts (EDs) used for reporting data in many parts of the
United States for the 1970 and 1980 censuses and in all areas
before 1970.

{\bf \large Procedures}

The data set was extracted from the 2000, TIGER/Line files.

{\bf \large Maintenance}

The data set originated from the Department of Commerce, Census
Bureau, Geography Division. Tract boundaries are released from the
Census Bureau every 10 years.

\begin{landscape}
\begin{longtable}{llrrrrrc}
\input{tables/snoh_census_block_group00}
\end{longtable}
\end{landscape}
\newpage
