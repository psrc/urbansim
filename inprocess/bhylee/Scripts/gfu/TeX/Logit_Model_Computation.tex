\documentclass[12pt]{article}

\usepackage[sumlimits]{amsmath}
\usepackage{txfonts}
\usepackage[T1]{fontenc}
\usepackage[latin1]{inputenc}
\usepackage{hyperref}
\usepackage{alg}

\voffset=-1in
\hoffset=-1in
%
% Set for US Letter
\textheight=9.0in
\textwidth=6.5in
% need to also set dvips config and pdflatex config to US letter
%
%
\topmargin=0.5in
\oddsidemargin=1in
\evensidemargin=1in
%
\frenchspacing
%
%
% xxxxxxxxx  nnnnnnnnnnnnnnnn
%       nnnnnnnnnnnnnnnnnnnnn
\newenvironment{longitem}
	{\begin{list}{$\bullet$}
				{\labelwidth=2.6cm\leftmargin=2.9cm\labelsep=0.3cm
				\itemsep=0cm}}
	{\end{list}}
%
% Vector command (sets vectors in bold roman, not italic)
\newcommand{\vc}[1]{\boldsymbol{\mathrm{#1}}}
%
% Shortcuts
\newcommand{\e}{\mathrm{e}}
%
\itemsep=0in
%\href{mailto:stuff@things}{link_text}
%
% Change some things for the algtab environment
\renewcommand{\algforto}[2]{\textbf{for } #1 \textbf{ to } #2\\\algbegin}
\newcommand{\algendfor}{\algend \textbf{end for}\\}
\newcommand{\algendif}{\algend \textbf{end if}\\}
\newcommand{\algendwhile}{\algend \textbf{end while}\\}
\newcommand{\alglet}{\mbox{\textbf{let }}}
\newcommand{\algtry}[1]{\textbf{try } #1}
%
%
\begin{document}

%
%
%
\section*{Logit Models}

The discrete choice models in UrbanSim are simulated with logit models.
In such models, the utility of alternative $i$ for decision maker $n$ is
given by:
\begin{equation}
U_{ni} = u_{ni} + \epsilon_{ni},
\end{equation}
where the systematic part of the utility is given by the
linear-in-parameters function
\begin{equation}
u_{ni} = \sum_{k=1}^{K} \beta_{ik}x_{nik},
\end{equation}
and the random part is assumed to be independently and identically
distributed with the type 1 extreme
value distribution (Gumbel), $\epsilon_{ni}\sim G(0,1)$. For brevity we
use vector notation, and write
\begin{equation}
u_{ni} = \vc{X}_{ni}\vc{\beta}_{i}.
\end{equation}
To summarize:
\begin{longitem}
\item[$n$] is the number of the decision maker, $n=1,\ldots,N$,
\item[$i$] is the number of the alternatives, $i=1,\ldots,I$,
\item[$\vc{X}_{ni}$] is a $1\times K$ row vector of $K$ observed variables,
\item[$x_{nik}$] is the $k$-th variable from $X_{ni}$,
\item[$\vc{\beta}_{i}$] is a $K\times 1$ column vector of user estimated
coefficients,
\item[$\beta_{ik}$] is the $k$-th coefficient from $\beta_{i}$.
\end{longitem}


The logit model probability of decision maker $n$ choosing alternative $i$ out of a
set of $I$ mutually exclusive alternatives is then
\begin{equation}
P_{ni} = \frac{\e^{\vc{X}_{ni}\vc{\beta}_{i}}}
{\sum_{i'=1}^{I}\e^{\vc{X}_{ni'}\vc{\beta}_{i'}} }.
\label{eq:logitprob}
\end{equation}

\subsection*{Practical Aspects of Logit Probability Computation}

The logit probability can be calculated directly from
(\ref{eq:logitprob}). However, that requires raising $\e$ to possibly
large positive or negative powers, and summing up a possibly large
denominator. This can result in over- or underflow in the computation.

Overflow (underflow) occurs when the floating point value becomes
larger (smaller) in absolute value than the computer memory allocated to the
number can store. This happens at roughly $\pm\e^{709}$ for large 
positive or negative numbers
(the largest number that can be stored in a Java \texttt{double}
is $1.7976931348623157\cdot 10^{308}$), 
and at roughly $\pm\e^{-708}$ for small positive or negative numbers.

The logit computation can exceed these boundaries in two ways. 
\begin{enumerate}
\item Individual utilities can go out of bounds when exponentiated:\\
	$\mathrm{abs}(\vc{X}_{ni}\vc{\beta}_{i})>708$ for any $n$ and $i$,
\item The denominator can go out of bounds:\\
	$\sum_{i'=1}^{I}\e^{\vc{X}_{ni'}\vc{\beta}_{i'}}>\e^{709}$
\end{enumerate}

When the logit computation exceeds the boundaries it is generally an
indication that the variable value being used is outside the range that
was available in the data for that variable during estimation. To
analyze such occurrences it is best to activate a logit model
logging feature that prints to the screen any
large individual utility terms, i.e. any
term $\mathrm{abs}(\beta_{ik}x_{nik}) > 100$. 
The printout contains the decision maker index (e.g. household id, grid id),
the alternative index (e.g. the grid id, development type transition),
the value of the coefficient ($\beta_{ik}$), the value of the variable ($x_{nik}$),
and the value of the term ($\beta_{ik}x_{nik}$).

We can avoid calculating the denominator because the ratio of the logit
probabilities for any two alternatives is independent of the denominator for a 
particular decision maker:
\begin{equation}
	\frac{P_{ni}}{P_{nj}} = \frac{\e^{u_{ni}}}{\e^{u_{nj}}} 
	= \e^{u_{ni}-u_{nj}},
\end{equation}
which gives that
\begin{equation}
  P_{ni} = \e^{u_{ni}-u_{nj}}P_{nj}.
\end{equation}
This allows us to calculate the probabilities
of each alternative in turn and handle overflows more easily since the 
calculation of the possibly large denominator is avoided. 
The logit computation is then performed as described below:

\begin{algtab}
\alglet $\varepsilon>0$ be the smallest noticeable difference from zero
    in a Java \texttt{double} \\
\alglet $\Omega\gg\varepsilon$ be the largest possible number that can be stored
    in a Java \texttt{double} \\
\algforto{$g=1$}{$g=G$}
	// for each group of identical decision makers with identical alternatives\\
	\algforto{$i=1$}{$i=I_g$}
		// for each alternative $i$ of decision maker group $g$\\
		// calculate the utility of $i$ for $g$\\
		\algtry{(\alglet $u_{gi} = \sum_{k=1}^{K} \beta_{ik}x_{gik}$)}\\
		\algif{try clause causes positive overflow}
			\alglet $u_{gi} = \Omega$\\
		\algelseif{try clause causes negative overflow}
			\alglet $u_{gi} = -\Omega$\\
		\algelsif{try clause causes underflow, i.e. too close to zero}
			\alglet $u_{gi} = 0$\\
		\algendif
	\algendfor
	find the largest utility value, $u_{\max}=u_{gj} \geq u_{gi}$ where
	$i=1,\ldots,I_g$ and $i\neq j$ \\
	\alglet $\mathit{temp} = u_{g1}$ \\
	\alglet $u_{g1} = u_{\max}$ \\
	\alglet $u_{gj} = \mathit{temp}$ \\
	// invariant: the maximum utility is first and
	the previous first utility is wherever the maximum utility was \\
%	sort the alternatives by utility, $u_{gi}$, from lowest to
%		largest\\
%	renumber indices so that $i=1$ is the index for the alternative
%   		with minimum utility and $i=I_g$ is the index for the alternative with
%   		maximum utility. 
%  		The order of equal values does not matter\\
%  	// invariant: $u_{g1} \leq \ldots \leq u_{gi} 
%  		\leq \ldots u_{gI_g}$ \\
	\alglet $P^*_{g1} = 1$ // arbitrarily assign the probability of the
     		alternative with the largest utility \\
   	\alglet $D = P^*_{g1}$ \\
	\algforto{$i=2$}{$i=I_g$}
		// for each alternative \\
		// calculate non-normalized probabilities \\
%		\algif{$u_{gi} == u_{g,i-1}$}
%			\alglet $P^*_{gi} = P^*_{g,i-1}$ \\
%		\algelsif{($P^*_{g,i-1} == \varepsilon$) 
%	    	\algor ($\vc{X}_{gi}\vc{\beta}_{i}
%	      		-\vc{X}_{gi+1}\vc{\beta}_{i+1}<-708$)}
%	      	\alglet $P^*_{gi} = \varepsilon$ \\
%	    \algelse
%	    	\algtry (\alglet $P^*_{gi} = 
%	    		\e^{\vc{X}_{gi}\vc{\beta}_{i} - \vc{X}_{gi+1}\vc{\beta}_{i+1}}
%				\cdot P^*_{gi+1}$)\\
%			\algif{try causes underflow}
%				\alglet $P^*_{gi} = \varepsilon$ \\
%			\algendif
%		\algendif

		\algtry{(\alglet $U = \vc{X}_{gi}\vc{\beta}_{i} - 
			\vc{X}_{g1}\vc{\beta}_{1}$ )} \\
		// invariant: $U \leq 0$ \\
		\algif{try clause causes underflow, i.e. $U\lessapprox 0$}
			\alglet $P^*_{gi} = 1$ \\
		\algelseif{ try clause causes negative overflow, i.e. $U <
		\Omega$, \algor $U < -708$}
			\alglet $P^*_{gi} = \varepsilon$ \\
		\algelse
			\alglet $P^*_{gi} = \e^{U}$ \\
		\algendif
		\alglet $D = D + P^*_{gi}$ \\
		// $D$ will not overflow unless $I_g > \Omega$ \\

	\algendfor
	\algforto{$i=1$}{$i=I_g$}
		// normalize probabilities \\
		\alglet $P_{gi} = \frac{P^*_{gi}}{D}$ \\
	\algendfor
	// invariant: have probabilities for all alternatives
    	$i=1,\ldots,I_g$ for decision maker group $g$ and $\sum_{i=1}^{I_g}
    	P_{gi}=1\pm\delta$, where $\delta>0$ is a tolerated rounding error \\

	\algforto{$n=1$}{$n=N_g$}
		// Select a particular alternative $i$ as the choice of individual decision
		maker $n$ in decision maker group $g$ \\
		\alglet $r\sim U(0,1)$ be a number drawn from a uniform random distribution
			between 0 and 1, inclusive \\
		\alglet $i=1$ \\
		\algwhile{$i< I_g$ \algand $r > P_{ni}$}
			\alglet $r = r - P_{ni}$ \\
			\alglet $i = i + 1$ \\
		\algendwhile
		// invariant: $i$ is the index of the chosen alternative for decision
			maker $n$\\
		assign the choice $i$ to individual $n$ \\
	\algendfor
\algendfor
	
\end{algtab}




\end{document}


